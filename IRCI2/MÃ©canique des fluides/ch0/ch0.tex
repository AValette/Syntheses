\chapter{Introduction}

\section*{Les deux approches}
    Afin d'aborder la mécanique des fluides, deux approches sont possibles :
    \begin{enumerate}
    \item Approche de d'Alembert
    \item Approche d'Euler
    \end{enumerate}
    
    La première approche, celle de d'Alemblert consiste à retrouver les équations de mouvements
    et d'après y appliquer les théorèmes généraux (Cf. \textit{Mécanique Rationnelle II}) en
    partant d'une série de postulats : les postulats de d'Alembert. De façon synthétique :
    $$ \text{Travaux/puissances virtuels}  \Rightarrow \text{Équations
    de mouvements} \Rightarrow \text{Théorèmes généraux}$$
    La seconde approche, celle d'Euler, est celle qui sera suivie tout au long de ce cours. Elle
    se présente de façon synthétique de la sorte :
    $$ \text{Théorèmes généraux}  \Rightarrow \text{Équations
    de mouvements} \Rightarrow \text{Travaux/puissances virtuels}$$
    
    
\section*{Hypothèses de continuité}  
    Un milieu est \textit{continu} si on peut définir mathématiquement des densités de propriétés
    physiques qui sont des \textit{fonctions continues} des coordonnées spatiales.\\ L'échelle des
    problèmes à traiter peut avoir son importance : depuis l'espace, le sable peut être vu comme
    un milieu continu, mais pas à "échelle humaine".\\
    
    Il existe une série d'hypothèses de continuité. Une des plus importantes peut s'énoncer : 
    \textit{deux points matériels infiniment voisins à l'instant $t_0$ restent infiniment voisins
    à tout instant $t > t_0$ et réciproquement.} Selon ce principe, un cylindre n'est pas continu ;
    je peux en effet rejoindre les extrémites.\\
    
    Il existe bien d'autres hypothèses de continuité :
    \begin{itemize}
    \item Continuité de la transformation par rapport au temps.
    \item Tout ensemble continu de points matériels à l'instant $t_0$ reste continu à l'instant $t$
    et réciproquement. Un ensemble fermé reste fermé.
    \item Les points matériels à l'intérieur d'une surface fermée restent à l'intérieur de cette
    surface déformée (surface matérielle) qui ne contiendra aucun autre point matériel.
    \item A l'intérieur d'une surface matérielle fermée, la quantité de masse est constante au cours
    du temps
    \item Les points matériels constituant la frontière d'un milieu continu à un instant $t_0$, en 
    forment la frontière en tout autre instant $t$.
    \end{itemize}
    
    Ces hypothèses montrent que des trous, fissures, failles, sillages et des chocs sont des phénomènes
    ne respectant pas les hypothèses de continuité.
    
\section*{Notion de base et hypothèses}
    La masse est distribuée de façon continue : 
    \begin{equation}
     \rho = \lim\limits_{V \rightarrow 0} \frac{M}{V}
    \end{equation}
    où $\rho$ est la masse volumique, $M$ la masse et $V$ le volume. Cette relation nous montre que
    le béton armé n'est pas continu car il faudrait considérer deux $\rho$ différents.\\
    
    Il existe deux types de forces :
    \begin{description}
    \item[À distance]; ce sont les forces réparties en volume. Le poids d'une table ne vient pas d'une
    force sur la table mais de la gravité.
    \item[De contact]; ce sont les forces réparties en surface, par exemple moi appuyant sur une
    table. Je ne peux ainsi "causer" que ce type de forces, et jamais celles à distance.
     \end{description}
     
\section*{Définition de la résultante cinétique}
    La quantité de mouvement élémentaire est définie comme $\vec v dm = \vec{v}\rho dV$. Celle-ci 
    permet de définir la résulstante cinétique :
    \begin{equation}
    \vec{\mathcal{R}} = \int_M \vec{v}\ dm = \int_M \vec{v}\rho\ dV
    \end{equation}
    
\section*{Définition du moment cinétique}
    Le moment cinétique (ou "moment des quantités de mouvement") est défini par rapport à un point 
    fixe. Son expression est :
    \begin{equation}
    \vec{\mathcal{M}} = \int_M \vec{r}\times\vec{v}\ dm = \int_M \vec{r}\times\vec{v}\rho\ dV 
    \end{equation}