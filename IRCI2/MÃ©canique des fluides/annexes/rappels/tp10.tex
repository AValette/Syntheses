%%%%%%%
%  TP 10   %
%%%%%%%

\section*{TP 10 : Fluides visqueux}
\subsection*{Loi de comportement}
La relation générale dans le cours est
\begin{equation}
	\tau _{ij} = -p\delta _{ij}+\lambda \delta _{ij} V_{kk} +2\mu V_{ij}
\end{equation}
En faisant \textbf{l'hypotèse de Stokes} : $\lambda = -\frac{2}{3}\mu$
\begin{equation}
	\tau _{ij} = -p\delta _{ij}+2\mu (V_{ij} -\frac{1}{3} \delta_{ij}V_{kk})
\end{equation}
où $\mu$ est le coefficient de vicosité dynamique $[kg/m.s]$ 
Définissons également le coefficient de viscosité \textbf{cinélmattique} $\nu$ $[m^2/s]$ tel que 
\begin{equation}
	\nu = \frac{\mu}{\rho}
\end{equation}

\subsection*{Equation de Navier-Stokes}
En faisant l'hyspothèse que le fluide est un fluide visqueux newtonien
\begin{equation}
	\rho (\partial _0 v_i + v_{kk}\partial _k v_i) = f_i-\partial _i p + \mu \partial _j \partial _j v_i
\end{equation}

\subsection*{Ecoulement laminaire/permanent}
\begin{itemize}
	\item Laminère : lignes de courant parallèles entre elles et rectilignes ($Re < 2000$)
	\item Turbulent : Apparition de remous ($Re > 2000)$
\end{itemize}
où $Re$ est le nombre de Reynolds définit comme
\begin{equation}
	Re = \frac{uD}{\nu}
\end{equation}
avec $u$ la vitesse débitante, $D$ le diamètre et $\nu$ le coefficient de viscosité cinématique

\subsection*{Loi de Poiseuille}
En faisant les hypothèses suivantes :
\begin{itemize}
	\item Ecoulement permanent unidimentionnel
	\item Fluide incompressible visqueux newtonien
	\item Pertes de charges linéaires. \textbf{Attention : on ne peut pas utiliser Poiseuille aux endroits de variation brusque !}
\end{itemize}
On a 
\begin{equation}
	Q = \frac{\pi R^4 \Delta \hat{p}}{8\mu L} \qquad avec \qquad \hat{p} = p+\rho g z \quad et \quad \Delta \hat{p} = \hat{p}_1 - \hat{p}_2
\end{equation}
où $R$ est le rayon de la conduite, $\Delta \hat{p}$ la différence de pression, $L$ la longueur de l'écoulement, $Q$ le débit et $\mu$ le coefficient de viscosité