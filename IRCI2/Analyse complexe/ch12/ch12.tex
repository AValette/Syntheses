\chapter{Transformée de Laplace 2}
\section{Calcul de la transformée de Laplace et de la transformée inverse}
	
\subsection{Calcul de la transformée de Laplace}
\subsubsection{A partir de la défintion}
Voici un petit tableau qu'il peut être intéressant de connaître par cœur :
\begin{center}
	\begin{tabular}{|c|c|c|}
		\hline 
		$x(t)$                                               & $X(p)$                        & RDC            \\ 
		\hline 
		$\delta(t)$                                          & 1                             & $\mathbb{C}$   \\ 
		\hline 
		$\nu(t)$                                             & $\frac{1}{p}$                 & $Re(p)>0$      \\ 
		\hline 
		$t^n\nu(t), n \in \mathbb{N}$                        & $\frac{n!}{p^{n+1}}$          & $Re(p)>0$      \\ 
		\hline 
		$e^{-at}\nu(t), a \in \mathbb{C}$                    & $\frac{1}{p+a}$               & $Re(p)>-Re(a)$ \\ 
		\hline 
		$\sin\omega t\ \nu(t), \omega \in \mathbb{R}$        & $\frac{\omega}{p^2+\omega^2}$ &                
		$Re(p)>0$ \\ 
		\hline $\cos\omega t\ \nu(t), \omega \in \mathbb{R}$ & $\frac{p}{p^2+\omega^2}$      &                
		$Re(p)>0$ \\ 
		\hline 
	\end{tabular} 
\end{center}
		
\subsubsection{A partir des propriétés}
Quatre exemples utilisant les propriétés sont repris dans les slides 5 à 7. Je ne 
les détaille pas ici car il s'agit de matière de TP.

\subsection{Calcul de la transformée de Laplace inverse}
		
\subsubsection{Par décomposition en fractions simples (Rouge!)}
Soit $X(p), \mathfrak{Re}(p) > \alpha$ une fraction rationnelle dont le degré du 
numérateur est inférieure au degré du numérateur.
\begin{enumerate}
	\item Si tous les pôles sont de multiplicité \textbf{unitaire} : 
	      \begin{equation}
	      	X(p) = \sum_{i=1}^n \dfrac{\alpha_i}{(p-\beta_i)}\ \text{pour }\ \mathfrak{Re}(p) >
	      	\alpha\ (=\max(\mathfrak{Re}(\beta_i)))
	      \end{equation}
	      Par propriété de la linéairité et vu la RDC :
	      \begin{equation}
	      	x(t) = \sum_{i=1}^n \alpha_i e^{\beta_it}\nu(t)
	      	\label{eq:Sl8}
	      \end{equation}
	\item S'il y a des pôles de multiplicité supérieure à 1 :\\
	      Soit $\beta$, un pôle de multiplicité $\gamma$ :
	      \begin{equation}
	      	X_\beta(p) = \sum_{l=1}^\gamma \dfrac{\alpha_k}{(p-\beta)^k}\ \text{pour }\ \mathfrak{Re}(p) >
	      	\mathfrak{Re}(\beta)
	      \end{equation}
	      Par le quatrième exemple (slide 7/37) de calcul de transformée de Laplace :
	      \begin{equation}
	      	x_\beta(t) = \mathcal{L}^{-1}(X_\beta(p)) = e^{\beta t}\sum_{k=1}^\gamma \alpha_k
	      	\dfrac{t^{k-1}}{(k-1)!}\nu(t)
	      \end{equation}
\end{enumerate}


\subsubsection{Par application d'une formule d'inversion basée sur le théorèmes des 
résidus}
Nous allons séparer la trnasformée en deux parties, une se trouvant à droite de ses
pôles et l'autre à gauche de ses pôles.
Enonçons deux hypothèses : 
\begin{enumerate}
	\item Soit $X(p)$ une fraction rationnelle à coefficients réels, analytique sur $\alpha^+ <
	      \mathfrak{Re}(p) < \alpha^-$. Décomposons $X(p)$ sous la forme (gauche et droite):
	      \begin{equation}
	      	X(p) = X_g(p) +X_d(p)
	      \end{equation}
	      de sorte que les pôles de $X_g(p)$ ont une partie réelle inférieure ou égale à $\alpha^+$ 
	      et ceux de $X_d(p)$ ont une partie réelle supérieure ou égale à $\alpha^-$.
	\item Il existe $M_g, b_g, c_g,M_d, b_d, c_d \in \mathbb{R}^+$ tels que :
	      \begin{equation}
	      	\begin{array}{lll}
	      		|X_g(p)| < \frac{M_g}{|p|^{c_g}} & \text{lorsque} & |p| > b_g \\
	      		|X_d(p)| < \frac{M_d}{|p|^{c_d}} & \text{lorsque} & |p| > b_d 
	      	\end{array}
	      \end{equation}
\end{enumerate}
Si les hypothèses sont vérifiées, ont peut appliquer la méthode d'inversion :\\
		
\retenir{\textsc{Méthode d'inversion}\\
	\begin{equation}
		\begin{array}{ll}
			x(t) & = \frac{1}{2\pi i}\lim\limits_{\omega\rightarrow\infty}\int_{\sigma-i\omega}^{                 
			\sigma+i\omega} X(p)e^{pt}dp\\
			     & = \sum_{p\in p_{-g}} \text{Res}_p\{X_g(p)e^{pt}\}\nu(t) - \sum_{p\in p_{-d}} \text{Res}_p\{X_d 
			(p)e^{pt}\}\nu(-t), \ \ \ \  t\neq0, \alpha^+ < \sigma < \alpha^-
		\end{array}
	\end{equation}
	où $p_{-g}, p_{-d}$ désignent l'ensemble des pôles de $X_g(p),X_d(p)$.}\ \\
		
Pour le calcul des résidus $t$ doit être vu comme un paramètre.  La première sommation de 
la deuxième égalité donne la partie causale de la transfo de Laplace inverse et l'autre 
sommation, pour $t > 0$, la partie anticausale.\\
En obtient cette expression en procédant comme à la séance 7 des exercices, c'est-à-dire en
choisissant un arc "fermant" le contour en s'arrangeant pour que la contribution le long de
cet arc soit nul\footnote{La démonstration est dans les notes}.\\
		
\textbf{Attention !} Dans le cas d'une fonction $H(p) = \tilde{H}(p)e^{-p\tau}$ avec $\tau 
\in \mathbb{R}$ et $\tilde{H}(p)$ vérifiant nos deux hypothèses, il est conseillé de suivre
ces deux étapes :
\begin{enumerate}
	\item Appliquer la méthode d'inversion à $\tilde{H}(p)$
	\item Utiliser la propriété du glissement dans le temps pour en déduire $h(t) = \mathcal{L}^{-1}
	      (H(p))$.
\end{enumerate}



\section{Théorèmes taubériens}

\subsection{Théorème de la valeur initiale}
Trois hypothèses (regroupée en "hypothèse A") permettent d'assurer la convergence de la transformée
de Laplace :
\begin{enumerate}
	\item Il existe un réel $\sigma$ pour lequel 
	      \begin{equation}
	      	\int_{-\infty}^{\infty} x(t)e^{-\sigma t}dt
	      \end{equation}
	      converge absolument (première condition de Dirichlet)
	\item $x(t)$ possède un nombre fini d'extréma dans tout intervalle fini de valeur de $t$
	\item $x(t)$ possède un nombre fini de discontinuités de première espèce dans tout intervalle
	      fini de valeurs de $t$.
\end{enumerate}
	
\theor{\textsc{valeur initiale}\\
	Considérons une fonction causale\footnote{Une fonction causale est une fonction définie sur 
	l'ensemble des réels qui est nulle pour $t < 0$.} $x(t)$ vérifiant l'hypothèse A. 
	S'il existe un nombre complexe
	$\alpha$ tel que 
	\begin{equation}
		\lim\limits_{t\rightarrow 0^+} x(t) = \alpha,
	\end{equation}
	alors
	\begin{equation}
		\lim\limits_{\sigma\rightarrow \infty} \sigma X(\sigma) = \alpha,
	\end{equation}
	où $\sigma \in \mathbb{R}, X(p) = \mathcal{L}(x(t))$.}\ 
	
	
\corollaire{ \textsc{Important (rouge)}\\
	Soit $X(p)$ une fraction rationnelle dont le degré du numérateur est inférieur au degré du 
	dénominateur. Soit $x(t)$ la fonction causale telle que $x(t) = \mathcal{L}^{-1}(X(p))$, alors
	\begin{equation}
		\lim\limits_{\sigma \rightarrow \infty} \sigma X(\sigma) = \lim\limits_{t\rightarrow 0^+} x(t)
	\end{equation}
}
	

\subsection{Théorème de la valeur finale}	
\theor{\textsc{valeur finale}\\
	Soit une fonction $x(t)$ vérifiant l'hypothèse $A$ et telle que $x(t)=0$ pour $-\infty < t <
	T, T \in \mathbb{R}$. S'il existe un $\alpha$ réel tel que 
	\begin{equation}
		\lim\limits_{t\rightarrow\infty} x(t) = \alpha,
	\end{equation}
	alors
	\begin{equation}
		\lim\limits_{\sigma\rightarrow0^+} \sigma X(\sigma) = \alpha
	\end{equation}}\
Ce théorème est le "petit frère". On "passe" d'une information $x(t)$ à $X(p)$, mais est-t-il
possible de passer de $X(p)$ à $x(t)$ ? Oui, avec le corollaire suivant :\\

\corollaire{ \textsc{Important (rouge)}\\
	Il faut avant tout respecter deux hypothèses : avoir une fraction rationnelle multipliée par 
	une exponentielle et l'analycité assurant la convergence (vient de l'expression de la transfo
	inverse \autoref{eq:Sl8}) ;
	\begin{enumerate}
		\item Soit $X(p) = G(p)e^{ap}$ où $a \in \mathbb{R}$ et $G(p)$ est une fraction rationnelle dont
		      le degré du numérateur est inférieur au degré du dénominateur.
		\item Supposons que $pX(p)$ est \textbf{analytique pour tout $p$ tel que $\mathfrak{Re}(p)\geq0$.}
	\end{enumerate}
	Si $x(t)$ est la fonction nulle pour $t < -a$ telle que $x(t) = \mathcal{L}^{-1}(X(p))$, alors
	\begin{equation}
		\lim\limits_{t\rightarrow\infty}x(t) = \lim\limits_{\sigma\rightarrow 0^+} \sigma X(\sigma)
	\end{equation}}\
\textbf{Attention !} L'hypothèse portant sur l'analycité est impérative sans quoi on peut déduire
des conclusions totalement erronées du corollaire du théorème de la valeur finale.
	
	
	
\section{Transformée de Laplace unilatérale}
L'intérêt de la transformée de Laplace unilatérale est la résolution d'EDL avec des CI non homogènes.
Ceci permet l'étude de système non initialement au repos.
	
\subsection{Définition}
Comme d'habitude, on commence par ré-écrire les conditions de Dirichlet "adaptées" (la première
est en rouge).\\
Soit une fonction $x(t)$ qui vérifie les conditions suivantes :
\begin{enumerate}
	\item Il existe un $\alpha \in \mathbb{R}$ tq $\int_0^\infty |x(t)|e^{-\sigma t} < \infty\ \ 
	      \forall \sigma > \alpha$.
	\item $x(t)$ possède un nombre fini d'extrema dans tout intervalle fini de valeurs de $t$, avec
	      $t\geq 0$
	\item $x(t)$ possède un nombre fini de discontinuités de première espèce dans tout intervalle
	      fini de valeurs de $t$, avec $t\geq 0$.
\end{enumerate}
La \textbf{transformée de Laplace unilatérale}
\begin{equation}
	\mathcal{X}(p) = \mathcal{L}_u(x(t)) = \int_0^\infty x(t)e^{-pt}dt
\end{equation}
converge pour $\mathfrak{Re}(p)>\alpha$.
	
\subsection{Transformée inverse}
\retenir{\begin{equation}
	x(t) = \mathcal{L}_u^{-1}(\mathcal{X}(p)) = \frac{1}{2\pi i}\int_{\sigma-i\infty}^{\sigma
		+i\infty} \mathcal{X}(p)e^{pt}dp
	\end{equation}
	existe $\forall \sigma> \alpha$ et $t>0$	 sauf aux points de discontinuité de $x(t)$.\\
	On la note
	\begin{equation}
		x(t) \overset{\mathcal{L}_u}{\longleftrightarrow} \mathcal{X}(p)
	\end{equation}}
	
Notons une propriété : \textit{pour une fonction causale, la transformée de Laplace unilatérale  
est	égale à la transformée de Laplace bilatérale.}
	
	
\subsection{Application à l'impulsion de Dirac}
Le zéro doit-il appartenir à l'intervalle d'intégration ? En adaptant la définition, on trouve
(en rouge encadré) :
\begin{equation}
	\mathcal{X}(p) = \mathcal{L}_u(x(t)) = \int_{0^-}^\infty x(t)e^{-pt}dt
\end{equation}
L'application de ceci est directe : $\mathcal{L}_u(\delta(t)) = 1$.
	
	
\subsection{Propriétés}
\subsubsection{Transformée de Laplace unilatérale de la dérivée d'une fonction}
En intégrant par parties 
\begin{equation}
	\int_{0^-}^\infty \frac{dx}{dt}e^{-pt}dt = x(t)e^{-pt}|_{0^-}^{\infty} + p\int_{0^-}^{
		\infty} x(t)e^{-pt}dt
\end{equation}
soit (en rouge, encadré),
\begin{equation}
	\mathcal{L}_u\left(\frac{dx}{dt}\right) = p\mathcal{X}(p) - x(0^-)
\end{equation}
\textbf{pour une RDC incluant $\mathfrak{Re}(p)>\alpha$} où $\alpha$ est l'abscisse de
convergence absolue de $\mathcal{X}(p)$.
		
\subsubsection{Dérivée seconde}
Juste encadré cette fois hyhy :
\begin{equation}
	\mathcal{L}_u\left(\frac{d^2x}{dt^2}\right) = p^2\mathcal{X}(p) - px(0^-) - \left.\frac{
		dx}{dt}\right|_{t=0^-}
\end{equation}
pour une RDC incluant $\mathfrak{Re}(p) > \alpha$.
		
\subsubsection{Rapidos}
\begin{itemize}
	\item Convolution et propriété du glissement en $t$ requièrent fonctions causales.
	\item Transformée de Laplace bilatérale $X(p)$ peut être remplacée par transformée 
	      unilatérale $X(p)$ dans les théorèmes de la valeur initiale et de la valeur finale 
	      ainsi que dans leurs corollaires.
\end{itemize}

\subsection{Exemple - Circuit RC}
Comme en BA1 et en électricité, ça devrait aller (slide 30/31).
	

\section{Résolution d'équations aux dérivées partielles}
Toujours comme en BA1 et en électricité, ça devrait aller (slide 33/37).
