%%%%%%%%%%%%%%%%%%
% Ch1 : Architecture atomique %
%%%%%%%%%%%%%%%%%%

\chapter{Architecture atomique}
\section{Atome : fiche technique}
	\noindent Commençons par un bref rappel des propriétés principales : 
	\begin{itemize}
	\item Le noyeau est formé de \textbf{protons} et \textbf{neutrons}, appelés \textbf{nucléons} et entouré d'\textbf{électrons} qui se déplacent dans des \textbf{orbitales} dictées par la fonction d'onde.
	\item Le \textbf{nombre atomique Z} désigne le nombre d'électrons = le nombre de protons. 
	\item Le nombre de neutrons peut différer pour un même atome donnant lieu à des \textbf{isotopes}.
	\item La masse d'un nucléon est de $\mathbf{1.66 \, 10^{-24} \,g}$ et celle de l'électron de $\mathbf{9.1 \, 10^{-31} \,g}$ (rapport 1/1800).
	\item La charge d'un électron est de $\mathbf{1.602 \, 10^{-19} \, g}$.
	\item L'unité de \textbf{mole} correspond à $\mathbf{6.02 \, 10^{23}}$ \textbf{particules} (nombre d'avogadro) et une mole de nucléons pèse $\mathbf{1\, g}$.
	\end{itemize}
	
\section{Comportement ondulatoire de l'électron}

\subsection{Dualité onde/corpuscule}
	\noindent Etudié en long et en large dans le cours de \emph{Physique quantique II}, on sait grâce à \textbf{de Broglie} qu'une onde est associée à toute particule et sa longueur d'onde est donnée par 
	\begin{equation}
	\lambda = \frac{h}{p} = \frac{h}{mv} \qquad \mathbf{h = 6.62 \, 10^{-34} \, J.s} \mbox{ (contante de Planck)}	
	\end{equation}
	\noindent En 1927, Davisson et Germer mettent en évidence expérimentalement cet aspect grâce au phénomène d'\textbf{interférence}. \\
	En 1926, \textbf{Schrödinger} proposera que l'onde associée à un électron en mouvement résulte de la variation périodique d'une fonction $\psi$ appelée \textbf{fonction d'onde}. Les électrons seront alors dans des orbitales occupant un volume de l'espace et non plus dans des orbites circulaires.  
	
	\noindent De plus, les états énergétiques correspondent à des ondes stationnaires qu'on peut caractériser par l'équation 
	\begin{equation}
	\frac{d^2}{dx^2}a + \frac{4\pi ^2}{\lambda ^2}a = 0
	\label{equation:1.2}
	\end{equation}	 
	qui régit également les oscillations d'une corde fixée à ses deux extrémités, à une dimension. L'amplitude $a$ varie en fonction de la position $x$ sur la corde pour un $\lambda$ donné. \\
	En se rappelant que $E_{cin} = E-V = \frac{1}{2}mv^2$ et en remplaçant l'amplitude de \eqref{equation:1.2} par $\psi$, on retrouve l'équation de Schrödinger à une dimension 
	\begin{equation}
	-\frac{\hbar ^2}{2m}\frac{d^2}{dx^2}\psi = (E-V)\psi
	\end{equation}
	Rappelons-nous qu'on obtient la \textbf{probabilité de présence} avec $\int |\psi| ^2 \, dV.$
	
\subsection{Orbitales et nombres quantiques}
	\noindent On sait donc que chaque atome possède un certain nombre d'orbitales électroniques caractérisées par des valeurs de l'énergie. Seules les orbitales de plus \textbf{basses énergies} sont occupées. Ces dernières sont définies par 4 nombres quantique :
	\begin{itemize}
	\item \textbf{Le nombre quantique principal n} $(n>0)$ qui fixe la \textbf{taille} de l'orbital. On associe les lettres K, L, M,\dots \ à $n=1,2,3, \dots$
	\item \textbf{Le nombe quantique azimutal l} $(0<l<n-1)$ qui fixe la forme de l'orbitale. Les sous-niveaux énergétique $l = 0,1,2, \dots$ sont désingnés par les lettres s (sphérique), p, d, f,\dots
	\item \textbf{Le nombre quantique magnétique } $\mathbf{m_l}$ $(-l <m_l<l)$
	\end{itemize}