%%%%%%
%  TP11 %
%%%%%%

\section*{TP 11 - 12 : Loi de la résultante cinétique}
\subsection*{Enoncé}
La dérivée de la résultante cinétique est égale à la somme des forces appliquées
\begin{equation}
	\left( \int _V \rho v_i dV \right) ^\bullet = \int f_i dV + \oint  _S T_i^{(n)} dS
\end{equation}
Rappel : $f^\bullet = \partial _0 f + v_k \partial _k f$
\begin{equation}
	\int _V \partial _0 (\rho v_i) dV = \int f_i dV + \oint _S (T_{ij} - \rho v_i v_j )n_j dS
\end{equation}

\subsection*{Hypothèses simplificatrices}
\begin{itemize}
	\item Ecoulement permanent : $\partial _0 (\rho v_i) = 0$
	\item Poids propres négligé : $\int _V f_i dV = 0$
	\item Fluide parfait : $\tau _{ij} = -p \delta _{ij}$
\end{itemize}
Calcul de l'action d'un fluide sur un obstacle (on pose ceci lorqu'on à l'intégrale de surfaces sur les bords)
\begin{equation}
	r_i = - \int _{S \ obstacle} T_i ^{(n)} dS
\end{equation}

\subsection*{Avantages}
On peut calculer l'action d'un fluide sur un obstacle sans connaître 
\begin{itemize}
	\item la symétrie exacte de l'obstacle (la surface d'entrée et de sortie suffisent)
	\item la répartition des pressions le long de la surface de contact fluide/obstacle (pression d'entrée et de sortie suffisent)
\end{itemize}

\subsection*{Etapes}
\begin{enumerate}
	\item Redessiner la portion du système et y repérer les vecteurs de vitesses entrante et sortante, le système d'axe et les vecteurs normaux à l'entrée, à la sortie et sur les parois de la portion de circuit
	\item Exprimer les coordonnées des différents vecteurs selon le sytème d'axe (vitesses et vecteurs normaux)
	\item Ecrire la loi, la simplifier et isoler les $r_i$ défini plus haut
	\item Déterminer les composantes dans les axes de $r_i$
\end{enumerate}