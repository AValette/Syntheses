\chapter{Convolution - Réponse d'un SLP}
\section{Réponse d'un SLP}
	\subsection{Représentation d'un signal continu en termes d'impulsions}
	Avant toute chose, l'impulsion de Dirac peut elle aussi \^etre approximée :
	\begin{equation}
	\delta_\epsilon(t) = \left\{\begin{array}{ll}
	\frac{1}{\epsilon} & 0\leq t \leq \epsilon\\
	0 & \text{sinon}
	\end{array}\right.
	\end{equation}
	La méthode utilisée ici est la m\^eme que celle utilisée pour obtenir les 
	intégrales de Riemann. Considérons un signal $u(t)$ approché par la 
	fonction en escalier $\hat{u}(t)$ :
	\begin{equation}
	\hat{u}(t) = \sum_{k=-\infty}^\infty u(k\epsilon)\delta_\epsilon(t-k\epsilon)\epsilon
	\end{equation}
	où un seul terme est non nul dans la somme, pour chaque valeur de $t$. 
	Lorsque $\epsilon\rightarrow 0, \hat{u}(t) \rightarrow u(t)$ et donc :
	\begin{equation}
	u(t) = \lim\limits_{\epsilon\rightarrow 0} \sum_{k=-\infty}^\infty u(k
	\epsilon)\delta_\epsilon(t-k\epsilon)\epsilon
	\end{equation}
	Cette somme tend gentillement vers une intégrale 
	\begin{equation}
	u(t) = \int_{-\infty}^\infty u(\tau)\delta(t-\tau)\ d\tau
	\end{equation}


	\subsection{Réponse impulsionnelle d'un SLP et convolution}
	La démarche est de calculer la réponse $\hat{y}(t)$ d'un SLP au signal
	$\hat{u}(t)$ puis d'ensuite faire tendre $\epsilon$ vers zéro pour 
	déduire la réponse $y(t)$ à $u(t)$.\\
	Notons $\hat{h}_{k\epsilon}(t)$ la réponse d'un système linéaire continu 
	au signal d'entrée $\delta_\epsilon(t-k\epsilon)$. Par les propriétés de 
	superposition et d'homothétie, je peux exprimer $\hat{y}(t)$ comme la 
	somme des réponse de chacune de ses impulsions. La somme approchée est 
	donnée par : 
	\begin{equation}
	\hat{y}(t) = \sum_{k=-\infty}^\infty u(k\epsilon)\hat{h}_{k\epsilon}
	\epsilon
	\end{equation}
	Lorsque $\epsilon\rightarrow0, \hat{u}(t)\rightarrow u(t)$ et $\hat{y}
	(t) \rightarrow y(t)$. Si $\epsilon$ est suffisamment petit, la durée de 
	l'impulsion $\delta_\epsilon(t-k\epsilon)$ n'importe pas. Notons $h_\tau(t
	)$ la réponse du système au temps $t$ à l'entrée $\delta(t-\tau)$ (l'
	impulsion à l'instant $\tau$). Il vient : 
	\begin{equation}
	\begin{array}{ll}
	y(t) &= \lim\limits_{\epsilon\rightarrow0} \sum_{k=-\infty}^\infty u(k\epsilon)
	\hat{h}_{k\epsilon}(t)\epsilon\\
	 &= \int_{-\infty}^\infty u(\tau)h_\tau(t)\ d\tau
	\end{array}
	\label{eq:RepSLP}
	\end{equation}
	
		 \subsubsection{Convolution}
		 Si en plus d'\^etre linéaire, le système est permanent, alors $h_\tau(
		 t) = h(t-\tau)$ où $h(t)$ est la réponse du système à l'entrée $\delta
		 (t)$.\\
		 \textbf{Définition} : réponse impulsionnelle d'un SLP := réponse du
		 système au signal d'entrée $\delta(t)$, notée $h(t)$. En substituant 
		 $h(t)$ par $h(t-\tau)$ dans l'équation \autoref{eq:RepSLP} on 
		 trouve\footnote{Par définition} :
		 \begin{equation}
		 y(t) = \int_{-\infty}^\infty u(\tau)h_\tau(t-\tau)\ d\tau
		 \end{equation}
		 c'est à dire la \textbf{convolution} ou \textbf{produit de convolution} 
		 de $u(t)$ et $h(t)$.\\
		 On utilise comme notation pour le produit de convolution $y(t) = u(t)*
		 h(t)$. Tout ce qui est nécessaire pour la sortie, c'est la réponse 
		 impulsionnelle. Cette notion caractérise entièrement le SLP.
	
	\subsection{Réponse indicielle d'un SLP}
	La réponse indicielle d'un SLP correspond à sa réponse l'entrée $u(t) = \nu
	(t)$ (fonction d'Heaviside).\\
	Grâce à la commutativité de la convolution, je peux écrire mon intégrale de
	la sorte :
	\begin{equation}
	s(t) = \int_{-\infty}^\infty h(\tau)\nu(t-\tau)\ d\tau = \int_{-\infty}^t h(
	\tau)\ d\tau
	\label{eq:RepIndic}
	\end{equation}
	où $s$ est appelé la réponse indicielle. Cette forme se justifie par le fait que 
	pour les $\tau > t$, l'intégrale est nulle. L'équation \autoref{eq:RepIndic} 
	permet d'écrire :
	\begin{equation}
	h(t) = \frac{ds(t)}{dt}
	\end{equation}
	En général, il est plus facile de déterminer $s(t)$ que $h(t)$ (
	expérimentalement).


\section{Système décrit par une équation différentielle ordinaire}
	\subsection{Formulation du problème}
	Considérons la loi décrivant le comportement d'un circuit RC :
	\begin{equation}
	\frac{dv_c(t)}{dt}+\frac{1}{RC}v_c(t) = v_s(t)
	\end{equation}		
	Cette équation est-elle suffisante pour déterminer $v_c(t)$ de façon 
	explicite ? On observe que :
	\begin{itemize}
	\item Une EDO fournit une description implicite du système au lieu d'une 
	expression explicite de la sortie en fonction de l'entrée.
	\item La résolution d'une EDO nécessite des conditions auxiliaires
	\item Une EDO décrit uniquement une contrainte liant l'entrée et la
	sortie d'un système.
	\end{itemize}
	En posant ainsi des conditions (souvent initiale) et en procédant 
	comme vu en \textit{Analyse I, II} on peut trouver la solution.
	
	\subsection{EDO et SLP}
	Considérons maintenant une EDO d'ordre $n$ où $y$ et $u$ sont fonctions 
	de $t$ :
	\begin{equation}
	y^{(n)} + a_1y^{(n-1)} + \dots + a_{n-1}y' + a_ny = b_0u^{(m)}+
	b_1u^{(m-1)} + \dots + b_{m-1}u' + b_mu
	\end{equation}
	Elle représente un SLP si les CI correspondent à un système initialement
	au repos.\\
	\textbf{Def :} un système est initialement au repos si lorsque $u(t)=0\ 
	\forall t<t_0, y(t) = 0\ \forall t<t_0$. Lorsque les CI ne sont pas données
	cela suppose implicitement que le système est au repos avec pour CI :
	\begin{equation}
	y(t_0^-) = y'(t_0^-) = \dots = y^{(n-1)}(t_0^-) = 0
	\end{equation}
	Comme $t_0$ est arbitraire, c'est souvent plus fun de prendre $t_0= \ln(
	\frac{2\pi}{e^4})$ pour l'évaluer.\\
	
	Déterminer la solution $y(t),\ t\geq0$ d'un tel problème pour l'entrée 
	$u(t) = \delta(t)$ n'est pas toujours simple : la transformée de Laplace
	fournira une méthode simple et systématique pour résoudre le problème.
	
\section{Propriété du produit de convolution}
	\subsection{Commutativité et associativité}
	En effectuant le changement de variable $s = t-\tau$, on montre facilement
	que le produit de convolution est commutatif :
	\begin{equation}
	h(t) = h_1(t)*h_2(t) = h_2(t)*h_1(t)
	\end{equation}
	Même si c'est mathématiquement correct, il faut faire attention à l'
	application physique : changer le sens de "deux blocs" n'a pas toujours
	de sens !	
	
	\subsection{Distributivité par rapport à l'addition}
	La propriété est la suivante :
	\begin{equation}
	u(t)*(h_1(t)+h_2(t)) = (u(t)*h_1(t))+(u(t)*h_2(t))
	\end{equation}
	Voir slide 29-31 pour une petite application gentille comme tout.
	
	\subsection{Existence d'un élément neutre pour le produit de convolution}
	L'élément neutre pour le produit est l'impulsion de Dirac. En effet :
	\begin{equation}
	u(t) = \int_{-\infty}^\infty u(\tau)\delta(t-\tau)d\tau = u(t)*\delta(t)
	\end{equation}

































