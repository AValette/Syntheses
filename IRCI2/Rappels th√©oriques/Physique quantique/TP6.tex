
\section*{TP6 : Système hydrogénoïde}

\begin{itemize}
	\item Unité : le Rydeberg
	      \begin{equation}
	      	a_0 = \frac{4\pi \epsilon _0 \hbar ^2}{m_e e^2} = 0,529 .10^{-7} \, mm \qquad Ryd = \frac{\hbar ^2}{2m_e a_0^2} = \frac{e^2}{8 \pi \epsilon_0 a_0} = 13,6 \, eV
	      \end{equation}
	      		
	\item Fonction d'onde 
	      \begin{equation}
	      	\psi _{Hyd} (\vec{R}) = R_{nl} (r) Y^m_l (\theta , \varphi )
	      \end{equation}
	      où, pour l'état 1s de l'hydrogène seulement (les autres sont donnés par une formule dégueu qui sera sans doute donnée)
	      \begin{equation}
	      	R_{nl} = 2\left( \frac{Z}{a_0} \right) ^{3/2} e^{-\frac{Zr}{a_0}}
	      \end{equation}
	      		
	\item Intégrale de normalisation \\
	      Lorsqu'on nous demande de vérifier si les fonctions $R_{nl}$ et $Y^m_l$ sont normées, il faut intégrer respectivement selon 
	      \begin{equation}
	      	r^2 dr \qquad et \qquad \sin \theta \, d\theta \, d\varphi
	      \end{equation}
	\item Energie quantifié 
	      \begin{equation}
	      	E_n = -\frac{Z^2}{n^2}Ryd
	      \end{equation}
	      avec 
	      \begin{itemize}
	      	\item $n = n_R +l +1 \geq 1$
	      	\item $l \in \mathbb{N} = 0,1,2, \dots \Rightarrow$ couches s, p, d, f, g, $\dots$		
	      	\item $m \in [-l,l]$
	      \end{itemize} 
	      	
	\item Système hydrogénoïde
	      \begin{equation}
	      	a_0, Ryd \qquad \Leftrightarrow \qquad a_\mu = \frac{m_e}{\mu } a_0, Ryd_\mu = \frac{\mu }{m_e} Ryd
	      \end{equation}
	      avec $\frac{1}{\mu} = \frac{1}{m_1}+\frac{1}{m_2}+ \dots$
\end{itemize}