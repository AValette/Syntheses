\chapter{Seconde quantification}
Considérons un état $\ket{\psi}$ et un opérateur $\hat{A}$. Nous avions précédemment discuté de l'image
de Schrödinger et d'Heisenberg mais dans les deux cas, nous avions à notre disposition un vecteur d'état 
et un observable.\\

Pour la seconde quantification, nous allons uniquement nous baser sur  les opérateurs linéaires, mais aussi 
les opérateurs de création et d'inhalation. Ce formalisme a plusieurs avantage: il prend compte 
"automatiquement" du fait que la fonction soit symétrique ou non, il s'utilise de la même façon que 
l'on travaille avec des bosons et des fermions mais aussi il est possible de faire varier le nombre 
de particules $N$ sans recommencer tous les calculs : $\hat{N}$ devient un observable (grosse différence 
par rapport à la première quantification). Notons que ce formalisme convient très bien au traitement 
des champs et des particules identiques.\\

Commençons par définir un ensemble de $N$ état
\begin{equation}
\{\ket{i}\} = \{\ket 1,\ket 2,\dots\}
\end{equation}
Adoptons une notation efficace :
\begin{equation}
\ket{i_1, i_2,\dots,i_N} \equiv \ket{i_1}_1\ket{i_2}_2\dots\ket{i_N}_N
\end{equation}
où $i_N$ correspond ici à la $N^e$ particule dans un état $N$. Nous formons donc une "liste" qui nous 
informe sur quelle particule occupe quel état. Pour former un état totalement symétrique
ou anti-symétrique, il suffit d'y appliquer $\hat{S}_\pm$ :
\begin{equation}
\hat{S}_\pm \ket{i_1, i_2,\dots,i_N} = \dfrac{1}{\sqrt{N!}}\sum(-1)^p\hat{p}
\end{equation}
Si on applique l'opérateur permutation à ceci
\begin{equation}
\hat{P}\hat{S}_\pm \ket{i_1, i_2,\dots,i_N} = (\pm 1)^p\dfrac{1}{\sqrt{N!}}\sum(-1)^p\hat{p}
\end{equation}
Ceci dépend de la parité : si l'on considère des fermions, nous aurons l'apparition d'un signe négatif. Cet 
objet à la propriété d'être totalement symétrique ou totalement antisymétrique.\\

Considérons que dans notre liste d'états occupés, le quatrième niveau est occupé par deux particules (la 
particule 3 et la particule 4)
\begin{center}
Inclure schéma cours 4, sous (21)
\end{center}
On défini le nombre d'occupation $n_i=m$ comme le nombre de particule dans un état $\ket{i}$. Le nombre de 
termes distincs est donné par
\begin{equation}
\dfrac{N!}{n_1!n_2!\dots}
\end{equation}
Ceci signifie que lorsque l'on écrit l'état totalement symétrique des bosons, le pré-facteur devient
\begin{equation}
\ket{\psi_{N,boson}} = \dfrac{1}{\sqrt{n_1!n_2!\dots}}\hat{S}_+\ket{i_1, i_2,\dots,i_N}
\end{equation}
et pour les fermions
\begin{equation}
\ket{\psi_{N,fermion}} = \dfrac{1}{\sqrt{n_1!n_2!\dots}}\hat{S}_-\ket{i_1, i_2,\dots,i_N}
\end{equation}
La subtilité est au niveau du nombre d'occupation qui ne peut valeur que 0 ou 1 pour les fermions ($n_i=0,1$) 
et tout pour les bosons ($n_i=0,1,2,\dots$).


\section{Système de $N$ bosons}
\subsection{Espace de Fock, opérateurs création/annihilation, opérateur nombre}
\subsubsection{Espace de Fock}
Toute l'idée est que lorsque l'on considère un système de $N$ bosons, nous n'allons pas le décrire comme
nous l'avions fait avec notre notation efficace. A la place, nous allons lister les orbitales et renseigner, 
pour chacune d'entre-elles, quelles particules les occupent.
\begin{equation}
\text{Etat de Fock }\ \equiv \ket{n_1,n_2,\dots} \equiv \dfrac{1}{\sqrt{n_1!\dots n_N!}}\hat{S}_+
\ket{i_1}_1\ket{i_2}_2\dots\ket{i_N}_N
\end{equation}
\danger\ A gauche, le numéro désigne les orbitales (il y en a une infinité, d'où les \dots). Par contre, le 
numéro du membre de droite correspond à la particules (que nous avons en nombre fini, $N$). Nous avons 
comme contrainte ($n_i=0,1,\dots$)
\begin{equation}
\sum_i n_i = N
\end{equation}
L'ensemble de tous les états de Fock forme l'espace de Fock : ceci forme une base complète d'espaces complètement 
symétrique. Cette base est donc complètement symétrique.\\

Considérons deux états de Fock, chacun caractérisé par une liste d'orbitale. Ces états sont bien orthogonaux
\begin{equation}
\bra{n_1,n_2,\dots}\ket{n_1',n_2',\dots} = \delta_{n_1,n_1'}\delta_{n_2,n_2'}\dots
\end{equation}
Nous avons également la relation de fermeture
\begin{equation}
\sum_{n_1,n_2} = \ket{n_1,n_2,\dots}\bra{n_1,n_2,\dots} =\hat{1}
\end{equation}

\subsection{Opérateur création}
L'opérateur création se note $\hat{a_i}^\dagger$ où la présence du dagger se justifie par le fait que son conjugué sera 
l'opérateur d'annihilation. Par définition
\begin{equation}
\hat{a}_i^\dagger \ket{n_1\dots n_i\dots} = \sqrt{n_i+1}\ket{n_1\dots n_i+1\dots}
\end{equation}
Regardons ce qu'est l'adjoint de cet opérateur afin d'obtenir l'expression de l’opérateur d'annihilation. Focalisons-nous, 
comme pour la définition, sur l'orbitale $i$
\begin{equation}
\bra{\dots n_i'\dots}\hat{a_i} = \sqrt{n_i'+1}\bra{\dots n_i'+1\dots}
\end{equation}
Fermons cette relation
\begin{equation}
\begin{array}{ll}
\bra{\dots n_i'\dots} \hat{a_i}\ket{\dots n_i\dots} &\DS= \sqrt{n_i'+1}\underbrace{\bra{\dots n_i'+1\dots} \hat{a_i}\ket{\dots
 n_i\dots}}_{
\delta_{n_i,n_{i+1}}}\\
&\DS= \sqrt{n_i}\ \delta_{n_i,n_{i+1}}
\end{array}
\end{equation}
Si l'on effectue une somme, celle-ci commencera à 1. Dès lors\footnote{Éclaircir}, $n_i'$ commence à $n_i-1$ 
\begin{equation}
\hat{a_i}\ket{\dots n_i\dots} = \sum_{n_i=0}\ket{\dots n_i'\dots}\underbrace{\bra{\dots n_i'\dots}\hat{a_i}\ket{\dots n_i\dots}}_{
si\ n_i=0 : \sqrt{n_i}\ \delta_{n_i, n_i'+1}}
\end{equation}
On appelle ceci l'opérateur d'annihilation. On peut compléter notre définition (\textbf{possible confusion}) :
\begin{equation}
\hat{a_i}\ket{\dots n_i\dots} = \left\{\begin{array}{ll}
\sqrt{n_i}\ket{\dots n_i-1\dots} & \text{ si } n_i\geq 1\\
0 & \text{ si } n_i = 0
\end{array}\right.
\end{equation}
Nos deux opérateurs sont alors défini comme
\begin{equation}
\left\{\begin{array}{ll}
\DS\hat{a}_i^\dagger \ket{n_1\dots n_i\dots} &= \sqrt{n_i+1}\ket{n_1\dots n_i+1\dots}\vspace{2mm}\\
\DS\hat{a}_i\ket{n_1\dots n_i\dots} &= \sqrt{n_i}\ket{\dots n_i-1\dots}\qquad\qquad (n_i\geq 1)
\end{array}\right.
\end{equation}

\subsection{Relation de commutations}
Trois relations de commutations sont particulièrement intéressantes
\begin{equation}
[\hat{a_i},\hat{a_j}]=0,\qquad\qquad[\hat{a_i}^\dagger,\hat{a_j}^\dagger]=0,\qquad\qquad [\hat{a_i},\hat{a_j}^\dagger]=\delta_{i,j}
\end{equation}
On voit donc que l'on peut créer ou annihiler deux particules dans l'ordre que l'on souhaite mais ce n'est pas le cas si 
on souhaite en créer une pour ensuite l'annihiler. Montrons la première commutation
\begin{equation}
\begin{array}{ll}
\DS\hat{a_i}\hat{a_j}\ket{\dots n_i\dots n_j\dots} &\DS= \hat{a_i}\sqrt{n_j}\ket{\dots n_i\dots n_j-1\dots}\\&\DS= \sqrt{n_i}\sqrt{n_j}\ket{\dots n_i-1\dots n_j-1\dots}\\
\DS&= \DS\hat{a_j}\hat{a_i}\ket{\dots n_i\dots n_j\dots} 
\end{array}
\end{equation}
La deuxième commutation est une conséquence directe de la première. En efet
\begin{equation}
[\hat{a_i},\hat{a_j}]^\dagger=[\hat{a_i}^\dagger,\hat{a_j}^\dagger]=0
\end{equation}
Intéressons-nous maintenant à la troisième relation. Dans le cas où $i\neq j$
\begin{equation}
\begin{array}{ll}
\hat{a_i}\hat{a_j}^\dagger\ket{\dots n_i\dots n_j\dots} &\DS = \hat{a_i}\sqrt{n_j+1}\ket{\dots n_i\dots n_j+1\dots}\\
&\DS = \sqrt{n_i+1}\sqrt{n_j+1}\ket{\dots n_i+1\dots n_j+1\dots}\\
&\DS = \hat{a_j}^\dagger\hat{a_i}\ket{\dots n_i\dots n_j+1\dots}
\end{array}
\end{equation}
Le seul cas intéressant est donc celui où $i=j$. Soit pour une orbitale
\begin{equation}
\begin{array}{ll}
(\hat{a_i}\hat{a_i}^\dagger -\hat{a_i}^\dagger\hat{a_i})\ket{\dots n_i\dots} &=\DS \hat{a_i}\sqrt{n_i+1}\ket{\dots n_i+1\dots}-
\hat{a_i}^\dagger\sqrt{n_i}\ket{\dots n_i-1\dots}\\
&=\DS \sqrt{n_i+1}\sqrt{n_i+1}\ket{\dots n_i\dots}-\sqrt{n_i}\sqrt{n_i}\ket{\dots n_i\dots}\\
&=\ket{\dots n_i\dots}
\end{array}
\end{equation}
d'où $\hat{a_i}\hat{a_i}^\dagger -\hat{a_i}^\dagger\hat{a_i} = \hat{1}$. Ce résultat est fondamental pour la seconde quantification.





\subsection{Écriture générale d'un état à $N$ fermions et des opérateurs à 1 et 2 corps}