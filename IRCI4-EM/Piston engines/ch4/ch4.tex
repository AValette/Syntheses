
\chapter{Spark ignition engine}
\section{Mixture preparation in SI engine}
	\wrapfig{6}{l}{5}{0.25}{ch4/1}{ch4/1}
	The objective is to obtain an homogeneous gaseous composition with the right equivalence ratio. We will see that in direct injection inhomogeneous gas is obtained. 
	
\subsection{The carburetor}
	It is now outdated because of the regulation imposed on the exhaust gases and has been replaced by fuel injection except\ \\
	 
	\wrapfig{4}{r}{5}{0.35}{ch4/2}{ch4/2}
	for small engines.  As a first approximation we can consider that the carburetor pump the fuel into the air flow using the Bernouilli principle (compressible): when the air velocity increases the static pressure decreases and the fuel flow increases. 
	
\subsubsection{Importance of equivalence ratio for combustion}
	\wrapfig{7}{l}{4}{0.5}{ch4/3}{ch4/3}
	The best burning condition is obtained when stoechiometry (15kg air for 1kg fuel). But rich and lean mixture can also burn not in the best conditions. In SI engine the amount of mixture going to the engine is controlled by the throttle. A venturi creates a decrease in pressure for the fuel to be pushed.  
		
	\ \\
	
	\wrapfig{7}{r}{6}{0.3}{ch4/4}{ch4/4}
	The reality deviates from the incompressible model. Indeed, for air velocity increase the pressure drop is more important and the injected fuel is larger as fuel is incompressible (eq. ratio higher in high air flow). In addition the speed in venturi is limited to Mach 1 to avoid shocks and the fuel height pressure has to be compensated in order to have injection, thus for low speeds there is no injection. 
	
\subsubsection{Compensating jet}
	\wrapfig{8}{l}{5}{0.3}{ch4/5}{ch4/5}
	This is introduced in order to solve the problem of the increasing equivalence coefficient when air flow increases. The fuel jet is divided into 2 jets, the main directly connected to the venturi and the second connected to the venturi through an \textbf{emulsion tube}. As the air flow increases, the one into the emulsion tube too and this decreases the amount of compensating fuel. 
	
	\ \\
	
	\wrapfig{6}{r}{4}{0.3}{ch4/6}{ch4/6}
	Here we can see the effect of this, we have a more constant eq. ratio. As we have decrease the flow of the main jet and the compensating jet is not working at high air speed, the mixture is kept constant. When acceleration, rich mixture is needed. For this an additional system sensing the acceleration provides extra fuel. 
	
\subsubsection{Idling fuel line}
	\minifig{ch4/7}{ch4/8}{0.35}{0.35}{0.3}{0.3}
	The remaining problem is at very low air flow rate because the pressure drop at the venturi is very low. To compensate this, one takes advantage from the flow near the throttle where a local pressure drop is created when nearly closed and uses so the \textbf{idling fuel line}.  The last trick to include is for \textbf{cold start}. The added \textbf{choke valve} is situated before the venturi in order to create a choke. 
	
\section{Indirect fuel injection}
	