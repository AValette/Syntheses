%%%%%%%%%%%%%
%  Ch1 : Generalities  %
%%%%%%%%%%%%%

\chapter{Generalities}
\section{Fundamental laws}
	We will first begin with a reminder of the previous \textit{Fluid Mechanics I} course equations, in particular, the 3 basic principles : \\
	
	\begin{itemize}
		\item[•] \textbf{Mass conservation :} \textit{The mass of a closed system remains constant in time.}\\
		This is much a definition of a closed system than a principle. We have to notice that related to Einstein law of relativity, $E = mc^2$, mass must vary with energy. But if we exclude nuclear reactions, our approximation is valid. Indeed, the square of light velocity has a greater impact on energy than the mass term. If the energy exchange is huge like in nuclear reaction, mass vary, but in smaller energies domain (combustion for example), the mass can be considered as constant. \\
		
		\item[•] \textbf{Newton's law :} \textit{the time rate of change of momentum of a closed system is equal to the sum of the forces applied on the system.} \\
		
		\item[•] \textbf{First principle of thermodynamics :} \textit{the time rate of change of the total energy of a closed system is equal to the sum of the power of the forces applied on the system and the thermal power provided to the system.}
	\end{itemize}		