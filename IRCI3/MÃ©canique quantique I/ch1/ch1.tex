\chapter{Notation de Dirac}
\section{Vecteurs d'état et espace de Hilbert}
Le \textit{vecteur d'état} se dénomme \textit{ket} et est noté :
\begin{equation}
\begin{array}{ll}
\ket{\psi} & \in \mathcal{E}\\
&\in \mathcal{E}_H
\end{array}
\end{equation}
où $\mathcal{E}$ est l'espace des états et $\mathcal{E}_H$ l'espace de Hilbert. 
Notons que $\mathcal{E} \subset \mathcal{E}_H$. Par abus de langage, nous 
désignerons souvent l'espace des états comme étant l'espace de Hilbert, ce qui 
n'est en toute rigueur pas exact ($\mathcal{E}_H$ contient des états non-physiques).
L'espace de Hilbert est un espace complet (si on définit une suite d'état, celle-ci 
convergera vers un état) muni d'un produit scalaire (défini à la section suivante).\\

Pourquoi définir un vecteur d'état ? En physique classique l'état d'un système 
ne pose pas de problème particulier. A l'inverse, en physique quantique, la notion 
même d'état pose déjà un problème, contraignant l'utilisation de vecteurs d'état. La raison 
physique de leur utilisation vient du principe d'incertitude d'Heisenberg. En effet, 
il nous est impossible de décrire la particule par le couple position/impulsion d'où 
la motivation à l'utilisation de ces vecteurs.\\

A la base de la physique, le \textbf{principe de superposition} nous dit que la 
combili (de coefficients complexes) de deux vecteurs d'états, soit deux kets, est 
de nouveau un ket, soit un état 100\% admissible.
\begin{equation}
\ket{\psi_1}, \ket{\psi_2} \in\mathcal{E},\qquad \ket{\lambda_1\psi_1+\lambda_2\psi_2} 
\equiv \lambda_1\ket{\psi_1}+\lambda_2\ket{\psi_2}\qquad \forall \lambda_1,\lambda_2\in
\mathbb{C}
\end{equation}
Il s'agit de la \textit{linéarité de la physique quantique} avec laquelle on peut, par 
exemple, décrire le phénomène d'interférences.


\section{Produit scalaire entre deux kets}
Le produit scalaire entre deux kets se note
\begin{equation}
\bra{\psi_2}\ket{\psi_1}
\end{equation}
\newpage
Les propriétés de bases de ce produit scalaires sont bien connues :
\begin{equation}
\begin{array}{llll}
\bullet & \bra{\psi}\ket{\psi} &> 0\\
\bullet & \bra{\psi_1}\ket{\psi_2} &= \bra{\psi_2}\ket{\psi_1}^*\\
\bullet & \bra{\psi}\ket{\lambda_1\psi_1+\lambda_2\psi_2} &= \lambda_1\bra{\psi_1}\ket{\psi_2}
+\lambda_2\bra{\psi_1}\ket{\psi_2}\quad \forall \lambda_i\in\mathbb{C}.\quad
& \text{Linéarité (à gauche)}\\
\bullet & \bra{\lambda_1\psi_1+\lambda_2\psi_2}\ket{\psi} &= \bra{\psi}\ket{\lambda_1
\psi_1+\lambda_2\psi_2}^* \quad & \text{Antilinéarité (à gauche)}\\
&  &= (\lambda_1\bra{\psi}\ket{\psi_1}+\lambda_2\bra{\psi}\ket{\psi_2})^*\\
&  &= \lambda_1^*\bra{\psi_1}\ket{\psi} +  \lambda_2^*\bra{\psi_2}\ket{\psi}\\
\bullet & \|\psi\| = \|\ket{\psi}\| = \sqrt{\bra{\psi}\ket{\psi}}>0
\end{array}
\end{equation}

Il est intéressant de s'intéresser à la \textit{"représentation"} d'un ket au sein 
d'un espace de Hilbert. Considérons l’exemple suivant (qui reviendra souvent).\\

\textsc{Exemple}\\
Considérons un espace de Hilbert de dimension $n$. Les vecteurs d'états, les ket, ne sont rien 
d'autres que des vecteurs colonnes dans cet espace de dimension $n$. Soit
\begin{equation}
\ket{u} = \left(\begin{array}{c}
u_1\\
u_2\\
\vdots\\
u_n
\end{array}\right),\qquad\qquad \ket{v} = \left(\begin{array}{c}
v_1\\
v_2\\
\vdots\\
v_n
\end{array}\right),\qquad u_i,v_i\in\mathbb{C}
\end{equation}
Le produit scalaire entre ces deux ket est donné par
\begin{equation}
\bra{v}\ket{u} = \sum_{i=1}^n v_i^*u_i = \underbrace{(v_1^*\ v_2^*\ \dots\ v_n^*)}_{(*)}
\left(\begin{array}{c}
u_1\\
u_2\\
\vdots\\
u_n
\end{array}\right)
\end{equation}
On va définir $(*)$ comme étant l'objet dual \textit{"complémentaire au ket"}, $\bra{v}$ que l'on 
nomme \textit{bra}. Ce bra appartient à un espace dual, ce qui est le sujet de la 
section suivante.


\section{Espace dual $\mathcal{E}^*$, vecteur "bra"}
Le bra est une forme linéaire : c'est une application qui va de l'espace des état 
(ou de Hilbert, pas de différence dans ce cours) vers l'ensemble des nombres complexes.
\begin{equation}
\varphi : \ket{\psi}\in\mathcal{E} \leadsto \varphi(\ket{\psi})\ \in \mathbb{C}
\end{equation}
Cette forme linéaire fait correspondre à chaque état un nombre complexe. La superposition 
est également vérifiée d'où le "linéaire".
\begin{equation}
\varphi(\ket{\lambda_1\psi_1 + \lambda_2\psi_2}) = \lambda_1\varphi(\ket{\psi_1})+
\lambda_2\varphi(\ket{\psi_2})\qquad \forall \lambda_1,\lambda_2\in\mathbb{C}
\end{equation}
où $\varphi \in \mathcal{E}^*$.\\

Il semble dès lors intéressant d'introduire un nouvel "objet" : 
\begin{equation}
\left\{\begin{array}{ll}
\varphi \in \mathcal{E}^*\\
\bra{\varphi}
\end{array}\right.
\end{equation}
Il s'agit de l'ensemble de toutes les formes linéaires, ensemble qui forme un espace dual. 
L'intérêt de ces nouveaux objets réside dans un isomorphisme : on associe à un ket de l'espace des états 
un et un seul bra de l'espace dual.\\

Ceci étant dit, il faut caractériser et montrer comment cette application agit sur les espaces. 
\begin{equation}
\forall \ket{\psi} \in \mathcal{E},\qquad \underline{\varphi(\ket{\psi}) =\bra{\varphi}\ket{\psi}}
\end{equation}

L'avantage de cette notation est qu'elle permet deux visions complémentaires des choses :
soit on voit cela comme un produit scalaire entre deux kets, soit on voit cela comme une
forme linéaire, le bra, s'appliquant à un ket.\\

Comme précisé, il s'agit d'une forme \textbf{linéaire} :
\begin{equation}
\begin{array}{ll}
\varphi(\ket{\lambda_1\psi_1+\lambda_2\psi_2}) &= \bra{\varphi}\ket{\lambda_1\psi_1+\lambda_2\psi_2}\\
&= \lambda_1\bra{\varphi}\ket{\psi_1}+\lambda_2\bra{\varphi}\ket{\psi_2}\\
&= \lambda_1\varphi(\ket{\psi_1})+\lambda_2\varphi(\ket{\psi_2})
\end{array}
\end{equation}
L'espace dual est également un espace de Hilbert : toutes les propriétés de 
linéarité seront retrouvées. Ainsi, toute combili (complexe) de forme 
apparentent à $\mathcal{E}^*$ forme une troisième forme appartenant à 
$\mathcal{E}^*$.
\begin{equation}
\text{Si } \bra{\varphi_1},\bra{\varphi_2}\in\mathcal{E}^*, \text{ alors }\ 
\lambda_1\bra{\varphi_1} + \lambda_2\bra{\varphi_2} \in\mathcal{E}^*\qquad 
\forall \lambda_i\in\mathbb{C}
\end{equation}
On peut ainsi démontrer que $\mathcal{E}^*$ est un espace vectoriel.
Un bra se définit par son action sur tout ket :
\begin{equation}
\begin{array}{ll}
\forall \ket{\psi} : (\lambda_1\bra{\varphi_1} + \lambda_2\bra{\varphi_2})\ket{\psi} 
&= \lambda_1\bra{\varphi_1}\ket{\psi}+\lambda_2\bra{\varphi_2}\ket{\psi} \\
&= \lambda_1\bra{\psi}\ket{\varphi_1}^*+\lambda_2\bra{\psi}\ket{\varphi_2}^*\\
&= (\lambda_1^*\bra{\psi}\ket{\varphi_1}+\lambda_2^*\bra{\psi}\ket{\varphi_2})^*\\
&= \bra{\psi}\ket{\lambda_1^*\varphi_1+\lambda_2^*\varphi_2}^*\\
&= \bra{\lambda_1^*\varphi_1+\lambda_2^*\varphi_2}\ket{\psi}
\end{array}
\end{equation}
Nous avons donc bien un espace vectoriel (ce qui est clairement visualisable 
dans l'équation ci-dessous). La dernière relation applique un certain bra à 
n'importe que $\psi$. En terme de bra, on peut alors écrire
\begin{equation}
\underline{\lambda_1\bra{\varphi_1} + \lambda_2\bra{\varphi_2} = \bra{\lambda_1^*\varphi_1
+\lambda_2^*\varphi_2}}\qquad \lambda_1,\lambda_2\in\mathbb{C}
\end{equation}


On vient de voir qu'à n'importe quel bra je peux associer un ket. Il serait 
dès lors intéressant de trouver le ket correspondant à ce bra. Mais avant, on va définir 
la notion d'opérateur s'appliquant dans l'espace de Hilbert. \\

Il est possible de se représenter de façon plus précise ce qu'est un bra en 
se souvenant de l'exemple donné avec un espace de Hilbert de dimension $n$. Dans 
un tel espace, un bra n'est qu'un vecteur ligne complexe conjugué.

\subsection{Opérateurs linéaires (agissant dans $\mathcal{E}$)}
Un opérateur linéaire est une application qui fait correspondre un ket à un ket, à 
la différence de la forme qui fait correspondre un complexe à un ket.

\begin{equation}
\ket{\psi} \in \mathcal{E} \leadsto \hat{A}\ket{\psi} \in \mathcal{E}
\end{equation}
Il est coutume d'indiquer les opérateurs linéaires par un chapeau. La sainte 
superposition reste d'actualité :
\begin{equation}
\hat{A}\ket{\lambda_1\psi_1+\lambda_2\psi_2} = \lambda_1\hat{A}\ket{\psi_1}+
\lambda_2\hat{A}\ket{\psi_2}
\end{equation}
Pas mal de propriétés valent la peine d'être énoncées :
\begin{equation}
\begin{array}{llll}
\bullet & (\hat{A}+\hat{B})\ket{\psi} &= \hat{A}\ket{\psi}+\hat{B}\ket{\psi}\\
\bullet & (\hat{A}.\hat{B})\ket{\psi} &= \hat{A}(\hat{B}\ket{\psi})\qquad & \text{ Opérateur 
produit $\hat{A}.\hat{B}$}
\end{array}
\end{equation}
Nous pouvons voir cet opérateur produit comme une notation efficace. Il ne faut 
cependant pas perdre à l'idée que, en toute généralité, $\hat{A}$ et $\hat{B}$ 
ne commutent pas. On définit alors le commutateur :
\begin{equation}
[\hat{A},\hat{B}] = \hat{A}\hat{B} - \hat{B}\hat{A}
\end{equation}
Comme $\hat{A}$ et $\hat{B}$ sont des opérateurs, la différence des opérateurs 
est toujours un opérateur, le commutateur est bien un opérateur. Il jouit des 
propriétés suivantes :
\begin{equation}
\begin{array}{lll}
\bullet & [\hat{B},\hat{A}] &= -[\hat{A},\hat{B}]\\
\bullet & [\hat{A},\hat{B}+\hat{C}] &= [\hat{A},\hat{B}]+[\hat{A},\hat{C}]\\
\bullet & [\hat{A},\hat{B}.\hat{C}] &= \hat{B}.[\hat{A},\hat{C}]+[\hat{A},
\hat{B}].\hat{C}
\end{array}
\end{equation}
On peut montrer qu'un opérateur linéaire peut se représenter comme une 
matrice. Pour l'illustrer, reconsidérons notre précédent exemple.\\

\textsc{Exemple}\\
Soit un espace de Hilbert de dimension finie $n$. Soit
\begin{equation}
\ket{u} = \left(\begin{array}{c}
u_1\\
u_2\\
\vdots\\
u_n
\end{array}\right),\qquad \ket{v} = \hat{A}\ket{u}\quad ; \left(\begin{array}{c}
v_1\\
v_2\\
\vdots\\
v_n
\end{array}\right) = \underbrace{\left(\begin{array}{ccc}
a_{11} & \dots & a_{1n}\\
\vdots &\ddots &\vdots\\
a_{n1} & \dots & a_{nn}
\end{array}\right)}_{\hat{A}}\left(\begin{array}{c}
u_1\\
u_2\\
\vdots\\
u_n
\end{array}\right)
\end{equation}

Le fait que le commutateur entre deux opérateurs soit non-nul indique que le
produit entre les matrcies correspondantes est non-commutatif.\\

De par cette représentation, on peut aisément comprendre que la non-commutation 
vient du fait que les différentes lignes et colonnes de $\hat{A}$ ne peuvent 
être commutées. Intéressons-nous aux éléments de la matrice de cet opérateur.



\section{"Élément de matrice" d'un opérateur $\hat{A}$}

Soient
\begin{equation}
\ket{\psi} \text{ et } \left\{\begin{array}{ll}
\ket{\varphi} &\in \mathcal{E}\\
\bra{\varphi} &\in \mathcal{E}^*
\end{array}\right.,
\end{equation}
comme précédemment, définissons un nouvel "objet" :
\begin{equation}
 \quad \bra{\varphi}\hat{A}\ket{\psi} = \bra{\varphi}(
\hat{A}\ket{\psi}) = (\bra{\varphi}\hat{A})\ket{\psi}
\end{equation}

%Les parenthèses permettent de voir ça "tel un produit scalaire". Revenons 
%à notre précédent problème : quel est finalement ce ket ? Lors de l'écriture 
%d'un élément de matrice, il est intéressant de pouvoir le voir comme un 
%opérateur appliqué à un ket. Une autre vision est celle d'un opérateur 
%qui agit sur un bra, définissant un nouveau bra qui cette fois, agit sur 
%$\psi$. Revenons à notre exemple.\\

Ce nouvel objet peut se comprendre de deux manières : on peut le voir comme
un opérateur s'appliquant à un ket pour former un nouveau ket que l'on
multiplie scalairement avec un bra, ou on peut le voir comme un opérateur
agissant sur un bra pour former une nouvelle forme qui s'appliquera au ket $\psi$.
Notons qu'en réalité il ne s'agit pas du même opérateur car les espaces vectoriels
de départ et d'arrivée ne ont pas les mêmes dans les deux cas. Cependant ils possèdent
la même forme en représentation matricielle et l'abus de notation est très pratique.
Notons également qu'un opérateur est entièrement caractérisé par ses éléments de matrice.\\

Revenons à notre exemple.

%\newpage
\textsc{Exemple}\\
Soit
\begin{equation}
\ket{u} = \left(\begin{array}{c}
u_1\\
u_2\\
\vdots\\
u_n
\end{array}\right), \qquad \bra{v} = (v_1^*\ v_2^*\ \dots\ v_n^*), \qquad 
\hat{A}\ket{u} = \left(\begin{array}{ccc}
a_{11} & \dots & a_{1n}\\
\vdots &\ddots &\vdots\\
a_{n1} & \dots & a_{nn}
\end{array}\right)\left(\begin{array}{c}
u_1\\
u_2\\
\vdots\\
u_n
\end{array}\right)
\end{equation}
Nous avons alors
\begin{equation}
\bra{v}\left(\hat{A}\ket{u}\right) = \underbrace{(v_1^*\ v_2^*\ \dots\ v_n^*)\left(\begin{array}{ccc}
a_{11} & \dots & a_{1n}\\
\vdots &\ddots &\vdots\\
a_{n1} & \dots & a_{nn}
\end{array}\right)}_{\bra{?}\ket{u}}\left(\begin{array}{c}
u_1\\
u_2\\
\vdots\\
u_n
\end{array}\right)
\end{equation}

\begin{equation}
\bra{v}\left(\hat{A}\ket{u}\right) = (v_1^*\ v_2^*\ \dots\ v_n^*)\underbrace{\left(\begin{array}{ccc}
a_{11} & \dots & a_{1n}\\
\vdots &\ddots &\vdots\\
a_{n1} & \dots & a_{nn}
\end{array}\right)\left(\begin{array}{c}
u_1\\
u_2\\
\vdots\\
u_n
\end{array}\right)}_{\left(\hat{A}\ket{u}\right)}\\
\end{equation}



\section{Opérateur adjoint}
A tout opérateur $\hat{A}$, on peut associer un nouvel opérateur noté $\hat{A}^\dagger$ (prononcé "dagger"). 
Soit $\ket{\psi}$ :
\begin{equation}
\hat{A} \text{ agit dans }\mathcal{E} ; \ket{\psi'} = \hat{A}\ket{\psi}
\label{eq:16}
\end{equation}
Chaque ket est associé à un bra ; dans ce cas ci il s'agit de $\bra{\psi'}$ et
$\bra{\psi}$. Existe-t-il une relation entre ces bra ? Mais à quoi cet objet 
correspond-t-il ? Un bra est une forme linéaire, il faut déterminer comment 
agit $\bra{\psi'}$ sur n'importe quel ket de l'espace.
\begin{equation}
\begin{array}{lll}
\forall \ket{\phi} \in \mathcal{E} : \bra{\psi'}\left(\ket{\varphi}\right) &\equiv 
\bra{\psi'}\ket{\varphi} & \qquad \text{Prop. p.scal.}\\
&= \bra{\varphi}\ket{\psi'}^*\\
&= \bra{\varphi}\hat{A}\ket{\psi}^* & \qquad \text{Def. de }\psi', \text{ def. op. adj.}\\
&= \bra{\psi}\hat{A}^\dagger\ket{\varphi}& \qquad \text{(*)}
\end{array}
\end{equation}
Pour arriver à $(*)$, on peut remplacer $\hat{A}$ par son adjoint si l'on 
permute les termes et considère le complexe conjugué.
La conclusion de tous cela - modulo la définition de l'opérateur adjoint - est 
que l'on voit que l'on peut réécrire le $\bra{\psi'}$ en terme de $\bra{\psi}$.
\begin{equation}
\underline{\bra{\psi'} = \bra{\psi}\hat{A}^\dagger}
\end{equation}
Cette relation ressemble assez fortement à \autoref{eq:16} ou $\hat{A}\rightarrow
\hat{A}^\dagger$.
De façon générale on peut voir qu'un opérateur linéaire peut être entièrement 
caractérisé par ses éléments de matrice, exactement comme une matrice est 
caractérisée par tous ses éléments. Pour parvenir à ce résultat, nous avons 
utilisé la définition d'un opérateur adjoint :
\begin{equation}
\forall \ket{\psi} \text{ et } \ket{\varphi} \in \mathcal{E},\qquad 
\bra{\psi}\hat{A}^\dagger \ket{\varphi} = \bra{\psi}\hat{A}\ket{\varphi}^*
\end{equation}

\textsc{Exemple}\\
Comme toujours, prenons notre espace de Hilbert de dimension finie $n$.
\begin{equation}
\bra{v}\hat{A}\ket{u} = \underbrace{(v_1^*\ v_2^*\ \dots\ v_n^*)\left(\begin{array}{ccc}
a_{11} & \dots & a_{1n}\\
\vdots &\ddots &\vdots\\
a_{n1} & \dots & a_{nn}
\end{array}\right)}_{(*)}\left(\begin{array}{c}
u_1\\
u_2\\
\vdots\\
u_n
\end{array}\right)
\end{equation}
Le "but" est que $(*)$ devienne notre nouveau bra $(w_1\ w_2\ \dots\ w_n)$ :
\begin{equation}
\left(\begin{array}{c}
w_1\\
w_2\\
\vdots\\
w_n
\end{array}\right) = \underbrace{\left(\begin{array}{ccc}
a_{11}^* & \dots & a_{n1}^*\\
\vdots &\ddots &\vdots\\
a_{1n}^* & \dots & a_{nn}^*
\end{array}\right)}_{\hat{A}^\dagger}\left(\begin{array}{c}
v_1\\
v_2\\
\vdots\\
v_n
\end{array}\right)
\end{equation}
%Pour obtenir le bra, nous avons réalisé une opération semblable à celles 
%réalisées en algèbre linéaire, à savoir pris le complexe conjugé de la matrice 
%conjugué après inversion et transposée du vecteur.\footnote{Mieux expliciter plz}.


Pour obtenir le bra, nous avons utilisé une méthode semblable à celle
réalisée en algèbre linéaire : nous avons obtenu la matrice de l'opérateur
adjoint en prenant la transposée de la matrice conjuguée.
L'opérateur adjoint n'est rien d'autre que la matrice adjointe.
Notons que la conjugaison et la transposition ne dépendent que de la base
choisie et que l'existence de l'opérateur adjoint est garantie si on peut l'obtenir
dans au moins une base. Le fait de permuter les lignes et les colonnes ne faisait 
qu'inverser les bra et ket.\\
Il en découle des propriétés intéressantes :
\begin{multicols}{2}
\begin{itemize}
\item[$\bullet$] $\left(\hat{A}^\dagger\right)^\dagger = \hat{A}$
\item[$\bullet$] $\left(\lambda\hat{A}\right)^\dagger = \lambda^*\hat{A}$
\item[$\bullet$] $\left(\hat{A}+\hat{B}\right)^\dagger = \hat{A}^\dagger+\hat{B}^\dagger$
\item[$\bullet$] $\left(\hat{A}.\hat{B}\right) = \hat{B}^\dagger.\hat{A}^\dagger$
\end{itemize}
\end{multicols}
A titre d'exercice, démontrons la dernière propriété
\begin{equation}
\begin{array}{ll}
\bra{\psi}\left(\hat{A}.\hat{B}\right)^\dagger\ket{\varphi} &= \bra{\varphi}\hat{A}.
\hat{B}\ket{\psi}^*\\
&= \left(\left(\bra{\varphi}\ket{\hat{A}}\right)\left(\bra{\hat{B}}\ket{\psi}\right)\right)^*\\
&= (\bra{\psi}\hat{B}^\dagger)(\hat{A}^\dagger\ket{\varphi})\\
&= \bra{\psi}\hat{B}^\dagger\hat{A}^\dagger\ket{\varphi}
\end{array}
\end{equation}

\section{Opérateurs hermitiens/auto-adjoints}
Par définition, $\hat{A}$ est auto-adjoint si
\begin{equation}
\hat{A} = \hat{A}^\dagger
\end{equation}
Dès lors
\begin{equation}
\begin{array}{ll}
\bra{\psi}\hat{A}\ket{\varphi} &= \bra{\varphi}\hat{A}\ket{\psi}^*\\
\bra{\psi}\hat{A}\ket{\psi} &= \bra{\psi}\hat{A}\ket{\psi}^* \rightarrow \underline{\in \mathbb{R}}
\end{array}
\end{equation}
Énonçons quelques propriétés intéressantes\footnote{Les opérateurs de quantités observables sont
auto-adjoints.}
\begin{equation}
\begin{array}{lll}
\forall \hat{A},\hat{B} \text{ hermitiens },\qquad & \hat{A}+\hat{B} & \text{hermitien}\\
& \hat{A}.\hat{B} & \text{hermitien ssi } [\hat{A},\hat{B}] = 0
\end{array}
\end{equation}
On peut justifier la dernière propriété de la façon suivante :
\begin{equation}
\begin{array}{lll}
(\hat{A}.\hat{B})^\dagger &= \hat{B}^\dagger.\hat{A}^\dagger \\
&= \hat{A}^\dagger.\hat{B}^\dagger & \text{vrai ssi } [\hat{A}^\dagger,\hat{B}^\dagger]=0=-
\underbrace{[\hat{A},\hat{B}]^\dagger}_{=0}\\
&= \hat{A}.\hat{B}
\end{array}
\end{equation}
Le produit position et impulsion n'est pas un opérateur hermitien (ces deux états ne 
commutent pas) ; ce n'est donc pas une quantité observable en physique quantique.\\




%COURS 2
%%%%%%%%%%%%%%%%%%%%%%%%%%%%%%%%%%%%%%%%%%%%%%%%%%%%%%%%%%%%%%%%%%%%%%%%%%%%%%%%%%%%%
Pour obtenir le complexe conjugué avec les notations de Dirac, il suffit de lire à 
l'envers pour obtenir ce que l'on souhaite :
\begin{equation}
\begin{array}{ll}
\hat{A}\ket{\psi} &\rightarrow \bra{\psi}\hat{A}^\dagger\\
\bra{\varphi}\hat{B}&\rightarrow \hat{B}^\dagger \ket{\varphi}\\
\bra{\psi}\hat{A}\ket{\varphi} &\rightarrow  \bra{\varphi}\hat{A}^\dagger\ket{\psi} =
\bra{\psi}\hat{A}\ket{\varphi}^*
\end{array}
\end{equation}
Un autre exemple, un peu moins trivial est de considérer l'opérateur suivant
\begin{equation}
\ket{u}\bra{v} \rightarrow (\ket{u}\bra{v})^\dagger= \ket{v}\bra{u}
\end{equation}
Avec la notation $\ket{u}\bra{v}\ket{\varphi}$, on se rend 
compte qu'il s'agit bien d'un opérateur agissant sur l'état $\ket{\varphi}$. Afin de 
s'en rendre compte, développons ceci à titre d'application
\begin{equation}
\begin{array}{lll}
\bra{\varphi}\underline{\left(\ket{u}\bra{v}\right)^\dagger}\ket{\psi} &= (\bra{\psi}(\ket{u}\bra{v})
\ket{\varphi})^* & (*)\\
&=(\bra{\psi}\ket{u})^*(\bra{v}\ket{\varphi})^*\\
&=(\bra{u}\ket{\psi})(\bra{\varphi}\ket{v})\\
&=(\bra{\varphi}\ket{v})(\bra{u}\ket{\psi}) & \text{Commutativité}\\
&=\bra{\varphi}\underline{(\ket{v}\bra{u})}\ket{\psi}
\end{array}
\end{equation}
Il est possible de voir $(*)$ de deux façons différentes. On peut le comprendre comme 
un objet (opérateur) dont on prend l'élément de matrice entre $\psi$ et $\varphi$ (comme 
le suggère les parenthèses). On peut également le voir comme deux produits scalaire dont 
on fait le produit simple (en omettant cette fois-ci les parenthèses).


\subsection{Base Hilbertienne}
Une base hilbertienne est une base de l'espace de Hilbert. Il en existe deux types particuliers:
la base discrète de dimension finie et la base continue.

\subsubsection{Base discrète}
Nous parlons de l'espace de Hilbert et donc d'un espace des états, soit encore un ensemble 
de ket qui sous-tendent l'espace :
\begin{equation}
\left\{\ket{u_i}\right\}
\end{equation}
Ces vecteurs de base sont orthonormés
\begin{equation}
\bra{u_i}\ket{u_j} = \delta_{ij}
\end{equation}
Le but d'une telle base est d'exprimer n'importe quel ket, n'importe quel état, comme une 
combili des vecteurs de cette base. En toute généralité, on peut écrire un ket comme une 
somme sur $i$ de coefficients multiplicatifs $C_i$ (qui joueront le rôle d'amplitude de 
probabilité, mais ils sont avant tout des coefficients de Fourier)
\begin{equation}
\ket{\psi} = \sum_i C_i\ket{u_i}
\label{eq:2.5}
\end{equation}
En remarquant que
\begin{equation}
\bra{u_j}\ket{\psi} = \sum_iC_i\underbrace{\bra{u_j}\ket{u_i}}_{\delta_{ij}} = C_j
\end{equation}
On peut réécrire \autoref{eq:2.5} :
\begin{equation}
\begin{array}{lll}
\ket{\psi} &= \sum_i \bra{u_i}\ket{\psi}\ket{u_i} & \text{Notations de Dirac}\\
&= \sum_i \ket{u_i}\bra{u_i}\ket{\psi} & \text{(Somme d') Op. lin. appliqué(e) à $\psi$}\\
&\underbrace{\left(\sum_i \ket{u_i}\bra{u_i}\right)}_{\mathbb{1}}\ket{\psi} & \forall\psi
\end{array}
\end{equation}

On voit apparaître la relation de fermeture. Dès que l'on a une base complète, la somme des projecteurs
$\ket{u_i}\bra{u_i}$ donnera l'opérateur identité. On appelle alors la \textbf{relation 
de fermeture} :
\begin{equation}
\sum_i \ket{u_i}\bra{u_i} = \hat{\mathbb{1}}
\end{equation}
On peut alors définir l'opérateur \textit{projecteur} $\hat{P_i}$ :
\begin{equation}
\hat{P_i} = \ket{u_i}\bra{u_i}
\end{equation}
Cet opérateur possède deux propriétés remarquables. La première est qu'il est hermitien, 
c'est-à-dire $\hat{P_i} = \hat{P_i}^\dagger$.
\begin{equation}
\hat{P_i}^\dagger = (\ket{u_i}\bra{u_i})^\dagger = \ket{u_i}\bra{u_i} = \hat{P_i}
\end{equation}
La seconde est qu'il est idempotent. Autrement dit, plus d'une application successive ne 
change rien au résultat obtenu : $\hat{P_i}^2 = \hat{P_i}$.
\begin{equation}
\begin{array}{ll}
\hat{P_i}^2 &= (\ket{u_i}\bra{u_i})(\ket{u_i}\bra{u_i})\\
&= \ket{u_i}\underbrace{\bra{u_i}\ket{u_i}}_{=1}\bra{u_i}\\
&= \hat{P_i}
\end{array}
\end{equation}
On interprète le projecteur comme on le ferait dans l'espace euclidien\footnote{Inclure 
graphe}. La relation de fermeture peut ainsi être réécrite :
\begin{equation}
\sum_i \hat{P_i} = \hat{\mathbb{1}}
\end{equation}
Le développement suivi ici est valable pour toute base. Prenons l'exemple d'un ket
\begin{equation}
\text{ket }\ \ket{\psi} = \sum_i c_i\ket{u_i},\qquad c_i = \bra{u_i}\ket{\psi}
\end{equation}
On peut faire de même pour un bra. Pour définir un bra, il faut premièrement définir 
un ket puis prendre son élément dual.
\begin{equation}
\begin{array}{lll}
& \ket{\varphi} &= \sum_i b_i\ket{u_i},\qquad b_i = \bra{u_i}\ket{\varphi}\\
\text{bra } & \bra{\varphi} &= \sum_i b_i^* \bra{u_i}
\end{array}
\end{equation}
Comment faire pour exprimer un produit scalaire ? Il suffit de faire apparaître 
l'opérateur identité et jouer avec les notations de Dirac
\begin{equation}
\begin{array}{lll}
\bra{\varphi}\ket{\psi} &= \bra{\varphi}\mathbb{1}\ket{\psi}\\
&= \sum_i \underbrace{\bra{\varphi}\ket{u_i}}_{b_i^*}\underbrace{\bra{u_i}\ket{\psi}}_{c_i} &
\text{Relation de fermeture}\\
&= \sum_i b_i^*c_i & (*)
\end{array}
\end{equation}
Dans $(*)$, $b_i^*$ est un vecteur ligne et $c_i$ un vecteur colonne. Si l'espace de Hilbert 
est complet, il s'agit là d'un produit scalaire.\\
Cela fonctionne également pour un opérateur, en effectuant la même astuce mathématique:
\begin{equation}
\begin{array}{ll}
\hat{A} &= \hat{\mathbb{1}}\hat{A}\hat{\mathbb{1}}\\
&=\sum_{i,j} \ket{u_i}\underbrace{\bra{u_i}\hat{A}\ket{u_j}}_{A_{i,j}}\bra{u_j}
\end{array}
\end{equation}
Appliquons $\hat{A}$ sur un ket
\begin{equation}
\begin{array}{ll}
\ket{\psi'} &= \hat{A}\ket{\psi}\\
&\equiv \sum_i a_i\ket{u_i}
\end{array},\qquad \begin{array}{ll}
a_i &= \bra{u_i}\ket{\psi'}\\
&=\bra{u_i}\hat{A}\mathbb{1}\ket{\psi} = \sum_j \underbrace{\bra{u_i}\hat{A}\ket{u_j}}_{A_{i,j}}
\underbrace{\bra{u_j}\ket{\psi}}_{c_j}
\end{array}
\end{equation}
où l'on a utilisé $\ket{\psi'} = \hat{A}\ket{\psi}$ et $a_i = \sum_jA_{ij}c_j\  "=" 
\left(\begin{array}{c}
\vdots\\
\vdots\\
\vdots
\end{array}\right) = \left(\begin{array}{ccc}
  & \dots &  \\
\vdots &\ddots &\vdots\\
  & \dots &  
\end{array}\right)\left(\begin{array}{c}
\vdots\\
\vdots\\
\vdots\\
\end{array}\right)$.


\subsubsection{Base continue}
On va ici partir d'une famille où cette fois ci l'indice sera "continu". 
\begin{equation}
\{\ket{u_\alpha}\}\quad \alpha\in\mathbb{R}
\end{equation}
On peut comme précédemment fabriquer des états orthogonaux. Ce qui va jouer 
le rôle orthonormalisation standard est le relation
\begin{equation}
\underline{\bra{u_\alpha}\ket{u_{\alpha'}} = \delta(\alpha-\alpha')}
\end{equation}
où $\delta$ est la fonction de Dirac. La subtilité est que $u_\alpha$ n'est 
pas toujours un état physique\footnote{La fonction de Dirac étant un pic de hauteur infinie
mais d'aire unitaire, il est difficile de la retrouver dans la nature...}.
 Cependant, il peut toujours être utilisé pour 
décrire un état qui lui, est bien physique. Comme pour le cas continu, il 
est possible d'exprimer le ket dans la base. Les sommes seront ainsi 
remplacées par es intégrales et les coefficients de Fourier par une fonction 
jouant le même rôle.
\begin{equation}
\ket{\psi} = \int d\alpha\ C(\alpha)\ket{u_\alpha}
\label{eq:2.12}
\end{equation}
où les $C(\alpha)$ renseignent sur le poids. Il est possible, comme précédemment, 
de déterminer ceux-ci en multipliant ce ket par un autre élément de la base.
\begin{equation}
\bra{u_{\alpha'}}\ket{\psi} = \int d\alpha\ C(\alpha) \underbrace{
\bra{u_{\alpha'}}\ket{u_\alpha}}_{=\delta(\alpha-\alpha')} = C(\alpha')
\end{equation}
On peut alors ré-écrire \autoref{eq:2.12}
\begin{equation}
\begin{array}{ll}
\ket{\psi} &= \int d\alpha\ \bra{u_\alpha}\ket{\psi}\ket{u_\alpha}\\
&= \int d\alpha\ \ket{u_\alpha}\bra{u_\alpha}\ket{\psi}\\
&=\displaystyle\left(\int d\alpha\ \ket{u_\alpha}\bra{u_\alpha}\right)\ket{\psi}
\qquad \forall\psi
\end{array}
\end{equation}
Le terme entre parenthèse n'est, par identification, rien d'autre que l'opérateur 
identité $\hat{\mathbb{1}}$ que l'on peut également voir comme un opérateur projecteur
$\hat{P_\alpha}$. La relation de fermeture s'écrit alors
\begin{equation}
\int d\alpha\ \underbrace{\ket{u_\alpha}\bra{u_\alpha}}_{\hat{P_\alpha}} = \hat{
\mathbb{1}}
\end{equation}
Comme nous l'avons fait pour le cas de la base discrète, montrons comment écrire 
un bra, ket, opérateur linéaire, \dots
\begin{equation}
\begin{array}{llll}
\text{ket } & \ket{\psi} &= \int d\alpha\ c(\alpha)\ket{u_\alpha} & \text{ où } 
c(\alpha) = \bra{u_\alpha}\ket{\psi}\\
& \ket{\varphi} &= \int d\alpha\ b(\alpha)\ket{u_\alpha} & \text{ où } b(\alpha) 
= \bra{u_\alpha}\ket{\varphi}\\
\text{bra } & \bra{\varphi} &= \int d\alpha\ b^*(\alpha)\bra{u_\alpha}
\end{array}
\end{equation}
Voyons maintenant pour le produit scalaire
\begin{equation}
\begin{array}{ll}
\bra{\varphi}\ket{\psi} = \bra{\varphi}\mathbb{1}\ket{\psi} &= \int d\alpha\ 
\underbrace{\bra{\varphi}\ket{u_\alpha}}_{b^*(\alpha)}\underbrace{\bra{u_\alpha}
\ket{\psi}}_{c(\alpha)}\\
&= \int d\alpha\ b^*(\alpha)c(\alpha)
\end{array}
\end{equation}
Par une analyse semblable pour les opérateurs. 
Définissons $\hat{A}$ :
\begin{equation}
\hat{A} = \mathbb{1}\hat{A}\mathbb{1} = \iint d\alpha\ d\alpha'\ \ket{u_\alpha}
\underbrace{\bra{u_\alpha}\hat{A}\ket{u_{\alpha'}}}_{A(\alpha,\alpha')}\bra{u_{\alpha'}}
\end{equation}
où $A(\alpha,\alpha')$ correspond aux éléments de matrice. \\
%Pour finir, intéressons nous au cas de l'opérateur adjoint\footnote{TT : détaille plus 
%ce qu'on essaye de montrer, c'est un peu pas clair :/}
Pour finir, examinons ce qui se passe quand on décompose en vecteurs de base un ket
auquel on a appliqué un opérateur :
\begin{equation}
\begin{array}{ll}
\ket{\psi'} = \hat{A}\ket{\psi} = \int d\alpha\ \overbrace{a(\alpha)}^{coef de Fourier}\ket{u_\alpha},
\qquad a(\alpha) &= \bra{u_\alpha}\ket{\psi'} = \bra{u_\alpha}\hat{A}\ket{\psi}\\
&= \int d\alpha'\ \underbrace{\bra{u_\alpha}\hat{A}\ket{u_{\alpha'}}}_{A(\alpha,
\alpha')}\underbrace{\bra{u_{\alpha'}}\ket{\psi}}_{c(\alpha')}
\end{array}
\end{equation}
%Nous nous intéressions à savoir ce qu'était $a(\alpha)$, c'est maintenant chose faite
On voit que le coefficient de Fourier du ket $\psi'$ est l'intégrale des produits des éléments de matrice
de l'opérateur avec les coefficient de Fourier du ket $\psi$ :
\begin{equation}
\underline{a(\alpha) = \int d\alpha'\ A(\alpha,\alpha')c(\alpha')}
\end{equation}
Dans le cas discret, nous avions $a_i = \sum_j A_{ij}c_j$. Ici le résultat est identique, 
mais sous forme intégrale. Cette expression illustre bien à quel point la notation de 
Dirac est compacte.


\subsection{Exemple de base de représentation continue : la base position}
Il existe bien évidemment plusieurs bases. L'une d'entre elles se nomme 
\textbf{base position}
\begin{equation}
\left\{\ket{\vec{r}}\right\}\qquad \vec{r}\in\mathbb{R}^3
\end{equation}
N'importe quel ket pourra s'exprimer dans cette base position. Le relation 
fermeture pour la base position nous dis que l'intégration sur tout l'espace
du projecteur pour une position donnée donnée l'identité.\footnote{Projeter
un ket sur un vecteur de la base position $\ket{r}$ revient quasiment à évaluer la 
probabilité que la particule se trouve au point r. On comprend intuitivement
que la particule doit se trouver quelque part dans l'espace et donc que
la somme des probabilités pour qu'elle se trouve en chacun des points doit
être égale à un.}
\begin{equation}
\int d\vec{r}\ \ket{\vec{r}}\bra{\vec{r}} = \vec{\mathbb{1}}
\end{equation}
Compte-tenu de cette relation, on peut exprimer les ket au moyen des $\ket{r}$
\begin{equation}
\ket{\psi} = \int d\vec{r}\ \ket{\vec{r}}\underbrace{\bra{\vec{r}}\ket{
\psi}}_{\psi(\vec{r})}
\end{equation}
où $\psi(\vec{r})$ est la \textit{fonction d'onde}\footnote{La fonction d'onde
utilisée en mécanique ondulatoire n'est donc rien d'autre que le "poids" qu'on associe
à chaque $\ket{r}$ quand on décrit l'état $\ket{\psi}$ de la particule. On retrouve
bien le fait que la fonction d'onde est liée à la probabilité qu'une particule se trouve
en un point de l'espace.}. En plaçant la relation 
d'identité à droite du bra, on obtient
\begin{equation}
\bra{\varphi} = \int d\vec{r}\ \underbrace{\bra{\varphi}\ket{\vec{r}}}_{
\varphi^*(\vec{r})}
\bra{\vec{r}}
\end{equation}
où $\varphi^*(\vec{r})$ est le complexe conjugué de la fonction d'onde du 
ket associé.
% En effet, nous avons\footnote{Dans quel but ? TT?}
Le produit scalaire entre deux vecteurs d'état devient donc :
\begin{equation}
\begin{array}{ll}
\bra{\varphi}\ket{\psi} = \bra{\varphi}\mathbb{1}\ket{\psi} &= \int d\vec{r}\ 
\bra{\varphi}\ket{\vec{r}}\bra{\vec{r}}\ket{\psi}\\
&= \int d\vec{r}\ \varphi^*(\vec{r})\psi(\vec{r})
\end{array}
\end{equation}
qui n'est rien d'autre que le produit scalaire usuel pour des fonctions complexes (cf cours
d'analyse).
Considérons le cas où $\ket{\psi}=\ket{\vec{r'}}$
% \footnote{TT : complète plz, j'sais plus pq on fait ça}
pour trouver la fonction d'onde correspondant à un vecteur de la base position.
\begin{equation}
\begin{array}{ll}
\ket{\vec{r'}} &= \displaystyle \int d\vec{r}\ \ket{\vec{r}}\underbrace{\bra{\vec{r}}\ket{\vec{r'}}}
_{\psi_{\vec{r'}}(\vec{r)}}\\
&\rightarrow\psi_{\vec{r'}}(\vec{r}) =\displaystyle \delta(\vec{r}-\vec{r'})
\end{array}
\end{equation}
Le delta de Dirac est dépourvu de sens physique, celui-ci représentant un étant 
infiniment localisé. Or, nous savons que de telle solutions divergent. Malgré tout, 
même sans être un état physique, 
il est pratique pour former une base.



\section{Observable}
Une \textit{observable} est un opérateur linéaire hermitien $\hat{A}$ associé à 
une grandeur physique observable $A$. A toute grandeur physique observable, il 
est possible de lui associer un opérateur linéaire  hermitien. Si $\ket{\psi}$
est un vecteur propre de $\hat{A}$,
\begin{equation}
\hat{A}\ket{\psi} = \lambda\ket{\psi}
\end{equation}
où $\lambda$ est une valeur propre pouvant être dégénérée (la dimension 
du sous-espace propre associé à $\lambda$ est appelée la \textit{dégénérescence}) 
ou non dégénérée.\\
Une classification importante se base sur le \textit{spectre}. Le spectre est
l'ensemble de toutes les valeurs propres possibles $\{\lambda\}$ de l'opérateur. Si celui-ci 
est discret, on retrouvera un système lié (particule dans une boîte par exemple). On 
peut également avoir des spectres continus (systèmes libres), utiles dans la 
\textit{théorie des collisions/diffusion}.\\

\subsubsection{Propriétés d'une observable}

%Intéressons-nous aux propriétés de notre base.
\begin{enumerate}
\item Toute valeur propre $\lambda$ est réelle.
\begin{equation}
\begin{array}{ll}
A\ket{\psi} &= \lambda \ket{\psi}\qquad\qquad\qquad \left(\hat{A}^\dagger = 
\hat{A}\right)\\
\bra{\psi}A\ket{\psi} = \lambda\underbrace{\bra{\psi}\ket{\psi}}_{=1}
\end{array}
\end{equation}
En prenant le complexe conjugué, on trouve
\begin{equation}
\lambda^* = \bra{\psi}A^\dagger\ket{\psi} = \bra{\psi}A\ket{\psi}=\lambda\in
\mathbb{R}
\end{equation}

\item Les vecteurs propres associés à deux valeurs propres distinctes sont 
orthogonaux.\\
Soit $\lambda_1\neq\lambda_2$
\begin{equation}
\left\{\begin{array}{ll}
A\ket{\psi_1} &=\lambda_1\ket{\psi_1}\\
A\ket{\psi_2} &=\lambda_2\ket{\psi_2}
\end{array}\right.
\end{equation}
En multipliant la première ligne par $\bra{\psi_2}$ et la seconde par $\bra{\psi_1}$
\begin{equation}
\left\{\begin{array}{ll}
\bra{\psi_2}A\ket{\psi_1} &=\lambda_1\bra{\psi_2}\ket{\psi_1}\\
\bra{\psi_1}A\ket{\psi_2} &=\lambda_2\bra{\psi_1}\ket{\psi_2}
\end{array}\right.
\label{eq:Syst}
\end{equation}
Considérons le complexe conjugué de la deuxième équation de \autoref{eq:Syst} 
\begin{equation}
\bra{\psi_2}A^\dagger\ket{\psi_1} = \lambda_2^*\bra{\psi_2}\ket{\psi_1}\qquad
\leftrightarrow
\qquad\bra{\psi_2}A\ket{\psi_1} = \lambda_2\bra{\psi_2}\ket{\psi_1}
\end{equation}
Le membre de gauche de cette expression est identique au membre de gauche de 
la première équation de \autoref{eq:Syst}. En effectuant la différence :
\begin{equation}
0 = (\lambda_1-\lambda_2)\bra{\psi_2}\ket{\psi_1},\qquad \lambda_1\neq\lambda_2 
\quad\Rightarrow\quad \bra{\psi_2}\ket{\psi_1} = 0
\end{equation}
\end{enumerate}
Dans le cas ou $\lambda_n$ est $g_n$ fois dégénérée, on utilisera la notation
\begin{equation}
\hat{A}\ket{\psi_n^i} = \lambda_n\ket{\psi_n^i}\qquad \lambda_n \rightarrow i=
1,2,\dots, g_n
\end{equation}
Ceci va nous amener à la décomposition spectrale des opérateurs ; il est 
toujours possible de les représenter dans une base. Une base particulièrement 
intéressante est celle composée des vecteurs propres de l'observable qui nous 
intéresse.

\subsubsection{Décomposition spectrale de A}
Comme introduit en fin de section précédente, deux indices sont utilisés : 
un pour désigner la valeur propre et l'autre pour la dégénérescence.
\begin{equation}
\hat{A}\ket{\psi_n^i} = \lambda_n\ket{\psi_n^i}\qquad \lambda_n \rightarrow i=
1,2,\dots, g_n
\end{equation}
La relation d'orthonormalité est toujours d'application
\begin{equation}
\bra{\psi_n^i}\ket{\psi_{n'}^{i'}} = \delta_{n,n'}\delta_{i,i'}
\end{equation}
Notons que l'algorithme de Gram-Schmidt permet de toujours pouvoir trouver une
base orthonormée au sein d'un sous-espace dégénéré.\\
Rappelons-nous que la relation de fermeture nous indique que si on somme les 
projecteurs sur tous les vecteurs de la base, on retrouvera l'identité. Dans 
notre cas, la relation de fermeture s'énonce (pour toute base, il 
est possible d'écrire une relation équivalente)
\begin{equation}
\sum_n\sum_{i=1}^{g_n} \ket{\psi_n^i}\bra{\psi_n^i} = \hat{\mathbb{1}}
\end{equation}

\textsc{Théorème de décomposition spectrale}\\
\begin{equation}
\begin{array}{ll}
\hat{A} &= \hat{A}\mathbb{1} = \sum_n\sum_{i=1}^{g_n} \hat{A}\ket{\psi_n^i
}\bra{\psi_n^i}\\
\hat{A} &= \sum_n \lambda_n \overbrace{\sum_{i=1}^{g_n} \ket{\psi_n^i}\bra{
\psi_n^i}}^{\text{Somme de proj. = proj.}}\\
\hat{A} &= \sum_n \lambda_n \hat{P}_n
\end{array}
\end{equation}
%mise en page un peu moche :(
où $\hat{P}_n$ est le projecteur sur le sous-espace propre de rang $g_n$ associé à $n$.
 Montrons que les $\hat{P}_n$ sont des projecteurs orthogonaux

\begin{equation}
\begin{array}{ll}
\hat{P_n}\ket{\psi^{i'}_{n'}} = \sum_n \sum_i \ket{\psi^i_{n}}
 \underbrace{\bra{\psi^i_{n}}\ket{\psi^{i'}_{n'}}}_{\delta_{n n'} \delta_{i i'}} 
= \delta_{n n'} \ket{\psi^{i'}_{n'}}\\
\hat{P_n}\hat{P_{n'}} = \sum_{i'} \hat{P_n} \ket{\psi^{i'}_{n'}}\bra{\psi^{i'}_{n'}}
 = \delta_{n n'} \sum_{i'}\ket{\psi^{i'}_{n'}}\bra{\psi^{i'}_{n'}}
 = \delta_{n n'}\hat{P_{n'}}\\
\rightarrow \hat{P_n}\hat{P_{n'}} = \delta_{n n'}\hat{P_{n'}}
\end{array}
\end{equation}




\textsc{Observable à spectre continu}\\
Considérons un triplet d'opérateur\footnote{Deuxième relation montre la valeur 
propre. L'intérieur du ket fait référence à la valeur propre associée.}
%On n'adopterait pas une autre notation pour désigner la valeur propre 
%(genre $\overline{r}$) ? Un peu ambigu.}
\begin{equation}
\hat{\vec{r}} = (\hat{x},\hat{y},\hat{z}),\qquad \hat{\vec{r}}\ket{\vec{r}} = 
\vec{r}\ket{\vec{r}}
\end{equation}
Effectuons la décomposition spectrale de $\hat{\vec{r}}$
\begin{equation}
\hat{\vec{r}} = \int d\vec{r}\ \vec{r}\ \ket{\vec{r}}\bra{\vec{r}}
\end{equation}
Ceci est l'intégrale d'un opérateur, donnant lieu à un opérateur (en réalité 
trois, car il s'agit d'un triplet d'opérateur).La décomposition spectrale peut 
s'écrire comme ceci. \\
A la place de l'opérateur $\hat{\vec r}$, considérons l'opérateur $\hat{\vec{\mathbb{1}}}$
 : comme il n'a que une seule valeur propre, nous retrouvons
\begin{equation}
\hat{\vec{\mathbb{1}}} = \int d\vec{r}\ \ket{\vec{r}}\bra{\vec{r}}
\end{equation}
Ceci montre que le relation de fermeture n'est rien d'autre que la 
décomposition spectrale d'un opérateur particulier.\\

La base impulsion est liée à la décomposition en valeur propre de l'opérateur 
impulsion (triplet d'opérateur) :
\begin{equation}
\hat{\vec{p}} = \left(\hat{p_x},\hat{p_y},\hat{p_z}\right),\qquad \hat{
\vec{p}}\ket{\vec{p}} = \vec{p}\ket{\vec{p}}
\end{equation}
Connaître exactement l'impulsion implique une position étalée sur l'infini 
d'énergie infinie, ce n'est pas physique.
\footnote{Phrase bizarre... En plus, on n'a pas encore parlé de l'incertitude
à ce stade-ci du cours...} 
Il est possible d'effectuer la décomposition spectrale
\begin{equation}
\hat{\vec{p}} = \int d\vec{p}\ \vec{p} \ket{\vec{p}}\bra{\vec{p}}
\end{equation}
La fonction d'onde peut elle aussi s'exprimer dans la base impulsion
\begin{equation}
\bra{\vec{p}}\ket{\psi}
\end{equation}

En réalité, les bases position et impulsion sont liées par une transformée de Fourier
et il est possible d'effecter le changement de base grâce à $\bra{\vec{r}}\ket{
\vec{p}})$ mais ceci fera l'objet d'un autre chapitre du cours.
%Il sera possible d'utiliser le changement de base de Fourier ($\bra{\vec{r}}\ket{
%\vec{p}})$ pour exprimer cette fonction d'onde dans la base position 
%$\bra{\vec{r}}\ket{\psi}$.\\

Afin d'introduire la notion d'\textit{ECOC}, rappelons la décomposition spectrale
\begin{equation}
\hat{A}\ket{\psi_n'} = a_n\ket{\psi_n'}\quad i=1,\dots,g_n\qquad\rightarrow\qquad
\hat{A} = \sum_na_n\hat{P_n}\quad\text{ où }\ \hat{P_n}=\sum_i\ket{\psi_n^i}\bra{
\psi_n^i}
\end{equation}
On peut montrer que \textit{lorsque les observables d'un ensemble commutent
deux à deux, il existe au moins une base de vecteurs propres communs à ces observables.} 
\begin{equation}
\left\{\hat{A},\hat{B},\hat{C},\dots\right\}\quad\rightarrow\quad \left\{n_a,
n_b,n_c,\dots\right\}
\end{equation}
où $n_a$ est le nombre quantique associé à $\hat{A}$, etc. Cette base de vecteurs
propres communs constituera une base privilégiée pour l'étude de problèmes
faisant intervenir cet ensemble d'observables. A priori, dans ces conditions,
la base obtenue ne sera pas forcément unique. On parle ainsi d'\textit{ECOC} 
(ensemble \textbf{complet} d'observables qui commutent deux à deux) : si la 
valeur propre de chacune des observable est fixée, il existe un unique état de 
la base. \\

Par exemple, si on étudie la dépendance angulaire
\begin{equation}
\left\{\hat{L}^2, \hat{L_z}\right\} \rightarrow \ket{l,m}
\end{equation}
Ces deux opérateurs commutent. On retrouve par ailleurs le nombre quantique 
orbital ainsi que le nombre quantique magnétique. Pour une particule dans un 
potentiel central 
\begin{equation}
\left\{\hat{H},\hat{L}^2, \hat{L_z}\right\} \rightarrow \ket{n_r,l,m}
\end{equation}
où $n_r$ est le nombre quantique radial associé à $\hat{H}$.