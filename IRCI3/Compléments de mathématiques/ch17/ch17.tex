\chapter{Polynômes orthogonaux, fonctions spéciales et résolution d'ED 
par séries de puissances}

\setcounter{section}{16}
\section{Polynômes orthogonaux}
\setcounter{subsection}{1}
\subsection{Espaces à produit scalaire}
Pour toute fonction $w$ continue et positive sur $[a,b]$ :
\begin{equation}
	<a,b> := \int_a^b w\ f\ g^*
\end{equation}
Celui-ci défini un produit scalaire hermition sur l'EV $C^0([a,b], 
\mathbb{K})$ et un presque" produit scalaire sur\footnote{Fonction de 
	carré sommable si $\int_a^b |f|^2 w < +\infty$.} $L^2([a,b],\mathbb{
	K})$.\\
	
Soit $E$ un espace pré-hilbertien et $E_n$ un sous-espace de dimension 
$n$ de $E$. Notons $pr_{E_n}\ f$ la projection de $f$ sur $E_n$.

\proposition{$pr_{E_n}\ f$ existe ($n<\infty$) et est la meilleure 
	approximation de $f\in E$ par un élément de $E_n$ pour $\|\ \|$.}
	
	
\subsection{Polynômes orthogonaux}
Considérons une fonctions poids $w \in C^0(]a,b[,\mathbb{R}_0^+)$ 
telle que $\forall n \in \mathbb{N} : \int_a^b |x|^n w(x)dx$ converge. 
Soit $E$ l'EV des fonctions continues dans $]a,b[$ et de carré sommable 
pour le produit scalaire
\begin{equation}
	<f,g> := \int_a^b f(x)g^*(x)w(x)dx
\end{equation}
Notons $\|\ \|_2$ la norme $L_2$, la norme en moyenne quadratique. Donc 
\begin{equation}
	E = C^0(]a,b[,\mathbb{R})\cap L_2(]a,b[, \mathbb{R})
\end{equation}
Un polynôme $p_n$ de degré $n$ est dit \textbf{unitaire} (ou \textbf{
	monique}) ssi le coefficient de $x^n$ dans $p_n(x)$ vaut 1.\\
	
\theor{\ \\
	$\forall w, \exists 1!$ suite de polynôme unitaire $(p_n)_{n\in\mathbb{N}}
	$ tels que
	\begin{equation}
		\left\{\begin{array}{l}
		\text{deg } p_n = n\\
		p_n \perp p_m\ \text{ si } n \leq m
		\end{array}\right.
	\end{equation}}
	
\begin{proof}\ \\
	$\forall n$ les polynômes $1,x,x^2,\dots,x^n$ sont L.I. dans $\mathcal{P}$. 
	Il s'agit d'une suite de polynômes unitaires de degré 0,1,dots, $n$. En 
	appliquant Gram-Schmidt on obtient une suite satisfaisant aux hypothèses.
	\begin{equation}
		\begin{array}{ll}
			p_0 & := 1                                                     \\
			p_n & := x^n - \text{proj }_{\mathcal{P}_{n-1}} x^n            \\
			    & = x^n - \sum_{k=0}^{n-1} \frac{<x^n, p_k>}{<p_k,p_k>}p_k 
		\end{array}
	\end{equation}
\end{proof}
Voir le théorème 17.31 page 113 également, un peu la flemme.
	
	
	
\theor{\ \\
	$\forall w|_{[a,b]}, p_n$ possède $n$ zéros distincts dans $]a,b[$.}\ 
	
\begin{proof}\ \\
	Soit $x_1, \dots, x_k$ les zéros de $p_n$ dans $]a,b[$ et leurs 
	multiplicités $m_1, \dots, m_k$. On a que $m_1 + \dots + m_k 
	\leq n = $deg $p_n$. Définissons $\epsilon_i := \left\{\begin{array}{ll}
	0 & \text{si $m_i$ pair}\\
	1 & \text{si $m_i$ impair}
	\end{array}\right.$ et construisons le produit suivant 
	\begin{equation}
		q(x) :=\Pi_{i=1}^k (x-x_i)^{\epsilon_i}\qquad \text{ deg } q \leq k \leq n
	\end{equation}
	Je multiplie $p_n$ par $q$ : si l'exposant est impair je l'augmente d'une 
	unité et s'il est pair je ne fais rien. Le polynôme $p_nq$ a pour zéros 
	$x_1, \dots, x_k$ avec comme multiplicité $m_1+\epsilon_1+\dots+m_k+
	\epsilon_k$, toutes paires. Dans $]a,b[\backslash\{x_1,\dots,x_k\}, p_nq$ 
	est de signe constant\footnote{??}. On a alors 
	\begin{equation}
		<p_n, q> = \int_a^b p_nqw \neq 0
	\end{equation}
	Or $p_n \perp \mathcal{P}_{n-1}$ et deg $q \leq n$. On en déduit que deg $q 
	= n \Longrightarrow k = n$ et tous les $m_i = 1$.
\end{proof}
	
	
\theor{\ \\
	$\forall f \in E, \forall n \in \mathbb{N}, \exists 1!$ polynôme $q_n \in 
	\mathcal{P}_n$ tel que $\| f-q_n\| = \min \{\|f-p\|_2; p \in \mathcal{P}_n
	\} =: d_2(f, \mathcal{P}_n)$}\ \\
On dit que $q_n$ est le polynôme de meilleure approximation quadratique de 
$f$ à l'ordre $n$.
\begin{proof}
	Cf. Analyse II
\end{proof}
	
\theor{\ \\
	Si $]a, b[$ est \textbf{borné} (!), alors $\forall f \in E : \lim\limits_{n
		\rightarrow	\infty} \| f - q_n \|_2 = 0$}
Cela signifie que l'ensemble $(p_k)_{k\in\mathbb{N}_0}$ est complet 
relativement à $E$ ; le développement de Fourier va converger quadratiquement.
	
\begin{proof}\ \\
	Pour se simplifier la vie, on travaille dans un fermer et $f$ est continue en 
	$a$ et $b$ : $f \in C^0([a,b])$. Considérons $r_n$, un polynôme de degré au 
	plus $n$ étant la meilleure approximation \textsc{uniforme} de $f$ dans 
	$\mathcal{P}_n$ (alors que $q_n$ est le champion en quadratique). On a donc
	\begin{equation}
		\begin{array}{ll}
			\|f - q_n\|_2^2 \leq \|f-r_n\|_2^2 & = \int_a^b |f-r_n|^2w              \\
			                                   & \leq \| f-r_n\|_\infty^2\int_a^b w 
		\end{array}
	\end{equation}
	En effet\footnote{On peut majorer cette fonction par le suprémum de ce carré 
	et le sortir de l’intégrale. (On majore l'intégrale comme d'hab?)}, par 
	définition $\|f-r_n\|_\infty^2 = \sup_{[a,b]} |f-r_n|^2$. Par le théorème de 
	Weirestrass $\| f - r_n\|_\infty^2 \rightarrow 0$ quand $n \rightarrow\infty$.
\end{proof}
	



\section{Théorème d'approximation de Weierstrass}
\subsection{Fonctions continues comme limites uniformes de polynômes}
\theor{\ \\
	Soit $f : [a,b] \rightarrow \mathbb{R}$ une fonction continue. Pour tout $
	\epsilon>0$, il existe une fonction polynomiale $p$ telle que $\|f-p\|_\infty 
	< \epsilon$}\ \\
Autrement dit, $\forall \epsilon > 0, \exists$ polynôme $p(x)$ tel que 
\begin{equation}
	\forall x \in [a,b] : p(x) - \epsilon \leq f(x) \leq p(x)+\epsilon
	+
\end{equation}
	
\subsection{Suites de Dirac}
On définira l'outil de Dirac 
\begin{equation}
	\varphi_n(x) := \left\{\begin{array}{ll}
	0 & \text{ si } x \notin [-1,1]\\
	\mu_n(1-x^2)^n & \text{ si } x \in [-1,1]	
	\end{array}\right.
\end{equation}
où $\mu_n$ est un réel tel que $\int_{-1}^1 \varphi_n = 1$ et donc 
\begin{equation}
	\int_{-\infty}^\infty \varphi_n = 1
\end{equation}
On peut prouver qu'il $\exists\mathcal{C} : \mu_n \leq \mathcal{C}\sqrt{n}$.\\
	
	
\subsection{Polynôme $p_n$ d'approximation "à la Dirac"}
Supposons que $a$ et $b$ sont compris entre 0 et 1 (par changement de variable 
par exemple), ceci n'est pas restrictif. Définitions le produit de convolution 
suivant (et utilisons l'outil de Dirac) :
\begin{equation}
	\begin{array}{ll}
		\forall \xi \in [a,b] : p_n(\xi) & := \int_0^1 f(x)\varphi_n(x-\xi)dx    \\
		                                 & = \mu_n\int_0^1 f(x)(1-(x-\xi)^2)^ndx 
	\end{array}
\end{equation}
où $(1-(x-\xi)^2)^n$ est un polynôme en $\xi$ de degré $2n$.\\
Deux cas :
\begin{enumerate}
	\item Si $x \notin V_\xi$, alors $\varphi_n(x-\xi).f(x) \approx 0$ au point 
	      que l'$\int = 0$. En effet, $\varphi_n(x-\xi)$ est très centrée autour de 
	      $\xi$. Si on en est loin, le produit est forcément proche de zéro.
	\item Si $x \in V_\xi$, on remplace brutalement $f(x)$ par $f(\xi)$ et on peut 
	      le sortir de l'intégrale. Il ne reste que l'intégrale de $\varphi_n$ près de $x$ 
	      et celle ci $\approx 1$
	      \begin{equation}
	      	\int_{V_\xi} f(x).\varphi_n(x-\xi)dx \approx f(\xi)\int_{V_0}\varphi_n \approx 
	      	f(\xi)
	      \end{equation}
\end{enumerate}
	
\setcounter{section}{8}
\section{L'ELD de Tchebychev}
\subsection{Présentation trigonométrique des polynômes de Tchebychev}
On peut utiliser la formule du binôme de Newton sur le deuxième membre de la 
formule d'Euler $\cos(n\theta) + i\sin(n\thetaà = (\cos\theta + i\sin\theta)^n
$. On obtient alors : 
\begin{equation}
	\cos^n\theta + n\cos^{n-1}\theta(i\sin\theta) + \left(\begin{array}{c}
	n\\
	2
	\end{array}\right) \cos^{n-2}\theta(i\sin\theta)^2 + \dots + (i\sin\theta)^n
\end{equation}
En identifiant les parties réelles des deux membres :
\begin{equation}
	\cos(n\theta) = \text{ combili}(\cos^{n-2k}\theta\underbrace{(i\sin\theta)^{2
		k}}_{(-1)^k(1-\cos^2\theta)^k})
\end{equation}
soit un polynôme en $\cos\theta$.\\
On définit le $n^e$ polynôme de Tchebychev à partir de $\cos n\theta =: T_n(\cos
\theta)$ où $T_n$ est défini sur $[-1,1]$. Autrement dit :
\begin{equation}
	T_n(x) = \cos(n\arccos x)
\end{equation}
Ceci revient par exemple à considérer $T_n(\cos\theta)$ comme le développement de $\cos n\theta$ sous forme de polynôme en $\cos\theta$.
Le polynôme $T_n : \mathbb{R}\rightarrow\mathbb{R}$ de degré $n$ en $x\in\mathbb{R}$ 
est obtenu en prolongeant $T_n|_{[-1,1]}$. Il n'est \textbf{pas} unitaire et est 
normé pour la norme sup.
	
\begin{equation}
	\|T_n\|_{\infty,[-1,1]} = \sup_{x \in [-1,1]}|T_n(x)| = \sup_{\theta \in [0,\pi]} 
	|\cos n\theta| = 1
\end{equation}		
	
La fonction $y = \cos(n\theta)$ est solution de 
\begin{equation}
	\frac{d^2y}{d\theta^2} + n^2y = 0
\end{equation}
En posant $x:= \cos\theta \rightarrow \tilde{y}(x) = y(\cos n\theta)$ on obtient 
l'\textbf{EDL de Tchebychev}
\begin{equation}
	(1-x^2)\frac{d^2\tilde{y}}{dx^2} - x\frac{d\tilde{y}}{dx} + n^2\tilde{y} = 0
\end{equation}
Une telle EDL (si $n^2 \in \mathbb{R}^+$) admet une solution $\tilde{y}$ dont la 
dérivée est bornée pour $x\rightarrow 1 \Leftrightarrow n^2$ est un carré parfait 
$\Leftrightarrow \tilde{y}$ est un polynôme de Tchebychev de degré $n$.
	
\setcounter{subsection}{3}
\subsection{Quelques propriétés des polynômes de Tchebychev}
Voir syllabus page 54 et slide 26.\\
Les polynômes unitaires de Tchebychev sont $2^{1-n}T_n$, c'est-à-dire que $\|T_n
\|_{\infty, [-1,1]} = \|\cos(n\theta)\|_{\infty, [0,\pi]} = 1$. Donc 
\begin{equation}
	\|2^{1-n}T_n\|_{\infty,[-1,1]} = 2^{1-n} = \left(\frac{1}{2}\right)^{n-1}
\end{equation}
Ce polynôme réalise le minimum pour cette norme, c'est le "champion" : celui 
possédant la plus petite distance dans l'espace des polynômes unitaires est 
donné par Tchebychev.\\
On en tire la \textit{Proposition du \textbf{minimax}}\\
\proposition{Dans $\mathcal{P}\mathcal{U}_n$, les polynômes unitaires de degré 
	$n > 0$ :
	\begin{enumerate}
		\item $2^{1-n}T_n$ minimise $\|\ \|_{\infty,[-1,1]}$ dans l'ensemble des 
		      polynômes unitaires.
		\item $\|2^{1-n}T_n\|_{\infty,[-1,1]} = 2^{1-n} = \left(\frac{1}{2}\right)^{n-1}$
	\end{enumerate}
}
\begin{proof} [Démonstration par l'absurde]\ \\
	Supposons $P$, un polynôme unitaire dont la norme est plus petite, encore 
	meilleur que celui de Tchebychev : $\|P\|_{\infty, [-1,1]} < 2^{1-2}$. L'équation
	\begin{equation}
		2^{1-n}T_n(x) = 2^{1-n}\cos n\theta
	\end{equation}
	prends les valeurs $2^{1-n}, -2^{1-n}, 2^{1-n}, \dots, (-1)^n2^{1-n}$ aux points 
	$\theta = 0, \pi/n, 2\pi/n, \dots, n\pi/n$. Comme $P$ est la meilleure 
	approximation, la fonction $Q(x)$ ci-dessous est forcément positive (l'autre 
	terme étant forcément plus grand)
	\begin{equation}
		Q(x) := 2^{1-n}T_n(x) - P(x)
	\end{equation}
	$Q(x)$ a le même signe que $2^{1-n}$ en ces points, donc $Q$ à au moins $n$ 
	zéros sur $[-1,1]$. Or, deg $Q \leq n-1.$ (car $2^{1-n}T_n$ et $P$ sont unitaires, 
	du coup le terme avec le facteur 1 disparaît d'office : degré $n-1$). On a 
	contradiction.
\end{proof}
	

\setcounter{section}{18}
\section{Quadratures et polynômes orthogonaux}
\setcounter{subsection}{6}
\subsection{Quadrature par interpolation}
On préfère le mot quadrature à interpolation. Ces méthodes consistent à 
approcher l'intégrale $\int_a^b fw$ par une formule d'interpolation
\begin{equation}
	\int_a^b fw \approx \sum_{k=0}^l A_k f(x_k)
\end{equation}
où les $A_k$ sont les coefficients d'interpolation.
\subsection{Méthode de quadrature $M_{l,X}$}
Supposons que l'on possède $l$ points $x_0,\dots, x_l$. Soit $P_l(x_k) \approx 
f(x)$, le polynôme d'interpolation de degré $\leq l$ par les $(x_k, f(x_k))$. 
Au différents points connus, $f(x_k) = P_l(x_k) \forall k =0,\dots,l$.\\
On associe un \textit{polynôme de Lagrange} aux points $x_0,\dots,x_l$. Soit 
$L_k(x) :=$ le polynôme de degré $l$ tel que $\left\{\begin{array}{ll}
L_k(x_k) &= 1\\
L_k(x_j) &= 0\ \forall j \neq k
\end{array}\right.$ On l'écrit 
\begin{equation}
	L_l(x) = \prod_{j=0,\dots,l;j\neq k} \frac{x-x_j}{x_k-x_j}
\end{equation}
Le \textbf{polynôme d'interpolation} est combili des polynômes de Lagrange 
\begin{equation}
	P_l(x) = \sum_{k=0}^l f(x_k)L_k(x)
\end{equation}
Par exemple, pour $l=1 : P_1(x) = f(x_0)L_0(x) + f(x_1)L_1(x)$. Cette fonction 
vérifie bien les points donnés : $P_1(x_0) = f(x_0)*1 + 0, \dots$. Cette fonction 
$P_l(x)$ va alors servir à approcher des fonctions dont on connaît une série 
de points.\\
Ici l'idée est d'approcher $\int_a^b fw$ par $\int_a^b P_lw$ :
\begin{equation}
	\begin{array}{ll}
		\int_a^b fw \approx \int_a^b P_l(x)w(x) dx & = \int_a^b \sum_{k=0}^l f(x_k)              
		L_k(x)w(x)dx\\
		                                           & = \sum_{k=0}^l f(x_k)\int_a^b L_k(x)w(x) dx \\
		                                           & = \sum_{k=0}^l f(x_k)\ A_k := M_{l,X}(f)    
	\end{array}
\end{equation}
Il s'agit de la méthode $M_{l,\{x_0,\dots,x_l\}}$. L'approximation est fournie 
par la somme ci-dessous où les $A_k$ sont calculés indépendamment de $f$.\\
	
\lemme{L'ordre de toute méthode de quadrature $M_{l,X}$ (par interpolation) est 
	$\geq l$.}\ \\
Cela signifie que si on prend un polynôme de degré au plus $l$, l'approximation 
sera parfaite.
	
\begin{proof}\ \\
	Les points $x_0, \dots, x_l$ sont donnés. Si $f$ est un polynôme de degré $\leq 
	l$, alors $f$ \textbf{est} le polynôme d'interpolation de $f$ par les points 
	$x_0,\dots, x_l$. Donc
	\begin{equation}
		\int_a^b fw = \sum_{k=0}^l f(x_k)A_k
	\end{equation}
\end{proof}
	
\subsection{Méthode de Gauss}
\proposition{La méthode de quadrature par interpolation $M_{l,X}$ est d'ordre 
	$2l+1$ ssi les points d'interpolations $x_0,\dots,x_l$ sont les zéros du 
	$(l+1)^{\text{ème}}$ polynôme (noté $P_{l+1}$) dans la suite des polynômes 
	unitaires orthogonaux 	de degrés croissants pour le produit scalaire $<f,g> 
	= \int_a^b fgw$.}\ \\
Si $w=1$, on parlera de la \textit{méthode de Gauss}.
	
\begin{proof}\ \\
	Définissons le polynôme unitaire $\alpha$ de degré $l+1$ : $\alpha(x) := (x-x_0)
	(x-x_1)	\dots(x-x_l) \perp_w \mathcal{P}_l$. Par hypothèse si l'ordre $\geq 2l+1$ 
	j'obtiens la solution exacte :
	\begin{equation}
		\forall f \in \mathcal{P}_{2l+1} : \int_a^b fw = \sum_{k=0}^l f(x_k)A_k\qquad 
		\text{ où } A_k = \int_a^b L_k(x)w(x)dx
	\end{equation}
	Notons que comme $\alpha'(x) = \prod_{j\neq l} (x_k-x_j)$, on a 
	\begin{equation}
		L_k(x) = \frac{\alpha(x)/(x-x_k)}{\alpha'(x)}
	\end{equation}
	Si $Q \in \mathcal{P}_l$ un polynôme quelconque\footnote{Attention, $\mathcal{P}_l$ 
		ne désigne pas seulement les polynômes unitaires!} de degré $\leq l : \alpha Q \in 
	\mathcal{P}_{2l+1}$, d'où
	\begin{equation}
		\int_a^b \alpha Q w = \sum_{k=0} \underbrace{\alpha(x_k)}_{=0}Q(x_k)A_k = 0
	\end{equation}
	$\Longrightarrow \forall Q \in \mathcal{P}_l : \alpha \perp_w Q \Rightarrow 
	\alpha \perp_w \mathcal{P}_l \Rightarrow \alpha = P_{l+1}$.
\end{proof}
	
	
\proposition{Toute méthode de quadrature par interpolation en $x_0, \dots, 
	x_l$ est d'ordre $\leq 2l+1$.}	\ \\
	
\begin{proof} [Démonstration par l'absurde]\ \\
	Par le même raisonnement : $\alpha \perp_m \mathcal{P}_{l+1}$ et donc $\alpha 
	\perp_m \alpha$, ce qui est absurde.
\end{proof}
	
\theor{\ \\
	Les méthodes de quadrature par interpolation polynomiales en $l+1$ points 
	sont d'ordre $\sigma : l \leq \sigma \leq 2l+1$.}\ \\
	
\begin{proof}
	Voir les trois lemmes précédents.
\end{proof}
	
\theor{\ \\
	La méthode $M_{l,X}$ est d'ordre maximal $(2l+1)$ ssi les points d'interpolation 
	$x_0, \dots, x_l$ sont les zéros du polynôme $P_{l+1}$ de degré $l+1 \perp_m 
	\mathcal{P}_l$ pour le produit scalaire\dots}\ \\

\begin{proof}
	Voir lemmes précédents	
\end{proof}
	
\proposition{Si $\alpha \perp_m \mathcal{P}_l$, alors la méthode est d'ordre 
	$\geq 2l+1$.}
	
\begin{proof}
	Voir page 136.
\end{proof}
	
	
\section{Résolution par séries : généralités}
	\subsection{EDL à coefficients analytiques}
	Considérons la série entière en $(x-x_0)$ :
	\begin{equation}
	\sum_{k=0}^\infty a_k(x-x_0)^k
	\end{equation}
	converge dans un voisinage de $x_0$ autour de $x_0$. Si l'on multiplie cette 
	série par $(x-x_0)^r$, on parlera de \textit{série de Frobenius}.\\

	Une fonction $f$ de classe $C^\infty$ au voisinage de $x_0$ est \textbf{analytique} 
	en $x_0$ ssi sa série de Taylor de $f$ autour de $x_0$ converge vers $f$ au 
	voisinage de $x_0$.\\
	
	\retenir{\ \\La série $\displaystyle \sum_{k=0}^\infty$ est \textbf{d'exposant $r$} ssi $
	\forall k > 0 : a_k = 0\quad \text{ et }\quad a_r \neq 0$? Son terme dominant pour $
	x\rightarrow0$ est le terme de plus bas degré (il donne une première approximation 
	du comportement pour cette limite.}
	
	\proposition{\begin{equation}
	\frac{P(x)}{Q(x)} = \frac{x^{r_P}(a_0+a_1x+\dots)}{x^{r_Q}(b_0+b_1x+\dots)}
	\end{equation}
	est analytique en 0 $\Leftrightarrow r_Q \leq r_P \Leftrightarrow$ 0 n'est pas 
	racine de $Q$ ou la multiplicité de 0 dans $Q \leq$ multiplicité de 0 dans $P$.}
	
	
	\subsection{EDL et série-solution}	
	Considérons l'équation (à droite, sous sa forme normale)
	\begin{equation}
	P(x)y'' + Q(x)y' + R(x)y = 0,\qquad y''+p(x)y'+q(x)y=0
	\end{equation}
	Cette équation est facilement généralisable aux ordres supérieurs, pareil pour la
	version non-homogène. On va chercher une solution au voisinage d'un point $x_0\in
	\overline{\mathbb{R}}$ 	de la forme
	\begin{equation}
	y(x) = \sum_{k=0}^\infty a_k(x-x_0)^k
	\end{equation}
	Si $x_0=0$, il s'agit de la série de Maclaurin. Si l'on désire, on peut effectuer 
	le changement $t :=x-x_0$ pour travailler autour de 0. Si $x_0=\infty$, poser 
	$t=1/x$ et travailler autour de zéro.
	
	\subsection{Point ordinaire, singulier, régulier ou non}
	Considérons deux prototypes, une EDLcc et une EDLee :
	\begin{enumerate}
	\item $y''+p_0y'+q_0y = 0$
	\item $x^2y''+xp_0y'+q_0y = 0$
	\end{enumerate}
	Remplaçons $p_0$ et $q_0$ par une fonction analytique en zéro, à savoir leur 
	développement de Maclaurin. Pour le point 0, trois cas sont possibles :
	\begin{enumerate}
	\item \textit{Ordinaire} : $y'' +\left(\sum_{k=0}^\infty p_kx^k\right)y' +
	\left(\sum_{k=0}^\infty q_kx^k\right)y = 0$
	\item \textit{Singulier régulier} : $x^2y'' + x\left(\sum_{k=0}^\infty p_kx^k
	\right)y'+\left(\sum_{k=0}^\infty q_kx^k\right)y=0$
	\item \textit{Singulier irrégulier} : $P(x)y'' + Q(x)y' + R(x)y = 0$
	\end{enumerate}
	\retenir{\ \\
	Soit la définition générale : $P(x)y'' + Q(x)y' + R(x)y = 0$. 
	\begin{itemize}
	\item[$\bullet$] $x_0$ est un \textbf{point ordinaire} ssi $P(x_0)\neq0$.
	\item[$\bullet$] $x_0$ est un \textbf{point singulier} ssi $P(x_0)=0$.
	\item[$\bullet$] $x_0$ est un \textbf{point singulier régulier} ssi
		\begin{itemize}
		\item[*] Soit $x_0$ est un zéro simple de $P$.
		\item[*] Soit $x_0$ est un zéro double de $P$, mais aussi un zéro de $Q$.
		\end{itemize}
	\end{itemize}}\ 
		
	Notons que dans ce cours, on travaillera toujours au voisinage de $x_0=0$ par 
	changement de variable.
	
	
\setcounter{section}{2}
\section{La théorie autour d'un point ordinaire}
	\setcounter{subsection}{1}
	\subsection{Unicité : formules de récurrence pour l'EDL normale}
	On cherche une solution de la forme $y(x) = \sum_{k=0}^\infty a_kx^k$ de l'
	équation $y''+p(x)y'+q(x)y=0$. On remplace :
	\begin{equation}
	\sum_{k=2}^\infty k(k-1)a_kx^{k-2} + \left(\sum_{k=0}^\infty p_kx^k\right)
	\left(\sum_{k=0}^\infty ka_kx^{k-1}\right) + \left(\sum_{k=0}^\infty q_kx^k
	\right)\left(\sum_{k=0}a_kx^k\right)=0
	\end{equation}
	Il faut changer un peu les termes pour commencer partout en $k=0$ : 
	$\displaystyle \sum_{k=2}^\infty k(k-1)a_kx^{k-2} = \sum_{k=0}^\infty (k+2)
	(k+1)a_{k+2}x^k$ et $\displaystyle\sum_{k=0}^\infty ka_kx^{k-1} 
	= \sum_{k=0}^\infty (k+1)a_{k+1}x^k$.\\
	En regroupant les termes en $x^k$ et en effectuant le produit de Cauchy : 
	\begin{equation}
	\sum_{k=0}^\infty\left((k+2)(k+1)a_{k+2}+\sum_{n=0}^\infty((n+1)a_{n+1}p_{k-n}
	+a_nq_{k-n})\right) = 0
	\end{equation}
	On trouve alors l'équation de récurrence $(R_k)$ :
	\begin{equation}
	\underbrace{(k+2)(k+1)}_{\neq0}a_{k+2}+\sum_{n=0}^\infty((n+1)a_{n+1}
	p_{k-n}+a_nq_{k-n}=0
	\end{equation}
	Grâce à cette équation, si $a_0, a_1$ sont donnés, on peut univoquement 
	déterminer tout le reste : cela prouve l'unicité des $a_k$. L'ennui est 
	que beaucoup de calculs ont été fait et il faudrait être assurer de trouver 
	un résultat avant ça.
	
	\subsection{Théorème de convergence de Fuchs}
	Ce théorème nous donne un \textbf{minorant} du rayon de convergence de la série.
	On l'énonce ici en toute généralité (et pas autour de 0).\\
	\theor{(admis)\\
	Soient $p$ et $q$ analytiques en $x_0$, de rayons de convergences respectifs 
	$\rho_p$ et $\rho_q$. Soient $a_0,a_1$, deux réels arbitraires. \\
	La solution de $y''+py'+qy=0$ satisfaisant à $y(x_0)=a_0$ et $y'(x_0)=a_1$ (CI de 
	Cauchy) est analytique en $x_0$ et peut s'écrire
	\begin{equation}
	\sum_{k=0}^\infty a_k(x-x_0)^k
	\end{equation}
	qui converge (au moins) sur $]x_0-\rho_m,x_0+\rho_m[$ où $\rho_m :=\min\{\rho_p,
	\rho_q\}$.}
	
	\setcounter{subsection}{4}
	\subsection{Récurrences pour l'EDL non-normale}
	Si l'on a une EDL $P(x)y''+Q(x)y'+R(x)y=0$ où $P, Q, R$ sont des polynômes, il 
	n'est pas utile de se ramener à $y''+p(x)+q(x)y=0$ : on ne s'encombre pas avec 
	les séries de Taylor
	
	
\setcounter{section}{4}
\section{EDL et polynômes de Hermite}
	\subsection{Les séries-solutions canoniques de Hermite}
	Voici l'\textbf{EDL de Hermite}, définie sur $\mathbb{R}$
	\begin{equation}
	y''-2xy'+2py=0\qquad\text{où }p\in\mathbb{R}
	\end{equation}
	On peut en tirer une équation de récurrence. En effet, tout point $x_0$ est 
	ordinaire, les coefficients $a_k$ de toute série solution $\sum_{k=0}^\infty 
	a_kx^k$ satisfont la relation de récurrence
	\begin{equation}
	a_{k+2} = -\frac{2(p-k)}{(k+2)(k+1)}a_k
	\end{equation}
	On peut en tirer deux séries-solutions canoniques : une paire et une impaire.
	Si $p\in\mathbb{N}$, alors une de ces solutions est un polynôme de degré 
	$n$ (à un multiple scalaire près) alors que l'autre est infinité de termes 
	non nuls (pas top-top).
	
	\subsection{Les polynômes de Hermite}
	En choisissant le multiple scalaire\footnote{??} comme valant $2^n$, on 
	définit le $n^{eme}$ polynôme de Hermite, noté $H_n$
	\begin{equation}
	\begin{array}{ll}
	H_n(x) &= \sum_{k=0}^{\lfloor\frac{p}{2}\rfloor} (-1)^k\frac{n!}{k!(n-2k)!}
	(2x)^{n-2k}\\
	&= 2^nx^n + \dots\\
	&= \text{le polynome de Hermite de degré $n$.}
	\end{array}
	\end{equation}
	
	Les section 17.5.3/4 donnent la fonction génératrice ainsi que quelques 
	propriétés intéressantes. Je ne l'inclus pas ici, mais à lire.
	
	\setcounter{subsection}{6}
	\subsection{L'EDP de l'oscillateur harmonique quantique 1D}
	Page 22-25 et slides 27-28. C'est plus une illustration qu'autre chose.
	
	
\setcounter{section}{6}
\section{Théorie à droite d'un point singulier régulier}
	\subsection{Série de Frobenius et exposant de la singularité}
	Considérons une EDLee $x^2y''+xp_0y'+q_0y=0$. Plus généralement, si
	0 est un point singulier régulier 
	\begin{equation}
	x^2y''+x\left(\sum_{k=0}^\infty p_kx^k\right)y'+\left(\sum_{k=0}^\infty 
	q_kx^k\right)y=0
	\end{equation}
	L'\textbf{espoir} et que si $r$ est solution de l'équation indicielle (
	cette solution est l'\textit{exposant de la singularité})
	$r(r-1)+p_0r+q_0=0$, alors $x^r$ est solution de l'EDLee et on peut espérer 
	une solution pour notre cas général de la forme
	\begin{equation}
	x^r\sum_{k=0}^\infty a_kx^{k} = \sum_{k=0}^\infty a_kx^{k+r}
	\end{equation}
	soit une \textbf{série de Frobenius} d'exposant $r$. Si $a_0\neq0$ on peut 
	l'écrire sous la forme
	\begin{equation}
	a_0x^r\left(1+\sum_{k=1}^\infty \frac{a_k}{a_0}x^k\right)
	\end{equation}
	qui sera dite \textbf{canonique} quand $a_0=1$. On verra que si $r_1,r_2$ sont 
	les deux exposants de la singularité on obtiendra deux solutions L.I. sauf 
	dans certains cas malheureux.
	
	\subsection{Convergence et dérivation des séries de Frobenius}
	La série $x^r\sum_{k=0}^\infty a_kx^{k}$ converge $\Leftrightarrow\sum_{k=0}^\infty 
	a_kx^{k+r}$ convergent et leurs sommes sont égales. La dérivée de cette série 
	converge uniformément sur $]0,R[$, on peut donc la dériver terme à terme.
	
	\setcounter{subsection}{3}
	\subsection{Calcul des coefficients de solutions formelles de type Frobenius}
	Comme précédemment, on dérive et on remplace ! 
	\begin{equation}
	\begin{array}{ll}
	S(x) &= \sum_{k=0}^\infty a_kx^{r+k}\\
	xS'(x) &= \sum_{k=0}^\infty (r+k)a_kx^{r+k}\\
	x^2S''(x) &= \sum_{k=0}	^\infty(r+k)(r+k-1)a_kx^{r+k}
	\end{array}
	\end{equation}		
	On remarque que grâce aux $x, x^2$ il ne faut plus rien modifier : on effectue 
	le produit de Cauchy et on déduit la relation de récurrence $(R_k)$ :
	\begin{equation}
	(r+k)(r+k-1)a_k + \sum_{l=0}^k(a_l(r+l)p_{k-l}+a_lq_{k-l})=0
	\end{equation}
	où encore 
	\begin{equation}
	((r+k)(r+k-1)+(r+k)p_0+q_0)a_k = -\sum_{l=0}^{k-1}a_l((r+l)p_{k-l}+q_{k-l})=0
	\end{equation}
	$R_0$ est ainsi l'équation indicielle et $r$ est forcément solution (sinon 
	il n'existera pas de solution de Frobenius). En déballant la somme pour 
	une valeur de $k$, je peux obtenir les premiers $a_k$.\\
	
	Définissons le premier membre de l'équation, $\mathcal{P}_0(r) := r(r-1)+p_0r
	+q_0$, comme étant le polynôme indiciel de degré 2 et $\mathcal{P}_n(r) := p_n
	r+q_n$ le polynôme de degré $\leq 1$. On peut réécrire $R_k$:
	\begin{equation}
	\mathcal{P}_0(r+k)a_k = -\sum_{m=0}\mathcal{P}_{k-l}(r+l)a_l
	\end{equation}
	où le premier terme est non nul, si $r+k\neq r_1,r_2$.\\
	Rappelons que $a_0=1$ et que pour $R_0 \Longrightarrow\mathcal{P}_0(r)=0 
	\Leftrightarrow r = r_1,r_2$.\\
	
	\textbf{Si}\footnote{??} $r_1\in\mathbb{R}$ et $r_1-r_2 \neq \mathbb{Z}$ et 
	$a_0=1$ alors les $(R_k)$ déterminent univoquement les $a_k(r_1)$ et les $a_k
	(r_2)$. Les deux séries solutions sont alors
	\begin{equation}
	\begin{array}{l}
	x^{r_1}(1+\sum_{k=1}^\infty a_k(r_1)x^k)\qquad(=y_1(x))\\
	x^{r_2}(1+\sum_{k=1}^\infty a_k(r_2)x^k)\qquad(=y_2(x))	
	\end{array}
	\end{equation}
	Ces deux solutions ayant un comportement différent pour $x\rightarrow0$, elles 
	sont forcément L.I.\\
	
	\textbf{Si} $r_1\notin \mathbb{R} : r_1 = \alpha +i\beta, r_2 = r_1^*$ comme les
	coefficients sont réels. On trouve alors en remplaçant $r_1$ dans la série solution 
	trouvée précédemment pour trouver la série solutions à valeurs dans $\mathbb{C}$. 
	Pour le reste ça ne change pas et l'on pourra tout déterminer à partir de $a_0 (=1)$.
	
	\textbf{Si} $r_1+r_2+\tilde{k}$ avec $\tilde{k}\in\mathbb{N}_0$, alors $R_{\tilde{k}}$ 
	pour $r=r_2$ est : $0.a_{\tilde{k}} = \underbrace{-\sum_{l=0}^{\tilde{k}-1}\mathcal{P}_{
	\tilde{k}-l}(r_2+l)a_l}_{?=0}$.\\
	Si c'est nul, alors $a_{\tilde{k}}$ est indéterminé et l'on peut choisir $a_{\tilde{k}}
	=0$ pour déterminer tout le reste.  La série-solution vaut alors
	\begin{equation}
	x^{r_2}\left(1+\sum_{k\neq\tilde{k},0}^\infty a_kx^k\right)
	\end{equation}
	Si c'est non nul, il n'existe pas de série de Frobenius d'exposant $r_2$ et on peut 
	trouver une solution du type (admis)
	\begin{equation}
	b_1y_1(x)\ln(x) + x^{r_1}\left(1+\sum_{k=1}^\infty c_kx^k\right)
	\end{equation}
	Pour le cas $r_1=r_2$, voir slide 39.
	
	\setcounter{subsection}{9}
	\subsection{Théorème de Frobenius (admis)}
	\theor{\ 
	\begin{itemize}
	\item[$\bullet$] Toutes les séries solutions formelles trouvées (section 7) convergent 
	dans $]0,\rho[$.
	\item[$\bullet$] Leurs sommes sont des solutions de l'EDL 
	\begin{equation}
	x^2y''+x\left(\sum_{k=0}^\infty p_kx^k\right)y'+\left(\sum_{k=0}^\infty q_kx^k\right)
	y=0
	\end{equation}
	\end{itemize}
	où $\rho := \min\{\rho_p,\rho_q\}$.}\ 


\section{L'EDL hypergéométrique de Gauss}
	\subsection{Définition}
	