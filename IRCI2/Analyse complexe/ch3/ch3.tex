\chapter{Intégrales 1}

\section{Préliminaires}

\subsection{Dérivée d'une fonction complexe d'une variable réelle}
Soit $w(t) = u(t) +iv(t)$ ou $u$ et $v$ sont réelles. On trouve naturellement 
\begin{equation}
	w'(t) = u'(t) + iv'(t)
\end{equation}
pour autant que les dérivées $u'$ et $v'$ existent en $t$.\\
    
\prop{Soit $z_0 \in \mathbb{C}$ :
	\begin{itemize}
		\item $[z_0w(t)]' = z_0w'(t)$
		\item Règles de dérivation + et * se généralisent ici.
	\end{itemize}}
    
\subsection{Intégrale définie d'une fonction complexe d'une variable réelle}
En gardant les mêmes pré-requis qu'à la précédente section, l'intégrale définie de $w(t)$ sur
l'intervalle $a \leq t \leq b$ vaut :
\begin{equation}
	\int_a^b w(t) dt = \int_a^b u(t) dt + i\int_a^b v(t)dt
\end{equation}
    
Si $U$ et $V$ sont des primitives de $u$ et $v$, l'intégrale de $w(t)$ n'est que l'intégrale 
d'une fonction réelle qui admet une primitive que l'on peut évaluer additionné à une intégrale
d'une fonction réelle qui admet également une primitive, multipliée par $i$ :
\begin{equation}
	\int_a^b w(t)dt = [U(b) + iV(b)] - [U(a) + iV(a)]
\end{equation}
 
Attention, l'intégrale d'une fonction complexe ne donne pas l'aire sous la courbe de cette fonction.   
    
\section{Intégrale d'une fonction complexe d'une valeur complexe}
\subsection{Notions de chemin élémentaire et de chemin admissible}
\subsubsection{Chemin élémentaire}
Un \textit{chemin élémentaire} (ou \textit{arc différentiable}) est défini par les équations
paramétriques :
\begin{equation}
	\left\{\begin{array}{ll}
	x &= \varphi(t)  \\
	y &= \psi(t) 
	\end{array}\right.\ \ \ \ \ \ \ t_0\leq t_1
\end{equation}
où $\varphi(t)$ et $\psi(t)$ sont de classe $C^1$ dans $t_0 \leq t\leq t_1$.\\
On utilise souvent comme expression équivalente : 
\begin{equation}
	z = \Phi(t) = \varphi(t) + i\psi(t)\ \ \ \ \ t_0 \leq t\leq t_1
\end{equation}
    
\exemple{Considérons le point $A : z = 2+i$. Le chemin $O\rightarrow A$ peut avoir comme  
	équation paramétrique :
	\begin{equation}
		\left\{\begin{array}{lll}
		x &= 2t &= \varphi(t)\\
		y &= t &= \psi(t)
		\end{array}\right.\ \ \ \ \ \ \ 0\leq t \leq 1\
	\end{equation}
	L'autre manière de le définir correspond à :
	\begin{equation}
		z = \Phi(t) := (2+i)t
	\end{equation}}
    
\subsubsection{Différents types de chemins}
Trois chemins valent le détour (pfpfp) :
\begin{enumerate}
	\item \textit{Chemin admissible} : chemin décomposable en un nombre fini de chemin
	      élémentaire. Par exemple, un segment de droite suivi d'un demi-cercle puis à nouveau d'
	      un segment de droite est un chemin dit admissible.
	\item \textit{Chemin simple} : $z(t_a) \neq z(t_b)$ si $t_a \neq t_b$. Cela signifie qu'
	      il ne peut y avoir de "boucle" dans un chemin simple si ce n'est (dans le cas d'un chemin
	      fermé) le point ou se referme celui-ci.
	\item \textit{Ensemble simplement connexe $R$} : ce sera le cas si tout chemin simple fermé
	      n'englobe que les points de $R$. Deux cercles dans un plan n'est pas connexe. Si j'ai un
	      cercle dans un cercle et que "l'entre-deux cercle" est mon ensemble, il sera bien connexe
	      mais pas simplement car mon "petit cercle central" englobe des points qui ne font pas partie
	      de mon ensemble.
\end{enumerate}
    
\subsection{Intégrale le long d'un chemin élémentaire}
\subsubsection{Définition (!) :}
Soit $\left\{\begin{array}{l}
- f(z)\ continue\\
- bornes\ z_0\ et\ z_1\\
- C : \textit{chemin élémentaire de $z_0$ à $z_1$}\\
- z = \Phi(t)\ \ \ avec\ t_0 \leq t \leq t_1
\end{array}\right.$ 
\begin{equation}
	\int_\mathcal{C} f(z) dz \equiv \int_{t_0}^{t_1} f(\Phi(t))\Phi'(t) dt
\end{equation}
NB : pour le peu que le chemin soit identique, si l'on change les équations paramétriques on 
obtiendra le même résultat (les équations paramétriques ne sont pas uniques).\\
    
\prop{\begin{enumerate}
	\item On peut sortir toute constante (même $\mathbb{C}$) de l'intégrale.
	\item L'intégrale d'une somme est la somme des intégrales.
	\item Inverser le sens du chemin revient à inverser le signe de l'intégrale.
	\end{enumerate}}\ \\
    
\retenir{Si le chemin admissible $\mathcal{C}$ est formé des chemins élémentaires $C_1$ et $C_2$ :
	\begin{equation}
		\int_\mathcal{C} f(z) dz = \int_{C_1} f(z) dz + \int_{C_2} f(z) dz
	\end{equation}}\ \\
    
\exemple{Considérons la même fonction $z = 2+i$ en guise de chemin avec la paramétrisation
	définie ci-dessus :
	$z = (2+i)t \ (0\leq t \leq 1) $. Calculons $\int_\mathcal{C}z^2dz$ :
	\begin{equation}
		\int_\mathcal{C}z^2dz = \int_0^1 (2+i)^2t^2 \ (2+i)dt = (2+i)^3\left[\frac{t^3}{3}\right]_0^1
		= (8+12i-6-i)\frac{1}{3} = \frac{2}{3} + \frac{11}{3}i
	\end{equation}
	    
	Décomposons le chemin $O\rightarrow A$ en deux parties : $O \rightarrow B$ et $B\rightarrow A$ :
	\begin{equation}
		OB : \left\{\begin{array}{l}
		x = 2i\\
		y = 0
		\end{array}\right. \rightarrow z= \Phi = 2t,\ \ \ \ \ \ \ \ BA :  \left\{\begin{array}{l}
		x = 2\\
		y = t
		\end{array}\right. \rightarrow z = \Phi = 2+it,\ \ \ \ \ \ \ (0\leq t \leq 1)
	\end{equation}
	Appliquons le cadre \textbf{A retenir} ci-dessus en calculant chacune des deux intégrale. Nous
	avons d'une part
	\begin{equation}
		\int_{OB} z^2 dz = \int_0^1 4t^2.(2t)dt = \frac{8}{3}
	\end{equation}
	D'autre part 
	\begin{equation}
		\int_{BA} z^2 dz = \int_0^1 (2+it)^2.i dt = \frac{11}{3}i-2
	\end{equation}
	En sommant ces deux contributions, on retrouve bien le même résultat que précédemment :
	\begin{equation}
		\int_{OB} z^2 dz +  \int_{BA} z^2 dz = \frac{2}{3} + \frac{11}{3}i
	\end{equation}}\ \\
Au vu du précédent exemple, on pourrait penser  que le chemin n'a pas d'importance. Ce n'est
cependant pas le cas. Illustrons avec un nouvel exemple :\\
\exemple{Tentons de calculer, avec les deux chemins représentés ci-contre, l'intégrale suivante:
	\begin{equation}
		\int_\mathcal{C} \frac{dz}{z}
	\end{equation}
	\begin{enumerate}
		\item \textit{Chemin ABC}\\
		      Une paramétrisation pour le chemin $ABC$ est $z = Re^{i\theta}$ où $R$ est constant et $0 \leq
		      \theta \leq \pi$. En appliquant la définition, on trouve :
		      \begin{equation}
		      	\int_a^\pi \frac{1}{R}e^{-i\theta}Rie^{i\theta} d\theta = i\pi
		      \end{equation}
		          
		\item \textit{Chemin ADB}\\
		      Une paramétrisation pour le chemin $ABC$ est $z = Re^{-i\theta}$\footnote{Les bornes de l'intégrale
		      	doivent être croissantes, d'ou l'inversion du signe de l'argument pour garder $0\leq \theta\leq \pi$}
		      où $R$ est constant et $0 \leq\theta \leq \pi$. En appliquant la définition, on trouve :
		      \begin{equation}
		      	\int_a^\pi \frac{1}{R}e^{i\theta}R(-i)e^{-i\theta} d\theta = -i\pi
		      \end{equation}
	\end{enumerate}} 
    
    
\subsection{Borne supérieure pour le module d'une intégrale le long d'un chemin}
Afin de démontrer un précieux théorèmes majorant notre intégrale le long d'un chemin, il faut 
avant tout énoncer un lemme :\\
    
\lemme{Si $w(t)$ est une fonction continue par morceaux à valeurs dans $\mathbb{C}$ définie
	sur $a \leq t \leq b$, alors
	\begin{equation}
		\left|\int_a^b w(t) dt\right| \leq \int_a^b |w(t)|dt
	\end{equation}}
\begin{proof}\ \\
	Considérons une intégrale $J$ et sa solution :
	\begin{equation}
		J = \int_a^b w(t).dt = r_0e^{i\theta_0}
	\end{equation}
	où $r_0, \theta_0$ sont des constantes réelles. Multiplions de part et d'autres par 
	$e^{-i\theta_0}$ : étant constant, il peut rentrer dans l'intégrale. Comme $r_0 \in
	\mathbb{R}$, l'intégrande doit forcément être réelle:
	\begin{equation}
		r_0 = \int_a^b w(t)e^{-i\theta_0}\ dt = \int_a^b \text{Re}[w(t)e^{-i\theta_0}]dt
	\end{equation}
	Comme la partie réelle d'un nombre complexe est toujours inférieure a son module, 
	on peut écrire :
	\begin{equation}
		\text{Re}[w(t)e^{-i\theta_0}] \leq |w(t)|
	\end{equation}
	On a donc\footnote{Passage obscur, à éclaircir.} :
	\begin{equation}
		\left|\int_a^b w(t) dt\right| \leq \int_a^b |w(t)|dt
	\end{equation}
\end{proof}
\theor{\label{theo:ML}\textsc{"ML"}\\
	Soit $f(z)$ continue en tout point du chemin admissible $\mathcal{C}$ (de $z_0 =
	\phi(t_0)$ à $z_1 = \phi(t_1)$) de longueur $L$.\\
	Si $|f(z)| \leq M, \forall z \in \mathbb{C}, M \in \mathbb{R}^+$, alors 
	\begin{equation}
		\left|\int_\mathcal{C}f(z)dz\right| \leq ML
	\end{equation}}
\begin{proof}
	Par définition de l'intégrale et application du lemme:
	\begin{equation}
		\left|\int_\mathcal{C} f(z) dz\right| = \left|\int_{t_0}^{t_1} f(\phi(t))\phi'(t)
		dt\right| \leq \int_{t_0}^{t_1} |f(\phi(t))||\phi'(t)| dt
	\end{equation}
	On considérant comme description du chemin élémentaire $\mathcal{C}$ 
	\begin{equation}
		z = x+iy = \phi(t) = \varphi(t) + i\psi(t)
	\end{equation}
	On peut calculer la dérivée de $\phi$ ainsi que son module : 
	\begin{equation}
		\begin{array}{lll}
			\phi'(t)   & = \frac{dx}{dt} + i\frac{dy}{dt} & \rightarrow \phi'(t)dt= dx + idy \\
			|\phi'(t)| & = \sqrt{dx^2 + dy^2}             & = ds                             
		\end{array}
	\end{equation}
	Après substitution dans l'intégrale :
	\begin{equation}
		\left|\int_\mathcal{C} f(z) dz\right| \leq M\int_0^L ds = ML
	\end{equation}
\end{proof}
    
    
    
    
    
    
\section{Primitive}
\subsection{Définition et propriété}
\textit{La primitive d'une fonction $f$ continue dans $D \cap \mathbb{C}$ :}
\begin{equation}
	F(z)\ t.q.\ F'(z) = f(z)\ \ \forall z \in D
\end{equation}
$F$ est primitive de $f$ pour autant que la dérivée de $F$ donne $f$. Bien sûr,
il existe une infinité de primitive comme dans le cas réel, qui ne sont distinctes
que d'une constante :\\
    
\prop{Soient $F(z)$ et $G(z)$ deux primitives de $f(z)$ dans $D \cap \mathbb{C}$:
	\begin{equation}
		F(z) - G(z) = c\ \ \ c\in\mathbb{C}, z\in D
	\end{equation}}
\begin{proof}\ \\
	Notons $h(z) = F(z) - G(z)$, toute deux primitive d'une même fonction. Sa dérivée
	vaut :
	\begin{equation}
		h'(z) =  F'(z) - G'(z) = f(z) - f(z) = 0
	\end{equation}
	Vu que $h$ est une fonction complexe, je peux l'écrire sous la forme $h(z) = 
	u(x,y) + iv(x,y)$. En exprimant alors les dérivées partielles ($\forall z =
	x + iy \in D$ ):
	\begin{equation}
		h' = \frac{\partial u}{\partial x} + i\frac{\partial v}{\partial x} = 0 \rightarrow
		\left\{\begin{array}{ll}
		\frac{\partial u}{\partial x} &= 0  \\
		\frac{\partial v}{\partial x} &= 0 
		\end{array}\right.
	\end{equation}
	Par les équations de Cauchy-Riemann $\frac{\partial u}{\partial y} =0$ et 
	$\frac{\partial v}{\partial y} = 0$. Ceci implique qu'à la fois $u$ et $v$ sont
	des constantes et donc $h(z)$ est constante $\Rightarrow h(z) = c \in C \; \forall
	z \in D$.
\end{proof}
    
\subsection{Intégrale indépendante du chemin}
Lors du dernier exemple, il apparaissait clairement que l'intégrale pouvait être
(ou ne pas être) dépendante du chemin. Ce théorème permet de savoir s'il faut ou
non tenir compte du chemin imposé :\\
    
\theor{Considérons une fonction $f(z)$ continu dans un domaine $D$. Si l'une des
	conditions suviantes est vérifiée, alors les autres le sont aussi.
	\begin{enumerate}
		\item $f(z)$ admet une primitive $F(z)$ dans $D$.
		\item Les intégrales de $f(z)$ le long de chemins admissibles de $z_1$ à $z_2$ 
		      contenus entièrement dans $D$ ont toutes la même valeur à savoir :
		      \begin{equation}
		      	\int_{z_1}^{z_2} f(z) dz = F(z)|^{z_2}_{z_1} = F(z_2)-F(z_1)
		      \end{equation}
		      où $F(z)$ est primitive du point (1).
		          
		\item Les intégrales de $f(z)$ sur des chemin admissibles fermés contenus entièrement
		      dans $D$ sont toutes égales à zéro.
	\end{enumerate}}
    
\newpage
\begin{proof}\ \\
	\textbf{(1) $\rightarrow$ (2)}\\
	\ \ \ \textit{Cas 1} : considérons un chemin élémentaire $\mathcal{C}$ de $z_1$ à
	$z_2$ dans $D$ tel que $z = \phi(t), a\leq t\leq b$ avec $\phi(a) = z_1$ et $\phi(
	2) = z_2$. Notons que : 
	\begin{equation}
		\frac{d}{dt}F(\phi(t)) = F'(\phi(t))\phi'(t) = f(\phi(t))\phi'(t)\ \ a \leq t\leq b
	\end{equation}
	Par défintion de l'intégrale de $f(z)$ et en utilisant \textbf{(3.3)} :
	\begin{equation}
		\int_\mathcal{C} f(z) dz = \int_a^b f(\phi(t))\phi'(t) dt = F(\phi(t))|_b^a = F(z_2)
		- F(z_1)
	\end{equation}
	Ce qui est indépendant de $\mathcal{C}$\\
	    
	\ \ \ \textit{Cas 2} : considérons un chemin admissible dans $D$ formé d'un nombre
	fini de chemins élémentaires $C_k$ de $z_k$ à $z_{k+1}$ pour $k=1,2\dots,n.$ Je
	'découpe' mon intégrale le long de $\mathcal{C}$ :
	\begin{equation}
		\int_\mathcal{C} f(z) dz = \sum_{k=1}^n \int_{C_k} f(z)dz = \sum_{k=1} \int_{z_k}^{z_{k+1}}
		f(z)dz = F(z_{n+1}) - F(z_1)
	\end{equation}
	Ce qui est indépendant de $\mathcal{C}$\\
	    
	\textbf{(2) $\rightarrow$ (3)}\\
	Soit $z_1,z_2$, deux points distincts appartenant à un chemin admissible fermé 
	$\mathcal{C}$ dans $D$. Comme mon intégrale ne dépend pas du chemin, je sépare
	$\mathcal{C}$ en $C_1,C_2$ : soit $C_1,C_2$, deux chemins de $z_1$ à $z_2$ tels
	que $\mathcal{C} = C_1-C_2$. En utilisant (2) :
	\begin{equation}
		\int_{C_1} f(z) dz = \int_{C_2} f(z) dz
	\end{equation}
	Ce qui implique : 
	\begin{equation}
		\int_{C_1} f(z) dz + \int_{-C_2} f(z) dz = \int_\mathcal{C}f(z) dz = 0
	\end{equation}\ \\
	    
	\textbf{(3) $\rightarrow$ (1)}\\
	Partons de (3) :
	\begin{equation}
		\int_\mathcal{C} f(z) dz = 0
	\end{equation}
	Quelque soit $\mathcal{C}$, je peux trouver une primitive associé à $f$ : 
	définissions $F(z) = \int_{z_0}^{z} f(s) ds$ avec $z_0,z \in D$ ($z_0$ est 
	arbtitraire) et montrons que  $F'(z) = f(z)\ \forall z \in D$. L'évaluation de la
	dérivée fait apparaître la différence suivante :
	\begin{equation}
		\begin{array}{ll}
			F(z+\Delta z) - F(z) & = \int_{z_0}^{z+\Delta z} f(s) ds - \int_{z_0}^z f(s)ds \\
			                     & = \int_{z}^{z+\Delta z} f(s) ds                         
		\end{array}
	\end{equation}
	Comme l'intégrale (3) est nulle, cela implique bien que le chemin parcouru n'a pas
	d'importance justifiant cette dernière ligne.\\
	Considérons le chemin d'intégration $z,z+\Delta z$ (segment de droite) :
	\begin{equation}
		\int_z^{z+\Delta z} ds = \Delta z
	\end{equation}
	Dans notre cas (comme $f(z)$ ne dépend pas de $s$), il en résulte :
	\begin{equation}
		f(z) = \frac{1}{\Delta z} \int_z^{z+\Delta z} f(z) ds
	\end{equation}
	Soustrayons de part et d'autre de l'égalité $f(z)$ pour avoir :
	\begin{equation}
		\dfrac{F(z+\Delta z) - F(z)}{\Delta z} - f(z) = \dfrac{1}{\Delta z}\int_z^{z+
			\Delta z} (f(s)-f(z)) ds
	\end{equation}
	Cherchons la borne supérieure pour :
	\begin{equation}
		\left|\dfrac{F(z+\Delta z) - F(z)}{\Delta z} - f(z)\right| = \dfrac{1}{|\Delta z|}
		|\int_z^{z+ \Delta z} (f(s)-f(z)) ds|
	\end{equation}
	Comme $f$ est continue en $z$, $\forall \epsilon > 0, \exists \delta$ tel que :
	\begin{equation}
		|f(s)-f(z)|<\epsilon\ \text{si }\ |s-z| < \delta
	\end{equation}
	Si $\Delta z$ tel que $|s-z|\leq |\Delta z | < \delta$, alors $|f(s)-f(z)|<\epsilon$
	et l'on peut appliquer le théorème ML :
	\begin{equation}
		\left|\dfrac{F(z+\Delta z) - F(z)}{\Delta z} - f(z)\right| < \dfrac{1}{|\Delta z|}
		\epsilon|\Delta z|
	\end{equation}
	Par définition de la limite :
	\begin{equation}
		\lim\limits_{\Delta z \rightarrow 0} \dfrac{F(z+\Delta z) - F(z)}{\Delta z} = f(z)
	\end{equation}
	Soit F'(z) = f(z), ce qui finit cette longue démonstration.
\end{proof}
\exemple{\begin{enumerate}
	\item Considérons un domaine $D \equiv |z| > 0$. On cherche à calculer :
	\begin{equation}
		\int_{z_1}^{z_2} \dfrac{dz}{z^2}
	\end{equation}
	Notre fonction étant $f(z) = 1/z^2$, sa primitive est $F(z) = -1/z$. Son intégrale 
	vaut alors :
	\begin{equation}
		\int_{z_1}^{z_2} \dfrac{dz}{z^2} = -\dfrac{1}{z_2} + \dfrac{1}{z_1}
	\end{equation}
	\item \begin{equation}
	\oint \dfrac{dz}{z^2} = 0
	\end{equation}
	Cette primitive étant partout définie, l'intégrale "donne $F-F$", ce qui donne bien
	zéro.
	\item \begin{equation}
	\oint_C \dfrac{dz}{z} =\ ?
	\end{equation}
	Ici, que vaut $F(z)$ ? $Log(z)$ ? La primitive n'étant pas défini sur l'axe des réels
	négatifs, on ne peut affirmer - avec ce théorème \textbf{ci} - que cette intégrale est
	nulle.
	\end{enumerate}}
    

\section{Théorème de Cauchy Goursat}
\theor{\textsc{Cauchy-Goursat}\\
	Soit $\mathcal{C}$, un chemin admissible fermé simple, $f(z)$ analytique en tout point
	de\footnote{Si le point est sur $\mathcal{C}$, cela ne va pas !} $\mathcal{C}\cup D$ 
	(où $D$ est un domaine intérieur à $\mathcal{C}$) :
	\begin{equation}
		\oint_\mathcal{C} f(z) dz = 0
	\end{equation}}\ \\

Plus "francisé" ce théorème dit que si $f$ est analytique en tout point à l'intérieur de
$D$, domaine défini par le chemin $\mathcal{C}$, l'intégrale le long du chemin fermé
est nulle.\\
Avant de démontrer ce théorème, rappellons l'identité de Green :
\begin{equation}
	\oint_\mathcal{C} P(x,y)dx + Q(x,y)dy = \iint_D \left(\dfrac{\partial Q}{\partial x} -
	\dfrac{\partial P}{\partial y}\right)dxdy
\end{equation}
\textbf{Attention !} On travaille ici avec l'hypothèse supplémentaire que les dérivées
partielles sont continues sur $\mathcal{C}\cup D$.

\begin{proof}
	Considérons un chemin $\mathcal{C}$ décrit par $z = x+iy = \varphi(t) + i\psi(t)$ pour
	$t_0 \leq t \leq t_1$ et $z(t_0) = z(t_1)$. Évaluons l'intégrale sur $\mathcal{C}$ de
	$f(z)$ et exprimons cette dernière en fonction de sa partie réelle et imaginaire. La 
	ligne ci-dessous n'est rien d'autre que la définition d'une intégrale en prenant compte
	de la paramétrisation.
	\begin{equation}
		\oint_\mathcal{C}f(z) dz = \int_{t_0}^{t_1} (u(x,y) + iv(x,y))(x'(t) + iy'(t)) dt
	\end{equation}
	On reconnaît la définition d'une intégrale curviligne :
	\begin{equation}
		\oint_\mathcal{C}f(z) dz = \oint (u+iv)(dx+idy) = \oint(udx - vdy) + i\oint (vdx + udy)
	\end{equation}
	En faisant l'hypothèse que les dérivées partielles sont continues, je peux appliquer l'
	identité de Green pour avoir :
	\begin{equation}
		\oint_\mathcal{C}f(z) dz = -\iint_D \left(\dfrac{\partial v}{\partial x} + 
		\dfrac{\partial u}{\partial y}\right) dxdy + i\iint_D\left(
		\dfrac{\partial u}{\partial x} - \dfrac{\partial v}{\partial y}\right)dxdy
	\end{equation}
	Par les équations de Cauchy-Riemann, le deuxième terme est nul.
\end{proof}

\exemple{Avec ce théorème, on peut directement dire que 
	\begin{equation}
		\oint \exp(5z)\ dz = 0
	\end{equation}
	Quelle est la différence avec le théorème vu en \textbf{3.2.2} ? C'est que ici, on n'a
	pas besoin d'une primitive. Ainsi, si l'on prend $\mathcal{C}$ le cercle de rayon 1 de
	centre (2,0) :
	\begin{equation}
		\oint_\mathcal{C} \frac{dz}{z} = 0
	\end{equation}
	Ce que nous n'avions pas réussi à obtenir avec le précédent théorème. Par contre, pour 
	l'intégrale :
	\begin{equation}
		\oint_\mathcal{C} \dfrac{dz}{z^2}
	\end{equation}
	je ne peux rien dire avec Cauchy-Goursat, la fonction étant non analytique en 0. Mais 
	cette fois, avec \textbf{3.2.2}, je peux dire qu'elle vaut 0 ! }

\subsection{Conséquence du théorème de Cauchy-Goursat}
\subsubsection{Première proposition}
Si $f(z)$ est analytique sur un chemin fermé $z_0z_2z_1z_3z_0$ et en tout point
intérieur, alors
\begin{equation}
	\int_{z_0z_2z_1} f(z) dz = \int_{z_0z_3z_1} f(z) dz
\end{equation}
$\Rightarrow$ la valeur de $\int_{z_0z_1}$ est indépendante du chemin. Notons
que la réciproque est fausse ; si la fonction n'est pas analytique, la valeur
de l'intégrale peut dépendre du chemin ou non !\\
\exemple{Voir slide 27.}
        
        
\subsubsection{Deuxième proposition}
Soit $C, C_0$ deux chemins admissibles, fermés, simples et orientés dan le sens
positif ($C_0$ intérieur à $C$). Si $f(z)$ est analytique dans $C$ excepté les
points intérieurs à $C_0$ alors :
\begin{equation}
	\oint_C f(z) dz = \oint_{C_0} f(z) dz
\end{equation}
        
Cela signifie que les points "à problème" de la fonction sur le domaine compris par les chemins $C$ et $C_0$ sont les mêmes, ceux-ci étant uniquement compris à l'intérieur de $C_0$.
        
\begin{proof}
	Slide 29.
\end{proof}
\exemple{On avait vu lors d'un précédent exemple (celui prouvant que le chemin 
	d'intégration avait parfois de l'importance que pour $\int_C \frac{dz}{z}$ :
	\begin{equation}
		\begin{array}{l}
			\int_{C_1} =\  i \pi \\
			\int_{C_2} = -i\pi   
		\end{array}
	\end{equation}
	On trouve alors :
	\begin{equation}
		\int_{C_1} - \int_{C_2} = 2\pi i
	\end{equation}
	$\Rightarrow$ résultat valable pour tout chemin admissible simple fermé entourant
zéro.}
        
        
\subsubsection{Troisième proposition}
Soient :
\begin{itemize}
	\item $\mathcal{C}$ un chemin admissible, ferme, simple et orienté dans le sens
	      positif.
	\item $C_k (k=1,\dots,n)$ des chemins admissibles fermés, simples, orientés dans
	      le sens positif, intérieurs à $\mathcal{C}$ et dont l'intérieur n'ont pas de 
	      points communs.
\end{itemize}
Si $f$ est analytique dans $\mathcal{C}$, sauf en des points intérieurs à $C_k$,
alors 
\begin{equation}
	\oint_\mathcal{C} f(z) dz = \sum_{k=1}^n \oint_{C_k} f(z) dz
\end{equation}
        