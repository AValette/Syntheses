
\section*{TP 2 : Postulats et modèles unidimentionnels}
\subsection*{Equation de Schrödinger}
\begin{itemize}
	\item Equation générale
	      \begin{equation}
	      	i\hbar \frac{d \psi}{dt} = H
	      \end{equation}
	      	
	\item Forme stationnaire
	      \begin{equation}
	      	H\psi = E\psi
	      \end{equation}
	      
	\item Hamiltonien $H = T + V$ (= opérateur) 
	      
	\item Opérateur énergie cinétique unidimentionnelle
	      \begin{equation}
	      	T = \frac{\vec{p}^2}{2m} = \frac{- \hbar ^2}{2m}\Delta = \frac{- \hbar ^2}{2m}\frac{d^2}{dx^2}
	      \end{equation}
	      
	\item Equation homogène : $\phi = c\psi$ est une fonction d'onde $\forall c \in \mathbb{C}$ (après normalisation)
	      
	\item Potentiel confinant 
	      \begin{equation}
	      	\lim _{x \rightarrow \pm \infty} = \infty
	      \end{equation}
	      Dans ce cas là, les énergies sont \textbf{discrètes} et les $\psi \in L^2(\psi )$ sont \textbf{liées}
	      \begin{equation}
	      	E_n = \frac{n^2\pi ^2 \hbar ^2}{2ma^2} \qquad \psi _n (x) = \sqrt{\frac{2}{a}} \sin (\frac{n\pi x}{a} )
	      \end{equation}
	      avec $a$ qui est la coordonnée du potentiel.
\end{itemize}

\subsection*{Notation de Dirac}
\begin{itemize}
	\item Définition 
	      \begin{equation}
	      	\braket{\psi | \psi} =  \int \psi ^* \psi \, d\vec{r} = \int |\psi |^2 \, d\vec{r}= 1 \quad \mbox{(si normée)}
	      \end{equation}
	      	
	\item Propriétés 
	      \begin{itemize}
	      	\item $\braket{\psi _i | \psi _j} = \delta _{ij}$
	      	\item Si $\ket{\phi} = c\ket{\psi}$ alors $\bra{\phi} = c^* \bra{\psi}$
	      	\item $P(E = E_i) = | \braket{\psi _i | \psi} |^2$
	      \end{itemize}
	      
	\item Valeur moyenne d'une observable A
	      \begin{equation}
	      	\braket{A} = \braket{\psi | A | \psi} = \int \psi ^* A \psi \, d\vec{r}
	      \end{equation}
	      Par exemple : $A = p_x = -i \hbar \frac{d}{dx} \quad \Rightarrow \quad  \braket{p_x} = \braket{\psi | p_x | \psi} = -i\hbar \int \psi^* \frac{d}{dx} \psi \, dx$
\end{itemize}