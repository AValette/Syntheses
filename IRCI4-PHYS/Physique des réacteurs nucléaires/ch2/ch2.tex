\chapter{Introduction à la physique des réacteurs}
\section{Introduction aux concepts}
\subsection{Hypothèses}
Nous allons faire ici trois grande hypothèses de modélisation
\begin{enumerate}
\item Les phénomènes quantiques sont négligeables (la longueur d'onde des neutrons est très petite 
par rapport au réacteur) et ce à toute énergie.
\item Il y a plus de dix ordres de grandeur entre la densité de neutrons thermique et la densité 
d'atomes : les interactions neutron-neutron sont négligeables et les interactions avec les noyaux 
vont être proportionnelles au nombre de neutrons. L'équation de bilan de neutrons sera donc linéaire.
\item On va considérer la distribution initiale des neutrons dans le recteur car il n'y a que peu 
de fluctuations dans la moyenne du flux.
\end{enumerate}
En résumé les neutrons sont vus comme des particules classiques n'interagissant pas les unes avec 
les autres.

\subsection{Densité de neutrons, flux et courant}
Commençons par énumérer les variables : trois pour la position $\vec{r}$, trois pour la vitesse 
$\vec{v} =v\vec{\Omega}$ où $\vec\Omega$ est la direction (coordonnées sphériques) et une temporelle
$t$.

\subsubsection{Densité neutronique angulaire}
On \textit{défini} la \textbf{densité neutronique angulaire} comme étant le nombre de neutrons dans 
un volume d$\vec{r}$ autour de $\vec{r}$, dans un intervalle de vitesse $[v,v+dv]$ et dans un 
élément d'angle solide d$\vec\Omega$ autour de $\vec{\Omega}$
\begin{equation}
N(\bar r,v,\bar \Omega ,t)\,d\bar rdvd\bar \Omega\qquad\qquad [m^{-4}.s]
\end{equation}
\danger\ La définition n'est pas la même si on utilise $\vec{v}$ à la place de $v$ : les unités 
seraient alors en $m^{-6}.s^3$.

\subsubsection{Densité neutronique}
La \textbf{densité neutronique} est le nombre de neutron par unité de volume et de vitesse quel 
que soit la direction de propagation, soit l'intégration sur toutes les directions possibles ($4\pi$ 
est habituel en angle solide, cela correspond à la sphère unité)
\begin{equation}
n(\bar r,v,t)\,\, = \;\int\limits_{4\pi }    N(\bar r,v,\bar \Omega ,t)d\bar \Omega 
\end{equation}  

\subsubsection{Flux angulaire}
La multiplication de la densité par la vitesse donne un flux. En partant de la densité neutronique 
angulaire on définit le \textbf{flux angulaire}
\begin{equation}
\varphi (\bar r,v,\bar \Omega ,t)\, \equiv \;v.N(\bar r,v,\bar \Omega ,t)
\end{equation}
Il ne s'agit pas d'un flux par surface par seconde mais un flux par unité de volume, soit encore un
flux par unité de temps, de surface et de vitesse.

\subsubsection{Flux neutronique total et flux intégré}
Pour considérer le \textbf{flux total}, il suffit d'intégrer sur toutes les directions
\begin{equation}
\varphi (\bar r,v,t)\, = \int_{4\pi }^{}    \varphi (\bar r,v,\bar \Omega ,t)d\bar \Omega \, \equiv \;v.n(\bar r,v,t)
\end{equation}
Si on regarde en plus sur toute les vitesses (intégration), on retrouve bien un flux par unité de 
surface et de temps : il s'agit du \textbf{flux intégré}
\begin{equation}
\varphi (\bar r,t)\,\, = \;\int_o^\infty  \varphi  (\bar r,v,t)dv
\end{equation}

\subsubsection{Densité de courant (angulaire)}
On obtient la \textbf{densité de courant} (angulaire) en multipliant le flux angulaire par un élément d'angle solide 
$\bar\Omega$
\begin{equation}
\bar J(\bar r,v,\bar \Omega ,t) = \varphi (\bar r,v,\bar \Omega ,t)\,\bar \Omega  = \bar vN(\bar r,v,\bar \Omega ,t)
\end{equation}
Le \textbf{courant net} s'obtient en intégrant la densité de courant multiplié par la normale sur 
toute la surface, pour toute direction et toute vitesse
\begin{equation}
\int_o^\infty     \int\limits_\Gamma  {\int\limits_{4\pi }    } \bar J(\bar r,v,\bar \Omega ,t).\bar ndSd\bar \Omega dv = \int_o^\infty     \int\limits_\Gamma  {\int\limits_{4\pi }   } \varphi (\bar r,v,\bar \Omega ,t)\,\bar \Omega .\bar ndSd\bar \Omega dv
\end{equation}
Cette équation nous permet de voir que la distribution ne peut pas être 
isotropique\footnote{Invariance des propriétés du milieu en fonction de la direction de propagation.}.
En effet, si la distribution était isotropique, le flux angulaire devrait être indépendant de 
$\Omega$ et vaudrait $1/4\pi$
\begin{equation}
\varphi (\bar r,v,\bar \Omega )\, = \frac{1}{{4\pi }}\varphi (\bar r,v)
\end{equation}
Si le flux ne dépend pas d'$\bar\Omega$, il ne reste que le produit scalaire entre $\bar\Omega$ et 
la normale extérieure sur tous les $\bar\Omega$ dans l'intégrale
\begin{equation}
\int\limits_\Gamma  {\int\limits_{4\pi }    } \varphi (\bar r,v,\bar \Omega )\,\bar \Omega .\bar ndSd\bar \Omega  = 0\qquad \forall \Gamma 
\end{equation}
En effet, $\bar n$ cache un $\sin\theta$ que l'on intègre de 0 à $\pi$ ce qui donne zéro. Ce 
résultat signifie que si il y a isotropie, il n'y a pas de courant net et donc pas de neutrons qui 
vont traverser une surface. Cette situation n'est pas réaliste. Cependant, si il y a anisotropie elle 
sera généralement assez faible (au premier ordre).


\subsection{Taux de réaction}
Nous essayons ici de définir une quantité essentielle : le nombre d'interactions d'un type donné 
que l'on va avoir par unité de volume et de temps, le \textbf{taux de réaction}
\begin{equation}
R = \left(\dfrac{\text{Nb. noyaux}}{cm^3}\right)\times\left(\dfrac{\text{Nb. neutrons}}{cm^2.s}\right)
\times\left(\text{Section efficace d'un noyau ($cm^2$)}\right)
\end{equation}
Des neutrons vont vers des "cibles" relativement fine. Le taux de réaction va être la surface de 
la cible (section transverse, efficace) multipliée par le flux de neutrons (le nombre de neutrons 
par $cm^2$ et par seconde qui arrivent sur une "cible" donnée) et ceci multiplié par le nombre de 
"cible" par unité de volume que l'on trouve dans la "cible totale"
\begin{equation}
R = N.\varphi(\bar r,t).\sigma(\bar r) = \Sigma(\bar r)\varphi(\var r, t)
\end{equation}
En effet, nous avions défini une section macroscopique. Si $\sigma$ est une probabilité par 
unité de longueur et que l'on multiplie ceci par une vitesse on trouve la probabilité par unité 
de temps du neutron avec le milieu. Si on multiplie encore ceci par une densité, on trouve bien 
quelque chose par unité de temps et de volume.\\

En utilisant la \textit{fréquence d'interaction} $\nu\Sigma$ ($s^{-1}$) et la densité 
neutroniques $n(\bar r,t)$ ($m^{-3}$), on peut ré-écrire
\begin{equation}
R = n(\bar r,t).\nu\Sigma(\bar r) = \Sigma(\bar r).\varphi(\bar r,t)
\end{equation}

Dans un cas général, les sections efficaces dépendent de l'énergie (et donc de la vitesse)
\begin{equation}
{R_*} = \int_o^\infty     v{\Sigma _*}(\bar r,v)n(\bar r,v,t)dv = \int_o^\infty     {\Sigma _*}(\bar r,v)\varphi (\bar r,v,t)dv
\end{equation}
Nous avons ici fait quelque chose de (très) mal la vitesse qui apparait dans $\Sigma$ et $\varphi$
porte le même signe mais celles-ci sont différentes. Quand on considère la section efficace, la vitesse est la relative entre le noyau et le neutron alors que pour le flux il s'agit de la vitesse 
absolue du neutron. Cette dernière vitesse fait l'hypothèse implicite que le noyau est immobile.\\

\subsubsection{Section différentielle efficace}
Supposons que l'on ai un scattering. Nous voulons décrire la situation où un neutron arrivant avec 
une vitesse $v$ dans la direction $\bar\Omega$ ressort à la vitesse $v'$ selon la direction
$\bar\Omega'$. Autrement dit, nous cherchons la probabilité conditionnelle d'une telle situation.\\

En intégrant sur $v'$ et $\bar\Omega'$, on sélectionne bien tous les neutrons qui ont la "bonne" 
vitesse et direction
\begin{equation}
\frac{{{\Sigma _s}(\bar r,v,\bar \Omega  \to v',\bar \Omega ')dv'd\bar \Omega '}}{{{\Sigma _s}(\bar r,v)}} = \frac{{{\Sigma _s}(\bar r,v,v',\bar \Omega .\bar \Omega ')dv'd\bar \Omega '}}{{{\Sigma _s}(\bar r,v)}}
\end{equation}
Il est possible de généraliser cette écriture en une fonction qui dépend de $v,v'$ et du produit
scalaire entre les deux directions. Celui-ci va faire intervenir un $\cos\theta$ qui va tenir 
compte de la direction (pas de raison qu'un électron ne soit pas diffusé selon un angle particulier).\\

En intégrant sur toutes directions et vitesses sortantes on trouve le scattering total
\begin{equation}
\int_{4\pi }   \int_o^\infty    {\Sigma _s}(\bar r,v,\bar \Omega  \to v',\bar \Omega ')dv'd\bar \Omega ' = \Sigma {}_s(\bar r,v)
\end{equation}
Où l'on définit le \textbf{noyau de scattering} $K(\bar r,v,\bar \Omega  \to v',\bar \Omega ')$ tel 
que
\begin{equation}
{\Sigma _s}(\bar r,v,\bar \Omega  \to v',\bar \Omega ') = \Sigma {}_s(\bar r,v).K(\bar r,v,\bar \Omega  \to v',\bar \Omega ')
\end{equation}
Dans un cas isotropique, on trouverait
\begin{equation}
{\Sigma _s}(\bar r,v,\bar \Omega  \to v',\bar \Omega ') = \frac{1}{{4\pi }}\Sigma {}_s(\bar r,v \to v')
\end{equation}



\subsection{Fluence, puissance et burnup}
La \textbf{fluence} est le nombre de particules traversant une unité d'aire pendant un temps donné. 
Ses unités sont le nombre de neutrons par unité de surface
\begin{equation}
\tau (\bar r,t)\,\, = \;\int_o^t \varphi  (\bar r,t')dt'
\end{equation}
Elle caractérise le taux d'irradiation.\\

La \textbf{puissance} est liée au nombre de réactions de fission que l'on va pouvoir observer dans 
un réacteur et \textbf{pas} directement proportionnelle au flux. Ainsi, $\Sigma_f\varphi$ donne 
le nombre d'interaction de fission par unité de volume et de temps en un point $\bar r$. Si on intègre 
sur $\bar r$ on obtient le nombre d'interactions par unité de temps. Si l'on multiplie ensuite par une 
certaine constante, on retrouve bien l'énergie dégagée par unité de temps, soit la puissance
\begin{equation}
P(t)\,\, \propto \;\int_V^{} {{\Sigma _f}(\bar r)\varphi } (\bar r,t)d\bar r
\end{equation}

Le \textbf{burnup} est l'énergie extraite d'une tonne de noyau lourd dans du combustible frais. Il 
reste en effet chaque fois de l'énergie mais elle n'est pas exploitable législativement. On le 
trouve comme
\begin{center}
Fluence $\times$ section efficace de fission $\times$ énergie par fission
\end{center}


\section{Équation de transport}
\subsection{Bilan neutronique}
Intéressons nous à la variation du nombre de neutrons par unité de temps, dans un volume $V$, à 
vitesse $dv$ autour de $v$ et dans la direction d$\bar\Omega$ autour de $\bar \Omega$.
\begin{equation}
\frac{{\partial \,}}{{\partial t}}\int\limits_V    N(\bar r,v,\bar \Omega ,t)d\bar r = \int\limits_V    \frac{{\partial N(\bar r,v,\bar \Omega ,t)}}{{\partial t}}d\bar r = \int\limits_V    \frac{1}{v}\frac{{\partial \varphi (\bar r,v,\bar \Omega ,t)}}{{\partial t}}d\bar r
\end{equation} 
Cette variation peut être causes par trois choses distinctes (respectivement le premier, deuxième 
et troisième terme ci-dessous)
\begin{enumerate}
\item Les sources
\item Les pertes dues à toutes les interactions (le fameux $\Sigma_t\varphi$)
\item Les pertes à travers la frontière (neutrons quittant le réacteur)
\end{enumerate}
\begin{equation}
\int\limits_V    S(\bar r,v,\bar \Omega ,t)d\bar r - \int\limits_V    {\Sigma _t}(\bar r,v)\varphi (\bar r,v,\bar \Omega ,t)d\bar r  - \int\limits_\Gamma     \bar J(\bar r,v,\bar \Omega ,t).d\bar S
\end{equation}
\newpage

En appliquant le théorème de Gauss, on trouve\\

\cadre{
\begin{equation}
\frac{1}{v}\frac{{\partial \varphi (\bar r,v,\bar \Omega ,t)}}{{\partial t}} = S(\bar r,v,\bar \Omega ,t) - {\Sigma _t}\varphi (\bar r,v,\bar \Omega ,t) - div(\bar J(\bar r,v,\bar \Omega ,t))
\end{equation}
où le premier terme sont les neutrons produits dans $d\bar r$ autour de $\bar r$ à vitesse $dv$ 
autour de $v$ et à direction $d\bar\Omega$ autour de $\bar\Omega$ et les deux derniers termes les 
pertes dans d$\bar r$\dots.}\ \\

On retrouve bien la forme générale d'une équation de conservation
\begin{equation}
\frac{{\partial \,}}{{\partial t}}\left( {density} \right) + div(\overline {current} ) = Sources
\end{equation}


\subsection{Équation de Boltzmann (ou de transport)}
\subsubsection{Sans neutrons retardés}
Il s'agit de l'équation de conservation appliquée aux neutrons (sans neutrons retardés). Pour 
l'instant, le terme de source n'a pas encore été considéré. Celui-ci se compose de plusieurs 
parties dont la plus simple est un terme source extérieur 
\begin{equation}
Q(\bar r,v,\bar \Omega ,t)
\end{equation}
Il faut voir dans quelles conditions nous allons trouver des neutrons qui viennent au point $\bar r$ 
s'ajouter à une vitesse $v$ et une direction $\bar\Omega$.  Parmi ceux-ci, certains proviennent de 
fissions
\begin{equation}
\int\limits_{4\pi }    \int_o^\infty     \nu {\Sigma _f}(\bar r,v')\varphi (\bar r,v',\bar \Omega ',t)dv'd\bar \Omega '
\end{equation}
Le nombre de neutrons produits par fission au point $\bar r$ est donner par le taux de fission 
$\Sigma_f\varphi$. Ceci est au point $\bar r$ mais ils peuvent se produire à toute vitesse et toute 
direction, il faut donc intégrer après avoir multiplié par $\nu$, le nombre moyen de neutrons émis 
par fission.  Cette intégrale est ainsi le \textit{nombre total de neutrons dus aux fissions au 
point $\bar r$ par unité de volume et de temps}.\\

Il est possible de découpler la partie $v'\bar \Omega'$ due aux neutrons qui ont causés les fissions
et $v\bar \Omega$ ceux provoqués par la fission, il est ainsi nécessaire de multiplier par le 
spectre Maxwellien de la distribution des vitesses. La fission est isotrope car les neutrons n'ont pas "souvenir" de leur direction de propagation, on retrouve un facteur $1/4\pi$
\begin{equation}
\frac{1}{{4\pi }}\chi (v)
\end{equation}

Il reste à traiter le scattering : l'objectif est de voir le nombre total de  neutrons qui subissent 
un scattering au point $\dot r$ en $v'\bar \Omega'$
\begin{equation}
\int\limits_{4\pi }    \int_o^\infty     {\Sigma _s}(\bar r,v',\bar \Omega ' \to v,\bar \Omega )\varphi (\bar r,v',\bar \Omega ',t)dv'd\bar \Omega '
\end{equation}

En résumé, on trouve
\begin{equation}
S(\bar r,v,\bar \Omega ,t) = \frac{1}{{4\pi }}\chi (v)\int\limits_{4\pi }    \int_o^\infty     \nu {\Sigma _f}(\bar r,v')\varphi (\bar r,v',\bar \Omega ',t)dv'd\bar \Omega +
\int\limits_{4\pi }    \int_o^\infty     {\Sigma _s}(\bar r,v',\bar \Omega ' \to v,\bar \Omega )\varphi (\bar r,v',\bar \Omega ',t)dv'd\bar \Omega ' + Q(\bar r,v,\bar \Omega ,t)
\end{equation}

La forme stationnaire de cette équation est à encadrer sur un mur de sa chambre :\\

\cadre{
\begin{equation}
\begin{array}{ll}
\DS \bar \Omega .\bar \nabla \varphi (\bar r,v,\bar \Omega ) &\DS+ {\Sigma _t}(\bar r,v)\varphi (\bar r,v,\bar \Omega ) - \int\limits_{4\pi }    \int_o^\infty     {\Sigma _s}(\bar r,v',\bar \Omega ' \to v,\bar \Omega )\varphi (\bar r,v',\bar \Omega ')dv'd\bar \Omega '\\
&\DS  = \frac{1}{{4\pi }}\chi (v)\int\limits_{4\pi }    \int_o^\infty     \nu {\Sigma _f}(\bar r,v')\varphi (\bar r,v',\bar \Omega ')dv'd\bar \Omega ' + Q(\bar r,v,\bar \Omega )
\end{array}
\end{equation}
Dans cette notation, les termes de sources (fission et sources extérieures) sont à droite.}\ \\

Lorsque l'on connaît cette équation, il est possible d'adopter la notation compacte suivante
\begin{equation}
K\varphi  = J\varphi  + Q
\end{equation}
où $K$ est l'o\textbf{opérateur de destruction-scattering}
\begin{equation}
\begin{array}{l}
(K\varphi )(\bar r,v,\bar \Omega ) = \bar \Omega .\bar \nabla \varphi (\bar r,v,\bar \Omega ) + {\Sigma _t}(\bar r,v)\varphi (\bar r,v,\bar \Omega )\\
\quad \quad \quad \quad \quad \;\; - \int\limits_{4\pi }    \int_o^\infty     {\Sigma _s}(\bar r,v',\bar \Omega ' \to v,\bar \Omega )\varphi (\bar r,v',\bar \Omega ')dv'd\bar \Omega '
\end{array}
\end{equation}
et $J$ l'\textbf{opérateur production}
\begin{equation}
(J\varphi )(\bar r,v,\bar \Omega ) = \frac{1}{{4\pi }}\chi (v)\int\limits_{4\pi }    \int_o^\infty     \nu {\Sigma _f}(\bar r,v')\varphi (\bar r,v',\bar \Omega ')dv'd\bar \Omega '
\end{equation}
Notons que la forme compacte non-stationnaire s'énonce
\begin{equation}
\frac{1}{v}\frac{{\partial \varphi }}{{\partial t}} = (J - K)\varphi  + Q
\end{equation}


\subsubsection{Avec neutrons retardés}
Hélas, tout ceci était trop simple. Il faut en effet tenir comptes des neutrons qui ne sont pas 
directement issu des fissions mais émis par des produits de fission qui les produisent par 
désexcitation radioactive. On se rappelle que les neutrons retardés sont émis par les précurseurs 
que l'on peut classer en six groupes (classement par spectre et période de demi-vie).\\

On s'intéresse à l'évolution temporelle de la concentration $C_i$ des précurseurs du groupe $i$
\begin{equation}
\frac{{\partial {C_i}(\bar r,t)}}{{\partial t}} = - {\lambda _i}{C_i}(\bar r,t)
 + \int\limits_{4\pi }    \int_o^\infty     {\Sigma _f}(\bar r,v)\varphi (\bar r,v,\bar \Omega ,t)dvd\bar \Omega 
\end{equation}
où $\lambda_i = \ln(2)/T_i$.\\

La premier terme, négatif, est lié à la décroissance radioactive. Les précurseurs se désintègrent 
avec une période de demi vie valant $2/\lambda_i$. Le second terme, positif, est lié aux fissions. 
On retrouve le taux de fission $\Sigma_f\varphi$ qui donne le nombre de neutrons par unité de volume 
et de temps mais tous ne donnent pas lieu à des précurseurs. Il faut multiplier ceci par $\nu$ pour 
avoir le nombre de neutrons émis par unité de volume et de temps puis encore multiplier ceci par 
$\beta$, la fraction de précurseurs. Ce second terme est alors fraction de neutrons dues à toutes 
les fissions en $\bar r$ dans le groupe $i$ par unité de volume et de temps.\\

Par similitude à l'opérateur de production, on défini l'opérateur de production des neutrons 
retardés du groupe $i$
\begin{equation}
{J_i}( \bullet ) = \frac{1}{{4\pi }}{\chi _i}(v)\int\limits_{4\pi }    \int_o^\infty     \nu {\Sigma _f}(\bar r,v') \bullet dv'd\bar \Omega '
\end{equation}
Onpeut définir un \textbf{opérateur de production totale} qui est l'opérateur de production 
précédemment décrit, mais modifié par les neutrons retardés
\begin{equation}
(J\varphi )(\bar r,v,\bar \Omega ,t) \to (1 - \beta )({J_o}\varphi )(\bar r,v,\bar \Omega ,t) + \sum\limits_{i = 1}^6    \left( {{\lambda _i}{C_i}(\bar r,t).\frac{{{\chi _i}(v)}}{{4\pi }}} \right)
\end{equation}
où le $J_0\varphi$ ne porte que sur les neutrons prompt, le second terme concerne les 
neutrons retardés. La fraction $\chi_i(v)/4\pi$ est présente car il n'y a pas de raison de considérer 
autre chose que de l'isotropie sur ces  neutrons retardés (les CI sont oubliés sur 10s).\\

On pourrait multiplier les six équations par $\chi_i(v)/4\pi$ pour faire apparaître les opérateurs 
de chaque groupes retardés. Posons alors
\begin{equation}
{F_i}(\bar r,v,t) \equiv \left( {{C_i}(\bar r,t).\frac{{{\chi _i}(v)}}{{4\pi }}} \right)
\end{equation}
Si l'on considère ceci, il faudra également avoir la dérivée temporelle de cette quantité. On trouve 
alors un système d'équation pour le problème de transport neutroniques avec neutrons retardés\footnote{Pq?} :\\

\cadre{
\begin{equation}
\frac{{\partial {F_i}(\bar r,v,t)}}{{\partial t}} =  - {\lambda _i}{F_i}(\bar r,v,t) + {\beta _i}({J_i}\varphi )(\bar r,v,t)\quad \quad ,\quad i = 1,\dots, 6
\end{equation}
\begin{equation}
\frac{1}{v}\frac{{\partial \varphi (\bar r,v,\bar \Omega ,t)}}{{\partial t}} = [(1 - \beta ){J_o} - K]\varphi (\bar r,v,\bar \Omega ,t) + \sum\limits_{i = 1}^6    {\lambda _i}{F_i}(\bar r,v,t) + Q(\bar r,v,\bar \Omega ,t)
\end{equation}}

Dans le cas stationnaire, $\partial_t = 0$. Comme il s'agit d'un situation d'équilibre, la 
décroissance radioactive est équivalente àla production de précurseurs $\lambda_i F_i$ va être 
remplacé par les $\beta_iJ_i$
\begin{equation}
[(1 - \beta ){J_o} - K + \sum\limits_{i = 1}^6    {\beta _i}{J_i}]\varphi (\bar r,v,\bar \Omega ) + Q(\bar r,v,\bar \Omega ) = 0
\end{equation}
L'opérateur de production est équivalent à celui de $J$ où l'on remplace $J$ par $J_0$ (pour les 
prompts) à une seule différence près, qui se trouve dans le spectre
\begin{equation}
{\chi _o}(v) \to \tilde \chi (v) \equiv (1 - \beta ){\chi _o}(v) + \sum\limits_{i = 1}^6    {\beta _i}{\chi _i}(v)
\end{equation}
La seule différence se trouve dans le spectre. Si l'on substitue $\chi_i$ à la place des 
$\chi_0$ on retrouve la même expression que précédemment\footnote{J'vois paas}. Le formalisme est 
donc équivalent, avec ou sans neutrons retardés, à condition de considérer le spectre de fission 
modifié. L'inclusion des neutrons retardés ne fait donc que modifier le spectre d'émission 
général des neutrons de fissions.



\subsection{Continuité et conditions limites}
Un réacteur nucléaire est la juxtaposition de milieux uniformes ($\Sigma$ indépendant de la 
position). Dès lors, comment combiner les solutions de
\begin{equation}
\bar \Omega .\bar \nabla \varphi (\bar r,v,\bar \Omega ) + {\Sigma _t}(\bar r,v)\varphi (\bar r,v,\bar \Omega ) = S(\bar r,v,\bar \Omega )
\end{equation}
Dans un cas où l'on possède différents milieux? Soit $\Gamma$ la frontière de discontinuité (on 
suppose qu'il n'y a pas de sources superficielles de neutrons). Plaçons nous sur la frontière 
extérieure du réacteur : $[-\epsilon,\epsilon]$ autour de $\bar r_s\in\Gamma$ dans la direction 
$\bar\Omega$
\begin{equation}
\int_{ - \varepsilon }^{ + \varepsilon }    \bar \Omega .\bar \nabla \varphi ({\bar r_s} + \xi \bar \Omega ,v,\bar \Omega )d\xi  + \int_{ - \varepsilon }^{ + \varepsilon }    {\Sigma _t}({\bar r_s} + \xi \bar \Omega ,v)\varphi ({\bar r_s} + \xi \bar \Omega ,v,\bar \Omega )d\xi  = \int_{ - \varepsilon }^{ + \varepsilon }    S({\bar r_s} + \xi \bar \Omega ,v,\bar \Omega )d\xi 
\end{equation}
Par continuité du flux sur $\Gamma$
\begin{equation}
\varphi ({\bar r_s} + \varepsilon \bar \Omega ,v,\bar \Omega ) - \varphi ({\bar r_s} - \varepsilon \bar \Omega ,v,\bar \Omega ) = O(\varepsilon )0
\end{equation}
Comme condition aux limites, considérons un réacteur dans le vide. Lorsqu'un neutron sort, il ne 
reviendra pas car il sera dans le vide. De même, rien ne va rentrer car rien ne vient du vide. Dès 
lors, en tout point de la surface extérieur, pour toute vitesse et toute direction entrante, le 
flux est nul (flux de neutrons rentrant nul)
\begin{equation}
\varphi ({\bar r_s},v,\bar \Omega ) = 0\quad \quad \forall {\bar r_s} \in \Gamma ,\;\bar n.\bar \Omega  < 0
\end{equation}

\subsection{Formes intégrales}
En oubliant provisoirement que la source dépend du flux, nous obtenons une simple équation 
différentielle du premier ordre
\begin{equation}
\frac{{d\varphi ({{\bar r}_o} + s\bar \Omega ,v,\bar \Omega )}}{{ds}} + {\Sigma _t}({\bar r_o} + s\bar \Omega ,v)\varphi ({\bar r_o} + s\bar \Omega ,v,\bar \Omega ) = S({\bar r_o} + s\bar \Omega ,v,\bar \Omega )
\end{equation}
où $s$ est la distance parcourue dans la direction d'un neutron. En considérant $v, \bar\Omega$ comme 
de simple paramètres, nous obtenons bien une variation du flux avec le libre parcours des neutrons.\\

Nous pouvons résoudre cette ED en décomposant en une "partie homogène" et une seconde "non-homogène". 
La partie homogène s'écrit
\begin{equation}
\dfrac{d\phi_s}{dt}+\Sigma_t(s)\phi(s) = 0\qquad\Leftrightarrow\qquad\phi_H(s) = e^{-\int \Sigma_t(s')ds'}
\end{equation}
En considérant comme équation non-homogène
\begin{equation}
\phi_{NH} : C(s).\phi_H(s)
\end{equation}
On en tire
\begin{equation}
\frac{dC(s)}{ds}\phi_H(s)+C(s)\frac{d\phi(s)}{dt} + \Sigma_t(s)C(s)\phi_H(s) = S(s)
\end{equation}
où $S(s)$ est la source (la somme des deux derniers terme est nulle). Il est possible de 
particulariser ce résultat avec les conditions initiales : si on recule suffisamment vers les 
$\bar r$ négatifs, on va sortir du réacteur. A un moment donné, la valeur du flux 
$\phi(\bar r_s, \bar r)$ va se retrouver en direction entrante et être nulle : on peut ainsi 
annuler la constante de l'équation homogène.\footnote{Je ne vois pas comment arriver ensuite à Lagrange.}. \\

Par la méthode de la variation des constantes de Lagrange
\begin{equation}
\varphi ({\bar r_o} + s\bar \Omega ,v,\bar \Omega ) = \int_{ - \infty }^s {  {e^{ - \int_{s'}^s    {\Sigma _t}({{\bar r}_o} + s\bar \Omega ,v)ds}}} S({\bar r_o} + s'\bar \Omega ,v,\bar \Omega )ds'
\end{equation}
En effectuant le changement de variable $\bar r - {\bar r_o} = s\bar \Omega $

\cadre{
\begin{equation}
\varphi (\bar r,v,\bar \Omega ) = \int_o^\infty  { {e^{ - \int_o^s   {\Sigma _t}(\bar r - s'\bar \Omega ,v)ds'}}} S(\bar r - s\bar \Omega ,v,\bar \Omega )ds
\end{equation}}\ \\

Interprétons ce résultat. Au point $\bar r$, nous prenons en considération tous les points qui se 
trouve à un certains libre parcours $s$ avant (en amont de $\bar r$). Le flux n'est rien autre que 
tous les points en amont pondérée par une exponentielle décroissante. Cette exponentielle n'est 
rien d'autre que la probabilité de survie à toute interaction ($\Sigma_t$) sur le libre parcours de 
0 à $s$.\\

Dit autrement, le flux en $\bar r$ n'est autre que la somme de toute les sources que l'on peut 
avoir au sein même du réacteur sur la demi-droite qui part de l'extrémité du réacteur jusqu'au point $\bar r$, sources pondérées par la probabilité de survivre à toute interactions\footnote{Typiquement le genre de question d'examen.}.\\

L'épaisseur (ou distance) optique est définie\footnote{Dans quel but?}
\begin{equation}
{\tau _v}(\bar r,{\bar r_o}) \equiv \int_o^s   {\Sigma _t}({\bar r_o} + s'\bar \Omega ,v)ds'
\end{equation}
Il doit être possible de passer à une intégrale de volume. Pour se faire, il faut se rappeler que 
l'élément de volume est constitué de la distance parcourue au carrée multipliée par $ds$ (ce qui 
est parcouru) et par l'élément d'angle solide $d\bar\Omega'$ (soit en réalité l'équivalent de $r^2.dr.d\theta$
\begin{equation}
d(\bar r - {\bar r_o}) = {\left| {\bar r - {{\bar r}_o}} \right|^2}dsd\bar \Omega '
\end{equation}
En effectuant le changement de variable
\begin{equation}
\begin{array}{ll}
\DS\varphi (\bar r,v,\bar \Omega ) &\DS= \int_o^\infty  {  \int\limits_{4\pi }    {e^{ - {\tau _v}(\bar r,{{\bar r}_o}(s))}}} S({\bar r_o}(s),v,\bar \Omega )\delta \left( {\bar \Omega  - \bar \Omega '} \right)dsd\bar \Omega '\vspace{2mm}\\
&\DS= \int_{{R^3}}^{} {  \frac{{{e^{ - {\tau _v}(\bar r,{{\bar r}_o})}}}}{{{{\left| {\bar r - {{\bar r}_o}} \right|}^2}}}} S({\bar r_o},v,\bar \Omega )\delta \left( {\bar \Omega  - \frac{{\bar r - {{\bar r}_o}}}{{\left| {\bar r - {{\bar r}_o}} \right|}}} \right)d{\bar r_o}
\end{array}
\end{equation}
En effectuant la division par \dots, on fait apparaitre l'élément de volume ($1/|\bar r -\bar r_0
|^2$) où l'on a fait rentrer $\bar\Omega'$ dans $d\bar r_0$. Comme nous avons  un terme qui 
correspond à la survie, il faut que $\bar r_0$ et $\bar r$ soient alignés dans la direction 
souhaitée : $\bar\Omega'-\vec{1_{r0}}$ car tous les points situés à $\bar r_0$ ne contribuent pas
au flux dans la direction $\bar\Omega$. Ce facteur dans le $\delta(x)$ limite donc les points sources
à ceux pouvant être alignés au flux. \\

Après intégration pour obtenir le flux total, on trouve
\begin{equation}
\varphi (\bar r,v) = \int_{{R^3}}^{} {  \frac{{{e^{ - {\tau _v}(\bar r,{{\bar r}_o})}}}}{{4\pi {{\left| {\bar r - {{\bar r}_o}} \right|}^2}}}} S({\bar r_o},v)d{\bar r_o}
\end{equation}
Le $\frac{1}{4\pi r^2}$ correspond au fait que le flux est une quantité qui évolue en $1/r^2$ ce 
qui n'est pas une première (sphère)\footnote{\danger\ Notons que la source est fonction du flux!}.

\subsubsection{Forme explicite de l'équation intégrale du flux angulaire}
Nous avons ainsi établi pour le flux
\begin{equation}
\varphi (\bar r,v,\bar \Omega ) = \int_{{R^3}}^{} {  \frac{{{e^{ - {\tau _v}(\bar r,{{\bar r}_o})}}}}{{{{\left| {\bar r - {{\bar r}_o}} \right|}^2}}}} S({\bar r_o},v,\bar \Omega )\delta \left( {\bar \Omega  - \frac{{\bar r - {{\bar r}_o}}}{{\left| {\bar r - {{\bar r}_o}} \right|}}} \right)d{\bar r_o}
\end{equation}
Et pour la source
\begin{equation}
\begin{array}{l}
S(\bar r,v,\bar \Omega ) = \frac{1}{{4\pi }}\chi (v)\int\limits_{4\pi }    \int_o^\infty     \nu {\Sigma _f}(\bar r,v')\varphi (\bar r,v',\bar \Omega ')dv'd\bar \Omega '\\
\quad \quad \quad  + \int\limits_{4\pi }    \int_o^\infty     {\Sigma _s}(\bar r,v',\bar \Omega ' \to v,\bar \Omega )\varphi (\bar r,v',\bar \Omega ')dv'd\bar \Omega ' + Q(\bar r,v,\bar \Omega )
\end{array}
\end{equation}
En en tire donc\footnote{"A laisser trainer en bord de table lorsque son halène sent la bouteille.}\\

\cadre{
\begin{equation}
\begin{array}{ll}
\DS \varphi (\bar r,v,\bar \Omega ) &= \int_{{R^3}}^{}    \frac{{{e^{ - {\tau _v}(\bar r,{{\bar r}_o})}}}}{{{{\left| {\bar r - {{\bar r}_o}} \right|}^2}}}\delta \left( {\bar \Omega  - \frac{{\bar r - {{\bar r}_o}}}{{\left| {\bar r - {{\bar r}_o}} \right|}}} \right)Q({{\bar r}_o},v,\bar \Omega )d{{\bar r}_o}\DS\\\DS&
 + \int_{{R^3}}^{}    \frac{{{e^{ - {\tau _v}(\bar r,{{\bar r}_o})}}}}{{{{\left| {\bar r - {{\bar r}_o}} \right|}^2}}}\delta \left( {\bar \Omega  - \frac{{\bar r - {{\bar r}_o}}}{{\left| {\bar r - {{\bar r}_o}} \right|}}} \right)\int\limits_{4\pi }    \int_o^\infty     [{\Sigma _s}({{\bar r}_o},v',\bar \Omega ' \to v,\bar \Omega ) + \frac{{\chi (v)}}{{4\pi }}\nu {\Sigma _f}({{\bar r}_o},v')]\\\DS&
\quad \quad \quad \quad \quad \quad \quad \quad \quad \quad \quad \quad \quad \quad \quad  \times \varphi ({{\bar r}_o},v',\bar \Omega ')dv'd\bar \Omega 'd\bar r
\end{array}
\label{eq:Ch2.1}
\end{equation}}\ \\

Malgré une forme un peu lourde, ce terme s'interprète facilement. La première intégrale considère 
tout ce qui vient d'une source en $\bar r_0$ sachant que ce point $\bar r_0$ est alligné avec 
le point $\bar r$ par la direction de propagation $\bar\Omega$ et que ces neutrons contribuant au 
flux ne subissent aucune interactions jusqu'au point $\bar r$. La seconde intégrale correspond elle 
au scattering.


\subsubsection{Noyau de transition}
Il s'agit d'un processus de transport exprimant la distribution de probabilité d'arriver aux 
coordonnées de la prochaine collision depuis les coordonnées de la précédente, le long du libre 
parcours moyen
\begin{equation}
T(\bar r',v',\bar \Omega ' \to \bar r,v,\bar \Omega ) = {\Sigma _t}(\bar r,v')\frac{{{e^{ - {\tau _v}(\bar r',\bar r)}}}}{{{{\left| {\bar r - \bar r'} \right|}^2}}}\delta \left( {\bar \Omega  - \frac{{\bar r - \bar r'}}{{\left| {\bar r - \bar r'} \right|}}} \right).\delta (v - v').\delta (\bar \Omega  - \bar \Omega ')
\end{equation}
Cette expression n'est pas belle, mais pas compliquée non plus. Pour passer de $\bar r', v'$ à 
$\bar r, v$ il faut que si la vitesse et et la propagation restent constantes afin de ne pas 
perturber $v$ et $\bar\Omega$ afin de garantir le bon alignement et l'absence d'interactions 
qui affecteraient la sortie entre la coordonnée de sortie précédente et la suivante. Il faut 
également garantir que $\bar r'$ et $\bar{r'}$ soient alignés avec la direction de propagation. 
Comme il s'agit d'une densité de probabilité, on retrouvera bien l'intégrale de $\Sigma_t$. Dès 
lors, la multiplication par $\frac{{{e^{ - {\tau _v}(\bar r',\bar r)}}}}{|\bar r - \bar r'|^2}$
exprime la densité de probabilité du du libre parcours avec un facteur venant de l'intégration sur le volume au dénominateur.


\newpage
\subsubsection{Noyau de collision}
Le but est de caractériser les \textbf{sorties} possibles étant donnés l'interaction. Il y a trois 
probabilités possibles
\begin{enumerate}
\item $\Sigma_c/\Sigma_t$
\item $\Sigma_f/\Sigma_t$
\item $\Sigma_s/\Sigma_t$
\end{enumerate}
On peut assez facilement caractériser ces probabilités
\begin{enumerate}
\item La probabilité que quelque chose sorte après une capture est tout simplement nulle.
\item Elle est donnée par le nombre de neutrons moyens émis par fission que l'on multiplie par le
spectre maxwellien et divise par $4\pi$, comme précédemment
\item Un terme de scattering
\end{enumerate}
Ceci nous donne
\begin{equation}
C(\bar r',v',\bar \Omega ' \to \bar r,v,\bar \Omega ) = \frac{{[{\Sigma _s}(\bar r',v',\bar \Omega ' \to v,\bar \Omega ) + \frac{{\chi (v)}}{{4\pi }}\nu {\Sigma _f}(\bar r',v')]}}{{{\Sigma _t}(\bar r',v')}}\delta (\bar r - \bar r')
\end{equation}
On trouve alors un pic sur la coordonnée spatiale à l'endroit de l'interaction.\\

En introduisant la notation compacte $P \equiv (\bar r,v,\bar \Omega )\;;\;P' \equiv (\bar r',v',
\bar \Omega ')$, on peut écrire
\begin{equation}
\left\{\begin{array}{ll}
\DS \int T(P'\to P)dP &= 1-e^{-\tau_v(\infty)}\vspace{2mm}\\
\DS \int C(P'\to P)dP &\neq 1
\end{array}\right.
\end{equation}
Pour un réacteur infini, la première équation vaut 1 car s'il est infini il parviendra toujours à 
atteindre une collision.  

\subsubsection{Densité de collisions}
Introduisons la \textit{densité entrante} $\psi(P)$d$P$ comme le nombre de neutrons entrant par 
unité de temps dans une collision de coordonnées $dP$ autour de $P$. Le nombre moyen de neutrons 
qui rentre est donné par le taux de réaction
\begin{equation}
\psi (P) = {\Sigma _t}(P)\varphi (P)
\end{equation}
Il est également possible de définir une densité de sortie $\zeta (P)dP$ comme étant le nombre moyen de neutrons sortants d'une interaction dans un volume d$P$ autour de $P$. \\

A l'aide de ces grandeurs, on peut écrire deux équations d'évolution
\begin{equation}
\left\{\begin{array}{ll}
\zeta (P) &\DS= Q(P) + \int\limits_{}   \psi (P')C(P' \to P)dP'\\
\psi (P) &\DS= \int\limits_{}   \zeta (P')T(P' \to P)dP'
\end{array}\right.
\end{equation}
La seconde équation exprime que la densité d'entrée est tout ce qui vient d'une interaction 
précédente et qui a été transporté le long d'un libre parcours (survie, \dots) jusqu'à $P$. La 
première equation est la somme de $Q(P)$\footnote{?} additionné à tout ce qui sort du noyau. Par 
substitution, on trouve\\

\cadre{
\begin{equation}
\begin{array}{ll}
\psi (P) &\DS= \int    Q(P')T(P' \to P)dP' + \int    \int\limits_{}    \psi (P)C(P \to P')T(P' \to P)dP'dP\\
&\DS\equiv I(P) + \int    \psi (P)K(P \to P)dP
\end{array}
\end{equation}}\ 

Même sans connaître le détail des noyaux $P$ et $C$, on obtient le transport des particules dans un 
milieu de façon assez intuitive. Le terme indépendant est le transport de ce qui est émis par une 
source extérieure auquel on ajoute un noyau combiné de $C$ et $T$ qui s'applique à cette densité
\footnote{Interprétation incomplète!}. Écrite en notation compacte, il ne faut pas être surpris 
qu'il ne s'agit rien d'autre que l'équation \eqref{eq:Ch2.1} multipliée par $\Sigma_t$ (soit 
$\Sigma_t(P)*\varphi(P)$) ! Une autre interprétation possible est de considérer que le transport 
se fait par un processus "choc par choc".


\section{Solutions formelles via les séries de Neumann}
Imaginons que dans la précédente équations, $\psi_0$ soit défini comme le fameux terme indépendant. 
A partir de celui-ci, on peut appliquer un $\psi_1$ qui est l'application du noyau $K$ à $\psi_0$, 
\dots 
\begin{equation}
\left\{\begin{array}{ll}
{\psi _o}(P) &\equiv I(P)\\
{\psi _j}(P)\, &\DS\equiv \int   {\psi _{j - 1}}(P')K(P' \to P)dP'\quad ,\;j = 1...\infty 
\end{array}\right.
\end{equation}
Les $\psi$ sont défini terme à terme où l'indice zéro est donc ce terme indépendant et l'indice 
$j$ le résultat d'une application supplémentaire du noyau composé $K$. Il est possible d'en tirer 
solution en série
\begin{equation}
\psi (P)\, \equiv \sum\limits_{j = 0}^\infty    {\psi _j}(P)
\end{equation}
Mais celle-ci n'est pas réaliste, une somme infinie n'a pas de sens physique. Cependant, elle sert 
de base par la solution algorithmique.