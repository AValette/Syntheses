\chapter{Semi-conducteurs à l'équilibre thermique}
Le but de ce chapitre est d'étudier comment la population électronique se 
répartit sur les différent niveaux d'énergies.

\section{Densités d'états en fonction de l'énergie}
Petite note : le potentiel chimique correspond au niveau de Fermi pour les
semi-conducteurs. Ceci étant dit, rappelons-nous des ellipsoïdes vus 
précédemment : ceux-ci sont gênants, on veut des sphères comme dans le 
modèle de Sommerfeld. Pour se faire, nous allons ramener l'expression 
donnant les surfaces isoénergétiques dans la direction (123) (direction qui 
diagonalise le tenseur de masse effective inverse) des dilatations d'axes. 
Ces dilatations d'axes nous permettront de retrouvé notre bien aimé modèle.\\

On peut définir $D(\epsilon)$ une densité d'états par rapport à l'énergie tel 
que $D(\epsilon)d\epsilon$ représente le nombre de places disponibles dans la bande 
pour des électrons d'un spin donné ayant une énergie entre $\epsilon$ et $\epsilon
+d\epsilon$. Pour se faire : 
\begin{itemize}
\item[$\bullet$] On calcule dans l'espace $\vec{k}$ le volume compris entre les 
surfaces isoénergétiques $\epsilon$ et $\epsilon+d\epsilon$
\item[$\bullet$] Multiplie par la densité des états de cet espace $(1/8\pi^3$ si 
l'on considère un volume unitaire)
\end{itemize}
Pour retrouver notre forme sphérique, on applique les dilatations d'axes suivantes :
\begin{equation}
\kappa_1 = \sqrt{\dfrac{m}{m_1}}k_1,\qquad\kappa_2 = \sqrt{\dfrac{m}{m_2}}k_2,\qquad
\kappa_3 = \sqrt{\dfrac{m}{m_3}}k_3.
\end{equation}
On retrouve une expression identique à celle trouvé au début du cours. La densité 
d'état dans l'espace $K$ vaut 
\begin{equation}
\dfrac{(m_1m_2m_3)^{1/2}}{m^{3/2}}\dfrac{1}{8\pi^3}
\end{equation}
On obtient finalement\footnote{??}
\begin{equation}
D(\epsilon) = \dfrac{2\pi}{h^3}2^{3/2}(m_1m_2m_3)^{1/2}(|\epsilon-\epsilon_0|)^{1/2}
\end{equation}
Cette expression est relative à un extremum : il faut la multiplier par le nombre 
de valées $\lambda$ dans le cas ou il y en aurait plusieurs. \\

$\bullet$ \textsc{Densité d'états dans une bande de conduction}\\
Par facilité, il est plus simple de travailler avec une seule masse pour n'avoir 
qu'un paramètre: la \textit{masse effective de densité d'états} :
\begin{equation}
m_c = \left(\lambda m_1^{1/2}m_2^{1/2}m_3^{1/2}\right)^{2/3}
\end{equation}
Cette expression tien compte des $m_i$ mais aussi du nombre de minima équivalent 
dans la bande via $\lambda$ : toutes les caractéristiques de la structure en bandes 
sont "dans" $m_c$. Notre expression devient
\begin{equation}
D_c(\epsilon) = \dfrac{2\pi}{h^3}(2m_c)^{3/2}(\epsilon-\epsilon_c)^{1/2}
\end{equation}


$\bullet$ \textsc{Densité d'états dans une bande de valence}\\
On peut, de façon analogue, écrire
\begin{equation}
D_v(\epsilon) = \dfrac{2\pi}{h^3}(2m_v)^{3/2}(\epsilon_v-\epsilon)^{1/2}
\end{equation}
avec $m_v$, la masse effective de densité d'état pour l'ensemble des trous,  
qui concentre toute la description de la structure en bande 
\begin{equation}
m_V = \left[m_{v1}^{3/2}+m_{v2}^{3/2}\right]^{3/2}
\end{equation}
ou $m_{vi}$ sont les masses effectives de densité d'état.