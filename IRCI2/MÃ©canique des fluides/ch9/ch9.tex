\chapter{Dynamique des fluides parfaits}

\section{Introduction}
Grande innovation ! A partir de maintenant ce ne sont plus les $V_{ij}$ qui sont 
nuls, mais $\mu$\footnote{Dans l'équation $\tau_{ij} = -p\delta_{ij} + 2\mu
(\dots)$} : un fluide parfait est un milieu isotrope dont la loi de comportement 
est :
\begin{equation}
\tau_{ij} = -p\delta_{ij}
\end{equation}
où $p$ est la pression, une inconnue qui sera donnée par une équation d'état (
$p/\rho = R\theta$ pour un gaz parfait).\\
\textbf{Attention !} Il ne faut pas confondre :
\begin{enumerate}
\item Un fluide parfait : défini par sa \textit{loi de comportement} de fluide de
viscosité nulle ($\mu = 0$)
\item Un gaz parfait : défini par son \textit{équation d'état} qui peut avoir de
la viscosité
\item Un gars parfait : moi, tout simplement.
\end{enumerate}
Ceci pour dire qu'un gaz parfait n'est donc \textbf{pas} nécessairement un fluide
parfait ! Dans ce chapitre, on considère donc un \textbf{fluide parfait sans 
viscosité}, quelque soit son équation d'état.


\section{Équation du mouvement}
	\subsection{Équations d'Euler}
	Partons des équations générales du mouvement :
	\begin{equation}
	\rho v_i^\bullet = f_i + \partial_j \tau_{ji}
	\end{equation}
	On développe la dérivée matérielle :
	\begin{equation}
	\rho[\partial_0v_i + v_k\partial_kv_i] = f_i + \partial_j\tau_{ji}
	\end{equation}
	Le tenseur des contraintes étant $\tau_{ij} = -p\delta_{ij}$, on trouve les
	équations d'Euler $1^{\text{ère}}$ forme :\\
	
	\prop{\textsc{Équation d'Euler - première forme}
	\begin{equation}
	\rho[\partial_0v_i + v_k\partial_kv_i] = f_i - \partial_i p
    \end{equation}}
	Ces équations sont parfaitement correctes mais non résolvable analytiquement 
	car non linéaires (à cause du terme $u\frac{\partial u}{\partial x}$ si on 
	développe la notation.)\\
	
	Le souci de cette forme est que la dérivée ne porte pas sur l'entièreté des 
	termes. Transformons :
	\begin{equation}
	\begin{array}{lll}
	\rho\partial_0 v_i & \Rightarrow&  \partial_0(\rho v_i) - v_i\partial_0\rho\\
	\rho v_k\partial_k v_i &\Rightarrow & \partial_k(\rho v_iv_k)-v_i\partial_k(\rho
	v_k)
	\end{array}
	\end{equation}
	Ce qui nous donne :
	\begin{equation}
	\partial_0(\rho v_i) - \underline{v_i\partial_0\rho}+ \partial_k(\rho v_iv_k)-
	\underline{v_i\partial_k(\rho v_k)} = f_i-\partial_ip
	\end{equation}
	En utilisant l'équation de la continuité $\partial_0\rho + \partial_k(\rho v_k)
	=0$ on trouve :\\
	
	\prop{\textsc{Équation d'Euler - deuxième forme}
	\begin{equation}
	\partial_0(\rho v_i) + \partial_k(\rho v_iv_k) + \partial_i p = f_i
	\end{equation}}	
	
	\subsection{La pression motrice}	
	Dans le cas le plus courant où la force de volume est la force de pensanteur, on
	a (avec $z$ l'axe vertical positif vers le haut) :
	\begin{equation}
	\begin{array}{ccc}
	\rho_0\vec{F} = -\rho_0g\vec{1_z} & \vec{F} = -\grad U & U = gz
	\end{array}
	\end{equation}
	L'équation du mouvement s'écrit :
	\begin{equation}
	\rho_0v_i^\bullet = \rho_0F_i - \partial_i p\ \ \ \ \text{ou }\ \ \ \rho_0
	\vec{v}^\bullet = -\grad \hat{p}	
	\end{equation}
	Où $\hat{p}$ est définit comme la pression motrice :
	\begin{equation}
	\hat{p} = p+\rho_0 gz
	\end{equation}
	
		\subsubsection{Exemple : écoulement unidimensionnel}
		Voir slides 18-21.
	
	\subsection{Équations de Lamb}
	L'idée est de repartir de l'équation du mouvement, mais de développer autrement 
	la dérivée matérielle en commençant par additionner et soustraire un terme et utiliser
	$v_{[i,k]} = \frac{1}{2}(\partial_kv_i - \partial_iv_k)$ :
	\begin{equation}
	\begin{array}{ll}
	v_i^\bullet &= \partial_0v_i + v_k\partial_kv_i\\
	 &= \partial_0v_i + v_k(\partial_kv_i - \partial_iv_k)+v_k\partial_iv_k\\
	 &= \partial_0v_i + v_k 2 v_{[i,k]} + v_k\partial_iv_k
	\end{array}
	\end{equation}
	Mais\footnote{Note : d'où ?} $\vec{\omega} = \frac{1}{2}\rot \vec{v} \rightarrow 
	v_{[i,k]} = \delta_{ipk}\omega_p$. En remplaçant :
	\begin{equation}
	v_i^\bullet = \partial_0v_i + 2 \delta_{ipk} \omega_p v_k + \frac{1}{2}\partial_i
	(v_kv_k)
	\end{equation}
	Ce qui nous donnes les...\\
	
	\prop{\textsc{Equations de Lamb}
	\begin{equation}
	\dfrac{\partial \vec{v}}{\partial t} + 2(\vec{\omega}\times\vec{v}) + \grad\left(
	\dfrac{v^2}{2}\right) = \vec{F}-\frac{1}{\rho}\grad p\ \ \ \ \text{avec }\ \ \ 
	\vec{f} = \rho\vec{F}
	\end{equation}}
	\textbf{Attention !} Il faut bien garder en tête que $v_kv_k \neq v_k^2$ !
	
	\subsection{Energie spécifique totale - charge}
	Considérons le cas particulier où les 3 conditions suivantes sont vérifiées :
	\begin{enumerate}
	\item Les forces dérivent d'un potentiel : $\vec{F} = -\grad U$.
	\item Le fluide est barotrope : $\rho = \rho(p)$
	\begin{equation}
	\grad P  = \frac{1}{\rho}\,\grad p\ \ \ \ \text{ou }\ \ P = \int_{p_0}^p \frac{dp}
	{\rho}
	\end{equation}
	\item Si le fluide est homogène et incompressible : $\rho = \rho_0 = C^{ste}$ :
	\begin{equation}
	P = \frac{p}{\rho_0}
	\end{equation}
	\end{enumerate}	
	Dans ce cas, en effectuant les changements renseignés ci-dessus, l'équation de
	Lamb devient :
	\begin{equation}
	\frac{\partial \vec{v}}{\partial t} + 2(\vec{\omega}\times\vec{v}) = -\grad\left[
	U + P +\frac{v^2}{2}\right]
	\end{equation}
	L'objectif est que le terme de gauche soit nul impliquant un gradient nul qui lui
	impliquerait que le terme entre crochet n'est qu'une constante (notre rêve).
	
		\subsubsection{L'énergie spécifique totale}
		Soit l'énergie spécifique totale (par unité de masse) :
		\begin{equation}
		\epsilon = U + P + \frac{v^2}{2}
		\end{equation}
		avec, pour chaqun des termes :
		\begin{itemize}
		\item $\frac{v^2}{2}$ ; l'énergie cinétique par unité de masse
		\item $U$ ; l'énergie potentielle par unité de masse
		\item $P$ ; l'énergie de pression par unité de masse
		\end{itemize}
		
		\subsubsection{La charge}
		On définit la charge $H$ comme :
		\begin{equation}
		\epsilon = g\left[z+\frac{P}{g}+\frac{v^2}{2g}\right] = gH
		\end{equation}
		C'est à dire :
		\begin{equation}
		H = z + \frac{P}{g}+\frac{v^2}{2g}
		\end{equation}
		\begin{itemize}
		\item $\frac{v^2}{2g}$ ; la hauteur de vitesse
		\item $z$ ; la hauteur géométrique
		\item $\frac{P}{g}$ ; la hauteur de pression
		\end{itemize}

	
\section{Ecoulement permanant d'un fluide parfait}
	\subsection{Théorème de Bernouilli 1}
	Ce théorème s'applique dans le cas d'un écoulement permanent d'un fluide parfait 
	soumis à des forces massiques dérivant d'un potentiel. Reprenons :
	\begin{equation}
	\epsilon = U + P + \frac{v^2}{2}
	\end{equation}
	où $\epsilon$ est constant le long d'une ligne de courant\footnote{Courbe dont les
	trangentes sont les vitesses.}.\\
	
	Pour établir le théorème, on part de l'équation de Lamb :
	\begin{equation}
	\frac{\partial\vec{v}}{\partial t} + 2(\vec{\omega}\times\vec{v}) = -\grad \epsilon
	\end{equation}
	Les dérivées temporelles sont nulles, étant dans le cas de l'écoulement permanent.
	Multiplions scalairement par $\vec{v}$ pour annuler le produit vectoriel $\rightarrow
	\vec{v}.\grad \epsilon = 0$. Cela montre que $\grad \epsilon$ est perpendiculaire
	aux surfaces et donc $\epsilon$ est constant le long d'une ligne de courant mais
	également, par un raisonnement similaire (en multipliant scalairement par $\vec{
	\omega}$), le long d'une ligne de tourbillon.
		
		\subsubsection{Application : le tube de Pitot}
		Le tube de Pitot permet de mesurer une vitesse par différence de pression. 
		Historiquement utilisé en 1730 pour mesurer la vitesse du courant de la Seine, 
		on l'utilise encore aujourd'hui pour calculer la vitesse des avions. Plus d'infos
		slide 42 !
		
	
	\subsection{Théorème de Bernouilli 2}
	Prenons comme hypothèses :
	\begin{itemize}
	\item Ecoulement permanent
	\item \textbf{Irrotationnel}
	\item Fluide parfait
	\item Soumis à des forces massiques dérivant d'un potentiel; $\varepsilon = U + P + 
	\frac{v^2}{2}$
	\item Constant dans tout l'écoulement
	\end{itemize}
	
	Pour établir le théorème, on part de l'équation de Lamb :
	\begin{equation}
	\frac{\partial\vec v}{\partial t}+2(\vec{\omega}\times\vec{v}) = -\grad \varepsilon
	\end{equation}
	L'écoulement étant permanant, la dérivée par rapport au temps est nulle, de même pour
	$\vec{\omega}$, étant dans le cas irrotationnel. On obtient donc :
	\begin{equation}
	\grad \varepsilon = 0\ \ \ \ \ \Rightarrow\ \ \ \varepsilon = C^{ste}\ \text{dans tout l'
	écoulement}
	\end{equation}
	
		\subsubsection{Application : vidange d'un réservoir}
		Petite application sympathique slide 48-50.
	
	
	
