\chapter{Méthodes d'approximations (Schrödinger indépendant du temps)}
Deux méthodes seront vue dans ce chapitre : la méthode des perturbations 
et la méthode des variations. \\

Pour les perturbations, on considère un hamiltonien proche du problème non perturbé. 
On part d'un problème proche ou l'on a un hamiltonien diagonalisable et, à partir de 
ce système, on va "allumer" la perturbation et se rapprocher de la situation réelle. 
Ceci sera possible en utilisant divers développement en série afin de se rapprocher 
du cas réel.\\

Pour la méthode des variations, on va définir une fonctionnelle énergie sur laquelle 
on va appliquer la méthode des variations de sortte à minimiser l'énergie par rapport 
à une famille de "fonction d'essai". On va ainsi définir une classe de fonction d'onde, 
les fonctions d'essai, et on va essayer de trouver quelle est celle qui s'approche le 
plus de la vrai fonction d'onde exacte. \\

On peut s'intéresser soit au problème stationnaire ou son évolution dans le temps. Pour 
la méthode des perturbations on verra comment l'utiliser pour l'équation stationnaire 
(ou indépendante du temps) ; soit comment utiliser cette méthode pour approximer les états 
propres) ou alors la dépendante du temps (comment peut-on approximer la dynamique d'un 
système toujours avec la méthode des perturbations). On en déduira la règle d'or de Fermi. 

\section{Méthode des perturbations - équation stationnaire}
	\subsection{Principe base - notation}
	On veut résoudre le problème aux valeurs propres 
	\begin{equation}
	\hat{H}\ket{\psi_n} = E_n\ket{\psi_n}
	\end{equation}
	où $\hat{H}$ est indépendant du temps, il s'agit de l'hamiltonien \textit{perturbé}.
	Il existe un autre hamiltonien qui est le \textit{non-perturbé}
	\begin{equation}
	\hat{H_0} : \text{hamiltonien non perturbé}
	\end{equation}
	Ce qui est intéressant, c'est que $\hat{H_0}$ est diagonalisable, ce qui n'est pas 
	nécessairement le cas de $\hat{H}$. Cet hamiltonien non perturbé donne l'équation 
	aux valeurs propres suivante
	\begin{equation}
	\hat{H_0}\ket{\psi_n^{(0)}} = E_n^{(0)}\ket{\psi_n^{(0)}}
	\end{equation}				
	Les solutions de cette équation donnent les solutions (désignées par $\ ^{(0)}$) du 
	problème non-perturbés. 	On peut réécrire l'équation que l'on veut résoudre en fonction 
	de l’hamiltonien non perturbé
	\begin{equation}
	\hat{H} = \hat{H_0} + \hat{W}\qquad \text{où }\ \hat{W} = \lambda\hat{H_1},\quad \lambda\in 
	\mathbb{R}
	\end{equation}
	où $\hat{W}$, un opérateur hermitien, représente la perturbation entre $\hat{H_0}$ et $\hat{H}$ 
	et $\lambda$ est un paramètre sans dimension. Toute l'idée est de dire que la perturbation 
	est petite : $\lambda\ll 1$ de sorte que l'on va pouvoir développer en série de puissances de 
	$\lambda$. Ainsi, sans trop s'écarter de $\hat{H_0}$, on va pouvoir s'intéresser à ce qui nous 
	intéresse dans le voisinage $\hat{H_0}$. L'équation que l'on veut résoudre devient alors
	\begin{equation}
	(\hat{H_0}+\lambda\hat{H_1})\ket{\psi_n} = E_n\ket{\psi_n}
	\end{equation}
	Insistons sur le fait que $E_n\ket{\psi_n}$ dépend du paramètre $\lambda$ : on peut voir ce terme
	comme une fonction analytique de $\lambda$ dans un certain domaine de convergence. L'hamiltonien 
	$\hat{H}_1$ est introduit pour caractériser l'écart entre $\hat{H}$ et $\hat{H_0}$. Il ne s'agit 
	que d'une ré-écriture de $\hat{W}$ que l'on cherche également à paramétrer. Les limites suivantes 
	sont vérifiées
	\begin{equation}
	\begin{array}{ll}
	\lim\limits_{\lambda\rightarrow0} E_n &= E_n^{(0)}\\
	\lim\limits_{\lambda\rightarrow0} \ket{\psi_n} &= \ket{\psi_n^{(0)}}
	\end{array}
	\end{equation}
	Écrivons maintenant ces fonction comme un développement en série de $\lambda$.
	\begin{equation}
	\left\{\begin{array}{ll}
	\ket{\psi_n} &= \ket{\psi_n^{(0)}}+\lambda\ket{\psi_n^{(1)}}+\lambda^2\ket{
	\psi_n^{(2)}}+\dots\\
	E_n &= E_n^{(0)}+\lambda E_n^{(1)}+\lambda^2E_n^{(2)}+\dots
	\end{array}\right.
	\end{equation}
	Par substitution, nous obtenons
	\begin{equation}
	\begin{array}{ll}
	(\hat{H_0}+\lambda\hat{H_1})\left(\ket{\psi_n^{(0)}}+\lambda\ket{\psi_n^{(1)}}
	+\lambda^2\ket{\psi_n^{(2)}}+\dots\right) &= \left(E_n^{(0)}+\lambda E_n^{(1)}
	+\lambda^2E_n^{(2)}+\dots \right)
	\left(\ket{\psi_n^{(0)}}+\dots\right.\\
	&\left.\dots+\lambda\ket{\psi_n^{(1)}}+\lambda^2\ket{\psi_n^{(2)}}+\dots\right),
	\qquad 	\underline{\forall\lambda}
	\end{array}
	\end{equation}		
	Le terme d'ordre 0 est connu mais les différentes corrections	$\left(\ket{\psi_n^{(1)}}, 
	E_n^{(1)}, 	\ket{\psi_n^{(2)}}, E_n^{(2)},\dots\right)$ sont inconnues. Comme ces deux 
	expressions doivent être valables pour tout lambda, on va pouvoir procéder à une 
	identification terme à terme
	\begin{equation}
	\begin{array}{lll}
	\lambda^0 &: \hat{H_0}\ket{\psi_n^{(0)}} &= E_n^{(0)}\ket{\psi_n^{(0)}}\\
	\lambda^1 &: \hat{H_0}\ket{\psi_n^{(1)}}+\hat{H_1}\ket{\psi_n^{(0)}} &= E_n^{(0
	)}\ket{\psi_n^{(1)}}+E_n^{(1)}\ket{\psi_n^{(0)}}\\
	\lambda^2 &: \hat{H_0}\ket{\psi_n^{(2)}}+\hat{H_1}\ket{\psi_n^{(1)}} &= E_n^{(0
	)}\ket{\psi_n^{(2)}}+E_n^{(1)}\ket{\psi_n^{(1)}}+E_n^{(2)}\ket{\psi_n^{(0)}}\\		
	&&\dots
	\end{array}
	\label{eq:9.6}
	\end{equation}
	La première ligne ($\lambda^0$) est trivialement vraie. Plus on descend en ordre, 
	plus on aura une précision dans les séries	de puissances. Il ne faut cependant pas 
	oublier la condition de normalisation 
	\begin{equation}
	\bra{\psi_n}\ket{\psi_n} = 1,\qquad \underline{\forall\lambda}
	\end{equation}
	On peut exprimer cette condition de normalisation en terme de 	puissances de $\lambda$.
	\begin{equation}
	\left(\bra{\psi_n^{(0)}}+\lambda\bra{\psi_n^{(1)}}+\lambda^2\bra{\psi_n^{(2)}}+\dots\right)
	\left(\ket{\psi_n^{(0)}}+\lambda\ket{\psi_n^{(1)}}+\lambda^2\ket{\psi_n^{(2)}}+\dots\right) = 1
	\label{eq:9.8}
	\end{equation}
	Nous pouvons à nouveau procéder par identification
	\begin{equation}
	\begin{array}{llll}
	\lambda^0 &: \bra{\psi_n^{(0)}}\ket{\psi_n^{(0)}} &=1\\
	\lambda^1 &: \bra{\psi_n^{(0)}}\ket{\psi_n^{(1)}} +\bra{\psi_n^{(1)}}\ket{\psi_n^{(0)}} &= 
	0 = 2\Re\left(\bra{\psi_n^{(0)}}\ket{\psi_n^{(1)}}\right)\\
	\lambda^2 &: \bra{\psi_n^{(0)}}\ket{\psi_n^{(2)}} +\bra{\psi_n^{(1)}}\ket{\psi_n^{(1)}}
	+\bra{\psi_n^{(2)}}\ket{\psi_n^{(0)}} &= 0 =2\Re\left(\bra{\psi_n^{(0)}}\ket{\psi_n^{(2)}}\right)		
	+\bra{\psi_n^{(2)}}\ket{\psi_n^{(0)}}\\
	&&\dots
	\end{array}
	\end{equation}
	La première ligne valant l'unité tout comme \eqref{eq:9.8}, les autres lignes doivent 
	forcément être nulles. On peut remarquer que certains termes sont apparaissent sous la 
	forme de conjugués, permettant de faire apparaître la partie réelle de ces termes.\\
	
	Pour trouver les $\lambda^\alpha$, il faut résoudre ce système de proche en proche (la 
	connaissance de la 	seconde ligne est nécessaire à la résolution de la $3^e, \dots$). Plus 
	on le résout "loin", plus la précision sera meilleure. Il est possible de voir $\ket{\psi_n}$
	comme un développement en série de $\lambda$ dont chaque terme peut être vu comme un 
	certain ket que l'on peut ré-exprimer dans une base. Une base naturelle est celle du 
	problème non-perturbé :
	\begin{equation}
	\ket{\psi_n^{(i)}} = \sum_k C_k\ket{\psi_k^{(0)}}
	\end{equation}
	où les $\ket{\psi_k^{(0)}}$ forment la base non-perturbée.
		
	\subsection{Perturbation d'un niveau non-dégénéré (1$^{er}$ et $2^e$ ordre)}
	Nous allons ici chercher à obtenir une expression de nos énergies et états perturbés. Nous 
	verrons plus tard que la dégénérescence joue un rôle important, mais commençons par nous 
	attardé sur le cas non-dégénéré.
		
		\subsubsection{Énergie au premier ordre}
		Soit notre problème non-perturbé
		\begin{equation}
		\hat{H_0}\ket{\psi_n^{0}} = E_n^{(0)}\ket{\psi_n^{0}}
		\end{equation}
		Nous allons partir de la première équation non triviale de \eqref{eq:9.6} (soit 
		celle de $\lambda^1$)) et la multiplier à gauche par $\bra{\psi_n^{(0)}}$
		\begin{equation}
		\underbrace{\bra{\psi_n^{(0)}}\hat{H_0}\ket{\psi_n^{(1)}}}_{E_n^{(0)}\underbrace{
		\bra{\psi_n^{(0)}}\ket{\psi_n^{(1)}}}_{=0}}+\bra{\psi_n^{(0)}}\hat{H_1}\ket{\psi_n^{(0)}} 
		= E_n^{(0)}\underbrace{\bra{\psi_n^{(0)}}\ket{\psi_n^{(1)}}}_{=0}+E_n^{(1)}\underbrace{
		\bra{\psi_n^{(0)}}\ket{\psi_n^{(0)}}}_{=1}
		\end{equation}
		On déduit directement que 
		\begin{equation}
		E_n^{(1)} = \bra{\psi_n^{(0)}}\hat{H_1}\ket{\psi_n^{(0)}}
		\end{equation}
		Ce résultat peut être écrit de façon plus générale
		\begin{equation}
		\begin{array}{ll}
		E_n &= E_n^{(0)}+\lambda E_n^{(1)}+\mathcal{O}(\lambda^2)\\
		&= E_n^{(0)}+\lambda \bra{\psi_n^{(0)}}\hat{H_1}\ket{\psi_n^{(0)}}+\mathcal{O}(\lambda^2)\\
		&= E_n^{(0)}+\ \bra{\psi_n^{(0)}}\underbrace{\lambda\hat{H_1}}_{\hat{W}}\ket{\psi_n
		^{(0)}}+\mathcal{O}(\lambda^2)\\		
		\end{array}
		\end{equation}
		On appelle
		\begin{equation}
		E_n^{(0)}+\ \bra{\psi_n^{(0)}}\hat{W}\ket{\psi_n	^{(0)}}+\mathcal{O}(\lambda^2)
		\end{equation}
		le \textit{déplacement de l'énergie} (au premier ordre)(en effet, rappelons que
		$\hat{W}=\hat{H}-\hat{H_0}$). Il s'agit de l'élément de matrice diagonal (valeur 
		moyenne) de la perturbation $\hat{W}$ dans l'état propre non-perturbé. Il s'agit 
		de la \textit{formule 	des perturbations}.\\		
		
		Nous allons maintenant regarder la perturbation au premier ordre de l'état 
		propre. Nous avons fait pour l'énergie, il faut faire de même pour l'état propre 
		correspondant.
		
		\subsubsection{État propre au premier ordre}
		La procédure est la même que pour l'énergie au premier ordre mais nous allons cette 
		fois refermer la première équation non triviale de \eqref{eq:9.6} avec 
		$\bra{\psi_k^{(0)}}$ où $k\neq n$. Le raisonnement est identique
		\begin{equation}				
		\underbrace{\bra{\psi_k^{(0)}}\hat{H_0}\ket{\psi_n^{(1)}}}_{E_k^{(0)}
		\bra{\psi_k^{(0)}}\ket{\psi_n^{(1)}}}+\bra{\psi_k^{(0)}}\hat{H_1}\ket{\psi_n^{(0)}} =
		E_n^{(0)}\bra{\psi_k^{(0)}}\ket{\psi_n^{(1)}}+E_n^{(1)}
		\underbrace{\bra{\psi_k^{(0)}}\ket{\psi_n^{(0)}}}_{=0\ \text{car }\perp}
		\end{equation}
		Comme précédemment, nous obtenons
		\begin{equation}
		\bra{\psi_k^{(0)}}\hat{H_1}\ket{\psi_n^{(0)}} = (E_n^{(0)}-E_k^{(0)})
		\bra{\psi_k^{(0)}}\ket{\psi_n^{(1)}}
		\end{equation}
		On peut développer $\ket{\psi_n^{(1)}}$ dans la base des états propre de
		l'état non perturbé
		\begin{equation}
		\ket{\psi_n^{(1)}} = \sum_k \ket{\psi_k^{(0)}}\underbrace{\bra{\psi_k^{(0)}}\ket{\psi_n^{(1)}}}_{
		C_k}	
		\end{equation}
		où l'on reconnaît la relation de fermeture. Pour $k\neq n$, nous obtenons
		\begin{equation}
		\bra{\psi_k^{(0)}}\ket{\psi_n^{(1)}} = \dfrac{\bra{\psi_k^{(0)}}\hat{H_1}\ket{\psi_n^{(0)}}}{
		E_n^{(0)}-E_k^{(0)}} = C_k
		\end{equation}
		En substituant
		\begin{equation}
		\ket{\psi_n^{(1)}} = \sum_k \dfrac{\bra{\psi_k^{(0)}}\hat{H_1}\ket{\psi_n^{(0)}}}{
		E_n^{(0)}-E_k^{(0)}} \ket{\psi_k^{(0)}}
		\end{equation}
		L'expression
		\begin{equation}
		\ket{\psi_n} = 	\ket{\psi_n^{(0)}}+\lambda\ket{\psi_n^{(1)}}+\lambda^2\ket{
		\psi_n^{(2)}}+\mathcal{O}(\lambda^2)
		\end{equation}
		devient alors
		\begin{equation}
		\ket{\psi_n} = \underline{\ket{\psi_n^{(0)}} + \sum_k	\dfrac{\bra{\psi_k^{(0)}}\overbrace{
		\lambda\hat{H_1}}^{	\hat{W}}\ket{\psi_n^{(0)}}}{E_n^{(0)}-E_k^{(0)}} \ket{\psi_k^{(0)}}} + 
		\mathcal{O}(\lambda^2)
		\end{equation}	
		On obtient une relation "semblable" à celle trouvée pour l'énergie avec l'élément 
		de matrice 	diagonal de la perturbation (soit la valeur moyenne). Ici, ce qui change, 
		c'est que l'on regarde tous les autres états propres non-perturbés. On peut remarquer 
		que pour calculer la perturbation du $n^e$ état on somme sur tous les \textbf{autres} états. 
		Si l'énergie était dégénérée, nous aurions un zéro au dénominateur : cette expression n'est 
		valable que pour des énergies non-dégénérées.
		
			\subsubsection{Exemple 1 : oscillateur harmonique}
			Considérons un oscillateur harmonique d'une autre raideur. Il s'agit d'un cas académique 
			(ce problème peut être résolu en posant $m(1+\lambda)\omega^2=\Omega^2$) mais il est 
			intéressant pour illustrer les différences obtenues entre la résolution analytique 
			exacte et la méthode des perturbations.
			\begin{equation}
			\hat{H} = \hat{H_0}+\lambda\hat{H_1}\quad\text{ où }\left\{\begin{array}{ll}
			\hat{H_0} &= \frac{p^2}{2m}+\frac{1}{2}m\omega^2x^2\\
			\hat{H_1} &= \frac{1}{2}m\omega^2x^2\lambda^2
			\end{array}\right.
			\end{equation}
			L'hamiltonien qui nous intéresse est ainsi donné par
			\begin{equation}
			\hat{H} = \frac{p^2}{2m}+\frac{1}{2}m(1+\lambda)\omega^2x^2
			\end{equation}
			Nous connaissons l'énergie de l'état non-dégénéré
			\begin{equation}
			E_n^{(0)} = \left(\frac{1}{2}+n\right)\hbar\omega
			\end{equation}
			Calculons le déplacement de l'énergie au premier ordre
			\begin{equation}
			\Delta E_n^{(0)} = \bra{\psi_n^{(0)}}\lambda\hat{H_1}\ket{\psi_n^{(0)}} = \lambda \langle V
			\rangle_{\psi_n^{(0)}} = \frac{\lambda}{2}E_n^{(0)}
			\end{equation}
			où nous avons appliqué le théorème du Viriel ($\langle K \rangle = \langle V \rangle =
			\frac{E_n}{2})$. Nous obtenons comme énergie
			\begin{equation}
			E_n = \left(\frac{1}{2}+n\right)\hbar\omega+\frac{\lambda}{2}\left(\frac{1}{2}+n\right)\hbar
			\omega + \mathcal{O}(\lambda^2) = \left(\frac{1}{2}+n\right)\hbar\omega\left\{1+\frac{\lambda}{2}
			+\mathcal{O}(\lambda^2)\right\}
			\end{equation}
			La solution exacte est
			\begin{equation}
			E_n = \left(\frac{1}{2}+n\right)\hbar\Omega = \left(\frac{1}{2}+n\right)\hbar\sqrt{1+\lambda}
			\end{equation}
			Le résultat obtenu par la méthode des perturbation est bien le développement en série en 
			puissance de $\lambda$ ($\sqrt{1+\lambda}\approx 1+\frac{\lambda}{2}$ pour $\lambda\to0$).

			
			\subsubsection{Exemple 2 : potentiel anharmonique}
			Considérons un exemple moins trivial que le précédent en rajoutant un potentiel 
			anharmonique, ici un terme en $x^4$.
			\begin{equation}
			\hat{H} = \hat{H_0}+\lambda\hat{H_1}\quad\text{ où }\left\{\begin{array}{ll}
			\hat{H_0} &= \frac{p^2}{2m}+\frac{1}{2}m\omega^2x^2\\
			\hat{H_1} &= \frac{m^2\omega^3}{\hbar^2}x^4
			\end{array}\right.
			\end{equation}
			Le déplacement de l'énergie au premier ordre est donné par
			\begin{equation}
			\Delta E_n^{(0)} = \bra{\psi_n^{(0)}}\lambda\hat{H_1}\ket{\psi_n^{(0)}}  = \frac{\lambda m^3}{\hbar}
			\bra{\psi_n^{(0)}}x^4\ket{\psi_n^{(0)}}
			\end{equation}
			Pour calculer ce déplacement, on peut utiliser les opérateurs de montée et de descente pour 
			écrire $x$ en fonction de ceux-ci
			\begin{equation}
 			\hat{x} = \sqrt{\dfrac{\hbar}{2m\omega}}(\hat{a}+\hat{a}^\dagger)
			\end{equation}
			Il faut dès lors calculer
			\begin{equation}
			\bra{n}(\hat{a}+\hat{a}^\dagger)^4\ket{n}
			\end{equation}
			Les seuls termes contribuant sont ceux qui provoque deux montées et deux descentes (par exemple
			$\hat{a}\hat{a}\hat{a}^\dagger\hat{a}^\dagger$). Mais arrêtons-nous la pour cet exemple.\\
			
		
		
		Revenons donc à nos moutons. Nous avions
		\begin{equation}
		\ket{\psi_n} = \ket{\psi_n^{(0)}} + \sum_k	\ket{\psi_k^{(0)}}\dfrac{\bra{\psi_k^{(0)}}\lambda\hat{H_1}
		\ket{\psi_k^{(0)}}}{E_n^{(0)}-E_k^{(0)}}  + 
		\mathcal{O}(\lambda^2)
		\end{equation}
		Pour quel cela fonctionne correctement, il faut que le terme correctionnel soit 
		relativement "petit" (et donc $\lambda$ petit). L'état non-perturbé est pondéré par un 
		coefficient 1 que l'on peut voir comme un vecteur colonne dont seul le premier élément 
		est non nul et vaut l'unité. Lorsque l'on allume la perturbation, ce vecteur colonne 
		se voit modifier 
		\begin{equation}
		\left(\begin{array}{c}
		1\\
		0\\
		0\\
		\vdots
		\end{array}\right)\qquad\rightarrow\qquad\left(\begin{array}{c}
		1\\
		\vdots\\
		\vdots\\
		\vdots
		\end{array}\right)
		\end{equation}
		où les $\dots$ représentent des "nouveaux" coefficients de pondération. La 
		condition de validité de la méthode dit que tous ces "nouveaux" coefficient 
		doivent être petit en norme. Ceci signifie que si on regarde un élément 
		non diagonale :
		\begin{equation}
		\left|\bra{\psi_n^{(0)}}\hat{W}\ket{\psi_n^{(0)}}\right| \ll |E_n^{(0)}-
		E_k^{(0)}|,\qquad \forall k \neq n
		\end{equation}
		
		Ceci étant dit, intéressons-nous quelque peu au second ordre.
		
		\subsubsection{Énergie au second ordre}
		Le point de départ est toujours le même :  nous repartons de notre 
		système obtenu par identification et considérons cette fois le terme
		quadratique. En multipliant à gauche par $\bra{\psi_n^{(0)}}$ :
		\begin{equation}
		\underbrace{\bra{\psi_n^{(0)}}\hat{H_0}\ket{\psi_n^{(2)}}}_{E_n^{(0
		)}\bra{\psi_n^{(0)}}\ket{\psi_n^{(1)}}}+\bra{\psi_n^{(0)}}\hat{H_1}\ket{\psi_n^{(1)}} =
		 E_n^{(0	)}\bra{\psi_n^{(0)}}\ket{\psi_n^{(1)}}+E_n^{(1)}\underbrace{
		 \bra{\psi_n^{(0)}}\ket{\psi_n^{(1)}}}_{0}+E_n^{(2)}
		\underbrace{\bra{\psi_n^{(0)}}\ket{\psi_n^{(0)}}}_{1}
		\end{equation}
		On trouve alors
		\begin{equation}
		E_n^{(2)} = \bra{\psi_n^{(0)}}\hat{H_1}\ket{\psi_n^{(1)}}
		\end{equation}
		On remarque que pour déterminer l'énergie à l'ordre deux, il est nécessaire de 
		connaître l'énergie à l'ordre zéro et à l'ordre un : il faut travailler de proche
		en proche. En substituant l'expression pour l'ordre $\ket{\psi_n^{(1)}}$ : 
		\begin{equation}
		E_n^{(2)} = \bra{\psi_n^{(0)}}\hat{H_1}\ket{\psi_n^{(1)}} = \sum_k
		\bra{\psi_n^{(0)}}\ket{\psi_k^{(0)}}
		\dfrac{\bra{\psi_k^{(0)}}\hat{H_1}\ket{\psi_n^{(0)}}}{	E_n^{(0)}-E_k^{(0)}} =
		\sum_k \dfrac{\left|\bra{\psi_k^{(0)}}\hat{H_1}\ket{\psi_n^{(0)}}\right|^2}
		{E_n^{(0)}-E_k^{(0)}}
		\end{equation}
		L'expression
		\begin{equation}
		E_n = E_n^{(0)}+\lambda E_n^{(1)}+\lambda^2E_n^{(2)}+\dots
		\end{equation}
		devient alors
		\begin{equation}
		E_n = \underbrace{\bra{\psi_n^{(0)}}\overbrace{\lambda\hat{H_1}}^{\hat{W}}\ket{\psi_n^{(0)}}}_{\Delta 
		E_n^{(1)}}
		 + 
		\underbrace{\sum_k \dfrac{\left|\bra{\psi_k^{(0)}}\overbrace{\lambda\hat{H_1}}^{\hat{W}}\ket{\psi_n^{(0)}}\right|^2}
		{E_n^{(0)}-E_k^{(0)}}}_{\Delta 
		E_n^{(2)}} +
		\mathcal{O}(\lambda^3)
		\end{equation}	
		Cette expression ne fait plus qu'apparaître les énergie et états propres de 
		l'état non perturbé. Remarquons que dans la correction au 	deuxième ordre pour 
		l'état fondamental est négative : le numérateur est toujours positif mais vu 
		que $E_n^{(0)}$ est l'état fondamental, les $E_k^{(0)}$ seront toujours plus grand 
		de sorte que le dénominateur soit toujours négatif.
		\begin{equation}
		\Delta E_n^{(2)} \leq 0
		\end{equation}
		
	\subsection{Perturbation d'un niveau dégénéré (1$^{er}$ ordre)}
	Considérons maintenant le cas où les énergies sont dégénérées et notons $i$ la 
	dégénérescence. En tenant compte de celle-ci, nous pouvons naturellement écrire
	\begin{equation}
	\hat{H_0}\ket{\psi_{n,i}^{(0)}} = E_n^{(0)}\ket{\psi_{n,i}^{(0)}},\qquad i=1,\dots, g_n
	\end{equation}
	On pourrait avoir une levée de la dégénérescence en allumant la 
	perturbation. Inversement, si nous avons quatre niveaux distincts, il peut être possible 
	d'obtenir une quadruple dégénérescence en éteignant progressivement la perturbation, 
	$\lambda\to 0$.
	\begin{equation}
	\lim\limits_{\lambda\to 0} \ket{\psi_{n,i}} \equiv \ket{\phi_{n,i}} = \sum_{k=1}^{g_n} C_k 
	\ket{\psi_{n,k}^{(0)}}
	\end{equation}
	En effet, il serait faux de croire que $\ket{\psi_{n,i}}$ va forcément tendre vers 
	$\ket{\psi_{n,i}^{(0)}}$, il n'y a pas de raison de tendre vers les états de base 
	choisis arbitrairement. Il n'y a pas une correspondance mais on sait qu'ils vont 
	tous tendre vers des état qui appartiennent à un certain espace propre. En allumant 
	la perturbation, on pourrait ainsi avoir une levée de la dégénérescence.	
	\begin{equation}
	\left\{\begin{array}{ll}
	\ket{\psi_{n,i}} &= \ket{\phi_{n,i}^{(0)}} + \lambda\ket{\psi_{n,i}^{(1)}} + \dots\\
	E_{n,i} &= E_n^{(0)} + \lambda E_{n,i}^{(1)} + \dots
	\end{array}\right.
	\end{equation}
	Ce qui change c'est que ici, à cause de la dégénérescence, il faut calculer l'ordre
	zéro de la correction $\ket{\phi_{n,i}^{(0)}}$ (ce qui était avant trivial) en plus 
	des autres termes. Insérons ceci dans l'équation de Schrödinger
	\begin{equation}
	\bra{\psi_{n,j}^{(0)}}H_1\ket{\phi_{n,i}^{(0)}} = E_{n,i}^{(0)}
	\bra{\psi_{n,j}^{(0)}}\ket{\phi_{n,i}^{(0)}}
	\end{equation}
	Il faut maintenant exprimer les états $\ket{\phi_{n,i}}$ dans la base. Nous avions déjà 
	écris	
	\begin{equation}
	\ket{\phi_{n,i}^{(0)}} = \sum_{k=1}^{g_n} C_k \ket{\psi_{n,k}^{(0)}}
	\end{equation}
	Le but est d'ici de déterminer les coefficients $C_k$ jusqu'ici inconnus afin d'obtenir 
	les états propres de l'ordre 0, qui permettront le calcul	de l'ordre 1,\dots \\
	Fixons $n$ et $i$ et remplaçons notre dernière expression dans l'avant dernière
	\begin{equation}
	\sum_{k=1}^{g_n} \bra{\psi_{n,j}^{(0)}}\hat{H_1}\ket{\psi_{n,k}^{(0)}}.C_k = E_{n,i}^{(0)}
	\sum_{k=1}^{g_n} \underbrace{\bra{\psi_{n,j}^{(0)}}\ket{\psi_{n,k}^{(0)}}}_{\delta_{jk}}.C_k
	\end{equation}
	On retrouve bien l'élément de matrice de la perturbation. On peut ré-écrire cela de façon 
	plus compacte
	\begin{equation}
	\sum\left[\underbrace{\bra{\psi_{n,j}^{(0)}}\hat{H_1}\ket{\psi_{n,k}^{(0)}}}_{\mathcal{M}_{jk}}-
	E_{n,i}\underbrace{\delta_{jk}}_{\mathbb{1}}\right]C_k=0	
	\end{equation}	
	Intéressons-nous à la différence avec le cas précédent. A l'ordre 1, il suffisait de prendre 
	l'unique élément de matrice $\bra{\psi_n^{(0)}}\hat{H_1}\ket{\psi_n^{(0)}}$. Ici, à cause 
	de la dégénérescence, il faut exprimer ça sous la forme d'une matrice de perturbation dans 
	l'espace propre du cas non-dégénéré. C'est cette matrice, d'ordre $g_n\times g_n$ qu'il 
	faudra diagonaliser : les $n$ valeurs propres donneront les $n$ déplacements en énergie 
	dus aux perturbations\footnote{Le cas non dégénéré n'est que le cas particulier de la 
	matrice $1\times1$}.
	\begin{equation}
	(\mathcal{M}_{jk}-E\delta_{jk})C_k=0
	\end{equation}
	
	
	

\section{Méthode des variations}
	\subsection{Fonctionnelle énergie, cas du niveau fondamental}	
		\subsubsection{Principe}
		Considérons un certain $\ket{\phi}\in\mathcal{E}_H$ que l'on nomme 
		\textit{ket d'essai}. Celui-ci va dépendre d'un ou plusieurs paramètres $\alpha$. 
		Cette méthode se base sur une fonctionnelle particulière : a chaque $\ket{\phi}$ on va 
		faire correspondre une 	valeur réelle. Cette "fonctionnelle particulière" n'est autre 
		que la fonctionnelle énergie $W$:
		\begin{equation}
		\text{Fonctionnelle énergie : }\ W\left(\ket{\phi}\right) \equiv \dfrac{\bra{\phi}\hat{H}\ket{\phi}}{
		\bra{\phi}\ket{\phi}}
		\end{equation}
		On peut voir celle-ci comme une \textit{"énergie moyenne de $\phi$"}. En particulier, 
		regardons l'application de la fonctionnelle sur un état propre
		\begin{equation}
		\underline{W(\ket{\psi_n}) = E_n}
		\end{equation}
		
		\subsubsection{Niveau fondamental}
		Quel que soit le ket sur lequel on applique notre fonctionnelle, on obtient toujours 
		une borne supérieure à l'énergie du niveau fondamental
		\begin{equation}
		\begin{array}{lll}
		W(\ket{\phi}) &\geq E_0&\qquad\ \  \forall\ket{\phi}\\
		W(\ket{\phi}) &=0 &\qquad\Leftrightarrow \ket{\phi}=\ket{\psi_0}
		\end{array}
		\end{equation}
		Montrons le. Pour cela, exprimons $\ket{\phi}$ sous la forme d'une combinaison linéaire des 
		états $\ket{\psi_n}$
		\begin{equation}
		\ket{\phi} = \sum_n C_n\ket{\psi_n}
		\end{equation}
		On montre alors aisément le résultat annoncé
		\begin{equation}
		\left\{\begin{array}{ll}
		\bra{\phi}\hat{H}\ket{\phi} &=\DS \sum_{n,n'} C_{n'}^*C_n 
		\underbrace{\bra{\psi_{n'}}\hat{H}\ket{\psi_n}}_{\delta_{n,n'}} = \sum_n |C_n|^2 E_n\\
		\bra{\phi}\ket{\phi} &=\DS \sum_n |C_n|^2\\
		\rightarrow &\phantom{0 }\DS\  \sum_n |C_n|^2 E_n \geq E_0\sum_n |C_n|^2 \Leftrightarrow \sum_n 
		\underbrace{|C_n|^2}_{\geq 0,\forall n}\underbrace{(E_n-E_0)}_{\geq 0,\forall n} \geq 0
		\end{array}\right.
		\end{equation}
		On peut ainsi avoir une assez bonne approximation du fondamental mais en l'approchant 
		par le haut : il s'agit bien d'une borne supérieure. Pour trouver le niveau fondamental, 
		il faut minimiser $W(\ket{\phi_\alpha}) = W(\alpha)$ : le problème sera de minimiser 
		$\alpha$. Commençons par un exemple très simple et académique : l'oscillateur harmonique.
		
		\textsc{Exemple 1 : l'oscillateur harmonique}\ \\
		L'hamiltonien de l'O.H. est donné par
		\begin{equation}
		\hat H = \frac{p^2}{2m}+\frac{1}{2}m\omega^2x^2
		\end{equation}
		Il faut choisir une fonction d'onde de $x$ qui dépend de $\alpha$ en guise de fonction 
		d'essai. Considérons 
		\begin{equation}
		\phi(x) = \sqrt{\dfrac{2\alpha^3}{\pi}}\dfrac{1}{x^2+\alpha^2}
		\end{equation}				
		Nous devons calculer (pas la peine de prendre le conjugué, la fonction d'essai est 
		entièrement réelle)
		\begin{equation}
		\begin{array}{ll}
		\bra{\phi_\alpha}\hat{H} \ket{\phi_\alpha} &=\DS \int_{-\infty}^\infty \phi_\alpha(x)\left(
		-\dfrac{\hbar^2}{2m}\dfrac{\partial^2}{\partial x^2} + \dfrac{1}{2}m\omega^2x^2\right)
		\phi_\alpha(x)\ \text{dx}\\
		&=\DS \dfrac{\hbar^2}{4m\alpha^2}+\dfrac{1}{2}m\omega^2\alpha^2
		\end{array}
		\end{equation}
		Nous devons minimiser cette fonction en $\alpha$
		\begin{equation}
		E_{approx} = \min_\alpha W(\alpha)
		\end{equation}
		En posant $x=\alpha^2$, on trouve
		\begin{equation}
		\begin{array}{lll}
		&f(x) &=\DS \frac{\hbar^2}{4mx}+\frac{1}{2}m\omega x\\
		\rightarrow& 0 =\DS \frac{\partial f}{\partial x} = -\frac{\hbar^2}{4m x^2}+\frac{1}{2}m\omega
		\end{array}
		\end{equation}
		Nous trouvons
		\begin{equation}
		x^2 = \dfrac{\hbar^2}{2m^2\omega^2} \quad\Leftrightarrow\quad \alpha^2 = \dfrac{\hbar}{\sqrt{2}m
		\omega}
		\end{equation}
		En substituant ceci, on trouve l'énergie approximée
		\begin{equation}
		\begin{array}{ll}
		E_{approx} &=\DS \min_\alpha W(\alpha)\\
		&=\DS \dfrac{\hbar^2}{4m}\frac{m\omega\sqrt{2}}{\hbar}+\frac{1}{2}m\omega\frac{\hbar}{\sqrt{2}m
		\omega}\\
		&=\DS\dfrac{\hbar\omega}{\sqrt{2}}	
		\end{array}
		\end{equation}
		On vérifie bien que 
		\begin{equation}
		E_0=\frac{\hbar\omega}{2} < E_{approx} = \frac{\hbar\omega}{\sqrt{2}}
		\end{equation}\ \\
		
		Il est possible de retrouver le principe d'incertitude de Heisenberg en se basant sur la relation
		\begin{equation}
		\bra{\phi}\hat{H}\ket{\phi} \geq E_0
		\end{equation}		
		Partons du cas de l'oscillateur harmonique
		\begin{equation}
		\frac{1}{2m}\bra{\phi}p^2\ket{\phi} + \frac{1}{2}m\omega^2\bra{\phi}x^2\ket{\phi} \geq 
		\frac{\hbar\omega}{2}
		\end{equation}
		Multiplions cette expression par $2m$ :
		\begin{equation}
		(m\omega)^2\bra{\phi}x^2\ket{\phi} - \hbar(m\omega) + \bra{\phi}p^2\ket{\phi} \geq 0\qquad 
		\forall (m\omega)
		\end{equation}
		Pour que ceci soit vrai, il faut que le discriminant de cette expression soit négatif ou 
		nul 
		\begin{equation}
		\Delta = \hbar^2-4\bra{\phi}x^2\ket{\phi}\bra{\phi}p^2\ket{\phi}\geq 0\qquad\Leftrightarrow
		\qquad \langle x^2\rangle\langle p^2\rangle \geq \frac{\hbar^2}{4}
		\end{equation}
		Où encore
		\begin{equation}
		\Delta x.\Delta p \geq \dfrac{\hbar}{2}
		\end{equation}
		Ce résultat est logique si l'on se souvient que le principe d'incertitude d'Heisenberg 
		découle du fait que l'on ne peut descendre plus bas que le fondamental sans quoi on 
		connaîtrait la position et l'impulsion avec trop de précision. Ce n'est donc pas une 
		surprise de déduire une telle expression du principe des variations.\\
				
		
		\textsc{Rmq2 : Relation entre perturbation 1er ordre et variations}\\
		Pour les perturbations aux premier ordre
		\begin{equation}
		\begin{array}{lll}
		\hat H = \hat{H_0} + \underbrace{\lambda\hat{H_1}}_{\hat{W}},\qquad E_0^{1er} &= E_0^{(0)}+
		\lambda E_0^{(1)}\\
		&= \bra{\psi_0^{(0)}}H_0\ket{\psi_0^{(0)}} + \lambda\bra{\psi_0^{(0)}}H_1\ket{\psi_0^{(0)}}\\
		&= \bra{\psi_0^{(0)}}H_0+\lambda H_1\ket{\psi_0^{(0)}}\\
		&= \bra{\psi_0^{(0)}}H\ket{\psi_0^{(0)}} \\
		&\equiv W(\ket{\psi_0^{(0)}}) &\geq E_0
		\end{array}
		\end{equation}
		En effet, il suffit de considérer que $\ket{\psi_0^{(0)}}$ joue le rôle de la fonction 
		d'essai. On montre également, par la même occasion, que le premier ordre des perturbations 
		surestime toujours l'énergie fondamentale (ce qui est cohérent avec le fait que, pour le 
		fondamental, la correction au second ordre est négative).
		
		
	\subsection{Théorème de Ritz}
	\textit{La fonctionnelle $W(\ket{\phi})$ est stationnaire au voisinage des états propres de 
	$\hat H$}. Dès lors, pour toute variation infinitésimale, la fonctionnelle sera nulle ssi le ket 
	d'essai est un des états propres de $\hat H$ : 
	\begin{equation}
	\delta W = 0,\quad \forall\ket{\delta\phi}\qquad \Leftrightarrow\qquad \ket{\phi} = 
	\text{état propre de }\ \hat{H}\ket{\psi_n}
	\end{equation}
	On part d'un ket d'essai et on lui applique une variation infinitésimale, de même pour $W$
	\begin{equation}
	\begin{array}{ll}
	\ket{\phi} &\longrightarrow \ket{\phi}+\ket{\delta\phi}\\
	W &\longrightarrow W +\delta W
	\end{array}
	\end{equation}
	On va alors effectuer tous les calculs au premier ordre du calcul des variation. On peut 
	réécrire la fonctionnelle en faisant passer le dénominateur de l'autre côté
	\begin{equation}
	\bra{\phi}\ket{\phi}W = \bra{\phi}\hat{H}\ket{\phi}
	\end{equation}
	Appliquons la variation infinitésimale
	\begin{equation}
	\bra{\delta \phi}\hat{H}-W\ket{\phi} + \bra{\phi}\hat{H}-W\ket{\delta\phi}=\bra{\phi}\ket{\phi}
	\delta W
	\end{equation}
	\begin{enumerate}
	\item Si la fonction d'essai est un état propre
	\begin{equation}
	\left.\begin{array}{ll}
	\ket{\phi}&=\ket{\psi_n}\\
	W &= E_n
	\end{array}\right.\Rightarrow \bra{\delta \phi}\underbrace{\hat{H}-E_n\ket{\psi_n}}_{=0}+
	\underbrace{\bra{\psi_n}\hat{H}-E_n}_{=0}\ket{\delta\phi}=\bra{\psi_n}\ket{\psi_n}\delta W
	\end{equation}
	Dès lors
	\begin{equation}
	\delta \omega = 0,\qquad \forall\ket{\delta\phi}
	\end{equation}
	\item Si la fonctionnelle est stationnaire
	\begin{equation}
	\begin{array}{lll}
	\delta W =0,\quad \forall \ket{\delta\phi} &\Rightarrow \bra{\delta \phi}\hat{H}-W\ket{\phi}+
	\bra{\phi}\hat{H}-W\ket{\delta \phi} =0&\qquad \forall\ket{\delta \phi}\\
	&\Rightarrow -i\bra{\delta \phi}\hat{H}-W\ket{\phi}+
	i\bra{\phi}\hat{H}-W\ket{\delta \phi} =0&\qquad \forall i\ket{\delta \phi}
	\end{array}
	\end{equation}
	En effectuant la différence (les $i$ se simplifient, ils ne sont utiles que pour insérer un 
	signe négatif)
	\begin{equation}
	\bra{\delta \phi}\hat{H}-W\ket{\phi} = 0, \qquad\forall \delta \phi
	\end{equation}
	On en déduit les implications suivantes
	\begin{equation}
	\hat{H}\ket{\psi} = W\ket{\phi}\qquad\Rightarrow\qquad \ket{\phi} = \ket{\psi_n}\quad;\quad 
	W=E_n
	\end{equation}
	\end{enumerate}	
	En 1., nous avons montré qu'un état propre implique un état stationnaire alors que en 2. 
	nous avons montré que si la fonctionnelle est stationnaire, c'est qu'il s'agit d'un 
	état propre. La méthode des variations nous fourni assez bonne estimation de l'énergie, 
	le terme d'erreur étant quadratique.
	\begin{equation}
	W\left(\ket{\psi_n}+\epsilon\ket{\xi}\right) = E_n+\mathcal{O}(\epsilon^2),\qquad \epsilon\ll1
	\end{equation}
	où $\epsilon\ket{\xi}$ est un terme d'erreur. \\
	
	Supposons que l'on ai déjà obtenu l'état fondamental $\ket{\psi_0}, E_0$. Pour une fonction 
	d'essai $\ket{\phi}$ \textbf{orthogonale} à $\ket{\psi_0}$, comme $W(\ket{\phi})\geq E_1$, on 
	obtient le premier état excité en minimisant la fonctionnelle.\\
	\begin{proof}\ \\
	\begin{equation}
	\dfrac{\bra{\phi}\hat{H}\ket{\phi}}{\sum |C_n|^2 E_n}\geq 
	\dfrac{E_1\bra{\phi}\ket{\phi}}{E_1\sum |C_n|^2}
	\end{equation}
	Comme
	\begin{equation}
	\ket{\phi} = \sum_{n=1}^\infty C_n\ket{\psi_n},\qquad C_0=0
	\end{equation}
	On montre que
	\begin{equation}
	\sum |C_n|^2(E_n-E_1)\geq 0
	\end{equation}
	\end{proof}
	Comment faire en sorte que $\ket{\phi}\perp\ket{\phi_0}$ ? En pratique, nous aurons a 
	disposition d'autres observables telles que $[\hat{A},\hat{H}]=0$, il suffira de choisir 
	une famille de fonction d'essai dans les fonctions propres d'une autres valeurs propre 
	de $\hat{A}$. Par exemple, si $l=0$ donne le fondamental, on choisira une fonction d'essai 
	avec $l=1$ pour garantir l'orthogonalité.
	
	
	