
\section*{TP 5 : Moments cinétiques orbital et de spin, polarisation de la lumière}

\subsection*{Moement cinétique orbital}
\begin{itemize}
	\item On a $\vec{L} = (L_x,L_y,L_z)$ qui est obtenu par
	      \begin{equation}
	      	\vec{L} = \vec{r} \times \vec{p}
	      \end{equation}
	      		
	\item Commutation
	      \begin{equation}
	      	[L_i,L_j] = i\hbar L_k \qquad et \qquad[L^2,L_i] = 0
	      \end{equation}
	      	
	\item Fonctions propres communes à $L^2$ et $L_z$
	      \begin{equation}
	      	Y^m_l (\theta , \phi ) = c_{norm} \cdot e^{im \varphi } \cdot \sin ^{|m|}(\theta ) \cdot p^{l-|m|}(\cos \theta )
	      \end{equation}
	      avec $l$ et $m$ respectivement le \textbf{nombre quantique de moment cinétique orbital} et \textbf{nombre quantique magnétique} (projeté sur l'axe z)
	      		
	\item Avec $l$ naturel et $m \in [-l,l] \Rightarrow (2l+1)$ valeurs de $m$
	      \begin{equation}
	      	\left\{
	      	\begin{aligned}
	      		L^2 Y^m_l (\theta , \varphi ) & = \hbar ^2 l(l+1) Y^m_l (\theta , \varphi ) \\
	      		L_z Y^m_l (\theta , \varphi ) & = \hbar m Y^m_l (\theta , \varphi )         
	      	\end{aligned}
	      	\right.
	      \end{equation}
\end{itemize}

\subsection*{Moment cinétique de spin}

\begin{itemize}
	\item On a $\vec{S} = (S_x,S_y,S_z)$ où les $S_j$ sont les \textbf{matrices de Pauli} 
	      \begin{equation}
	      	S_i = \frac{\hbar}{2}\sigma _i
	      \end{equation}
	      avec
	      \begin{equation}
	      	\sigma _x = 
	      	\left(
	      	\begin{array}{cc}
	      		0 & 1 \\ 
	      		1 & 0 
	      	\end{array}
	      	\right)
	      	\qquad
	      	\sigma _y = 
	      	\left(
	      	\begin{array}{cc}
	      		0 & -i \\ 
	      		i & 0  
	      	\end{array}
	      	\right)
	      	\qquad
	      	\sigma _x = 
	      	\left(
	      	\begin{array}{cc}
	      		1 & 0  \\ 
	      		0 & -1 
	      	\end{array}
	      	\right) 
	      \end{equation}
	      	
	\item Commutation
	      \begin{equation}
	      	[S_i,S_j] = i\hbar S_k \qquad et \qquad[S^2,S_i] = 0
	      \end{equation}
	      		
	\item Fonctions propres communes à $S^2$ et $S_z$
	      \begin{equation}
	      	\chi ^{m_s}_s \qquad (spineur)
	      \end{equation}
	      où $s$ est le \textbf{spin} et $m_s$ la projection du spin sur l'axe z. 
	      	
	\item pour l'électron, le protion et le neutron, le spin vaut $1/2$ et on les appelle \textbf{fermion}
	      	
	\item Avec $s = 1/2$ et $m_s \in [-s,s] \Rightarrow m_s = \pm 1/2 \Rightarrow(2s + 1)$ valeurs
	      \begin{equation}
	      	\left\{
	      	\begin{aligned}
	      		S^2 \chi _{m_s} & = \hbar ^2 s(s+1) \chi _{m_s} = \frac{3}{4}\hbar ^2 \chi _{m_s} \\
	      		S_z \chi _{m_s} & = \hbar m_s \chi _{m_s} = \pm \frac{1}{2}\hbar \chi _{m_s}      
	      	\end{aligned}
	      	\right.
	      \end{equation}
	      		
	\item Photon $\Rightarrow s = 1$ et on les appelle \textbf{Boson}
	      	
	\item Etat \textbf{up} (à gauche) et état \textbf{down} (à droite)
	      \begin{equation}
	      	\chi _{\frac{1}{2}} = 
	      	\left(
	      	\begin{array}{c}
	      		1 \\ 
	      		0 
	      	\end{array}
	      	\right) 
	      	\qquad
	      	\chi _{-\frac{1}{2}} =
	      	\left(						
	      	\begin{array}{c}
	      		0 \\ 
	      		1 
	      	\end{array} 
	      	\right)		
	      \end{equation}
	      		
\end{itemize}