\setcounter{chapter}{2}
\chapter{Cristallographie}
\section{Réseaux cristallins}

Un cristal est un solide dans lequel on peut 
observer un arrangement périodique des ions le composant.\\

Pour pouvoir décrire correctement les cristaux nous avons 
besoin d'introduire la notion de réseaux de Bravais. Ceux
-ci spécifient la géométrie sous-jacente des cristaux.
On peut définir un réseau de Bravais de deux manières équivalentes :\\

\begin{itemize}
\item Un réseau de Brvais est un ensemble infini de points discrets
 tel que l'arrangement et l'orientation de ces points apparaît de
 la même manière quel que soit le point de l'ensemble depuis lequel
 on regarde celui-ci.

\item Un réseau de Bravais consiste en tous les points dont les 
vecteurs-position $\vec{R}$ sont donnés
par $\vec{R} = n_1 \vec{a_1} + n_2 \vec{a_2} + n_3 \vec{a_3}$, où
$\vec{a_1}, \vec{a_2},\vec{a_3}$ est un triplet quelconque de
vecteurs non-coplanaires et $n_1, n_2, n_3$ sont des nombres entiers.
\end{itemize}

\retenir{La définition d'un réseau de Bravais}

Les vecteurs apparaissnt dans la deuxième définition sont appelés
\textbf{vecteurs primitifs} et on dit qu'il \textbf{sous-tendent}
ou \textbf{génèrent} le réseau.
Il est important d'insister sur le fait que non seulement l'arrangement
mais aussi l'orientation du réseau doit rester identique en changeant de 
point de repère. Ainsi, des réseaux cubique simple, cubique faces centrées
ou cubique centré sont des réseau de Bravais alors qu'une structure en nid
d'abeilles ne l'est pas.

\begin{center}
\textbf{fig 4.1, 4.2, 4.3}
\end{center}

Les cristaux réels ne sont évidemment pas infinis comme les réseaux de Bravais
mais la majorité des points sont suffisamment loins de la surface pour ne pas être
affectés par son existence. Nous serons toutefois amenés à considérer un solide fini
afin de pouvoir construire des fonctions d'ondes normalisées sur un volume $V$.
Dans ce cas, les valeurs possibles de $n_1, n_2, n_3$ seront $0 \leq n_1 \leq N_1,
0 \leq n_2 \leq N_2, 0 \leq n_3 \leq N_3$ avec $N = N_1.N_2.N_3$ le nombre de point
du réseau fini.\\

Bien que plus précise mathématiquement, la deuxième définition d'un réseau de Bravais
possède deux limitations majeurs : il existe une infinté de choix non-équivalents
pour l'ensemble de vecteurs primitifs et il n'est pas toujours évident de montrer
l'existence de vecteurs primitifs satisfaisant la définition du réseau.

\begin{center}
\textbf{fig 4.5, 4.9}
\end{center}


Les points d'un réseau qui sont les plus proches d'un point donné sont appelés
ses \textbf{plus proches voisin}. Etant donné la périodicité du réseau, chaque
point a le même nombre de plus proches voisins et ce nombre appelé \textbf{
nombre de coordination} est une propriété du réseau. A titre informatif, un réseau
cubique simple a un nombre de coordination de 6, un réseau cubique centré en a un de
8 et un réseau cubique-face centré en a un de 12.\\

Une cellule (unitaire) primitive est un volume spatial qui couvre tout l'espace
sans trou ni recouvrement lorqu'il est translaté par tous les vecteurs du réseau 
de Bravais.

\section{Le réseau réciproque}