\chapter{Les lois universelles des milieux continus}
\section{Conservation de la masse}
\subsection{Forme locale et intégrale de l'équation de continuité}
Par hypothèse de continuité, si je prends un volume $v$ et que je le suis dans son
mouvement je retrouverai les mêmes particules : la masse définie par $M = \int_V \rho\ 
dV$ ne varie pas, sa dérivée matérielle est ainsi nulle 
\begin{equation}
M^\bullet = \int_V [\partial_0\rho + \partial_i(\rho v_i)]\ dV = 0\ \ \ \ \forall V
\end{equation}
En peut en déduire l'\textit{équation de continuité de la masse}\footnote{Les deux
équations sont équivalentes.} :
\begin{equation}
\left\{\begin{array}{ll}
\partial_0\rho + \partial_i(\rho v_i) &=0  \\
\dfrac{\partial\rho}{\partial t} + \text{div}(\rho \vec v) &= 0 
\end{array}\right.
\end{equation}
Si $\rho$ est dans la divergence, c'est parce qu'il n'est pas nécessairement constant, 
comme dans un gaz par exemple. En utilisant l'équation définissant la dérivée matérielle
$\rho^\bullet = \partial_0\rho+ v_k.\partial_k\rho$, on trouve :
\begin{equation}
\left\{\begin{array}{ll}
\rho^\bullet +\rho\partial_i v_i &= 0   \\
\rho^\bullet +\rho\text{div}(\vec v) &=0 
\end{array}\right.
\end{equation}

\subsection{Autre forme locale de l'équation de continuité}
Repartons de $M^\bullet = \int_V [\partial_0\rho + \partial_i(\rho v_i)]\ dV = 0$ et 
appliquons Gauss à la deuxième intégrale :
\begin{equation}
\int_V [\partial_0\rho]\ dV + \oint_S \rho v_in_i\ dS = 0
\end{equation}
\begin{itemize}
\item Dans le premier terme, si $\rho$ varie la masse fait de même : cela ne peut 
provenir que si de la masse entre par la surface. C'est la vitesse d'accroissement de
la masse dans le volume $V$ du à la variation de $t$.
\item Le deuxième terme est le débit massique sortant au travers de la surface fermée 
$S$ entourant le volume $V$. Le débit est sortant car la normale est orientée à l'extérieur 
du volume.
\end{itemize}
On en conclut que tout accroissement de masse ne peut provenir \textbf{que} d'un débit
de masse entrant par la surface $S$.


\subsection{Cas particuliers}
    \subsubsection{Fluide incompressible}
    C'est le cas ou une même masse est toujours contenue dans un même volume : la 
    masse étant constante, on aura pour définition de l'incompressibilitité :
    \begin{equation}
    \rho^\bullet = 0
    \end{equation}
    En repartant de l'équation de continuité : 
    \begin{equation}
    \underbrace{\rho^\bullet}_{0} +\rho\,\text{div}(\vec v) = 0 \Rightarrow
    \rho\,\text{div}(\vec v) = 0 \Rightarrow v_{i,i} = 0 \Rightarrow V_{ii} = 0
    \end{equation}
    Comme la divergence est $\partial_i v_i$, soit $v_{i,i}$ en composante de la vitesse et
    $V_{ii}$ en composante du tenseur (def.).
    
    
        \subsubsection{Écoulement permanent}
        L'écoulement est dit permanent lorsque, en un point géométrique, les grandeurs ne varient pas en
        fonction du temps ; $\frac{\partial}{\partial t} = 0$. C'est par exemple, le point de vue d'un 
        gendarme au bord de la route : le matin et le soir, il voit la même photo. En repartant de l'
        équation de continuité:
        \begin{equation}
        \partial_0\rho + \partial_i(\rho v_i) = 0 \Rightarrow \partial_i(\rho v_i) = 0 \Rightarrow
        \text{div}(\rho\vec v) = 0
        \end{equation}
        Ce n'est pas la même chose que la dérivée matérielle en suivant les particules dans leurs 
        mouvement. Comme on considère un point fixe, la dérivée est bien partielle.\\
        
        Reprenons la forme précédente de l'intégrale en choisissant comme volume un tube de courant : $
        \int_V [\partial_0\rho]\ dV + \oint_S \rho v_in_i\ dS = 0$. Dans notre cas :
        \begin{equation}
        \oint_S \rho v_in_i\ dS = 0
        \end{equation}
        En séparant mon intégrale sur un contour fermé (où $n',n''$ sont tous deux sortants) :
        \begin{equation}
        \int_{S'} \rho'\vec v'.\vec n'\ dS' + \int_{S''} \rho''\vec v''.\vec n''\ dS'' = 0
        \end{equation}
        ou encore :
        \begin{equation}
        \int_{S'} -\rho'\vec v_n'\ dS' + \int_{S''} \rho''\vec v_n''\ dS'' = 0
        \end{equation}
        La vitesse étant tangente aux bords de mon tube, pour les vitesses latérales on trouve : 
        \begin{equation}
        \vec v.\vec n =0
        \end{equation}
        
        
    \subsection{Conclusion}
    La loi de conservation de la masse n'est \textbf{pas} une loi universelle ($E = mc^2$ nous le
    montre) ; c'est la loi de conservation de l'énergie qui en est une ! Cependant, dans le cadre
    de ce cours, on stipulera qu'elle est conservée (Les slides 11 - 14 illustrent que la faible
    variation de masse est négligeable).
    
    

\section{Loi de la résultante cinétique}
Précédemment, nous avions obtenu la loi de la résultante cinétique :
\begin{equation}
\left[\int_v \rho v_i\ dV\right]^\bullet = \int_v f_i\ dV + \oint_S T_i^{(n)}\ dS
\end{equation}
On peut développer l'expression de la dérivée matérielle dans l'intégrale de volume pour avoir :
\begin{equation}
\int_v [\partial_0(\rho v_i) + \partial_j(v_j\rho v_i)]\ dV = \int_v f_i\ dV + \oint_S \tau_{ij}n_j\ dS
\end{equation}
En appliquant Gauss au deuxième terme du membre de gauche et en le faisant passer de l'autre côté 
de l'égalité, on obtient :
\begin{equation}
\int_v \partial_0(\rho v_i)\ dV = \int_v f_i\ dV + \oint_S (\tau_{ij} - \rho v_i v_j)n_j\ dS
\end{equation}

    
    \subsection{Calcul de l'action exercée par un fluide sur un obstacle}
    L'idée est de considérer un volume $V$ de fluide tel que la surface fermée autour de ce volume
    contient au moins la surface de contact et aussi autre chose. En faisant ceci, il ne faut pas
    connaître la forme de l'obstacle, ni même la répartition du tenseur des contraintes sur celui-ci 
    \begin{equation}
    A_i = \int_{\text{Surface de contact}} T_i^{(n)}\ dS
    \end{equation}
    où $n$ est la normale intérieure au volume du fluide ! Il faudra donc inverser les signes lors
    de l'écriture de l'expression suivante, la normale étant ici extérieure au volume du fluide :
    \begin{equation}
    \int_v \partial_0(\rho v_i)\ dV = \int_v f_i\ dV + \oint_S (\tau_{ij} - \rho v_i v_j)n_j\ dS
    \end{equation}
    Un exemple d'application de cette loi est donnée aux slides 17-18.




\section{Loi du moment cinétique}
On pourrait l'utiliser en pratique pour connaître les points d'applications qui ne sont pas données
par la loi de la résultante cinétique mais ce n'est pas utilisé en pratique. Notons simplement que
cela permet de démontrer que le tenseur des contraintes est symétrique.


\section{Théorème de l'énergie cinétique}
Ce théorème stipule que la dérivée de l'énergie cinétique est égale à la somme des puissances de
toutes les forces (externes et internes). Pour obtenir cette loi, il faut partir des équations du
mouvement, les multiplier scalairement par $v_i$ et \textbf{ensuite} intégrer au volume $V$ :
\begin{equation}
\rho v_i^\bullet = f_i + \partial_j\tau_{ij} \Rightarrow \rho v_i^\bullet v_i = f_iv_i + v_i 
\partial_j\tau_{ij} \Rightarrow \dfrac{1}{2}\int_V\rho(v_iv_i)^\bullet\ dV = \int_V v_if_i\ dV + 
\int_V v_i\partial_j\tau_{ij}\ dV
\end{equation}

\textit{J'ai la flemme de commencer une nouvelle page : cf. slide 29-30 pour obtenir une nouvelle
forme du théorème, en utilisant le théorème de Gauss.}
