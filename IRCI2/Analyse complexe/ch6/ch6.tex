\chapter{Résidus et p\^oles}

\section{Résidu et théorème des résidus}

	\subsection{Point singulier isolé}
	On dira que $z_0$ est un point isolé de $f(z)$ si :
	\begin{itemize}
	\item $f$ n'est pas analytique en $z_0$ mais analytique en certains points
	autour de $z_0$.
	\item Il existe un $\epsilon$ tel que $f$ est analytique dans $0<|z-z_0|<
	\epsilon$.
	\end{itemize}
	
	\exemple{\begin{enumerate}
	\item La fonction $f(z) = \frac{z^3}{z^2(z^2+5)}$ possède trois points 
	singuliers isolés : $z=0,\pm i\sqrt{5}$.
	\item Pour $Log\ z$, zéro est un point singulier, mais pas isolé.
	\end{enumerate}}
	
	\subsection{Notion de résidu}
	Soit $z_0$ un point singulier isolé. Ceci implique qu'il existe un $R_2$ tel 
	que $f$ est analytique pour tout $z$ dans $|z-z_0| < R_2$, où $f(z)$ est 
	représenté par le développement en série de Laurent :
	\begin{equation}
	f(z) = \sum_{n=0}^\infty a_n(z-z_0)^n + \frac{b_1}{z-z_0}+\frac{b_2}{(z-z_0)^2}
	+ \dots + + \frac{b_n}{(z-z_0)^n} + \dots\ \ \ \ (0<|z-z_0|<R_2)
	\end{equation}
	avec $b_n = \frac{1}{2\pi i}\oint_\mathcal{C}\frac{f(z)}{(z-z_0)^{-n+1}}dz\ \
	\ (n=1,2,\dots)$ avec en particulier $b_1 = \frac{1}{2\pi i}\oint_\mathcal{C}
	f(z) dz$.\\
	
	\retenir{\textbf{Définition}\\
	Le résidu de $f$ au point singulier isolé $z_0$ est le coefficient de $\frac{1}{z-z_0}$ 
	dans le développement en série de Laurent de $f(z)$ en puissance de $(z-z_0)$ pour 
	$0<|z-z_0|<R_2$.\\
	On le notera :
	\begin{equation}
	b_1 = \text{Res}_{z=z_0}\ f(z)
	\end{equation}}\ \\
	
	Ce résidu permet de simplifier fortement le calcul d'intégrale. Nous verrons 
	également qu'il est possible de le calculer sans déballer le dev. de Laurent.
	
	\newpage
	\subsection{Théorème des résidus}
	Mais comment calculer des intégrales ? Avec le résidu défini ci-dessus ça devient 
	plus simple : il suffit de faire la somme des résidus aux différents points 
	singuliers (avec un facteur de $2\pi i$):\\
	
	\theor{\textbf{INCLURE SCHEMA SLIDE 7/19}\\
	Soit $\mathcal{C}$ un chemin admissible fermé, simple, orienté dans le sens positif 
	et $f$ analytique en tout point de $\mathcal{C}$ ainsi qu'à l'intérieur de $\mathcal{
	C}$ sauf en un nombre fini de points singuliers isolés $z_k\ (k=1,2,\dots,n)$ à l'
	intérieur de $\mathcal{C}$ :
	\begin{equation}
	\oint_\mathcal{C} f(z) dz = 2\pi i \sum_{k=1}^n \text{Res}_{z=z_k}\ f(z)
	\end{equation}}
	
	\begin{proof}\ \\
	Découle directement du théorème de Cauchy-Goursat, la troisième proposition :
	\begin{equation}
	\oint_\mathcal{C}f(z)dz - \sum_{k=1}^n \oint_{C_k} f(z) dz = 0
	\end{equation}
	Ceci impliquant :
	\begin{equation}
	\oint_{C_k} f(z)dz = 2\pi i \text{Res}_{z=z_k}\ f(z)
	\end{equation}
	\end{proof}
	
	\exemple{\textbf{1}
	 \begin{equation}
	\oint_C\frac{\cos(z)}{z^3}dz=2\pi i\,\underset{z=0}{Res}\,\overbrace{(\cos(z)z^{-3})}^{f(z)}\quad(C\equiv|z|=1)
	\end{equation}
	En sachant que \begin{eqnarray}
	f(z) &=& \frac{1}{z^3}\left(1-\frac{z^2}{2!}+\frac{z^4}{4!}+\dots\right)\quad(0<|z|<\infty)\\
	&=& \frac{1}{z^3}-\frac{1}{2z}+\dots
	\end{eqnarray}
	le résidu (de cette fonction) étant le coefficient de $1/z\rightarrow$ résidu = $-1/2$\\
	Ainsi, l'intégrale vaut \begin{equation}
	\oint_C\frac{\cos(z)}{z^3}dz = -\pi i\quad(C\equiv|z|=1)
	\end{equation}}\\
	
	\exemple{\textbf{2}
	 \begin{equation}
	\oint_C\overbrace{\frac{5z-2}{z(z-1)}}^{f(z)}dz=2\pi i\,(\underset{z=1}{Res}\,f(z)+\underset{z=0}{Res}\,f(z))\quad(C\equiv|z|=2)
	\end{equation}
	En sachant que \begin{eqnarray}
	f(z) &=& \frac{5z-2}{z(z-1)} = \frac{2}{z}+\frac{3}{z-1}\\
	&=& \frac{2}{z}+\sum_0^{\infty}a_nz^n\quad(0<|z|<1)\\
	&=& \frac{3}{z-1}+\sum_0^{\infty}b_n(z-1)^n\quad(0<|z-1|<1)
	\end{eqnarray}
	le résidu vaut donc respectivement pour $z=0$ et $z=1$, 2 et 3.\\
	Nous obtenons donc \begin{equation}
	\oint_C\frac{5z-2}{z(z-1)}dz=10\pi i\end{equation}}
	
\section{Calcul du résidu}
	\subsection{Trois types de points singuliers isolés}
	On utilise la partie principale du développement en série de Laurent. Considérons 
	$z_0$ un point singulier isolé : $\exists R_2 > 0 :$
	\begin{equation}
	f(z) = \sum_{n=0}^\infty a_n(z-z_0)^n + \frac{b_1}{z-z_0} + \frac{b_2}{(z-z_0)^2}+\dots 
	\end{equation}		
	La partie principale de $f$ en $z_0$ est alors : 
	\begin{equation}
	\frac{b_1}{z-z_0} + \frac{b_2}{(z-z_0)^2}+\dots + \frac{b_n}{(z-z_0)^n}+\dots
	\end{equation}
	Ceci étant maintenant clair comme de l'eau de roche, voici ce qui justifie le titre 
	de cette sous-section :\\
		
	\retenir{\begin{enumerate}
	\item S'il existe $m$ tq :
	\begin{equation}
	b_m \neq 0\ \ \text{et}\ \ b_{m+1}=b_{m+2}=\dots=0
	\end{equation}
	alors $z_0$ est un \textbf{pôle} d'ordre $m$ de $f(z)$. Si $m=1$ le pôle est dit \textit{simple}.
	\item Si $b_n=0 \forall n$ alors $z_0$ est un \textbf{point singulier artificiel} de $f(z)$.
	\item Si la partie principe contient un nombre infini de terme, $z_0$ est un \textbf{point 
	singulier essentiel} de $f(z)$.
	\end{enumerate}}
	
	
	\exemple{\textbf{1}
	\begin{equation}
	e^{\frac{1}{z}}=\sum_0^{\infty}\frac{1}{z^n}\frac{1}{n!}\quad(0<|z|<\infty)
	\end{equation}
	Nous avons donc un nombre $\infty$ de terme en puissance négative de $z\rightarrow z=0$
	 est un point singulier \textit{essentiel}.\\
	Prenons un autre exemple :\begin{eqnarray}
	\frac{\sinh(z)}{z^4} &=& \frac{1}{z^4}\left(z+\frac{z^3}{3!}+\frac{z^5}{5!}+\dots\right)
	\quad(0<|z|<\infty)\\
	&=&\frac{1}{z^3}+\frac{1}{6z}+\frac{z}{5!}
	\end{eqnarray}
	Nous avons donc un pôle de multiplicité d'ordre 3 (degré le plus haut des puissance 
	négative de z)
	}\ \\
	
	\exemple{\textbf{2}
	\begin{eqnarray}
f(z) &=& \frac{1-\cos(z)}{z^2}\quad (0<|z|<\infty)\\
  &=&\frac{1-(1-\frac{z^2}{2!}+\frac{z^4}{4!}+\dots)}{z^2}\\
  &=& \frac{1}{2!}-\frac{z^2}{4!}+\dots\quad(0<|z|<\infty)
 \end{eqnarray}
 Associer $f(0)=\frac{1}{2}\,|\,\lim\limits_{z\rightarrow 0}f(z)=\frac{1}{2}$
	}\ \\\\
	
	\theor{\textsc{R1}\ \\
	Un point singulier isolé $z_0$ de $f$ est un pôle d'ordre $m$ si et seulement si
	\begin{equation}
	f(z) = \frac{\phi(z)}{(z-z_0)^m}
	\label{eq:R1}
	\end{equation}
	où $\phi(z)$ est analytique et non nulle en $z_0$. En outre :
	\begin{equation}
	\begin{array}{lll}
	Res_{z=z_0}\ f(z) &= \phi(z_0) &si\ m = 1\\
	Res_{z=z_0}\ f(z) &= \frac{\phi^{(m-1)}(z_0)}{(m-1)!} &si\ m\geq 2
	\end{array}
	\end{equation}}\ \\
	Ce théorème donne le résidu de façon rapide, pour autant que l'on sait factoriser 
	$f(z)$ de la façon suivante. Il faut bien qu'il y ai un facteur 1 au dénominateur 
	sinon \textit{slenderman} arrive.
	
	\begin{proof}\ \\
	\textbf{Sens direct}\\
	Ecrivons le développement de Taylor au voisinage de $z_0$ en supposant que $f(z)$ 
	est de la forme \autoref{eq:R1}  :
	\begin{equation}
	\phi(z) = \phi(z_0)+ \frac{\phi'(z_0)}{1!}(z-z_0) + \dots + \frac{\phi^{(m-1)}(z_0)}{(m-1)!}
	(z-z_0)^{m-1} + \sum_{n=m}^\infty  \frac{\phi^{(n)}(z_0)}{n!}(z-z_0)^n
	\end{equation}
	Pour un voisinage de $z_0, |z-z_0|<\epsilon$, on obtient un developpement qui possède 
	une série de Laurent. Comme $\phi$ est nulle, on peut conclure que le terme en $1/(z-z_0)$
	est bien un pôle d'ordre $m$ en $z_0$.
	\ \\
	\ \\
	\textbf{Réciproquement}\\
	Supposons que $z_0$ soit un pôle d'ordre $m$ de $f$, c'est à dire :
	\begin{equation}
	f(z) = \sum_{n=0}^\infty a_n(z-z_0)^n + \frac{b_1}{z-z_0}+\dots+\frac{b_m}{(z-z_0)^m}\ \
	\ \ (0<|z-z_0|<R_2), b_m\neq0
	\end{equation}
	Définissions une fonction $\phi(z)$ définie par :
	\begin{equation}
	\phi(z) = \left\{\begin{array}{ll}
	(z-z_0)^mf(z) &si\ z\neq z_0\\
	b_m &si\ z=z_0
	\end{array}\right.
	\end{equation}
	Cette fonction possède le développement en série suivant :
	\begin{equation}
	\phi(z) = b_m + b_{m-1}(z-z_0) + \dots +b_2(z-z_0)^{m-2} + b_1(z-z_0)^{m-1} + 
	\sum_{n=0}^\infty a_n(z-z_0)^{m+n}
	\end{equation}
	dans $|z-z_0| < R_2$. Comme le développement existe, cela implique que $\phi(z)$ est 
	analytique dans $|z-z_0|<R_2$ (en particulier en $z_0$) et que $\phi(z_0)=b_m\neq0$.
	\end{proof}
	
	\exemple{
	Choisissons la fonction ci-dessous et appliquons le théorème R1
	\begin{equation}
	f(z) = \frac{z^3+2z}{(2z-i)^3}=\frac{[\overbrace{(z^3+2z)/2^3}^{\phi(z)}]}{(z-\frac{i}{2})^3}
    \end{equation}
    Calculons le résidu (à l'aide du théorème R1) en $z=1/2$, l'un des points singuliers de la fonction
    \begin{eqnarray}
    \underset{z=\frac{1}{2}}{Res}\,f(z) &=& \left.\frac{\phi''(z)}{2!}\right|_{z=\frac{1}{2}}\\
    &=& \left.	\frac{6z/2^3}{2!}\right|_{z=\frac{1}{2}}\\
    &=& \frac{3i}{16}
    \end{eqnarray}
	}\ \\	
	
	Il est parfois compliquer de factoriser le dénominateur pour le mettre sous la forme
	du \textsc{Theorème R1}. Si c'est le cas, on préfèrera le théorème suivant\footnote{
	Démonstration pas à connaitre.} :\\
	
	\theor{\textsc{R2}\\
	Soient deux fonctions $p$ et $q$ analytiques en $z_0$. Si :
	\begin{equation}
	\begin{array}{ccc}
	p(z_0)\neq 0 & q(z_0) = 0 & q'(z_0) \neq 0
	\end{array}
	\end{equation}
	alors $z_0$ est un pôle simple du quotient $\frac{p(z)}{q(z)}$ et 
	\begin{equation}
	Res_{z=z_0} \frac{p(z)}{q(z)} = \frac{p(z_0)}{q'(z_0)}
	\end{equation}}

\section{Zéro de $f(z)$}
\textbf{Def :} Soit $f$ analytique en $z_0$. Si $f(z_0) = 0$, il existe un entier $m$ tel que
$f^{(m)}\neq 0$ et $f'(z_0) = \dots = f^{(m-1)}(z_0) = 0$ alors $f$ possède un 
zéro d'ordre $m$ en $z_0$.\\

\textbf{Proposition :} $f$ possède un zéro d'ordre $m$ si et seulement si il existe une fonction $g$ analytique et non nulle en $z_0$ telle que :
\begin{equation}
f(z) = (z-z_0)^mg(z)
\end{equation}

\textbf{Lien avec pôle :} Soient $p$ et $q$ analytiques en $z_0$ avec $p(z_0)\neq 0$. Si
$z_0$ est zéro d'ordre $m$ de $q$, alors $z_0$ est un pôle d'ordre $m$ de $\frac{p}{q}$.

\exemple{Calculons la valeurs de la fonction ci-dessous en $z=0$ ainsi que ses dérivées
\begin{eqnarray}
f(z) &=& z(e^z-1)\\
f(0) &=& 0\\
f'(z) &=& e^z-1+ze^z\rightarrow f'(0)=0\\
f''(z) &=& e^z+e^z+ze^z\rightarrow f''(0)=2\\
\end{eqnarray}
pour $m=2$ on a :
\begin{equation}
f(z)=z^2g(z)
\end{equation}
Ainsi, $g(z)$ vaut :
 \begin{eqnarray}
g(z) &=& \frac{e^z-1}{z}=\frac{f(z)}{z^2}\quad (0<|z|<\infty)\\
 &=&  \frac{\cancel{1+}z+\frac{z^2}{2!}+\frac{z^3}{3!}+\dots\cancel{-1}}{z}\\
 &=& \left\{\begin{array}{ll}
 1+\frac{z}{2!}+\frac{z^3}{3!}+\dots & (0<|z|<\infty)\\
 1 & (z=0)
 \end{array}\right.\end{eqnarray}
}
