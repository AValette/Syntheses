\chapter{Systèmes de particules identiques}
\section{Origine du problème}
Imaginons que nous avons deux boules de billard en déplacement, que nous laissons évoler dans le temps. 
Une modélisation mathématique pourrait être
\begin{equation}
\left\{\begin{array}{ll}
\vec{r_1}(t) &= \vec{f}(t)\\
\vec{r_2}(t) &= \vec{g}(t)
\end{array}\right.
\end{equation}
Or, ceci n'est qu'une modélisation possible : la description reste identique si l'on permute les numéros. Ceci 
est vrai en physique classique, mais pas forcément dans le cadre de la mécanique quantique : lorsque l'on 
travaille avec des objets quantique, il faut traiter avec des paquets d'onde. Lorsque ceux-ci seront proche l'un 
de l'autre ils vont se recouvrir : les deux situations ci-dessous sont indiscernables
\begin{center}
Inclure figure cours 2, sous eq (13).
\end{center}


\subsection{Deux particules identiques sans interactions dans un O.H.}
Soit l'Hamiltonien de l'oscillateur harmonique à deux particules
\begin{equation}
\begin{array}{ll}
\hat{H} &\DS= \frac{p_1^2}{2m}+\frac{1}{2}m\omega^2\vec{r_1^2}+\frac{p_2^2}{2m}+\frac{1}{2}m\omega^2\vec{r_2^2}\vspace{2mm}\\
&= h^{(1)} + h^{(2)}
\end{array}
\end{equation}
L'énergie vaut 
\begin{equation}
E = (1+n_1+n_2)\hbar\omega,\qquad\ n_1,n_2\geq 0
\end{equation}

\subsubsection{État fondamental $E_0$}
L'énergie de l'état fondamental est donnée par $E_0=\hbar\omega$ et la fonction propre par 
\begin{equation}
\psi_0(x_1,x_2) = \phi_0(x_1)\phi_0(x_2)
\end{equation}

\subsubsection{Premier état excité $E_1$}
L'énergie et la fonction d'onde est cette fois donnée par
\begin{equation}
E_1 = 2\hbar \omega,\qquad \psi_1(x_1,x_2) = \phi_1(x_1)\phi_0(x_1)
\end{equation}
où $\phi_0$ désigne l'état fondamental. Nous pouvons avoir exactement l'opposé en permutant le rôle des deux 
particules ou encore en considérant une combinaison linéaire de ces deux situations
\begin{equation}
\begin{array}{ll}
\psi_1' &= \phi_0(x_1)\phi_1(x_2)\\
\psi_1'' &=\alpha \phi_1(x_1)\phi_0(x_2)+\beta \phi_0(x_1)\phi_1(x_2)
\end{array}
\end{equation}
Regardons ce qui se produit lorsque l'on applique l'Hamiltonien à cet état
\begin{equation}
\begin{array}{ll}
\DS(\hat{h}^{(1)}+\hat{h}^{(2)})(\alpha\phi_1\phi_0+\beta\phi_0\phi_1) &\DS= \alpha\frac{3\hbar \omega}{2}\phi_1\phi_0 + \beta
\frac{\hbar\omega}{2}\phi_0\phi_1 + \alpha\frac{\hbar\omega}{2}\phi_1\phi_0+\beta\frac{3\hbar\omega}{2}\phi_0\phi_1\vspace{2mm}\\ 
&\DS= 
2\hbar\omega(\alpha\phi_1\phi_0+\beta\phi_0\phi_1)
\end{array}
\end{equation}
Or
Nous pouvons ainsi interpreter $(\alpha\phi_1\phi_0+\beta\phi_0\phi_1)$ comme une fonction propre et $2\hbar\omega$ comme 
la fonction propre associée. Nous somme face à la \textbf{dégénérescence d'échange} qui est causée par l'échange de particules 
identiques : les situations sont indiscernables.\\

Intéressons nous à la valeur moyenne du produit des deux observables positions
\begin{equation}
\begin{array}{ll}
\langle x_1x_2\rangle_\psi &=\DS (\alpha^*\bra{\phi_1}\bra{\phi_0}+\beta^*\bra{\phi_0}\bra{\phi_1})x_1x_2
((\alpha\ket{\phi_1}\ket{\phi_0}+\beta\ket{\phi_0}\ket{\phi_1}))\\
&=|\alpha|^2\bra{\phi_1}x\ket{\phi_1}\bra{\phi_0}x\ket{\phi_0} + \alpha^*\beta \bra{\phi_1}x\ket{\phi_0}\bra{\phi_0}x\ket{\phi_1} \\&\ \ \ + \alpha\beta^*\bra{\phi_0}x\ket{\phi_1}\bra{\phi_1}x\ket{\phi_0} + |\beta|^2\bra{\phi_0}x\ket{\phi_0}\bra{\phi_1}x\ket{\phi_1}
\end{array}
\end{equation}
Or $bra{\phi_1}x\ket{\phi_1} = bra{\phi_0}x\ket{\phi_0} = 0$ : il n'y a pas de raison que la valeur moyenne soit "plus à gauche" 
ou "plus à droite (cf. cours de J.M.Sparenberg)\footnote{La fonction d'onde est symétrique ou antisymétrique, mais en module il y a autant de chance que la particule soit à gauche ou à droite.}.\\

Pour continuer le calcul, introduisons
\begin{equation}
x = \sqrt{\frac{\hbar}{m\omega}}X\quad\Leftrightarrow X = \frac{a+a^\dagger}{\sqrt{2}}\quad \Rightarrow \quad \bra{1}X\ket{0} =
\frac{1}{\sqrt{2}}\bra{1}a\ket{0} + \frac{1}{\sqrt{2}}\bra{1}a^\dagger\ket{0} = \frac{1}{\sqrt{2}}
\end{equation}
Nous avons alors
\begin{equation}
\bra{\phi_1}x\ket{\phi_0} = \sqrt{\dfrac{\hbar}{2m\omega}}
\end{equation}
Dès lors
\begin{equation}
\langle x_1x_2\rangle_\psi = \alpha^*\beta \frac{\hbar}{2m\omega}+ \alpha\beta^*\frac{\hbar}{2m\omega} = \frac{\hbar}{
2m\omega}\Re(\alpha^*\beta) \quad\Rightarrow\quad ?
\end{equation}
Le choix de $\alpha$ et $\beta$ est totalement arbitraire et la solution actuelle ne dépend en rien de l'observable 
initiale : il manque quelque chose sans quoi il n'est possible de rien prédire. Ce "quelque chose manquant" n'est rien 
d'autre que le principe d'exclusion de Pauli, nous reviendrons donc plus tard sur ce développement.

\section{Opérateur d'échange}
\subsection{Propriétés, valeurs propres, opérateur (anti)-symétriseur}
\subsection{Cas du spin $1/2$}
Soit $\ket{k}_1$ une base de $\mathcal{H}_1$, $\ket{n}_2$ une base de $\mathcal{H}_2$ et $\mathcal{H} = 
\mathcal{H}_1\times\mathcal{H}_2$. Nous pouvons écrire $\ket{\psi}$ comme
\begin{equation}
\ket{\psi} = \sum_{k,n} C_{k,n} \ket{k}_1\ket{n}_1
\end{equation}
Il faut maintenant introduire notre \textbf{opérateur d'échange}. Par définition
\begin{equation}
\hat{P}_{12}\ket{k}_1\ket{n}_2 = \ket{n}_1\ket{k}_2
\end{equation}
Appliquons cet opérateur comme suggéré ci-dessous
\begin{equation}
\hat{P}_{12} \ket{\psi}_1\ket{\phi}_2=\ket{\phi}_1\ket{\psi}_2
\end{equation}
Où encore, par décomposition dans la base de $\mathcal{H}$
\begin{equation}
\begin{array}{ll}
\DS\hat{P}_{12}\left(\sum_k \alpha_k\ket{k}_1\right)\left(\sum_n \beta_n\ket{n}_2\right) &=\DS \sum_k\sum_n \alpha_k
\beta_n \ket{n}_1\ket{k}_2\\
&=\DS \sum_n\sum_k \beta_k\ \alpha_n \ket{k}_1\ket{n}_2 = \ket{\phi}_1\ket{\psi}_2
\end{array}
\end{equation}
A partir de la définition, on peut montrer que
\begin{equation}
\hat{P}_{12}= \sum_{k,n} \ket{n}\bra{k}\ \ \otimes\ \ \ket{k}\bra{n}
\end{equation}

\subsubsection{Propriétés}
Il existe plusieurs propriétés, nous allons ici en présenter quatre 
\begin{enumerate}
\item $\hat{P}_{12}=\hat{P}_{21}$
\item $\hat{P}_{12}^2 = \hat{\mathbb{1}}$ (on fait apparaître quatre delta de Kronecker : résultat attendu car une double permutation d'indice revient à ne rien faire).
\item $\hat{P}_{12}$ est unitaire : $\hat{P}_{12}\hat{P}_{12}^{-1} = \hat{\mathbb{1}}$
\item $\hat{P}_{12}^\dagger = \hat{P}_{12} = \hat{P}_{12}^{-1}$ (voir séance d'exercices)
\end{enumerate}

\subsubsection{Valeurs propres et opérateur (anti)-symétriseur}
L'opérateur d'échange possède deux valeurs propres : $+1$ (\textit{symétrique}) et $-1$ (\textit{antisymétrique}). On 
définit alors deux opérateurs : l'opérateur symétriseur et l'opérateur anti-symétriseur
\begin{equation}
\left\{\begin{array}{ll}
\hat{S} &= \frac{1}{2}\left(1+\hat{P}_{12}\right)\\
\hat{A} &= \frac{1}{2}\left(1-\hat{P}_{12}\right)
\end{array}\right.
\end{equation}
Ces deux opérateurs vérifie les propriétés suivantes
\begin{equation}
\hat{S}^2 =\hat{S},\qquad \hat{A}^2=\hat{A},\qquad \hat{S}\hat{A}=\hat{A}\hat{S} = 0, \qquad \hat{S}+\hat{A}=\hat{\mathbb{1}}
\end{equation}
Les noms de ces opérateurs se comprennent facilement avec la propriété énoncée ci-dessous
\begin{equation}
\left\{\begin{array}{ll}
\hat{P}_{12}\hat{S} = \hat{S}\hat{P}_{12} = \hat{S}\\
\hat{P}_{12}\hat{A} = \hat{A}\hat{P}_{12} = -\hat{A}
\end{array}\right.
\end{equation}
Si on applique $\hat{P}_{12}$ sur un état, on obtient
\begin{equation}
\begin{array}{ll}
\DS\hat{P}_{12}\underbrace{\hat{S}\ket{\psi}}_{\ket{\zeta}}&\DS=\underbrace{\hat{S}\ket{\psi}}_{\ket{\zeta}}\\
\DS\hat{P}_{12}\underbrace{\hat{A}\ket{\psi}}_{\ket{\xi}}&\DS=-\underbrace{\hat{A}\ket{\psi}}_{\ket{\xi}}
\end{array}
\end{equation}
où $\ket{\zeta}\subset$ sous-espace symétrique et $\ket{\xi}\subset$ sous-espace anti-symétrique.






















