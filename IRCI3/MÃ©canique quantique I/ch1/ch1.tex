\chapter{Notation de Dirac}
Inclure les notes de Terence
\section{Vecteurs d'état et espace de Hilbert}
Le \textit{vecteur d'état} se dénomme \textit{ket} et est noté :
\begin{equation}
\begin{array}{ll}
\ket{\psi} & \in \mathcal{E}\\
&\in \mathcal{E}_H
\end{array}
\end{equation}
où $\mathcal{E}$ est l'espace des états et $\mathcal{E}_H$ l'espace de Hilbert. 
Notons que $\mathcal{E} \subset \mathcal{E}_H$. Par abus de langage, nous 
désignerons souvent l'espace des états comme étant l'espace de Hilbert, ce qui 
n'est en toute rigueur pas exact ($\mathcal{E}_H$ contient des états non-physiques).
L'espace de Hilbert est un espace complet (si on définit une suite d'état, celle-ci 
convergera vers un état) muni d'un produit scalaire (défini à la section suivante).\\

Pourquoi définir un vecteur d'état ? En physique classique l'état d'un système 
ne pose pas de problèmes particuliers. A l'inverse, en physique quantique, la notion 
même pose déjà un problème, contraignant l'utilisation de vecteurs d'état. La raison 
physique de leur utilisation vient au principe d'incertitude d'Heisenberg. En effet, 
il nous est impossible de décrire la particule par le couple position/impulsion d'où 
la motivation à l'utilisation de ces vecteurs.\\

A la base de la physique, le \textbf{principe de superposition} nous dit que la 
combili (de coefficients complexes) de deux vecteurs d'états, soit deux kets, est 
de nouveau un ket, soit un état 100\% admissible.
\begin{equation}
\ket{\psi_1}, \ket{\psi_2} \in\mathcal{E},\qquad \ket{\lambda_1\psi_1+\lambda_2\psi_2} 
\equiv \lambda_1\ket{\psi_1}+\lambda_2\ket{\psi_2}\qquad \forall \lambda_1,\lambda_2\in
\mathbb{C}
\end{equation}
Il s'agit de la \textit{linéarité de la physique quantique} avec laquelle on peut, par 
exemple, décrire le phénomène d'interférences.


\section{Produit scalaire entre deux kets}
Le produit scalaire entre deux kets se note
\begin{equation}
\bra{\psi_2}\ket{\psi_1}
\end{equation}
\newpage
Les propriétés de bases de ce produit scalaires sont bien connues :
\begin{equation}
\begin{array}{llll}
\bullet & \bra{\psi}\ket{\psi} &= 0\\
\bullet & \bra{\psi_1}\ket{\psi_2} &= \bra{\psi_2}\ket{\psi_1}^*\\
\bullet & \bra{\psi}\ket{\lambda_1\psi_1+\lambda_2\psi_2} &= \lambda_1\bra{\psi_1}\ket{\psi_2}
+\lambda_2\bra{\psi_1}\ket{\psi_2}\quad \forall \lambda_i\in\mathbb{C}.\quad
& \text{Linéarité (à gauche)}\\
\bullet & \bra{\lambda_1\psi_1+\lambda_2\psi_2}\ket{\psi} &= \bra{\psi}\ket{\lambda_1
\psi_1+\lambda_2\psi_2}^* \quad & \text{Antilinéarité (à gauche)}\\
&  &= (\lambda_1\bra{\psi}\ket{\psi_1}+\lambda_2\bra{\psi}\ket{\psi_2})^*\\
&  &= \lambda_1^*\bra{\psi_1}\ket{\psi} +  \lambda_2^*\bra{\psi_2}\ket{\psi}\\
\bullet & \|\psi\| = \|\ket{\psi}\| = \sqrt{\bra{\psi}\ket{\psi}}>0
\end{array}
\end{equation}

Il est intéressant de s'intéresser à la \textit{"représentation"} d'un ket au sein 
d'un espace de Hilbert. Considérons l’exemple suivant (qui reviendra souvent).\\

\textsc{Exemple}\\
Considérons un espace de Hilbert de dimension $n$. Les vecteurs d'états, les ket, ne sont rien 
d'autres que des vecteurs colonnes dans cet espace de dimension $n$. Soit
\begin{equation}
\ket{u} = \left(\begin{array}{c}
u_1\\
u_2\\
\vdots\\
u_n
\end{array}\right),\qquad\qquad \ket{v} = \left(\begin{array}{c}
v_1\\
v_2\\
\vdots\\
v_n
\end{array}\right),\qquad u_1,v_i\in\mathbb{C}
\end{equation}
Le produit scalaire entre ces deux ket est donné par
\begin{equation}
\bra{v}\ket{u} = \sum_{i=1}^n v_i^*u_i = \underbrace{(v_1^*\ v_2^*\ \dots\ v_n^*)}_{(*)}
\left(\begin{array}{c}
u_1\\
u_2\\
\vdots\\
u_n
\end{array}\right)
\end{equation}
On va définir $(*)$ comme étant un \textit{"complémentaire au ket"}, $\bra{v}$ que l'on 
nomme \textit{bra}. Ce bra appartient à un espace dual, ce qui est le sujet de la 
section suivante.


\section{Espace dual $\mathcal{E}^*$, vecteur "bra"}
Le bra est une forme linéaire : c'est une application qui va depuis l'espace des état 
(ou de Hilbert, pas de différence dans ce cours) vers $\varphi(\psi)$, un nombre complexe.
\begin{equation}
\varphi : \ket{\psi}\in\mathcal{E} \leadsto \varphi(\ket{\psi})\ \in \mathbb{C}
\end{equation}
Cette forme linéaire fait correspondre à chaque état un nombre complexe. La superposition 
est également vérifiée d'où le "linéaire".
\begin{equation}
\varphi(\ket{\lambda_1\psi_1 + \lambda_2\psi_2}) = \lambda_1\varphi(\ket{\psi_1})+
\lambda_2\varphi(\ket{\psi_2})\qquad \forall \lambda_1,\lambda_2\in\mathbb{C}
\end{equation}
où $\varphi \in \mathcal{E}^*$.\\

Il semble dès lors intéressant d'introduire un nouvel "objet" : 
\begin{equation}
\left\{\begin{array}{ll}
\varphi \in \mathcal{E}^*\\
\bra{\varphi}
\end{array}\right.
\end{equation}
Il s'agit de l'ensemble de toutes les formes linéaires, ensemble qui forme un espace dual. 
L'intérêt réside dans un isomorphisme : on peut associer à chaque état de l'espace des états 
un bra de l'espace dual.\\

Ceci étant dit, il faut caractériser et montrer comment cette application agit sur les espaces. 
\begin{equation}
\forall \ket{\psi} \in \mathcal{E},\qquad \underline{\varphi(\ket{\psi}) =\bra{\varphi}\ket{\psi}}
\end{equation}
Cette application peut ainsi être écrite comme un produit scalaire. Il s'agit de la forme 
linéaire $\varphi$ qui s'applique à $\psi$ et qui donne un nombre complexe. Il existe une 
autre façon de voir ceci. On peut le voir comme le produit scalaire entre deux ket ou encore 
comme un bra (forme linéaire qui appliquée à un ket qui donnera un complexe) et un ket.\\
 
Comme précisé, il s'agit d'une forme \textbf{linéaire} :
\begin{equation}
\begin{array}{ll}
\varphi(\ket{\lambda_1\psi_1+\lambda_2\psi_2}) &= \bra{\varphi}\ket{\lambda_1\psi_1+\lambda_2\psi_2}\\
&= \lambda_1\bra{\varphi}\ket{\psi_1}+\lambda_2\bra{\varphi}\ket{\psi_2}\\
&= \lambda_1\varphi(\ket{\psi_1})+\lambda_2\varphi(\ket{\psi_2})
\end{array}
\end{equation}
L'espace dual est également un espace de Hilbert : toutes les propriétés de 
linéarité seront retrouvées. Ainsi, toute combili (complexe) de forme 
apparentent à $\mathcal{E}^*$ forme une troisième forme appartenant à 
$\mathcal{E}^*$.
\begin{equation}
\text{Si } \bra{\varphi_1},\bra{\varphi_2}\in\mathcal{E}^*, \text{ alors }\ 
\lambda_1\bra{\varphi_1} + \lambda_2\bra{\varphi_2} \in\mathcal{E}^*\qquad 
\forall \lambda_i\in\mathbb{C}
\end{equation}
On peut ainsi démontrer que $\mathcal{E}^*$ est un espace vectoriel.
\begin{equation}
\begin{array}{ll}
\forall \ket{\psi} : (\lambda_1\bra{\varphi_1} + \lambda_2\bra{\varphi_2})\ket{\psi} 
&= \lambda_1\bra{\varphi_1}\ket{\psi}+\lambda_2\bra{\varphi_2}\ket{\psi} \\
&= \lambda_1\bra{\psi}\ket{\varphi_1}^*+\lambda_2\bra{\psi}\ket{\varphi_2}^*\\
&= (\lambda_1^*\bra{\psi}\ket{\varphi_1}+\lambda_2^*\bra{\psi}\ket{\varphi_2})^*\\
&= \bra{\psi}\ket{\lambda_1^*\varphi_1+\lambda_2^*\varphi_2}^*\\
&= \bra{\lambda_1^*\varphi_1+\lambda_2^*\varphi_2}\ket{\psi}
\end{array}
\end{equation}
Nous avons donc bien un espace vectoriel (ce qui est clairement visualisable 
dans l'équation ci-dessous). La dernière relation applique un certain bra à 
n'importe que $\psi$. En terme de bra, on peut alors écrire
\begin{equation}
\underline{\lambda\bra{\varphi_1} + \lambda_2\bra{\varphi_2} = \bra{\lambda_1^*\varphi_1
+\lambda_2^*\varphi_2}}\qquad \lambda_1,\lambda_2\in\mathbb{C}
\end{equation}


On vient de voir qu'à n'importe quel bras je peux associer un ket. Il serait 
dès lors intéressant de trouver le ket correspondant à ce bra. Mais avant, on va définir 
la notion d'opérateur s'appliquant dans l'espace de Hilbert. \\

Il est possible de se représenter de façon plus précise ce qu'est un bra en 
se souvenant de l'exemple donné avec un espace de Hilbert de dimension $n$. Dans 
un tel espace, un bra n'est qu'un vecteur ligne complexe conjugué.

\subsection{Opérateurs linéaires (agissant dans $\mathcal{E}$)}
Un opérateur linéaire est une application qui fait correspondre un ket à un ket, à 
la différence de la forme qui fait correspondre un ket à un complexe.

\begin{equation}
\ket{\psi} \in \mathcal{E} \leadsto \hat{A}\ket{\psi} \in \mathcal{E}
\end{equation}
Il est coutume d'indiquer les opérateurs linéaires par un chapeau. La sainte 
superposition reste d'actualité :
\begin{equation}
\hat{A}\ket{\lambda_1\psi_1+\lambda_2\psi_2} = \lambda_1\hat{A}\ket{\psi_1}+
\lambda_2\hat{A}\ket{\psi_2}
\end{equation}
Pas mal de propriétés valent la peine d'être énoncées :
\begin{equation}
\begin{array}{llll}
\bullet & (\hat{A}+\hat{B})\ket{\psi} &= \hat{A}\ket{\psi}+\hat{B}+\ket{\psi}\\
\bullet & (\hat{A}.\hat{B})\ket{\psi} &= \hat{A}(\hat{B}\ket{\psi})\qquad & \text{ Opérateur 
produit $\hat{A}.\hat{B}$}
\end{array}
\end{equation}
Nous pouvons voir cet opérateur produit comme une notation efficace. Il ne faut 
cependant pas perdre à l'idée que, en toute généralité, $\hat{A}$ et $\hat{B}$ 
ne commutent pas. On définit alors le commutateur :
\begin{equation}
[\hat{A},\hat{B}] = \hat{A}\hat{B} - \hat{B}\hat{A} \neq 0
\end{equation}
Comme $\hat{A}$ et $\hat{B}$ sont des opérateurs, la différence des opérateurs 
est toujours un opérateur, le commutateur est bien un opérateur. Il jouit des 
propriétés suivantes :
\begin{equation}
\begin{array}{lll}
\bullet & [\hat{B},\hat{A}] &= -[\hat{A},\hat{B}]\\
\bullet & [\hat{A},\hat{B}+\hat{C}] &= [\hat{A},\hat{B}]+[\hat{A},\hat{C}]\\
\bullet & [\hat{A},\hat{B}.\hat{C}] &= \hat{B}.[\hat{A},\hat{C}]+[\hat{A},
\hat{B}].\hat{C}
\end{array}
\end{equation}
On peut montrer qu'un opérateur linéaire peut se représenter comme une 
matrice. Pour l'illustrer, reconsidérons notre précédent exemple.\\

\textsc{Exemple}\\
Soit un espace de Hilbert de dimension $n$. Soit
\begin{equation}
\ket{u} = \left(\begin{array}{c}
u_1\\
u_2\\
\vdots\\
u_n
\end{array}\right),\qquad \ket{v} = \hat{A}\ket{u}\quad ; \left(\begin{array}{c}
v_1\\
v_2\\
\vdots\\
v_n
\end{array}\right) = \underbrace{\left(\begin{array}{ccc}
a_{11} & \dots & a_{1n}\\
\vdots &\ddots &\vdots\\
a_{n1} & \dots & a_{nn}
\end{array}\right)}_{\hat{A}}\left(\begin{array}{c}
u_1\\
u_2\\
\vdots\\
u_n
\end{array}\right)
\end{equation}
De par cette représentation, on peut aisément comprendre que la non-commutation 
vient du fait que les différentes lignes et colonnes de $\hat{A}$ ne peuvent 
être commutées. Intéressons-nous aux éléments de la matrice de cet opérateur.



\section{"Élément de matrice" d'un opérateur $\hat{A}$}
Comme précédemment, définissons un nouvel "objet" :
\begin{equation}
\ket{\psi} \text{ et } \left\{\begin{array}{ll}
\ket{\varphi} &\in \mathcal{E}\\
\bra{\varphi} &\in \mathcal{E}^*
\end{array}\right., \quad \bra{\varphi}\hat{A}\ket{\psi} = \bra{\varphi}(
\hat{A}\ket{\psi})
\end{equation}
Les parenthèses permettent de voir ça "tel un produit scalaire". Revenons 
à notre précédent problème : quel est finalement ce ket ? Lors de l'écriture 
d'un élément de matrice, il serait intéressant de pouvoir le voir comme un 
opérateur appliqué à un ket. Une autre vision est celle d'un opérateur 
qui agit sur un bra, définissant un nouveau bra qui cette fois, agit sur 
$\psi$. Revenons à notre exemple.

\newpage
\textsc{Exemple}\\
Soit
\begin{equation}
\ket{u} = \left(\begin{array}{c}
u_1\\
u_2\\
\vdots\\
u_n
\end{array}\right), \qquad \bra{v} = (v_1^*\ v_2^*\ \dots\ v_n^*), \qquad 
\hat{A}\ket{u} = \left(\begin{array}{ccc}
a_{11} & \dots & a_{1n}\\
\vdots &\ddots &\vdots\\
a_{n1} & \dots & a_{nn}
\end{array}\right)\left(\begin{array}{c}
u_1\\
u_2\\
\vdots\\
u_n
\end{array}\right)
\end{equation}
Nous avons alors
\begin{equation}
\bra{v}\left(\hat{A}\ket{u}\right) = \underbrace{(v_1^*\ v_2^*\ \dots\ v_n^*)\left(\begin{array}{ccc}
a_{11} & \dots & a_{1n}\\
\vdots &\ddots &\vdots\\
a_{n1} & \dots & a_{nn}
\end{array}\right)}_{\bra{?}\ket{u}}\left(\begin{array}{c}
u_1\\
u_2\\
\vdots\\
u_n
\end{array}\right)
\end{equation}



\section{Opérateur adjoint}
A tout opérateur $\hat{A}$, on peut associer un nouvel opérateur noté $\hat{A}^\dagger$. 
Soit $\ket{\psi}$ :
\begin{equation}
\hat{A} \text{ agit dans }\mathcal{E} ; \ket{\psi'} = \hat{A}\ket{\psi}
\label{eq:16}
\end{equation}
Chaque ket est associé à un bra ; dans ce cas ci il s'agit de $\bra{\psi'}$ et
$\bra{\psi}$. Existe-t-il une relation entre ces bra ? Mais à quoi cet objet 
correspond-t-il ? Un bra est une forme linéaire, il faut déterminer comment 
agit $\psi'$ sur n'importe quel ket de l'espace.
\begin{equation}
\begin{array}{lll}
\forall \ket{\psi} \in \mathcal{E} : \psi'\left(\ket{\varphi}\right) &\equiv 
\bra{\psi'}\ket{\varphi} & \qquad \text{Prop. p.scal.}\\
&= \bra{\varphi}\ket{\psi'}^*\\
&= \bra{\varphi}\hat{A}\ket{\psi}^* & \qquad \text{Def. de }\psi', \text{ def. op. adj.}\\
&= \bra{\psi}\hat{A}^\dagger\ket{\varphi}& \qquad \text{(*)}
\end{array}
\end{equation}
Pour arriver à $(*)$, on peut remplacer $\hat{A}$ par son adjoint si l'on 
permute les termes et considère le complexe conjugué.
La conclusion de tous cela - modulo la définition de l'opérateur adjoint - est 
que l'on voit que l'on peut réécrire le $\bra{\psi'}$ en terme de $\bra{\psi}$.
\begin{equation}
\underline{\bra{\psi'} = \bra{\psi}\hat{A}^\dagger}
\end{equation}
Cette relation ressemble assez fortement à \autoref{eq:16} ou $\hat{A}\rightarrow
\hat{A}^\dagger$.
De façon générale on peut voir qu'un opérateur linéaire peut être entièrement 
caractérisé par ses éléments de matrice, exactement comme une matrice est 
caractérisée par tous ses éléments. Pour parvenir à ce résultat, nous avons 
utilisé la définition d'un opérateur adjoint :
\begin{equation}
\forall \ket{\psi} \text{ et } \ket{\varphi} \in \mathcal{E},\qquad 
\bra{\psi}\hat{A}^\dagger \ket{\varphi} = \bra{\psi}\hat{A}\ket{\varphi}^*
\end{equation}

\textsc{Exemple}\\
Comme toujours, prenons notre espace de Hilbert de dimension $n$.
\begin{equation}
\bra{v}\hat{A}\ket{u} = \underbrace{(v_1^*\ v_2^*\ \dots\ v_n^*)\left(\begin{array}{ccc}
a_{11} & \dots & a_{1n}\\
\vdots &\ddots &\vdots\\
a_{n1} & \dots & a_{nn}
\end{array}\right)}_{(*)}\left(\begin{array}{c}
u_1\\
u_2\\
\vdots\\
u_n
\end{array}\right)
\end{equation}
Le "but" est que $(*)$ devienne notre nouveau bra $(w_1\ w_2\ \dots\ w_n)$ :
\begin{equation}
\left(\begin{array}{c}
w_1\\
w_2\\
\vdots\\
w_n
\end{array}\right) = \underbrace{\left(\begin{array}{ccc}
a_{11}^* & \dots & a_{1n}^*\\
\vdots &\ddots &\vdots\\
a_{n1}^* & \dots & a_{nn}^*
\end{array}\right)}_{\hat{A}^\dagger}\left(\begin{array}{c}
v_1\\
v_2\\
\vdots\\
v_n
\end{array}\right)
\end{equation}
Pour obtenir le bra, nous avons réalisé une opération semblable à celles 
réalisées en algèbre linéaire, à savoir pris le complexe conjugé de la matrice 
conjugué après inversion et transposée du vecteur.\footnote{Mieux expliciter plz}.
L'opérateur adjoint n'est rien d'autre que de la matrice adjointe. Le fait 
de permuter les lignes et les colonnes ne faisait qu'inverser les bra et ket.
Il en découle des propriétés intéressantes :
\begin{multicols}{2}
\begin{itemize}
\item[$\bullet$] $\left(\hat{A}^\dagger\right)^\dagger = \hat{A}$
\item[$\bullet$] $\left(\lambda\hat{A}\right)^\dagger = \lambda^*\hat{A}$
\item[$\bullet$] $\left(\hat{A}+\hat{B}\right)^\dagger = \hat{A}^\dagger+\hat{B}^\dagger$
\item[$\bullet$] $\left(\hat{A}.\hat{B}\right) = \hat{B}^\dagger.\hat{A}^\dagger$
\end{itemize}
\end{multicols}
A titre d'exercice, démontrons la dernière propriété
\begin{equation}
\begin{array}{ll}
\bra{\psi}\left(\hat{A}.\hat{B}\right)^\dagger\ket{\varphi} &= \bra{\varphi}\hat{A}.
\hat{B}\ket{\psi}^*\\
&= \left(\left(\bra{\varphi}\ket{\hat{A}}\right)\left(\bra{\hat{B}}\ket{\psi}\right)\right)^*\\
&= (\bra{\psi}\hat{B}^\dagger)(\hat{A}^\dagger\ket{\varphi})\\
&= \bra{\psi}\hat{B}^\dagger\hat{A}^\dagger\ket{\varphi}
\end{array}
\end{equation}

\section{Opérateurs hermitiens/auto-adjoints}
Par définition
\begin{equation}
\hat{A} = \hat{A}^\dagger
\end{equation}
Dès lors
\begin{equation}
\begin{array}{ll}
\bra{\psi}\hat{A}\ket{\varphi} &= \bra{\varphi}\hat{A}\ket{\psi}^*\\
\bra{\psi}\hat{A}\ket{\psi} &= \bra{\psi}\hat{A}\ket{\psi}^* \underline{\in \mathbb{R}}
\end{array}
\end{equation}
Énonçons quelques propriétés intéressantes
\begin{equation}
\begin{array}{lll}
\forall \hat{A},\hat{B} \text{ hermitiens },\qquad & \hat{A}+\hat{B} & \text{hermitien}\\
& \hat{A}.\hat{B} & \text{hermitien ssi } [\hat{A},\hat{B}] = 0
\end{array}
\end{equation}
On peut justifier la dernière propriété de la façon suivante :
\begin{equation}
\begin{array}{lll}
(\hat{A}.\hat{B})^\dagger &= \hat{B}^\dagger.\hat{A}^\dagger \\
&= \hat{A}^\dagger.\hat{B}^\dagger & \text{vrai ssi } [\hat{A}^\dagger,\hat{B}^\dagger]=0=-
\underbrace{[\hat{A},\hat{B}]^\dagger}_{=0}\\
&= \hat{A}.\hat{B}
\end{array}
\end{equation}
Le produit position et impulsion n'est pas un opérateur hermitien (ces deux états ne 
commutent pas) ; ce n'est donc pas une quantité observable en physique quantique.\\




%COURS 2
%%%%%%%%%%%%%%%%%%%%%%%%%%%%%%%%%%%%%%%%%%%%%%%%%%%%%%%%%%%%%%%%%%%%%%%%%%%%%%%%%%%%%
Pour obtenir le complexe conjugué avec les notations de Dirac, il suffit de lire à 
l'envers pour obtenir ce que l'on souhaite :
\begin{equation}
\begin{array}{ll}
\hat{A}\ket{\psi} &\rightarrow \bra{\psi}\hat{A}^\dagger\\
\bra{\varphi}\hat{B}&\rightarrow \hat{B}^\dagger \ket{\varphi}\\
\bra{\psi}\hat{A}\ket{\varphi} &\rightarrow  \bra{\varphi}\hat{A}^\dagger\ket{\varphi} =
\bra{\psi}\hat{A}\ket{\varphi}^*
\end{array}
\end{equation}
Un autre exemple, un peu moins trivial est de considérer l'opérateur suivant
\begin{equation}
\ket{u}\bra{v} \rightarrow (\ket{u}\bra{v})^\dagger= \ket{v}\bra{u}
\end{equation}
Avec la notation $\ket{u}\bra{v}\ket{\varphi}$, on se rend 
compte qu'il s'agit bien d'un opérateur agissant sur l'état $\ket{\varphi}$. Afin de 
s'en rendre compte, développons ceci à titre d'application
\begin{equation}
\begin{array}{lll}
\bra{\varphi}\underline{\left(\ket{u}\bra{v}\right)^\dagger}\ket{\psi} &= (\bra{\psi}(\ket{u}\bra{v})
\ket{\varphi})^* & (*)\\
&=(\bra{\psi}\ket{u})^*(\bra{v}\ket{\varphi})^*\\
&=(\bra{u}\ket{\psi})(\bra{\varphi}\ket{v})\\
&=(\bra{\varphi}\ket{v})(\bra{u}\ket{\psi}) & \text{Commutativité}\\
&=\bra{\varphi}\underline{(\ket{v}\bra{u})}\ket{\psi}
\end{array}
\end{equation}
Il est possible de voir $(*)$ de deux façons différentes. On peut le comprendre comme 
un objet (opérateur) dont on fait l'élément de matrice entre $\psi$ et $\varphi$ (comme 
le suggère les parenthèses). On peut également le voir comme deux produits scalaire dont 
on fait le produit simple (en omettant cette fois-ci les parenthèses).


\subsection{Base Hilbertienne}
Une base hilbertienne est une base de l'espace de Hilbert. Il en existe deux particulières:
la base discrète de dimension finie et la base continue.

\subsubsection{Base discrète}
Nous parlons de l'espace de Hilbert et donc d'un espace des états, soit encore un ensemble 
de ket :
\begin{equation}
\left\{\ket{u_i}\right\}
\end{equation}
Ces bases sont orthonormées
\begin{equation}
\bra{u_i}\ket{u_j} = \delta_{ij}
\end{equation}
Le but d'une telle base est d'exprimer n'importe quel ket, n'importe quel état, comme une 
combili des vecteurs de cette base. En toute généralité, on peut écrire un ket comme une 
somme sur $i$ de coefficients multiplicatifs $C_i$ (qui joueront le rôle d'amplitude de 
probabilité, mais ils sont avant tout des coefficients de Fourier)
\begin{equation}
\ket{\psi} = \sum_i C_i\ket{u_i}
\label{eq:2.5}
\end{equation}
En remarquant que
\begin{equation}
\bra{u_j}\ket{\psi} = \sum_iC_i\underbrace{\bra{u_j}\ket{u_i}}_{\delta_{ij}} = C_j
\end{equation}
On peut réécrire \autoref{eq:2.5} :
\begin{equation}
\begin{array}{lll}
\ket{\psi} &= \sum_i \bra{u_i}\ket{\psi}\ket{u_i} & \text{Notations de Dirac}\\
&= \sum_i \ket{u_i}\bra{u_i}\ket{\psi} & \text{(Somme d') Op. lin. appliqué(e) à $\psi$}\\
&\underbrace{\left(\sum_i \ket{u_i}\bra{u_i}\right)}_{\mathbb{1}}\ket{\psi} & \forall\psi
\end{array}
\end{equation}

Il s'agit de la relation de fermeture. Dès que j'ai une base complète, la somme des
$\ket{u_i}\bra{u_i}$ donnera l'opérateur identité. On appelle alors la \textbf{relation 
de fermeture} :
\begin{equation}
\sum_i \ket{u_i}\bra{u_i} = \hat{\mathbb{1}}
\end{equation}
On peut alors définir l'opérateur \textit{projecteur} $\hat{P_i}$ :
\begin{equation}
\hat{P_i} = \ket{u_i}\bra{u_i}
\end{equation}
Cet opérateur possède deux propriétés remarquables. La première est qu'il est hermitien, 
c'est-à-dire $\hat{P_i} = \hat{P_i}^\dagger$.
\begin{equation}
\hat{P_i}^\dagger = (\ket{u_i}\bra{u_i})^\dagger = \ket{u_i}\bra{u_i} = P_i
\end{equation}
La seconde est qu'il est idempotent. Autrement dit, plus d'une application successive ne 
change rien au résultat obtenu : $\hat{P_i}^2 = \hat{P_i}$.
\begin{equation}
\begin{array}{ll}
\hat{P_i}^2 &= (\ket{u_i}\bra{u_i})(\ket{u_i}\bra{u_i})\\
&= \ket{u_i}\underbrace{\bra{u_i}\ket{u_I}}_{=1}\bra{u_i}\\
&= \hat{P_i}
\end{array}
\end{equation}
On interprète le projecteur comme on le ferrait dans l'espace euclidien\footnote{Inclure 
graphe}. La relation de fermeture peut ainsi être réécrite :
\begin{equation}
\sum_i \hat{P_i} = \hat{\mathbb{1}}
\end{equation}
Le développement suivi ici est valable pour toute base. Prenons
Ceci est valable pour toute base. Prenons l'exemple d'un ket
\begin{equation}
\text{ket }\ \ket{\psi} = \sum_i c_i\ket{u_i},\qquad c_i = \bra{u_i}\ket{\psi}
\end{equation}
On peut faire de même pour un bra. Pour définir un bra, il faut premièrement définir 
un ket puis prendre son élément dual.
\begin{equation}
\begin{array}{lll}
& \ket{\varphi} &= \sum_i b_i\ket{u_i},\qquad b_i = \bra{u_i}\ket{\varphi}\\
\text{bra } & \bra{\varphi} &= \sum_i b_i^* \bra{u_i}
\end{array}
\end{equation}
Comment faire pour exprimer un produit scalaire ? Il suffit de faire apparaître 
l'opérateur identité et jouer avec les notations de Dirac
\begin{equation}
\begin{array}{lll}
\bra{\varphi}\ket{\psi} &= \bra{\varphi}\mathbb{1}\ket{\psi}\\
&= \sum_i \underbrace{\bra{\varphi}\ket{u_i}}_{b_i^*}\underbrace{\bra{c_i}\ket{\psi}}_{c_i} &
\text{Relation de fermeture}\\
&= \sum_i b_i^*c_i & (*)
\end{array}
\end{equation}
Dans $(*)$, $b_i^*$ est un vecteur ligne et $c_i$ un vecteur colonne. Si l'espace de Hilbert 
est complet, il s'agit la d'un produit scalaire.\\
Cela fonctionne également pour un opérateur, en effectuant la même astuce mathématique:
\begin{equation}
\begin{array}{ll}
\hat{A} &= \hat{\mathbb{1}}\hat{A}\hat{\mathbb{1}}\\
&=\sum_{i,j} \ket{u_i}\underbrace{\bra{u_i}\hat{A}\ket{u_i}}_{A_{i,j}}\bra{u_j}
\end{array}
\end{equation}
Appliquons $\hat{A}$ sur un ket
\begin{equation}
\begin{array}{ll}
\ket{\psi'} &= \hat{A}\ket{\psi}\\
&\equiv \sum_i a_i\ket{u_i}
\end{array},\qquad \begin{array}{ll}
a_i &= \bra{u_i}\ket{\psi'}\\
&=\bra{u_i}\hat{A}_\mathbb{1}\ket{\psi} = \sum_j \underbrace{\bra{u_i}\hat{A}\ket{u_j}}_{A_{i,j}}
\underbrace{\bra{u_j}\ket{\psi}}_{c_j}
\end{array}
\end{equation}
où l'on a utilisé $\ket{\psi'} = \hat{A}\ket{\psi}$ et $a_i = \sum_jA_{ij}c_j\  "=" 
\left(\begin{array}{c}
\vdots\\
\vdots\\
\vdots
\end{array}\right) = \left(\begin{array}{ccc}
  & \dots &  \\
\vdots &\ddots &\vdots\\
  & \dots &  
\end{array}\right)\left(\begin{array}{c}
\vdots\\
\vdots\\
\vdots\\
\end{array}\right)$.


\subsubsection{Base continue}
On va ici partir d'une famille ou cette fois ci l'indice sera "continu". 
\begin{equation}
\{\ket{u_\alpha}\}\quad \alpha\in\mathbb{R}
\end{equation}
On peut comme précédemment fabriquer des états orthogonaux, ce qui va jouer 
le rôle orthonormalisation standard est le relation
\begin{equation}
\underline{\bra{u_\alpha}\ket{u_{\alpha'}} = \delta(\alpha-\alpha')}
\end{equation}
où $\delta$ est la fonction de Dirac. La subtilité est que $u_\alpha$ n'est 
pas toujours un état physique. Cependant, il peut toujours être utilisé pour 
décrire un état qui lui, est bien physique. Comme pour le cas continu, il 
est possible d'exprimer le ket dans la base. Les sommes seront ainsi 
remplacées par es intégrales et les coefficients de Fourier par une fonction 
jouant le même rôle.
\begin{equation}
\ket{\psi} = \int d\alpha\ C(\alpha)\ket{u_\alpha}
\label{eq:2.12}
\end{equation}
où les $C(\alpha)$ renseignent sur le poids. Il est possible, comme précédemment, 
de déterminer ceux-ci en multipliant ce ket par un autre élément de la base.
\begin{equation}
\bra{u_{\alpha'}}\ket{\psi} = \int d\alpha\ C(\alpha) \underbrace{
\bra{u_{\alpha'}}\ket{u_\alpha}}_{=\delta(\alpha-\alpha')} = C(\alpha')
\end{equation}
On peut alors ré-écrire \autoref{eq:2.12}
\begin{equation}
\begin{array}{ll}
\ket{\psi} &= \int d\alpha\ \bra{u_\alpha}\ket{\psi}\ket{u_\alpha}\\
&= \int d\alpha\ \ket{u_\alpha}\bra{u_\alpha}\ket{\psi}\\
&=\displaystyle\left(\int d\alpha\ \ket{u_\alpha}\bra{u_\alpha}\right)\ket{\psi}
\qquad \forall\psi
\end{array}
\end{equation}
Le terme entre parenthèse n'est, pas identification, rien d'autre que l'opérateur 
identité $\mathbb{1}$ que l'on peut également voir comme un opérateur projecteur
$\hat{P_\alpha}$. La relation de fermeture s'écrit alors
\begin{equation}
\int d\alpha\ \underbrace{\ket{u_\alpha}\bra{u_\alpha}}_{\hat{P_\alpha}} = \hat{
\mathbb{1}}
\end{equation}
Comme nous l'avons fait pour le cas de la base discrète, montrons comment écrire 
un bra, ket, opérateur linéaire, \dots
\begin{equation}
\begin{array}{llll}
\text{ket } & \ket{\psi} &= \int d\alpha\ c(\alpha)\ket{u_\alpha} & \text{ où } 
c(\alpha) = \bra{u_\alpha}\ket{\psi}\\
& \ket{\varphi} &= \int d\alpha\ b(\alpha)\ket{u_\alpha} & \text{ où } b(\alpha) 
= \bra{u_\alpha}\ket{\varphi}\\
\text{bra } & \bra{\varphi} &= \int d\alpha\ b^*(\alpha)\bra{u_\alpha}
\end{array}
\end{equation}
Voyons maintenant pour le produit scalaire
\begin{equation}
\begin{array}{ll}
\bra{\varphi}\ket{\psi} = \bra{\varphi}\mathbb{1}\ket{\psi} &= \int d\alpha\ 
\underbrace{\bra{\varphi}\ket{u_\alpha}}_{b^*(\alpha)}\underbrace{\bra{u_\alpha}
\ket{\psi}}_{c(\alpha)}\\
&= \int d\alpha\ b^*(\alpha)c(\alpha)
\end{array}
\end{equation}
Comme précédemment, par une analyse semblable pour les opérateurs. 
Définissons $\hat{A}$ :
\begin{equation}
\hat{A} = \mathbb{1}\hat{A}\mathbb{1} = \iint d\alpha\ d\alpha'\ \ket{u_\alpha}
\underbrace{\bra{u_\alpha}\hat{A}\ket{u_{\alpha'}}}_{A(\alpha,\alpha')}\bra{u_{\alpha'}}
\end{equation}
où $A(\alpha,\alpha')$ correspond aux éléments de matrice. \\
Pour finir, intéressons nous au cas de l'opérateur adjoint\footnote{TT : détaille plus 
ce qu'on essaye de montrer, c'est un peu pas clair :/}
\begin{equation}
\begin{array}{ll}
\ket{\psi'} = \hat{A}\ket{\psi} = \int d\alpha\ \overbrace{a(\alpha)}^{?}\ket{u_\alpha},\qquad a(\alpha) &= \bra{u_\alpha}\ket{\psi'} = \bra{u_\alpha}\hat{A}\ket{\psi}\\
&= \int d\alpha'\ \underbrace{\bra{u_\alpha}\hat{A}\ket{u_{\alpha'}}}_{A(\alpha,
\alpha')}\underbrace{\bra{u_{\alpha'}}\ket{\psi}}_{c(\alpha)}
\end{array}
\end{equation}
Nous nous intéressions à savoir ce qu'était $a(\alpha)$, c'est maintenant chose faite
\begin{equation}
\underline{a(\alpha) = \int d\alpha\ A(\alpha,\alpha')c(\alpha)}
\end{equation}
Dans le cas discret, nous avions $a_i = \sum_j A_{ij}c_j$. Ici le résultat est identique, 
mais sous forme intégrale. Cette expression illustre bien à quel point la notation de 
Dirac est compacte.


\subsection{Exemple de base (représentation) continue}
Base position : 
\begin{equation}
15
\end{equation}
N'importe quel ket pourra s'exprimer dans cette base position. Le relation 
fermerture pour la base position nous dis que l'intégration l'espace, en 
prenant le projecteur pour une position donnée, cela donne l'identité.
\begin{equation}
16
\end{equation}
Compte-tenu de cette relation, on peut exprimer les ket tel que
\begin{equation}
17
\end{equation}
Si je viens mettre la relation d'identité à droite de mon bras
\begin{equation}
18
\end{equation}
J'ai défini la définition ???
\begin{equation}
19
\end{equation}
Le delta n'est pas physique car il représente un état infiniment localisé. Or 
de telles solutions diverge. Malgré tout, même si ce n'est pas un état 
physique c'est bien pratique pour former une base.





\section{7. Observable}
Il s'agit d'un observateur linéaire hermitien $\hat{A}$ associa à une grandeur 
physique observable $A$. Si l'on a une grandeur physique observable, je 
sais que je peux lui associer un opérateur linéaire hermitien. 
\begin{equation}
20
\end{equation}
Une autre classification importante est selon le spectre. Le spectre est
l'ensemble de toutes les valeurs propres possibles $\{\lambda\}$. Si celui-ci 
est discret, on retrouvera un système lié (particule dans une boîte). On 
peut également avoir des spectres continu (système libre), utile dans la 
\textit{théorie des collisions/diffusion}.\\

On peut s'intéresser aux propriétés de bases
\begin{equation}
21
\end{equation}
Ceci va nous amener à la décomposition spectrale des opérateurs; on peut 
toujours les réprésenter dans une base. Unbase particulièrement intégrassante 
est celle des vecteurs propres de l'observable qui nous intéresse; 

\subsubsection{Décomposition spectrale de A}
L'idée c'est que j'ai les vecteurs propres et un second indice si les valeurs propres sont dégénérées
\begin{equation}
22
\end{equation}
On a tjs la relation d'orthonormalité
\begin{equation}
23
\end{equation}
On peut toujours trouver une base grace à GS.

La relation de fermeture dis que si je somme le sprojecteurs sur tous les 
vecteurs de ma base j'aurai l'identité. La relation de fermeture sera donnée 
par (pour chaque base on peut écrire cette rel)
\begin{equation}
24
\end{equation}

Théorème spectral
\begin{equation}
25
\end{equation}
Si s'agit d'un projecteur de rang $g_n$. On va montrer que ces $P_n$ forment
des projecteurs orthogonaux
\begin{equation}
26
\end{equation}
...


OBSERVABLE SPECTRE CONTINU
\begin{equation}
27
\end{equation}
Décomposition spectrale de $\hat{\vec{r}}$
\begin{equation}
28
\end{equation}
J'ai une intégrale d'opérateur qui me donne un opérateur (bon en fait 3 car
j'ai un triplet d'opérateur). La décomposition spectrale peut s'écrire comme 
ceci. Si au lieu de parler de l'opérateur r je parle de 1, il n'a que une 
seule valeur propre, je retrouve
\begin{equation}
29
\end{equation}
Soit la relation de fermeture, qui n'est rien d'autre que la décomposition 
spectrale d'un opérateur particulier.\\

La base impulsion est liée à la décomposition en valeur propre de l'omérateur impulsion (triplet d'opérateur) :
\begin{equation}
30
\end{equation}
COnnaitre l'impulsion implique une position étalée sur l'infinie d'énergie 
infinie, ce n'est pas physique. Je peux considérere la décomposition 
sepctrale
\begin{equation}
31
\end{equation}
On peut exprimer la fonction d'onde dans la ase impulsion
\begin{equation}
32
\end{equation}
Je peux faire de meme dans la base position.
























