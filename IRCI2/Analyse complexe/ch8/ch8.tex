\chapter{Notion de signal et de système}
\section{Notion de signal}
	\subsection{Définition}
	Un signal est une fonction d’une ou de plusieurs variables indépendantes 
	qui contient de l’information sur certains phénomènes. Les signaux sont 
	souvent des fonctions représentant des grandeurs physiques (pression, 
	température, tension électrique, ...).
	
		\subsubsection{Signal continu et signal discret}
		\begin{itemize}
		\item Signal continu : $x(t), t \in [a,b] | a,b\in\mathbb{R}$
		\item Signal discret : $x(k), k \in [a,b] | a,b\in\mathbb{Z}$
		\end{itemize}
	
		\subsubsection{Signaux continus de base : $\exp,\sin,\cos$}
		Par exemple $C\exp(at)$ où $C = |C|\exp(i\theta), a = r+i\omega_0$. 
		On peut également écrire :
		\begin{equation}
		\begin{array}{ll}
		C\exp(at) &= |C|e^{rt}\exp(i(\omega_0t+\theta))\\
		 &= |C|e^{rt}(\cos(\omega_0t+\theta) + i\sin(\omega_0t+\theta))
		\end{array}
		\end{equation}
		Pour $r<0$, la sinusoïde sera amortie (comme la réponse d'un circuit
		RLC, suspension automobile (ressort + dashpot\footnote{On s'en fiche, 
		mais j'aimais bien le mot \textit{dashpot}.})
		
	\subsection{Fonction d'Heaviside}
	La fonction d'Heaviside $\nu(t)$ est définie :
	\begin{equation}
	\nu(t) = \left\{\begin{array}{ll}
	0 & t<0\\
	1 & t>0	
	\end{array}\right.
	\label{eq:Heaviside}
	\end{equation}
	
	\subsection{Impulsion de Dirac}
	On remarque que la "fonction" d'Heaviside n'en est pas vraiment une (pas 
	très injectif, n'est ce pas ?) : on va pouvoir l'approximer grâce à l'
	impulsion de Dirac\footnote{Un traitement rigoureux requiert l'utilisation 
	de la théorie des fonctions généralisées.}.\\
	Le lien avec la fonction d'Heaviside est donné par : 
	\begin{equation}
	\nu(t) = \int_{-\infty}^t \delta(t)\ d\tau
	\end{equation}
	L'\textit{impulsion de Dirac} est la dérivée première de la fonction d'
	Heaviside : $\delta(t) = \frac{d\nu(t)}{dt}$. Si $t>0$, on retrouve bien 
	le fameux 1 défini en \autoref{eq:Heaviside}, sinon l'impulsion n'est pas 
	dans l'intervale d'intégration et l'on retrouvera bien 0.\\
	L'approximation de la fonction d'Heaviside est la suivante :
	\begin{equation}
	\nu_\epsilon(t) = \left\{\begin{array}{ll}
	0 & t<0\\
	\frac{t}{\epsilon}	 & 0\leq t \leq \epsilon\\
	1 & t \geq \epsilon
	\end{array}\right.
	\end{equation}
	La dérivée $\delta_\epsilon = \frac{d\nu_\epsilon}{dt}$ correspond à une 
	impulsion de durée $\epsilon$ et de surface unitaire. On retrouve bien 
	$\delta(t) = \lim\limits_{\epsilon\rightarrow0} \delta_\epsilon(t)$.\\
	Pour la réprésentation graphique, on considère le cas ou la base tend 
	vers zéro et donc la hauteur est très élevée. \textbf{Attention !} Il 
	est faux de dire $\delta(0)=1$.
	
	
\section{Notion de système}
	\subsection{Définition}
	Un système est un dispositif qui répond à un ou plusieurs signaux d’entrées 
	en engendrant un ou plusieurs signaux de sortie. Il peut être considéré 
	comme un ensemble de composants interconnectés par lesquels les signaux 
	d’entrée sont transformés.
	
	\subsection{Classe des systèmes linéaires permanents (SLP)}
	Un système décrit par l'équation $y(t) = \varphi(u(t))$ est linaire s'il 
	vérifie les propriétés de superposition et d’homothétie, c'est-à-dire :
	\begin{enumerate}
	\item \textit{Propriété de superposition}.\\
	Si $y_1(t) = \varphi(u_1(t)), y_2(t) = \varphi(u_2(t))$, alors :
	\begin{equation}
	y_1(t)+y_2(t) = \varphi(u_1(t) + u_2(t))
	\end{equation}
	\item \textit{Propriété d’homothétie}.\\
	\begin{equation}
	ay_1(t) = \varphi(au_1(t))\ \ \ \ a \in \mathbb{C}
	\end{equation}
	\end{enumerate}	
	Ces deux propriétés peuvent être reprises en une seule expression :
	\begin{equation}
	ay_1(t) + by_2(t) = \varphi(au_1(t) + bu_2(t))\ \ \ a,b\in\mathbb{C}
	\end{equation}		
	
	
		\subsubsection{Définition d'un système permanent (Time inveriant 
		system)}
		Si $y(t)$ est la sortie correspondant à $u(t)$, alors $y(t-t_0)$ 
		est la sortie correspondant à $u(t-t_0) \forall\ \ t_0$.
		
		\exemple{Soit $y(t) = u(2t)$ non permanent (time varying). Si l'on
		impose un changement d'échelle temporelle (compression d'un facteur 2 
		dans le temps) alors tout décalage temporel de l'entrée est comprimé 
		d'un facteur 2.}
	
	
	
	
	
	
	
	
	
	
	