\section*{TP 1 : Notations indicielles}

\subsection*{Notation indicielles}
\begin{multicols}{2}
	\begin{itemize}
		\item	Indice libre : 
		      \begin{itemize}
		      	\item N'apparaît qu'une seule fois
		      	\item Réprésente une composante
		      	\item Doit apparaître dans tous les termes
		      \end{itemize}
		      		
		\item Indice muet : 
		      \begin{itemize}
		      	\item Apparaît deux fois
		      	\item Représente une sommation
		      \end{itemize}
	\end{itemize}
\end{multicols}

\subsection*{Symbole de Kronecker}
\begin{itemize}
	\item A 2 indices :
	      \begin{equation}
	      	\delta _{ij} = 
	      	\left\{
	      	\begin{aligned}
	      		  & 1 \mbox{ si } i = j    \\
	      		  & 0 \mbox{ si } i \neq j 
	      	\end{aligned}
	      	\right.
	      \end{equation}
	      	
	\item A 3 indices :
	      \begin{equation}
	      	\delta _{ijk} = 
	      	\left\{
	      	\begin{aligned}
	      		  & 1 \mbox{ si permutation paire des indices}     \\
	      		- & 1 \mbox{ si permutation impaires des indices } \\
	      		  & 0 \mbox{ sinon }                               
	      	\end{aligned}
	      	\right.
	      \end{equation}
	      
	\item Formule d'expulsion 
	      \begin{equation}
	      	\delta _{ijk} \delta _{ipq} = \delta _{jp} \delta _{kq} - \delta _{jq} \delta _{kp}
	      \end{equation}
\end{itemize}

\subsection*{Tenseur d'ordre 2}
\begin{multicols}{2}
	\noindent Le tenseur des contraintes est défini comme
	\begin{equation}
		\overline{T}^{(n)} = \overline{T}_i n_i = T_{ij} \overline{1x}_j \cdot n_i
	\end{equation}
	
	Matriciellement cela revient à
	\begin{equation}
		\overline{T}^{(n)} = \overline{n}\overline{\overline{T}}
	\end{equation}
\end{multicols}
Ce qui nous donne en fait la composante en j du tenseur 
\begin{equation}
	T_j^{(n)} = T_{ij} n_i \rightarrow \overline{T}^{(n)}= T_j^{(n)} \overline{1x}_j  
\end{equation}


\begin{itemize}
	\item Propriétés : 
	      \begin{itemize}
	      	\item Changement d'axe : $T'_{pq} = \alpha _{pi} \alpha _{qj} T_{ij}$ 
	      	\item Tenseur symétrique : $T_{ij} = T_{ji}$
	      	\item Tenseur antisymétrique : $T_{ij} = - T_{ji}$		
	      \end{itemize}
\end{itemize}

\subsection*{Divers}
\begin{multicols}{2}
	\begin{itemize}
		\item Gradient : $\partial _i \varphi$
		\item Divergence : $\partial _i \varphi _i = \varphi _{i,i}$
		\item Rotationel : $\delta _{ijk} \partial _j \varphi _k$
		\item Laplacien : $\partial _{ii} \varphi$
		\item Produit scalaire : $a \cdot b = a_i b_i$
		\item Produit vectoriel : $a \times b = \delta _{ijk} a_j b_k$
	\end{itemize}
\end{multicols}