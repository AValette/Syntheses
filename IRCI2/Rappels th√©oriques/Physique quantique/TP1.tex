\section*{TP 1 : Relativité et lois de conservation}

\begin{itemize}
	\item Energie d'une particule libre 
		\begin{equation}
			E = E_0 + T \qquad \mbox{avec } E_0 = m_0 c^2
		\end{equation}
		\begin{itemize}
			\item $e^- : E_0 \approx 500 \, keV$
			\item $p : E_0 \approx 1\, GeV$, $m \approx 1,67 .10^{-27}$ et $r \approx 0,8 \, fm$ \\
				 $\Rightarrow \mathbf{IBA} : T = 230 \, MeV, \mathbf{LHC} : T = 7 \, TeV$
		\end{itemize}
		
	\item Relation importante dans le \textbf{domaine ultrareltiviste}
		\begin{equation}
			E = \sqrt{m_0^2c^4 + p^2c^2}
		\end{equation}
		Pour un \textbf{photon} : $E = pc$
	
	\item Pour les particules dans le \textbf{domaine non relativiste}, on fait les calculs avec
		\begin{equation}
			T = \frac{1}{2}mv^2 \qquad et \qquad \vec{p} = m \vec{v}
		\end{equation}
		
	\item Longueur d'onde de De Broglie 
		\begin{equation}
			\lambda = \frac{c}{\nu} = \frac{h}{p} \qquad \mbox{avec } \hbar = 1,05 . 10^{-34}\, J.s
		\end{equation}
	\item Pendant une réaction, il y a conservation de \textbf{Q} (charge), \textbf{E} (énergie) et $\vec{\mathbf{p}}$ (impulsion).
	
	\item La masse du système de particules libres \textbf{est inférieure} à la somme des masses des contituants. Ceci est dû à l'énergie de liaison (ex : $E_L(h) = \mathbf{1 \, Ryd = 13,6 \, eV}$)
\end{itemize}