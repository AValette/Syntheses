\chapter{Seconde quantification}
Considérons un état $\ket{\psi}$ et un opérateur $\hat{A}$. Nous avions précédemment discuté de l'image
de Schrödinger et d'Heisenberg mais dans les deux cas, nous avions à notre disposition un vecteur d'état 
et un observable.\\

Pour la seconde quantification, nous allons uniquement nous baser sur  les opérateurs linéaires, mais aussi 
les opérateurs de création et d'inhalation. Ce formalisme a plusieurs avantage: il prend compte 
"automatiquement" du fait que la fonction soit symétrique ou non, il s'utilise de la même façon que 
l'on travaille avec des bosons et des fermions mais aussi il est possible de faire varier le nombre 
de particules $N$ sans recommencer tous les calculs : $\hat{N}$ devient un observable (grosse différence 
par rapport à la première quantification). Notons que ce formalisme convient très bien au traitement 
des champs et des particules identiques.\\

Commençons par définir un ensemble de $N$ état
\begin{equation}
\{\ket{i}\} = \{\ket 1,\ket 2,\dots\}
\end{equation}
Adoptons une notation efficace :
\begin{equation}
\ket{i_1, i_2,\dots,i_N} \equiv \ket{i_1}_1\ket{i_2}_2\dots\ket{i_N}_N
\end{equation}
où $i_N$ correspond ici à la $N^e$ particule dans un état $N$. Nous formons donc une "liste" qui nous 
informe sur quelle particule occupe quel état. Pour former un état totalement symétrique
ou anti-symétrique, il suffit d'y appliquer $\hat{S}_\pm$ :
\begin{equation}
\hat{S}_\pm \ket{i_1, i_2,\dots,i_N} = \dfrac{1}{\sqrt{N!}}\sum(-1)^p\hat{p}
\end{equation}
Si on applique l'opérateur permutation à ceci
\begin{equation}
\hat{P}\hat{S}_\pm \ket{i_1, i_2,\dots,i_N} = (\pm 1)^p\dfrac{1}{\sqrt{N!}}\sum(-1)^p\hat{p}
\end{equation}
Ceci dépend de la parité : si l'on considère des fermions, nous aurons l'apparition d'un signe négatif. Cet 
objet à la propriété d'être totalement symétrique ou totalement antisymétrique.\\

Considérons que dans notre liste d'états occupés, le quatrième niveau est occupé par deux particules (la 
particule 3 et la particule 4)
\begin{center}
Inclure schéma cours 4, sous (21)
\end{center}
On défini le nombre d'occupation $n_i=m$ comme le nombre de particule dans un état $\ket{i}$. Le nombre de 
termes distincs est donné par
\begin{equation}
\dfrac{N!}{n_1!n_2!\dots}
\end{equation}
Ceci signifie que lorsque l'on écrit l'état totalement symétrique des bosons, le pré-facteur devient
\begin{equation}
\ket{\psi_{N,boson}} = \dfrac{1}{\sqrt{n_1!n_2!\dots}}\hat{S}_+\ket{i_1, i_2,\dots,i_N}
\end{equation}
et pour les fermions
\begin{equation}
\ket{\psi_{N,fermion}} = \dfrac{1}{\sqrt{n_1!n_2!\dots}}\hat{S}_-\ket{i_1, i_2,\dots,i_N}
\end{equation}
La subtilité est au niveau du nombre d'occupation qui ne peut valeur que 0 ou 1 pour les fermions ($n_i=0,1$) 
et tout pour les bosons ($n_i=0,1,2,\dots$).


\section{Système de $N$ bosons}
\subsection{Espace de Fock, opérateurs création/annihilation, opérateur nombre}
\subsubsection{Espace de Fock}
Toute l'idée est que lorsque l'on considère un système de $N$ bosons, nous n'allons pas le décrire comme
nous l'avions fait avec notre notation efficace. A la place, nous allons lister les orbitales et renseigner, 
pour chacune d'entre-elles, quelles particules les occupent.
\begin{equation}
\text{Etat de Fock }\ \equiv \ket{n_1,n_2,\dots} \equiv \dfrac{1}{\sqrt{n_1!\dots n_N!}}\hat{S}_+
\ket{i_1}_1\ket{i_2}_2\dots\ket{i_N}_N
\end{equation}
\danger\ A gauche, le numéro désigne les orbitales (il y en a une infinité, d'où les \dots). Par contre, le 
numéro du membre de droite correspond à la particules (que nous avons en nombre fini, $N$). Nous avons 
comme contrainte ($n_i=0,1,\dots$)
\begin{equation}
\sum_i n_i = N
\end{equation}
L'ensemble de tous les états de Fock forme l'espace de Fock : ceci forme une base complète d'espaces complètement 
symétrique. Cette base est donc complètement symétrique.\\

Considérons deux états de Fock, chacun caractérisé par une liste d'orbitale. Ces états sont bien orthogonaux
\begin{equation}
\bra{n_1,n_2,\dots}\ket{n_1',n_2',\dots} = \delta_{n_1,n_1'}\delta_{n_2,n_2'}\dots
\end{equation}
Nous avons également la relation de fermeture
\begin{equation}
\sum_{n_1,n_2} = \ket{n_1,n_2,\dots}\bra{n_1,n_2,\dots} =\hat{1}
\end{equation}

\subsection{Opérateur création}
L'opérateur création se note $\hat{a_i}^\dagger$ où la présence du dagger se justifie par le fait que son conjugué sera 
l'opérateur d'annihilation. Par définition
\begin{equation}
\hat{a}_i^\dagger \ket{n_1\dots n_i\dots} = \sqrt{n_i+1}\ket{n_1\dots n_i+1\dots}
\end{equation}
Regardons ce qu'est l'adjoint de cet opérateur afin d'obtenir l'expression de l’opérateur d'annihilation. Focalisons-nous, 
comme pour la définition, sur l'orbitale $i$
\begin{equation}
\bra{\dots n_i'\dots}\hat{a_i} = \sqrt{n_i'+1}\bra{\dots n_i'+1\dots}
\end{equation}
Fermons cette relation
\begin{equation}
\begin{array}{ll}
\bra{\dots n_i'\dots} \hat{a_i}\ket{\dots n_i\dots} &\DS= \sqrt{n_i'+1}\underbrace{\bra{\dots n_i'+1\dots} \hat{a_i}\ket{\dots
 n_i\dots}}_{
\delta_{n_i,n_{i+1}}}\\
&\DS= \sqrt{n_i}\ \delta_{n_i,n_{i+1}}
\end{array}
\end{equation}
Si l'on effectue une somme, celle-ci commencera à 1. Dès lors\footnote{Éclaircir}, $n_i'$ commence à $n_i-1$ 
\begin{equation}
\hat{a_i}\ket{\dots n_i\dots} = \sum_{n_i=0}\ket{\dots n_i'\dots}\underbrace{\bra{\dots n_i'\dots}\hat{a_i}\ket{\dots n_i\dots}}_{
si\ n_i=0 : \sqrt{n_i}\ \delta_{n_i, n_i'+1}}
\end{equation}
On appelle ceci l'opérateur d'annihilation. On peut compléter notre définition (\textbf{possible confusion}) :
\begin{equation}
\hat{a_i}\ket{\dots n_i\dots} = \left\{\begin{array}{ll}
\sqrt{n_i}\ket{\dots n_i-1\dots} & \text{ si } n_i\geq 1\\
0 & \text{ si } n_i = 0
\end{array}\right.
\end{equation}
Nos deux opérateurs sont alors défini comme
\begin{equation}
\left\{\begin{array}{ll}
\DS\hat{a}_i^\dagger \ket{n_1\dots n_i\dots} &= \sqrt{n_i+1}\ket{n_1\dots n_i+1\dots}\vspace{2mm}\\
\DS\hat{a}_i\ket{n_1\dots n_i\dots} &= \sqrt{n_i}\ket{\dots n_i-1\dots}\qquad\qquad (n_i\geq 1)
\end{array}\right.
\end{equation}

\subsection{Relation de commutations}
Trois relations de commutations sont particulièrement intéressantes
\begin{equation}
[\hat{a_i},\hat{a_j}]=0,\qquad\qquad[\hat{a_i}^\dagger,\hat{a_j}^\dagger]=0,\qquad\qquad [\hat{a_i},\hat{a_j}^\dagger]=\delta_{i,j}
\end{equation}
On voit donc que l'on peut créer ou annihiler deux particules dans l'ordre que l'on souhaite mais ce n'est pas le cas si 
on souhaite en créer une pour ensuite l'annihiler. Montrons la première commutation
\begin{equation}
\begin{array}{ll}
\DS\hat{a_i}\hat{a_j}\ket{\dots n_i\dots n_j\dots} &\DS= \hat{a_i}\sqrt{n_j}\ket{\dots n_i\dots n_j-1\dots}\\&\DS= \sqrt{n_i}\sqrt{n_j}\ket{\dots n_i-1\dots n_j-1\dots}\\
\DS&= \DS\hat{a_j}\hat{a_i}\ket{\dots n_i\dots n_j\dots} 
\end{array}
\end{equation}
La deuxième commutation est une conséquence directe de la première. En effet
\begin{equation}
[\hat{a_i},\hat{a_j}]^\dagger=[\hat{a_i}^\dagger,\hat{a_j}^\dagger]=0
\end{equation}
Intéressons-nous maintenant à la troisième relation. Dans le cas où $i\neq j$
\begin{equation}
\begin{array}{ll}
\hat{a_i}\hat{a_j}^\dagger\ket{\dots n_i\dots n_j\dots} &\DS = \hat{a_i}\sqrt{n_j+1}\ket{\dots n_i\dots n_j+1\dots}\\
&\DS = \sqrt{n_i+1}\sqrt{n_j+1}\ket{\dots n_i+1\dots n_j+1\dots}\\
&\DS = \hat{a_j}^\dagger\hat{a_i}\ket{\dots n_i\dots n_j+1\dots}
\end{array}
\end{equation}
Le seul cas intéressant est donc celui où $i=j$. Soit pour une orbitale
\begin{equation}
\begin{array}{ll}
(\hat{a_i}\hat{a_i}^\dagger -\hat{a_i}^\dagger\hat{a_i})\ket{\dots n_i\dots} &=\DS \hat{a_i}\sqrt{n_i+1}\ket{\dots n_i+1\dots}-
\hat{a_i}^\dagger\sqrt{n_i}\ket{\dots n_i-1\dots}\\
&=\DS \sqrt{n_i+1}\sqrt{n_i+1}\ket{\dots n_i\dots}-\sqrt{n_i}\sqrt{n_i}\ket{\dots n_i\dots}\\
&=\ket{\dots n_i\dots}
\end{array}
\end{equation}
d'où $\hat{a_i}\hat{a_i}^\dagger -\hat{a_i}^\dagger\hat{a_i} = \hat{1}$. Ce résultat est fondamental pour la seconde quantification.

\subsection{Écriture générale d'un état à $N$ bosons et des opérateurs à 1 et 2 corps}
Reprenons nos deux opérateurs
\begin{equation}
\left\{\begin{array}{ll}
\DS\hat{a}_i^\dagger \ket{n_1\dots n_i\dots} &= \sqrt{n_i+1}\ket{n_1\dots n_i+1\dots}\vspace{2mm}\\
\DS\hat{a}_i\ket{n_1\dots n_i\dots} &= \sqrt{n_i}\ket{\dots n_i-1\dots}\qquad\qquad (n_i\geq 1)
\end{array}\right.
\end{equation}
Ce qui nous intéresse avec ceux-ci, c'est qu'ils permettent de décrire tous les états complètement symétriques.
Définisions l'\textbf{état bosonique du vide} (bosonic vacuum state)
\begin{equation}
\ket{0} \equiv\ket{0,0,0,\dots}
\end{equation}
Dans le cas où nous avons une particule bosonique, l'application de l'opérateur création donne
\begin{equation}
\hat{a_i}^\dagger\ket{0} = \ket{0,0,\dots 0,1,0,\dots}
\end{equation}
où le 1 apparaît à la $i^e$ composante. Si nous avons maintenant deux particules bosoniques :
\begin{equation}
\begin{array}{ll}
\DS\hat{a_i}^\dagger\hat{a_j}^\dagger\ket{0} &\DS= \ket{0\dots 010\dots 010\dots} = \dfrac{1}{\sqrt{\dots}}\hat{S_+}
\ket{\dots}\ket{\dots}\dots\ket{\dots}\\
&\DS=\hat{a_j}^\dagger\hat{a_i}^\dagger\ket{0}
\end{array}
\end{equation}
où les deux 1 occupent respectivement la $i^e$ et la $j^e$ composante. Nous avons ici fait 
apparaître l'opérateur symétriser pour insister sur le fait que cette forme est complètement symétrique. 
Même si cela n'apparaît pas explicitement a symétrie est cachée derrière le fait que 
$\hat{a_i}^\dagger\hat{a_j}^\dagger\ket{0} = \hat{a_j}^\dagger\hat{a_i}^\dagger\ket{0}$.\\

Imaginons que deux bosons occupent le niveau $i$. On peut décrire un tel état comme
\begin{equation}
\frac{1}{\sqrt{2}}(\hat{a_i}^\dagger)^2\ket{0}=\underline{\ket{0\dots 020\dots}}
\end{equation}
Ce qui est important à remarquer c'est que $\frac{1}{\sqrt{2}}(\hat{a_i}^\dagger)^2$ 
caractérise complètement l'état souligné ci-dessus. Si on s'intéresse à une orbitale en 
particulier 
\begin{equation}
\left.\begin{array}{ll}
\hat{a}^\dagger\ket{n} &= \sqrt{n+1}\ket{n+1}\\
\hat{a}^\dagger\ket{n-1} &= \sqrt{n}\ket{n}
\end{array}\right\}\quad\rightarrow\quad \ket{n} =\frac{\hat{a}^\dagger}{\sqrt{n}}
\end{equation}
Voyons ce que donne l'application de cette nouvelle forme
\begin{equation}
\begin{array}{ll}
\ket 0&\\
\ket 1&\DS= \frac{\hat{a}^\dagger}{\sqrt{1}}\ket 0\vspace{2mm}\\
\ket 2&\DS= \frac{\hat{a}^\dagger}{\sqrt{2}}\frac{\hat{a}^\dagger}{\sqrt{1}}\ket 0\vspace{2mm}\\
\ket 3&\DS= \frac{\hat{a}^\dagger}{\sqrt{3}}\frac{\hat{a}^\dagger}{\sqrt{2}}\frac{\hat{a}^\dagger}{\sqrt{1}}\ket 0\\
\vdots&\\
\ket n &=\DS \dfrac{(\hat{a}^\dagger)^n}{\sqrt{n!}}\ket0
\end{array}
\end{equation}
Ceci nous permet de décrire complètement un état en fonction de ces opérateurs. Un état bosonique totalement général 
s'exprime alors comme
\begin{equation}
\ket{n_1, n_2, n_3,\dots} = \dfrac{1}{\sqrt{n_1!n_2!n_3!\dots}}(\hat{a_1}^\dagger)^{n_1}(\hat{a_2}^\dagger)^{n_2}
(\hat{a_3}^\dagger)^{n_3}\dots\ket0
\end{equation}
Ceci n'est donc rien d'autre qu'une façon très générale de noter l'état d'un système de $n$ bosons, totalement en 
fonction de l'opérateur de création. Pour terminer notre description, nous devons encore introduire un opérateur :
l'\textbf{opérateur nombre}. Celui-ci se défini
\begin{equation}
\hat{n_i}=\hat{a_i}^\dagger\hat{a_i}
\end{equation}
La mesure de cet observable donne (valeur propre) le nombre de bosons occupant l'orbitale $i$. Ceci est facile à 
vérifier
\begin{equation}
\hat{n_i}\ket{\dots n_i\dots} = \hat{a_i}^\dagger\sqrt{n_i}\ket{\dots n_i-1\dots}=
\sqrt{n_i}\sqrt{n_i}\ket{\dots n_i\dots} = n_i\ket{\dots n_i\dots}
\end{equation}
où $n_i$ est la valeur propre associée à la fonction propre $\ket{\dots n_i\dots}$. On défini de façon claire 
l'opérateur \textbf{nombre total}
\begin{equation}
\hat{N} = \sum_i n_i = \sum_i \hat{a_i}^\dagger\hat{a_i}
\end{equation}
Avec la première quantification, il fallait fixer dès le début le nombre de particules puis résoudre le 
problème avec ce nombre fixé. Ce n'est plus le cas ici : montrons-le en considérant un problème assez 
simple. Soit l'Hamiltonien de $N$ particules
\begin{equation}
\hat{H} = \sum_\alpha^N \hat{h}^{(\alpha)} 
\end{equation}
Pour chaque particule, nous pouvons diagonaliser :
\begin{equation}
\hat{h}\ket i = \varepsilon_i\ket{i}
\end{equation}
Avec la seconde quantification, cette Hamiltonien peut s'écrire (ceci sera justifié plus tard)
\begin{equation}
\hat{H} = \sum_i \varepsilon_i \hat{n_i}
\end{equation}
où $n_i$ est le nombre de particules correspondant à une énergie $\varepsilon_i$. Ceci est cohérent 
avec le fait que l'Hamiltonien est bien l'opérateur énergie totale.


\subsubsection{Expression générale pour un corps}
On définit un opérateur de corps que l'on nomme $\hat{T}$ et qui peut être une énergie cinétique, 
un Hamiltonien,\dots
\begin{equation}
\hat{T} = \hat{t}^{(1)}+\hat{t}^{(2)}+\dots+\hat{t}^{(N)}=\sum_{\alpha=1}^N t^{(\alpha)}\qquad 
\text{ e.g. } \quad t^{(\alpha)} = \left\{\begin{array}{l}
p_\alpha^2/2m\\
V(\vec{r}_\alpha)\\
\dots
\end{array}\right.
\end{equation}
Cet "opérateur un corps" est écrit dans une certaine base $\{\ket{i}\}$ : utilisons la relation de fermeture pour 
obtenir
\begin{equation}
\begin{array}{ll}
\hat{t} &\DS= \sum_{ij} \ket{i}\overbrace{\bra{i}\hat{t}\ket{j}}^{t_{ij}}\bra{i}\vspace{2mm}\\
&=\DS\sum_{ij} t_{ij}\ket{i}\bra{j}
\end{array}
\end{equation}
Donc, l'opérateur total s'écrit
\begin{equation}
\hat{T} = \sum_{\alpha=1}^N t_{ij}\ket{i}_\alpha\bra{j}
\end{equation}
La première sommation somme sur les particules et las seconde sur les différentes états (il  n'y a pas 
de valeurs au dessus du $\Sigma$ : ayant un nombre infini d'orbitale, c'est l'infini). Ce qui est 
fondamental, c'est que $t_{ij}$ n'est pas dépendant de $\alpha$.\\

Physiquement, tout opérateur agissant sur $N$ particules bosonique indiscernables doit être symétrique, cela ne 
dépend pas d'une dénomination arbitraire des particules : l'observable $\hat{T}$ doit être symétrique à 
l'échange des particules. Or, tous les $t^{(\alpha)}$ correspondent à un même opérateur pour toutes les 
particules du système. Par hypothèse que les particules sont identiques, il n'y a pas de sens que les éléments $t$ puissent être 
différent et donc pas de sens que les éléments de matrices puissent être différent d'une particule à une autre.\\

Nous pouvons dès lors le permuter :
\begin{equation}
\hat{T} = \sum_{ij} t_{ij}\overbrace{\sum_{\alpha=1}^N \ket{i}_\alpha\bra{i}}^{?}
\end{equation}
Il nous reste à déterminer comment écrire cet opérateur en terme de seconde quantification. Pour se faire, nous allons
regarder comment celui-ci agit sur un état de Fock. Si nous parvenons ensuite à généraliser à n'importe quel état de 
Fock, nous l'aurons complètement caractérisé 
\begin{equation}
\left( \sum_{\alpha=1}^N \ket{i}_\alpha\bra{j}\right)\ket{\dots n_i\dots n_j\dots}
\end{equation}
où $\ket{i}_\alpha\bra{j}$ agit bien entendu sur les deux états. Par définition d'un état de Fock, si 
$n\neq j$
\begin{equation}
\left( \sum_{\alpha=1}^N \ket{i}_\alpha\bra{j}\right)\sqrt{\dots n_i\dots n_j\dots}\hat{S}_+
\ket{i_1}_1\ket{i_2}_2\dots\ket{i_N}_N
\end{equation}
Ceci n'est qu'un "liste" des particules renseignant que la particule 1 occupe l'état $i_1,\dots$ 
L'opérateur $\ket{i}_\alpha\bra{j}$ est symétrique et nous savons que $\hat{S}_+$ commute avec t
tout opérateur symétrique. Dès lors
\begin{equation}
\sqrt{\dots n_i\dots n_j\dots}\hat{S}_+\left( \sum_{\alpha=1}^N \ket{i}_\alpha\bra{j}\right)
\ket{i_1}_1\ket{i_2}_2\dots\ket{i_N}_N
\end{equation}
Nous savons que sur toutes les particules, $n_j$ d'entre-elles occupent l'état $j$ : il y aura 
$n_j$ terme de contribution égale. Chaque terme de cette sommation correspond alors à un $\alpha$. 
Ceci nous donne lieux à plusieurs termes distincts mais il n'y a pas lieu de s'y intéresser : cela 
donnera le même état de Fock après application de $\hat{S}_+$ :
\begin{equation}
\dfrac{\hat{S}_+}{\sqrt{\dots n_i!\dots n_j!\dots}}\left[\underbrace{\ket{i_1'}\ket{i_2'}\dots \ket{i_N'}+ 
\ket{i_1''}\dots \ket{i_N''}+\dots}_{N\text{ termes}}\right]
\end{equation}
Chaque état entre crochet n'est pas symétrique, mais la forme finale, elle, l'est bien. On peut 
appliquer la définition d'un espace de Fock pour obtenir
\begin{equation}
\dfrac{1}{\sqrt{\dots n_i\dots n_j\dots}}n_j\sqrt{\dots (n_i+1)!\dots (n_j+1)!}\ket{\dots n_i+1\dots 
n_j-1\dots}
\end{equation}
où $\sqrt{\dots (n_i+1)!\dots (n_j+1)!}$ est le pré-facteur venant de la définition de l'état de Fock.
En simplifiant les factorielles
\begin{equation}
\sqrt{\dfrac{n_i+1}{n_j}}n_j\ket{\dots n_i+1\dots n_j-1\dots}
\end{equation}
Dès lors
\begin{equation}
\left( \sum_{\alpha=1}^N \ket{i}_\alpha\bra{j}\right)\ket{\dots n_i\dots n_j\dots}
 = \sqrt{n_i+1}\sqrt{n_j}\ket{\dots n_i+1\dots n_j-1\dots}
\end{equation}
Si l'on se souvient de comment agit l'opérateur $\hat{a}^\dagger$ sur un état de Fock, on peut dire 
que cette dernière expression est équivalente à
\begin{equation}
\hat{a_i}^\dagger\hat{a_j}^\dagger\ket{\dots n_i\dots n_j\dots}
\end{equation}
Nous avons donc prouvé que
\begin{equation}
\sum_{\alpha=1}^N \ket{i}_\alpha\bra{j} = \hat{a_i}^\dagger\hat{a_j}^\dagger
\end{equation}
Ce qui définit notre opérateur jusqu'ici inconnu. Regardons maintenant le cas ou $i=j$. 
Il nous faut calculer
\begin{equation}
\left(\underline{\sum_{\alpha=1}^N \ket{i}_\alpha\bra{i}}\right)\ket{\dots n_i\dots} = n_i\ket{\dots 
n_i\dots} = \underline{\hat{a}^\dagger\hat{a}}\ket{\dots n_i\dots}
\end{equation}
Cette opérateur nous liste donc le nombre de particules en prenant comme valeur $n_i$ ou 0.\\

Dans le cadre de la seconde quantification, notre opérateur s'écrit alors
\begin{equation}
\hat{T} = \sum_{ij} t_{ij}\hat{a}^\dagger\hat{a}\qquad\text{ où }\quad t_{ij} = \bra{i}\hat{t}\ket{j}
\end{equation}
Tentons maintenant de justifier notre précédent exemple. Nous avons à calculer l'élément de matrice 
suivant où $t_{ij} = h_{ij}$
\begin{equation}
h_{ij} = \bra{i}\hat{h}\ket{j} = \varepsilon_i\delta_{ij}
\end{equation}
Dans notre cas, $\hat{T}=\hat{H}$ :
\begin{equation}
\hat{H} = \sum_{ij} \varepsilon_i\delta_{ij} \hat{a_i}^\dagger\hat{a_i} = \sum_i \varepsilon_i\hat{a_i}^\dagger
\hat{a_i}
\end{equation}
Ce qui justifie le résultat précédemment annoncé.\\
\\

\subsubsection{Expression générale pour deux corps}
Tout ce que nous avons fait va se généraliser assez naturellement :
\begin{equation}
\hat{F} = \frac{1}{2}\sum_{\alpha < \beta}^N \hat{f}(\alpha,\beta)\qquad \text{ e.g. }\quad 
\hat{f}^{(\alpha,\beta)} = V(\vec{r_\alpha}-\vec{r_\beta}|)
\end{equation}
à l'aide de la relation de fermeture, on peut écrire (quadruple somme)
\begin{equation}
\hat{f} = \sum_{ijkl} \ket{ij}\underbrace{\bra{ij}\hat{f}\ket{kl}}_{f_{ij,kl}}\bra{kl}
\end{equation}
où
\begin{equation}
f_{ij,kl} = \iint dxdy\ \phi_i^*(x)\phi_j^*(x)\ f(x,y)\ \phi_k(x)\phi_l(y)
\end{equation}
Nous avons alors
\begin{equation}
\hat{F} =\frac{1}{2}\sum_{\alpha\neq\beta}^N\sum_{ijkl} f_{ij,kl}\left( \ket{i}_\alpha\bra{k}\otimes
\ket{j}_\beta\bra{l}\right)
\end{equation}
En permutant
\begin{equation}
\hat{F} =\frac{1}{2}\sum_{ijkl} f_{ij,kl}\underbrace{\left(\sum_{\alpha\neq\beta}^N \ket{i}_\alpha\bra{k}\otimes
\ket{j}_\beta\bra{l}\right)}_{\hat{a_i}^\dagger\hat{a_j}^\dagger\hat{a_k}\hat{a_l}}
\end{equation}
Il est possible d'écrire chaque opérateur dans une forme compacte : le $t_{ij}$ précédemment obtenu
devient $f_{ij,kl}$, le reste bien de la symétrie d'état. Conventionnellement, on préfère noter
\begin{equation}
\hat{F} =\frac{1}{2}\sum_{ijkl} \bra{ij}\hat{f}\ket{kl}\ \hat{a_i}^\dagger\hat{a_j}^\dagger\hat{a_l}\hat{a_k}
\end{equation}
Ce raisonnement peut se suivre pour les fermions, mais cette dernière expression  ne serait pas égale à
la précédente (anti-symétrie). 



\section{Système de $N$ fermions}
\subsection{Espace de Fock, opérateurs création/annihilation, opérateur nombre}
Comme nous travaillons avec des Fermions, il nous faut une anti-symétrie générale
\begin{equation}
\hat{S}_-\ket{i_1}_1\ket{i_2}_2\dots\ket{i_N}_N = \dfrac{1}{\sqrt{N!}}\sum_p(-1)^p \ket{i_1}_1\ket{i_2}_2\dots\ket{i_N}_N
\end{equation}

\subsubsection{État de Fock}
Avant toute chose, commençons par introduire la notion d'espace de Fock. Comme pour les bosons, nous 
dressons ici une "liste". Par définition :
\begin{equation}
\text{Etat de Fock }\ \equiv \ket{n_1,n_2,\dots} \equiv \hat{S}_-\ket{i_1}_1\ket{i_2}_2\dots\ket{i_N}_N 
\end{equation}
Cet espace de Fock contient \textit{tout les état antisymétrique complet de $m$ fermions.} Ces différents 
états forment l'espace de Fock qui est cette fois-ci totalement antisymétrique. Ces états sont bien 
orthogonaux
\begin{equation}
\bra{n_1,n_2,\dots}\ket{n_1',n_2',\dots} = \delta_{n_1,n_1'}\delta_{n_2,n_2'}\dots
\end{equation}
Pour la relation de fermeture:
\begin{equation}
\sum_{n_1,n_2} = \ket{n_1,n_2,\dots}\bra{n_1,n_2,\dots} =\hat{\mathbb{1}}
\end{equation}

\subsubsection{Opérateurs de création et d'annihilation}
Il n'est pas possible d'avoir plus d'un fermion par orbitale. Dès lors $(\hat{a}_i^\dagger)^2=0$. Nous 
voudrions avoir une relation du genre
\begin{equation}
\hat{a_{i_1}}^\dagger\hat{a_{i_2}}^\dagger\dots\hat{a_{i_N}}^\dagger\ket{0} = \hat{S}_-\ket{i_1}_1
\ket{i_2}_2\dots\ket{i_N}_N
\end{equation}
Voyons ce qui se produit lorsque nous inter-changeons les deux premiers éléments
\begin{equation}
\begin{array}{ll}
\hat{a_{i_2}}^\dagger\hat{a_{i_1}}^\dagger\dots\hat{a_{i_N}}^\dagger\ket{0} &\DS= \hat{S}_-\ket{i_2}_1
\ket{i_1}_2\dots\ket{i_N}_N\vspace{2mm}\\
&=\DS-\hat{S}_-\ket{i_1}_1\ket{i_2}_2\dots\ket{i_N}_N\vspace{2mm}\\
&=\DS-\hat{a_{i_1}}^\dagger\hat{a_{i_2}}^\dagger\dots\hat{a_{i_N}}^\dagger\ket0
\end{array}
\end{equation}
Des lors
\begin{equation}
\hat{a_i}^\dagger\hat{a_j}^\dagger = -\hat{a_j}^\dagger\hat{a_i}^\dagger\quad\Leftarrow\quad
\hat{a_i}^\dagger\hat{a_j}^\dagger +\hat{a_j}^\dagger\hat{a_i}^\dagger= 0\quad\Leftarrow\quad 
\{ \hat{a_i}^\dagger, \hat{a_j}^\dagger\}=0
\end{equation}
Ces opérateurs ne commutent pas, mais ils \textit{anti-commutent} ce qui est normal car nous avons 
des fermions et cette propriété antisymétrique est ici recherchée. Le cas où $i=j$ est trivial
\begin{equation}
2\hat{a_i}^\dagger\hat{a_i}^\dagger = 0\qquad\Leftrightarrow\qquad (\hat{a_i}^\dagger)^2=0
\end{equation}

On souhaiterait avoir une forme similaire à celle obtenue pour les bosons afin de pouvoir les 
traiter de la même façon
\begin{equation}
\bra{\dots n_i\dots n_j\dots} = (\hat{a_1}^\dagger)^{n_1}(\hat{a_2}^\dagger)^{n_2}\dots\ket{0}
\end{equation}
où $\ket0\equiv \ket{0,0,0,\dots}$ l'\textbf{état fermique du vide} (fermion vacuum state). Comme 
nous savons que $n_i=0,1 \forall i$ (où $i$ désigne une orbitale), nous pouvons décrire l'application 
de l'opérateur création sur un état de Fock
\begin{equation}
\hat{a_i}^\dagger \ket{\dots n_i\dots} = (1-n_i)(-1)^{\sum_{j<i}n_j}\ket{\dots n_i+1\dots}
\end{equation}
où le $\DS (-1)^{\sum_{j<i}n_j}$ est du à l'anti-commutateur : si nous avons $\hat{a_1}^\dagger
\hat{a_3}^\dagger\ket0 = \ket{1010\dots}$ et que nous voulons créer un fermion dans la quatrième orbitale, 
on veut noter $\hat{a_4}^\dagger\hat{a_3}^\dagger\hat{a_1}^\dagger$ mais ceci n'est pas un état de 
Fock, la définition n'est pas respectée. Il nous faut donc évaluer une double permutation pour le 
mettre à la "bonne place" : $\hat{a_1}^\dagger\hat{a_3}^\dagger\hat{a_4}^\dagger\ket0=\ket{10110\dots}$ : 
ce nombre de permutation donne un signe positif s'il est pair et négatif sinon (comme nous avons vu 
ci-dessous) et c'est ce qui est pris en compte avec ce facteur\footnote{Voir schéma cours 5 sous (24)}.\\

Il nous faut maintenant définir l'opérateur d'annihilation. Prenons pour ça le conjugué
\begin{equation}
\bra{\dots n_i'\dots}\hat{a_i} = (1-n_i-)(-1)^{\sum_{j<i}n_j}\bra{\dots n_i'+1\dots}
\end{equation}
Refermons cette relation
\begin{equation}
\bra{\dots n_i'\dots}\hat{a_i}\ket{\dots n_i\dots} = (1-n_i-)(-1)^{\sum_{j<i}n_j}
\delta_{n_i'+1,n_i}
\end{equation}
En appliquant l'opérateur d'annihilation à un état de Fock et en appliquant la relation de fermeture,
on voit apparaître la relation que nous venons d'obtenir
\begin{equation}
\hat{a_i}\ket{\dots n_i\dots} = \sum_{n_i'=0} \ket{\dots n_i'\dots}
\underbrace{\bra{\dots n_i'\dots}\hat{a_i}\ket{\dots n_i\dots}}_{ (1-n_i-)(-1)^{\sum_{j<i}n_j}
\delta_{n_i'+1,n_i}}
\end{equation}
Or, si
\begin{equation}
\left\{\begin{array}{ll}
n_i = 0 &\rightarrow 0\\
n_i = 1 &\rightarrow (-1)^{\sum_{j<i}n_j}\ket{\dots n_i-1\dots}
\end{array}\right.
\end{equation}
Dès lors
\begin{equation}
\hat{a_i}\ket{\dots n_i\dots} = n_i(-1)^{\sum_{j<i}n_j}\ket{\dots n_i-1\dots}
\end{equation}



\subsection{Relations d’anti-commutation}
Les relations sont les mêmes que pour les bosons, à l'exception du fait que nous utilisons ici 
l'anticommutateur
\begin{equation}
\{\hat{a_i},\hat{a_j}\}=0,\qquad\qquad \{\hat{a_i}^\dagger,\hat{a_j}^\dagger\}=0,\qquad\qquad
\{\hat{a_i},\hat{a_j}^\dagger\} = \delta_{ij}
\end{equation}
Prouvons la première relation. Si $i=j$, nous avons trivialement $(\hat{a_i})^2=0$. Pour $i\neq j$
\begin{equation}
\left\{\begin{array}{lll}
\hat{a_i}\hat{a_j}\ket{\dots n_i\dots n_j\dots} &= n_in_j(\pm 1)\ket{\dots n_j-1\dots n_j-1} &= 
-\hat{a_j}\hat{a_i}\ket{\dots n_i\dots n_j\dots}\\
\hat{a_j}\hat{a_i}\ket{\dots n_i\dots n_j\dots} &= n_jn_i(\pm 1)\ket{\dots n_j-1\dots n_j-1} 
\end{array}\right.
\end{equation}
Le résultat est immédiat. Le second anti-commutateur est un résultat immédiat ce ce-dernier. Prouvons alors la troisième relation
\begin{equation}
\begin{array}{ll}
\hat{a_i}^\dagger\hat{a_i}\ket{\dots n_i\dots} &= n_i\ket{\dots n_i\dots}\\
\hat{a_i}\hat{a_i}^\dagger &=(1-n_i)\ket{\dots n_i\dots}
\end{array}
\end{equation}
Le "préfacteur" de la première relation vaut 1 si l'on a un fermion et 0 sinon alors que pour la 
seconde, ce préfacteur vaut 0 en présence de fermion et inversement. En sommant ces deux relation, 
on fait apparaître l'anti-commutateur
\begin{equation}
\{\hat{a_i},\hat{a_i}^\dagger\}\ket{\dots n_i\dots} = 1\ket{\dots n_i\dots}\qquad\forall \text{Fock state}
\end{equation}
On en tire que 
\begin{equation}
\{\hat{a_i},\hat{a_i}^\dagger\} = \hat{\mathbb{1}}
\end{equation}

\subsection{Écriture générale d'un état à $N$ fermions et des opérateurs à 1 et 2 corps}
Reprenons notre opérateur nombre. Pour les bosons, nous l'avions défini 
\begin{equation}
\hat{n_i} = \hat{a_i}^\dagger\hat{a_i}
\end{equation}
Appliquons cet opérateur à un état de Fock (on ne s'intéresse ici qu'à l'orbitale $i$)
\begin{equation}
\begin{array}{ll}
\DS\hat{a_i}^\dagger\hat{a_i}\ket{\dots n_i\dots} &\DS= \hat{a_i}^\dagger n_i(-1)^{\sum_{j<i}n_j}\ket{\dots
 n_i-1\dots}\\
&\DS=(1(n_i-1))(-1)^{\sum_{j<i}n_j} n_i(-1)^{\sum_{j<i}n_j}\ket{\dots n_i\dots}\\
&\DS=(2-n_i)n_i\ket{\dots n_i\dots}\\
&\DS=2n_i-\underbrace{n_i^2}_{n_i}\ket{\dots n_i\dots}\\
&=\DS n_i\ket{\dots n_i\dots}
\end{array}
\end{equation}
car $n_i=0,1$. Ce résultat est donc également applicable aux fermions. On peut définir semblablement 
l'opérateur nombre total
\begin{equation}
\hat{N}=\sum_i\hat{n_i}
\end{equation}


\subsubsection{Opérateur à un corps}
De façon similaire au cas bosonique, nous pouvons écrire
\begin{equation}
\hat{T} =\sum_{ij}\bra{i}t\ket{j}\hat{a_j}^\dagger\hat{a_j}
\end{equation}

\subsubsection{Opérateur à deux corps}
De même pour l'opérateur à deux corps
\begin{equation}
\hat{F} = \frac{1}{2}\sum_{ijkl} \bra{i,j}\hat{f}\ket{k,l}\hat{a_i}^\dagger\hat{a_j}^\dagger\hat{a_l}\hat{a_k}
\end{equation}
Si ici nous interchangeons exemple $\hat{a_l}\hat{a_k}$ un signe négatif apparaîtra du au fait que ces 
deux opérateurs anticommutent. Ce formalisme est très élégant car $\hat{N}$ et $\hat{F}$ peuvent être 
arbitraire les bosons/fermions se traitent de façon identique.


