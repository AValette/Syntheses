\chapter{Principes fondamentaux de la mécanique quantique}

 Nous allons ici reformaliser les bases en nous basant sur le formalisme de Dirac. 
 On souhaite décrire l'\textit{état} du système, la \textit{mesure} et
 l'\textit{évolution} temporelle. Nous avions vu au début du premier chapitre 
 que la façon de définir une mesure était quelque peu particulière. Nous 
 allons ici nous baser sur l'interprétation de \textsc{Copenhagen} (Niels Bohr) mais 
 il faut savoir qu'il y en existe d'autre (\textsc{Bohm} (interprétation de l'onde 
 pilote), interprétation des \textsc{mondes multiples}, \dots).\\

  Cette interprétation 
 pose des  problèmes d'interprétation mais est bonne d'un point de vue pragmatique (
 les résultats expérimentaux correspondent très bien avec la théorie). Dès lors, si 
 l'on n'essaye pas d'interpréter la chose, tout fonctionne très bien\\
 
 Dans ce cours nous allons nous limiter aux \textit{états purs} : idéalisation 
 de la description s'il n'y a aucun bruit. Il existe évidemment des 
 \textit{états mixtes} (voir cours MA1) qui tient compte du bruit. 
 
 \section{1er principe : État d'un système}
 \subsection{Premier principe}
 Un état sera défini par un ket $\ket{\psi(t)} \in \mathcal{E}_H$. Cet 
 état doit être normé ($\|\psi(t)\|^2 = 1 \forall t$). Ceci a pour 
 conséquence immédiate le principe de superposition, toute combili d'état 
 étant un état possible
 \begin{equation}
 \ket{\psi} = \sum_i\ket{\psi_i},\qquad \sum_i |c_i|^2 = 1
 \end{equation}
 En effet, ceci est nécessaire pour la normalisation de la fonction d'onde
 \begin{equation}
 \begin{array}{ll}
 \|\psi\| &= \sum_{ij} c_j^*c_i\overbrace{\bra{\psi_j}\ket{\psi_i}}^{\delta_{ij}}\\
 &= \sum_i |c_i|^2 \\
 &= 1
 \end{array}
 \end{equation}
 L'état d'un système est déterminé à une indétermination de phase près : on 
 défini la phase globale
 \begin{equation}
 e^{i\delta}\ket{\psi}
 \end{equation}
 Cet état sera totalement indistinguable de l'état $\ket{\psi}$. Cette phase 
 globale disparaît d'ailleurs dans le traitement plus général des états mixtes. 
 Cette phase globale n'a pas d'interprétation physique. Par contre un phase 
 locale pondérant les différents états d'une superposition est pourvue de 
 sens physique (interférences)
 \begin{equation}
 \sum_j c_je^{i\delta_j}\ket{\psi_j}\quad\neq\quad \sum_j c_j\ket{\psi_j}
 \end{equation}
 
 Avant de s'attaquer à un problème, il faut s'intéresser au nombre de 
 degrés de liberté du système.
 \begin{equation}
 \text{Degré de liberté}\quad \rightsquigarrow\quad \mathcal{H}
 \end{equation}
 où $\mathcal{H}$ est l'espace de Hilbert. A chaque degré de liberté, on 
 confond un espace de Hilbert donnant lieu à des produits tensoriels d'espace 
 de Hilbert.
 
 \subsection{Structure de l'espace de Hilbert}
 Prenons l'exemple d'une particule à une dimension. En terme de fonction d'onde
 (qui se traduit aisément en notation de Dirac)
 \begin{equation}
 \psi(x) = \sum c_n\phi_n(x)\quad \rightarrow\quad \ket{\psi} = \sum_n c_n
 \ket{\phi_n}
 \label{eq:3.7}
 \end{equation}
 En étudiant les fonctions propres de l'Hamiltonien on peut écrire la fonction 
 d'onde sous la forme \autoref{eq:3.7}.\\
 A deux dimensions
 \begin{equation}
 \psi(x,y) = \sum_{n,m} c_{n,m}\phi_n(x)\phi_m(y)\quad\rightarrow\quad
 \ket{\psi} = \sum_{n,m} c_{n,m} \underbrace{\ket{\phi_n}\otimes\ket{\phi_m}}_{(*)}
 \end{equation}
 où $(*)$ est un produit tensoriel entre deux ket. Il en résultera un autre ket, 
 mais appartenant à un espace de Hilbert plus grand (ceci est équivalent à 
 $|\phi_n\ket{\phi_m}$. Plus généralement, on peut également exprimer une base :
  cela pourrait être n'importe quel fonction de la base multipliée par une autre. Ce p
 produit forme une base des fonctions d'ondes à deux dimension. \\
 
 Arrêtons avec les exemples et considérons deux ket de deux espaces de Hilbert
 \begin{equation}
 \left.\begin{array}{ll}
 \ket{u} \in \mathcal{E}_H\\
 \ket{v} \in \mathcal{F}_H 
 \end{array}\right\}\quad\rightarrow\quad \ket{u}\otimes\ket{v} \in\mathcal{G}_H 
 \equiv \mathcal{E}H_\otimes\mathcal{F}_H
 \end{equation}
 où $\otimes$ désigne le produit tensoriel. De façon encore plus générale :
 \begin{equation}
 \left.\begin{array}{l}
 \text{Si } \ket{e_n} \text{forme une base de } \mathcal{E}_H\\
 \text{Si } \ket{f_n} \text{forme une base de } \mathcal{E}_F 
 \end{array}\right\}\Rightarrow \left\{\ket{e_n}\otimes\ket{f_n}\right\} \begin{array}{l}
 \text{ forme  une base de l'espace de Hilbert }\\
 \text{  de produit $\mathcal{E}_H\otimes\mathcal{E}_F$}
 \end{array}
 \end{equation}
 On peut ainsi écrire tout $\psi$
 \begin{equation}
 \ket{\psi} = \sum_{n,m} \ket{e_n}\otimes\ket{f_m}
 \end{equation}
 
 Il en découle une série de propriétés. Par exemple
 \begin{equation}
 \dim(\mathcal{E}_F\otimes\mathcal{F}_H) = \dim(\mathcal{E}_F).
 \dim(\mathcal{F}_H)
 \end{equation}
 Le produit sera simplement le produit scalaire espace par espace
 \begin{equation}
 11
 \end{equation}
 Même si cet état est parfaitement possible, certains états ne peuvent 
 \textbf{jamais} s'écrire sous cette forme la. C'est le principe d'\textit{intrication 
 quantique }: l'état quantique ne peut pas se voir comme un produit 
 tensoriel c'est-à-dire une situation ou on ne peut pas décrire les deux particules 
 de façon séparées. Il faudrait les décrire simultanément et l'on n'arriverait donc jamais \\
 à écrire $\psi$ sous cette forme.\\
 
 Petite note supplémentaire: on peut considérer cet exemple de produit tensoriel dans 
 le cas d'un système à deux degrés de liberté discrets.
 \begin{equation}
 \begin{array}{ll}
 \ket{u} &= \left(\begin{array}{c}
 u_1\\
 u_2
 \end{array}\right)\\
 &\\
  \ket{v} &= \left(\begin{array}{c}
 v_1\\
 v_2
 \end{array}\right) 
 \end{array}\qquad \ket{u}\otimes\ket{v} = \left(\begin{array}{c}
 u_1v_1\\
 u_1v_2\\
 u_1v_3\\
 u_2v_1\\
 u_2v_2\\
 u_2v_3  
 \end{array}\right)
 \end{equation}
 Le produit tensoriel donne bien lieu à toutes les combinaisons possibles.
 
 
 \section{2e principe : Mesure}
 \subsection{Observable}
 A toute grandeur physique mesurable $A$ on peut associer $\hat{A}$ un 
 opérateur linéaire hermitien qui agit dans $\mathcal{E}_H$. Ceci dit 
 que dès qu'on a une grandeur mesurable il doit exister un certain opérateur
 hermition. Ceci ne dit rien sur cet opérateur, mais les règles de 
 correspondances permettent de apsser d'un opérateur classique à quantique. 
 Tout ce qui existe en classique existe en quantique, l'inverse n'est pas
 vrai (spin).
 
 \subsection{Principe de quantification}
 Les seuls résultats de la mesure de l'observalbe $\hat{A}$ sont les 
 valeurs propres $a_n$ de l'observable. C'est vrai pour les systèmes liés 
 (boite bien quantifiée) et non liés (on est dans un cas plus classique ou 
 on a un continuum, tout est observable)
 
 \subsection{Principe de décomposition spectrale}
 La probabilité d'obtenir un résultat $a_n$ est donné par l'élément de 
 matrice diagonal de l'opérateur projection associé
 \begin{equation}
 13
 \end{equation}
 Pour rappel
 \begin{equation}
 14
 \end{equation}
 Il s'agit de la règle de Born. Mais qu'est ce qui se passe après la 
 mesure ?
 
 \subsection{Réduction du paquet d'onde}
 Juste après la mesure, on va se retrouver dans un nouvel état :
 \begin{equation}
 15
 \end{equation}
 Au niveau de l'interprétation : comment interpréter ces probabilités 
 qui apparaissent ? Problème toujours non résolu : les proba observées 
 sont liées à la connaissance du systeme (fct d'onde objet de type 
 théorie desproba) ou faut il comprendre fct d'onde comme objet 
 physique existant comme une onde EM ? Exist phys ou objet décrivant 
 la connaissance du syst ? \\
 
 la limite (de la fréquence d'apparition de l'apparition de $a_n$) =
 $\mathbb{P}(a_n)$ Pour une mesure particuliere la MQ ne donne pas précisement 
 la valeur observée mais juste le "chance" de pouvoir l'observer. C'est 
 la raison pour laquelle Einstein et co disait que la MQ était incimplete  : 
 les proba ne feraitque cacher un mecénisme sous jascean. Ajd on sait que 
 c'était faux, il n'y a pas de variable caché dont on ne connait pas la 
 mécanique (montrable expérimentalement) : ces propriétés sont intrinsèques 
 à cette théorie.\\
 
 Remarquons que la probabilité correspond bien à toutes les propriétés 
 usuelles
 \begin{equation}
 16
 \end{equation}
 
 \subsection{Reproductibilité de la mesure}
 Si on a un systeme dans un état quelconque, on fait une mesure et 
 on observa $a_n$. Pour que ça ai un sens, il faut que si on remesure 
 directement après la mesure soit identique. \\
 
 On définit la valeur moyenne de l'obs hat A
 \begin{equation}
 17
 \end{equation}
 Ceci désigne un valeur moyenne, c'est l'élément de matrice diagonal de 
 l'observable $\hat{A}$. On peut montrer qu'en partant de cette 
 expression de la valeur moyenne pour interpréter que seules les valeurs 
 propres apparaissent dans une mesure
 \begin{equation}
 18
 \end{equation}
 Les seuls état à donner une variance nuls sont les états propre. Au 
 deuxième coup la variance doit forcément etre nulle et ceci montre que 
 seul les valeurs propres peuvent etre observées e qu'il faut que juste 
 après la mesure on ai l'état propre de la grandeur mesurée. \\
 
 De façon générale, si on prend tjs pour acquis que la valeur moyenne 
 d'une quantité physique est donné par l'élément de matrice diagonal 
 de l'opérateur en question, alors
 \begin{equation}
 19
 \end{equation}
 La probabilité est donnée par le module carré du coefficient, tout ça 
 en faisant une seule hypothèse.\\
 
 \subsection{Relation d'incertitude de Heisenberg}
 Initialement nous avons un ket psi et considérons deux observables qui 
 a priori ne commutent pas
 \begin{equation}
 20
 \end{equation}
 On va préparer un grand nombre de systeme, sur un partie on va observer
 A ou A' et sur l'autre B et B' pour en déduire les variance. On veut 
 voir à quel point l'une est liée à l'autre et montrer qu'elles ne peuvent 
 pas etre toutes les deux petites. Définissons un nouvel op math 
 \begin{equation}
 21
 \end{equation}
 Ceci donnera un nombre strictement positif, les deux ne peuvent donc 
 pas être très petits. 