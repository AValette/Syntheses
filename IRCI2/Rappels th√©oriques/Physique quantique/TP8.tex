
\section*{TP 8 : Les atomes}
\begin{itemize}
	\item Moment cinétique $\vec{J}$ (opérateur vectoriel) : 2 nombre quantiques
		\begin{itemize}
			\item $j$ ($\geq 0$) : entier ou demi-entier (valeur propre de $J^2 = \hbar ^2 j(j+1)$)
			\item $m \in [-j,j]$ pour un total de $2j +1$ valeurs entières ou demi-entières (valeur propre de $J_z = \hbar m$)
		\end{itemize}
		
	\item Composition de moment cinétique
	\begin{equation}
		\vec{J} = \vec{J_1}+\vec{J_2} \Rightarrow |j_1-j_2| \leq j \leq |j_1+j_2|
	\end{equation}
	\begin{itemize}
		\item Structure fine : $\vec{J}=\vec{L}+\vec{S}$
		\item Structure hyperfine : $\vec{F} = \vec{J}+\vec{L}$ (spin du noyau)
	\end{itemize}
	
	\item Système hydrogénoïde : état = $(n,l,m,m_s \, (= \pm 1/2))$ avec $n = n_r+l+1$
	\begin{itemize}
		\item $l$ donné : $(2l+1)$ valeurs de $m$ et 2 valeurs de $m_s \Rightarrow 2.(2l+1)$ états
		
		\item $l = 0,1,2, \dots \Rightarrow s,p,d,f,g,\dots$
		
		\item Sous-couche ($nl$) fermée : $l = 0 \Rightarrow J(atome) = J(dernière$ $couche)$
		
		\item Notation spectroscopique (dernière couche atome)
		\begin{equation}
			^{2S+1}L_J
		\end{equation}
		\begin{itemize}
			\item $L$ et $S$ : moment cinétique orbital et spin \textbf{total} de la dernière couche
			
			\item Sur demande de Cédric Hannotier, lorsque l'exercice demande d'établir la structure de la dernière couche, on doit 
			\begin{enumerate}
				\item Ecrire la configuration (1s2s2P...) et répartir le nombre d'électrons de l'atome
				\item Relever le nombre d'électrons sur la dernière couche et déduire L d'après la lettre associée à la couche ($S : L=0, P : L=1$, ...)
				
				\item Dessiner un nombre $m_L \in [-L,L]$ de case représentant chque $m_L$ et placer les électrons selon le principe de Pauli
				\item En utilisant 
				\begin{equation}
				S = \sum m_S \qquad et \qquad L = \sum m_L
				\end{equation}
				Calculer ces derniers en sommant les nombres quantiques de chaque électrons (faire attention aux spin opposés pour $S$)
				
				\item Ecrire la notation $^{2S+1}L_J$ en n'oubliant pas que $J$ se compose selon l'inégalité triangulaire et peut donc avoir plusieurs valeurs
			\end{enumerate}
			\item Helium : toujours un électron en $1s$ (stabilité)
		\end{itemize}
	\end{itemize}
\end{itemize}