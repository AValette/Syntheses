\chapter{Méthode des approximations successives pour problèmes de Cauchy}

\setcounter{section}{3}
\section{Principe de contraction de Banach}
	 \subsection{Contraction (de constante $\alpha$)}
	 Le principe d'une contraction est que si deux points sont à une 
	 distance $d$, la distance entre leur image par la contraction 
	 sera inférieure à $\alpha d$ où $\alpha<1$.\\
	 	 
	 \defi{\textsc{Contraction de Banach}\\
	 Une application $T : E \rightarrow E$ est une \textbf{contraction de 
	 constante} $\alpha$ ssi
	 \begin{equation}
	 \left\{\begin{array}{ll}
	 (i) & \forall x, x' \in E : \| Tx-Tx'\| \leq \alpha \|x-x'\| \\
	 (ii) & \alpha <1
	 \end{array}\right.
	 \end{equation}}\ \\
	 Ainsi, une fonction lipschitzienne de constante strictement inférieure à
	 1 est une contraction.\\
	 
	 Il est intéressant de travailler dans un espace de Banach (par exemple $V$), 
	 c'est à dire un espace vectoriel réel normé (on aura besoin de la notion de 
	 distance) \textbf{complet}. $V$ est dit complet ssi toute suite dite de Cauchy 
	 converge. Pour rappel :
	 \begin{equation}
	 \forall \epsilon> 0, \exists N : \forall n \geq N, \forall l > 0 : \| 
	 u_{n+l}-u_n\| < \epsilon
	 \end{equation}
	
	On travaille souvent avec des fermés. Si $E$ est fermé dans $V$ alors 
	toute suite d’éléments de $E$ converge dans $E$ !
	
	\subsection{Théorème de contraction de Banach}
	\theor{\textsc{Principe de contraction de Banach}\\
	Si $T : E \rightarrow E$ est une contraction de constante $\alpha$, alors 
	\begin{enumerate}
	\item $T$ admet \textbf{un} et \textbf{un} seul point fixe $\tilde{x}$
	\item $\forall x \in E : T^n(x) \rightarrow \tilde{x}$
	\item $\forall x \in E : \| T^n(x)-\tilde{x}\| \leq \frac{\alpha^n}{-1\alpha}
	\|Tx-x\|$
	\end{enumerate}	}
	\newpage
	De façon francisée, cela signifie que :
	\begin{enumerate}
	\item On admet un point fixe
	\item D’où que l'on parte, pour chaque $x \in E$ ($T^1, \dots T^n$) cette suite
	converge vers ce point fixe
	\item Quel que soit $x \in E$, si j'ai appliqué $n$ fois la contraction je suis 
	à une distance du point fixe majorée par le longueur du premier pas multiplié par
	$\dots$ Comme $\alpha < 1$, cela tend fortement vers zéro
	\end{enumerate}
	
	\begin{proof}\ \\	
	Prouvons d'abord qu'il s'agit d'une suite de Cauchy.\\
	Soit $x_0 \in E, x_n := T^n(x_0)$. On peut écrire
	\begin{equation}
	\| x_{n+l} - x_n\| = \| T^{n+l}(x_0à-T^n(x_0)\|
	\end{equation}
	L'idée à exploiter est que $T^n(x_0) = T(T^{n-1}(x_0))$. Je peux alors appliquer 
	une majoration en sachant que appliquer $T$, c'est multiplier par $\alpha$
	\begin{equation}
	\begin{array}{ll}
	\| x_{n+l} - x_n\| &\leq \alpha \|T^{n+l-1}(x_0) - T^{n-1}(x_0)\|\\
	 &\leq \alpha^2 \|T^{n+l-2}(x_0) - T^{n-2}(x_0)\|\\	
	 & \leq \dots\\
	 &\leq \alpha^n \|T^{l}(x_0) - x_0\|
	\end{array}
	\end{equation}
	On a alors
	\begin{equation}
	\| x_{n+l} - x_n\| \leq \alpha^n . \|x_l-x_0\|\qquad\qquad (*)
	\end{equation}
	En appliquant l'inégalité triangulaire 
	\begin{equation}
	\begin{array}{ll}
	\|x_l-x_0\| &\leq \|x_l-x_{l-1}\| + \dots + \|x_2-x_1\| + \|x_1-x_0\|\\
	 &\leq \underbrace{(\alpha^{l-1} + \dots + \alpha^1 + \alpha^0)}_{\frac{1-\alpha^l}{
	 1-\alpha} \leq \frac{1}{1-\alpha}}\|x_1-x_0\|
	\end{array}
	\end{equation}
	On a alors
	\begin{equation}
	\|x_l-x_0\| \leq \frac{1}{1-\alpha}\|x_1-x_0\|\qquad\qquad (**)
	\end{equation}
	En rassemblant $(*)$ et $(**)$ :
	\begin{equation}
	\|x_{n+l}-x_n\| \leq \frac{\alpha^n}{1-\alpha}\|x_1-x_0\| \underbrace{<\epsilon}_{\text{
	dès que $n$ grand ($l\geq0$)}}
	\end{equation}
	Ceci démontre que $x_n$ est une suite de Cauchy.\\
	
	Cette suite converge, car par hypothèse $V,+,\| \|$ est complet. Comme $x_n$ est 
	une suite de Cauchy, $x_n \rightarrow \tilde{x}\in V$. Par hypothèse $E$ est 
	fermé dans $V$. Comme $\underbrace{x_n}_{\in E\ \forall n} \rightarrow \tilde{x}
	\in V \Rightarrow \tilde{x}\in E$.\footnote{$x_n$ est une suite d’éléments dans $E$, 
	elle converge forcément dans $E$ : sa "limite", $\tilde{x}\in E$.}\\
	
	Il faut maintenant prouver que $\tilde{x}$ est fixe : $T(\tilde{x}) = \tilde{x}$. 
	Par continuité de la contraction :
	\begin{equation}
	T(\lim\limits_{n\rightarrow\infty} x_n) \underbrace{=}_{T \in C^0} = \lim\limits_{n
	\rightarrow\infty} T(x_n) = \lim\limits_{n\rightarrow\infty} x_n = \tilde{x}
	\end{equation}
	
	
	Prouvons maintenant l'unicité de ce point fixe par l'absurde.\\
	Soit $\tilde{x}, \tilde{y}$ fixés par $T$ contraction de sorte que $\|\tilde{x}
	-\tilde{y}\| \neq 0$. Alors 
	\begin{equation}
	\|\tilde{x}-\tilde{y}\| = \| T(\tilde{x}) - T(\tilde{y})\| \leq\alpha\|\tilde{x}-
	\tilde{y}\| < \|\tilde{x}-\tilde{y}\|
	\end{equation}
	On peut éviter la contradiction si $\|\tilde{x}-\tilde{y}\| = 0$ impliquant l'
	unicité du point fixe.\\
		
	Il ne reste qu'a prouver la qualité de approximation :
	\begin{equation}
	\begin{array}{ll}
	\| x_n-\tilde{x}\| &= \|x_n - \lim\limits_{l\rightarrow\infty} x_{n+l}\|\\
	 & = \lim\limits_{l\rightarrow\infty} \|x_n-x_{n+l}\|\\
	 & \leq \left(\lim\limits_{l\rightarrow\infty}\right) \frac{\alpha^n}{1-\alpha} 
	 \|x_1-x_0\|
	\end{array}
	\end{equation}
	On peut donc dire que $\tilde{x} \approx x_n$ avec une erreur $\leq \frac{\alpha^n}{
	1-\alpha}\|x_1-x_0|$ soit la longueur du premier pas.
	\end{proof}
	



\setcounter{section}{0}
\section{EDO normale du premier ordre}
	\subsection{Attention }
	Le problème de Cauchy 
	\begin{equation}
	\left\{\begin{array}{ll}
	xy' + 2y &= 4x^2	\\
	y(0) &= 1
	\end{array}\right.
	\end{equation}
	n'admet pas de solution car 0 est un point singulier ! Il faut faire attention à 
	pas confondre une ED implicite avec une ED explicite ($y' = f(x,y)$).
	
	\subsection{Solution maximale et globale dans un cylindre}
	Considérons le problème de Cauchy
	\begin{equation}
	\left\{\begin{array}{ll}
	y' &= f(t,y)\\
	y_0 &= y(t_0)
	\end{array}\right.
	\end{equation}
	Si l'ED est scalaire, considérons un domaine rectangulaire et un cylindre s'il 
	s'agit d'un SD. Les différents types de solutions sont :
	\begin{enumerate}
	\item[$\bullet$] \textbf{Maximale} ; Une solution est dite maximale ssi elle ne 
	peut pas être prolongée en une autre solution, c'est à dire qu'on ne peut la 
	prolonger sur un intervalle plus grand (Pas de solution dans un domaine plus 
	étendu restant dans le domaine de $f$)
	\item[$\bullet$] \textbf{Globale} ; Une solution est globale ssi la solution 
	au problème de Cauchy est définie sur $I$ tout entier. 
	\item[$\bullet$] \textbf{Locale} ; Une solution est locale ssi il existe un 
	voisinage $\mathcal{V}$ du point de la C.I. tel que la fonction est définie 
	dans un sous-intervalle de $I$.
	\end{enumerate}

	\textsc{Exemple}. Soit l'EDO $y'=y^2$. Sa solution générale est $y(t) = 
	-\frac{1}{t+C}$. A cause de l'asymptote, toutes ces solutions sont maximales 
	mais seule la solution nulle est globale.
	
	\newpage
	\subsection{Régularité des solutions d'une EDO}
	\proposition{Si $\vec{\varphi}$ est solution de $\vec{y}' = \vec{f}(t,\vec{y})$ 
	pour $\vec{f}\in C^k$, alors $\vec{\varphi}\in C^{k+1}$}\ \\
	
	\begin{proof}\ \\
	Comme $\varphi$ est solution d'EDO, il est dérivable (et donc $C^0$).\\
	\begin{itemize}
	\item[$\bullet$] Supposons $k=0$
	\begin{equation}
	\left.\begin{array}{l}
	\varphi \in C^0\\
	f \in C^0
	\end{array}\right\} \Rightarrow t \mapsto f(t,\varphi(t)) \in C^0
	\end{equation}
	Or $f(t, \varphi(t)) = \varphi'(t) \Rightarrow \phi \in C^1$.
	
	\item[$\bullet$] Récurrence\footnote{??}. Supposons vrai pour $k-1$ et montrons vrai pour $k$
	\begin{equation}
	\left.\begin{array}{l}
	\text{Vrai pour } k-1\\
	C^k \subset C^{k-1}\\
	\vec f \in C^k
	\end{array}\right\} \Rightarrow \vec \varphi \in C^k \Rightarrow 
	t \mapsto \vec f(t,\vec \varphi(t)) \in C^k
	\end{equation}
	Or $\vec \varphi'(t) = \vec{f}(t,\vec{\varphi}(t))$, d'où $\vec \varphi' 
	\in C^k$ c'est-à-dire $\vec{\varphi}\in C^{k+1}$
	\end{itemize}
	\end{proof}
	
	
	\subsection{Équation intégrale d'un problème de Cauchy}
	\proposition{$\varphi$ est solution du problème de Cauchy sur $I$ 
	\begin{equation}
	\left\{\begin{array}{ll}
	y' &= f(t,y)\\
	y(t_0) &= y_0
	\end{array}\right.
	\end{equation}		
	ssi 
	\begin{equation}
	\left\{\begin{array}{ll}
	\varphi \in C^0(I)\\
	\forall t \in I : \varphi(t) = y_0+\int_{t_0}^t f(\tau,\varphi(\tau))d\tau
	\end{array}\right.
	\end{equation}}\	
	\begin{proof}\ \\
	\textit{Sens direct : } $\varphi'(t) = f(t,\varphi(t))$ par hypothèse, il 
	suffit d'intégrer les deux membres de $t_0$ à $t$.\\
	
	\textit{Sens indirect :}  $\varphi \in C^0 \Longrightarrow \tau \mapsto f(
	\tau,\varphi(\tau)) \in C^0$, d'où $\varphi'(t) = f(t, \varphi(t))$ en dérivant
	l'équation intégrale.
	\end{proof}
	
	
\section{Théorème de résolubilité locale}
	\subsection{Théorème}
	\theor{\textsc{Cauchy-Paeno-Arzela}\ \\
	Le problème de Cauchy 
	\begin{equation}
	\left\{\begin{array}{ll}
	y' &= f(t,y)\\
	y(t_0) &= y_0
	\end{array}\right.
	\end{equation}
	où $f \in C^0(\mathcal{U})$ ($\mathcal{U}$ étant ouvert) et $(t_0,y_0)\in\mathcal{U}$ 
	admet \textbf{au moins} une solution locale.
	}


\newpage
\section{L'opérateur intégral de Picard}
	\subsection{L'opérateur intégral de Picard}
	Inspiré par l'écriture intégrale d'un problème de Cauchy :
	\begin{equation}
	\vec \varphi(t) = \vec{y_0} + \int_{t_0}^t \vec{f}(\tau,\varphi(\tau))d\tau
	\end{equation}
	On définit l'opérateur intégral de Picard :
	
	\defi{\textsc{Opérateur intégral de Picard} \\
	\begin{equation}
	\begin{array}{ll}
	T : z \mapsto &T(z)\\
	 &T(z)|_t := \vec y_0 + \int_{t_0}^t \vec{f}(\tau,z(\tau))d\tau
	\end{array}
	\end{equation}
	}\ \\
	$T$ est l'opérateur intégral de Picard \textbf{associé au problème de Cauchy}
	ci-dessus. Les solutions de l'équation intégrale du problème de Cauchy sont donc 
	exactement les points fixes de cet opérateur, c'est-à-dire les fonctions 
	$\vec{\varphi}$ telles que $\vec \varphi = T(\vec \varphi)$.
	
	\setcounter{subsection}{8}
	\subsection{Erreur d'une approximation de la solution}
	Étudions la différence entre la solution exacte et l'approximation
	\begin{equation}
	\begin{array}{ll}
	|y(t) - y_n(t)| &= |y_0 + \int_{t_0}^t f(\tau,y(\tau))d\tau - y_0 - \int_{t_0}^t 
	f(\tau, y_n(\tau))d\tau|\\
	 &= |\int_{t_0}^t (f(\tau,y(\tau)) - f(\tau,y_n(\tau)))d\tau|\\
	 &\leq \sup_t |t-t_0| ; \sup_\tau |f(\tau,y(\tau)) - f(\tau,y_n(\tau))|
	\end{array}
	\end{equation}
	Pour que l'erreur soit petite, il faut que $t$ soit proche de $t_0$, que l'on 
	parte d'une bonne approximation et aussi que $f$ ne varie pas trop vite en sa 
	deuxième variable $y$ (c'est à dire $\sup |\partial f/\partial y|$ petit si 
	$f$ est "brave" (c'est-à-dire lipschitzienne)).
	
\setcounter{section}{4}
\section{Condition de Lipschitz}
	\subsection{Fonction totalement ou partiellement lipschitzienne}
	\defi{\textsc{Fonction lipschitzienne}\\
	$f : A \subseteq \mathbb{R} \rightarrow \mathbb{R}$ est dite 
	\textbf{lipschitzienne} ssi 
	\begin{equation}
	\exists M \in \mathbb{R}(=V) : \forall x, \tilde{x} \in A : |f(x) - f(\tilde{x})
	\leq M|x-\tilde{x}|
	\end{equation}}
	Ceci signifie que $M$ majore toutes les valeurs absolues de pentes de cordes 
	du graphe de $f$. Être lipschitzienne est plus fort qu'être continue, mais 
	cela n'implique pas la dérivabilité. Par contre si une fonction est dérivable 
	à dérivée bornée alors elle est lipschitzienne.\\
	
	On peut généraliser dans le cas où $V = \mathbb{R}^n\times\mathbb{R}^m$ en 
	considérant une condition de Lipschitz "partielle", avec $\vec{x}$ constant 

	\defi{\ \\
	$\vec{f} : \mathcal{U} \subseteq \mathbb{R}^n\times\mathbb{R}^m \rightarrow\mathbb{R}^p$ 
	est $\Lambda-$	\textbf{lipschitzienne} en $\vec{y}$ (sur $\mathcal{U}$) ssi 
	\begin{equation}
	\forall(x,y),(x,\tilde{y}) \in \mathcal{U} : ||f(x,y) - f(x,\tilde{y})|| \leq 
	\Lambda||y-\tilde{y}||
	\end{equation}
	On remarque bien qu'ici $x$ est fixé, constant.}
	
	
	\newpage
	\subsection{Fonctions localement lipschitzienne}
	Comme être lipschitzienne est fort contraignant, on ne demande parfois que 
	localement. La définition est presque identique, sauf que l'on se limite à 
	comparer les points de même abscisse en restant dans un voisinage $\mathcal{V}_0$.
	
	\defi{\ \\
	$\forall (x_0,y_0) \in \mathcal{U}, \exists \mathcal{V}_0$ voisinage de $(x_0,y_0), 
	\exists \Lambda_0 \in \mathbb{R}$ :
	\begin{equation}
	\forall (x,y), (x,\tilde{y}) \in \mathcal{V}_0 \cap \mathcal{U} : |f(x,y)-f(x,\tilde{y}|
	\leq \Lambda_0 |y-\tilde{y}|
	\end{equation}
	Ceci se généralise aux fonction vectorielles en changeant $|\ |$ par $\|\ \|$.}
	
	
	\setcounter{subsection}{3}
	\subsection{CNS pour Lipschitz : composante par composante}
	\proposition{$\vec{f}$ est lipschitzienne en $\vec{y}$ ssi $\forall i = 1,\dots, p :
	f_i$ est lipschitzienne en $\vec{y}$}\ 
\begin{proof}\ \\
$\vec f \quad \Lambda$-{\bf lipschitzienne}  en $\vec y$
\\
$
\begin{array}{ll}
\Longleftrightarrow & {\displaystyle \forall \: (\vec x , \vec y ) ,
 (\vec x, \tilde{\vec y} ) \in {\cal U} : || \vec f (\vec x, \vec y
 ) - \vec f ( \vec x , \tilde{\vec y} ) ||_p ^2 \leq \Lambda ^2 ||
 \vec y - \tilde{\vec y} ||^2 _m }\\
\Longleftrightarrow &{\displaystyle \forall \: (\vec x , \vec y )
, (\vec x, \tilde{\vec y} ) \in {\cal U} : \sum^p _{i=1} (  f_i
(\vec x, \vec y) -  f_i ( \vec x , \tilde{\vec y} ) )^2  \leq
\Lambda ^2 || \vec y - \tilde{\vec y} ||^2 _m } \\
\stackrel{?}{\Longleftrightarrow} &\forall i = 1 , \ldots , p,
\quad \forall \: (\vec x , \vec y ) , (\vec x, \tilde{\vec y} )
\in {\cal U} : (f_i (\vec x, \vec y) - f_i ( \vec x , \tilde{\vec
y} ) )^2  \leq \lambda ^2 _i || \vec y - \tilde{\vec y}
 ||^2 _m \\
\Longleftrightarrow  &\forall i = 1 , \ldots , p , \quad f_i
\mbox{ est } \lambda_i  \mbox{ -lipschitzienne.}

\end{array}
$ \\

L'équivalence surmontée d'un point d'interrogation
s'établit comme suit : \\

$\Longrightarrow$ est vraie si on a posé $\lambda_i : = \Lambda
$. \\

$ \Longleftarrow $ est vraie si on a posé ${\displaystyle
\Lambda : = \sqrt{\sum_{i=1}^{p} \lambda^2 _i}}$. 
\end{proof}
	Cette démonstration est naturelle dans le sens ou si $\vec{f}$ est vectorielle 
	et lipschitzienne de même constante, chacune de ses composantes l'est également. 
	Notons que dès que l'on a une constante de Lipschitz, tout nombre supérieur à 
	celle-ci est également une constante de Lipschitz.
	
	
	\subsection{$C^1$ garantit localement Lipschitz}
	Le plus simple est de travailler avec des dérivées. Si les dérivées d'ordre 
	$p$ sont bornées alors la fonction est lipschitzienne.\\
	\lemme{\ \\
		Si $\vec{f}$ est différentiable sur int($\mathcal{U}$) et continue sur 
		$\mathcal{U}$ et que ses fonctions dérivées $\frac{\partial\vec f}{\partial 
		y_1},\dots,\frac{\partial\vec f}{\partial y_m}$ sont bornées sur int($\mathcal{U}$),
		alors $\vec{f}$ est lipschitzienne en $\vec{y}$ sur $\mathcal{U}$.}\ 		
		\begin{proof}\ \\
		Supposons que $f$ est scalaire (grâce à la précédente proposition). Grâce au 
		théorème des accroissement fini on peut écrire la différence selon la première 
		égalité si $\exists \vec{c} \in ]\vec{y}\vec{\tilde{y}}[\subset \mathbb{R}^m$ :		
\begin{equation}
\begin{array}{ll}
| f (\vec x , \vec y ) - f ( \vec x , \tilde{\vec y} ) | & =
{\displaystyle \left | < \vec {\nabla} f  \right |_{( \vec x , \
\vec c)} , (\vec x , \vec y ) -  ( \vec x , \tilde{\vec y} ) > | }\\

  & = {\displaystyle | \left . \left . < \left (  \frac{\partial f}
{\partial y_1}, \ldots , \frac{\partial f}{\partial y_m} \right )
\right |_{( \vec x , \vec c)} ,  ( y_1 - \tilde{y}_1, \ldots, y_m -
 \tilde{y}_m ) > \right . | }\\

 & ={\displaystyle | \left . \left . \sum_i \frac{\partial f}
{\partial y_i} \right |_{( \vec x , \vec c)} ( y_i - \tilde{y_i} )
\right .  | }\\

  &{\displaystyle \leq \sqrt{\sum_i \left . \frac{\partial f }
{\partial y_i} \right | ^2 _{(\vec x , \vec c )}} \quad  || \vec y -
\tilde{\vec y} ||_m }
\end{array}
\end{equation}
	On pourrait directement écrire la 
	troisième égalité. Comme il n'y a pas de différence entre $\vec{x}$ et $\vec{x}$, 
	seul intervienne les dérivées par rapports aux $y_i$. La dernière inégalité s'
	obtient gr\^ace à l'inégalité de Cauchy-Schwarz.\\
	Si toute les dérivées $\dfrac{\partial f}{\partial y_i}$ sont en valeurs absolue 
	majorée par $M$, alors $f$ est $\sqrt{mM^2}$-lipschitzienne en $\vec{y}$.
		\end{proof}
		
	
	\proposition{Si $\vec{f}\in C^1(\mathcal{U})$, alors $\vec{f}$ est localement 
	lipschitzienne sur int($\mathcal{U}$).}
	
	\begin{proof}
	Non vu? 
	\end{proof}
	
	\setcounter{subsection}{2}
	\subsection{Fonctions à variables séparées de Lipschitz}
	\proposition{Si \begin{itemize}
	\item[$\bullet$] $f: A \subseteq \mathbb{R} \rightarrow \mathbb{R} : x \mapsto 
	f(x)$ est continue, 
	\item[$\bullet$] $g: B \subseteq \mathbb{R} \rightarrow \mathbb{R} : y \mapsto 
	g(x)$ est localement lipschitzienne,
	\end{itemize}
	alors 
	\begin{equation}
	F : A\times B \rightarrow \mathbb{R} : (x,y) \mapsto \rightarrow f(x).g(y)
	\end{equation}
	est localement lipschitzienne en $y$}\ 	
	\begin{proof}\ \\
	Une telle fonction est-elle lipschitzienne? On va essayer de la majorer. Comme 
	$x$ est constant, je peux écrire 
	\begin{equation}
	|f(x)g(y) - f(x)g(\tilde{y})| = |f(x)|.|g(y)-g(\tilde{y})|
	\end{equation}
	Je peux toujours majorer de la sorte
	\begin{equation}
	|f(x)g(y) - f(x)g(\tilde{y})| \leq \sup_{\mathcal{V}_{x_0}} |f|.\lambda_0.
	|y-\tilde{y}|
	\end{equation}
	Si $\sup_{\mathcal{V}_{x_0}} = M_0\in\mathbb{R}$, $F$ est $\lambda_0 M_0$-
	lipschitzienne en $y$.\footnote{J'applique la "définition" de Lipschitz pour 
	la partie en $|g(y)-g(\tilde{y})|$.}
	\end{proof}
	
	
	
	\setcounter{subsection}{5}
	\subsection{Fonctions linéaires en $\vec{y}$ à coefficients continus en $t$}
	Considérons la fonction
	\begin{equation}
	\vec{f}(t,\vec{y}) := A(t)\vec{y} + \vec{b}(y)
	\end{equation}
	où $A(y)$ est $\forall t$ une application linéaire $\mathbb{R}^m\rightarrow
	\mathbb{R}^m$. Une telle fonction est-elle lipschitzienne? On va essayer de 
	la majorer. Par linéarité, je peux mettre en évidence un facteur $\|y-z\|$ 
	pour prendre l'image d'un vecteur normé par $A(t)$ ($\vec{y},\vec{z}\in 
	\mathbb{R}^m$)
	\begin{equation}
	\begin{array}{ll}
	\vec{f}(t,\vec{y}) - \vec{f}(t,\vec{z}) &= A(t)\vec{y}+\vec{b}-(A(t)\vec{z}+\vec{b})\\
	 &= A(t)(\vec{y}-\vec z) = \|\vec{y}-\vec{z}\|A(t)\left(\frac{\vec{y}- 
	 \vec{z}}{\|\vec{y}-\vec{z}\|}\right)
	\end{array}
	\end{equation}
	Notons l'utilisation d'une petit artifice de calcul à la deuxième ligne, on à 
	multiplié par $\frac{\|\vec y - \vec{z}\|}{\|\vec y - \vec{z}\|}.$
	On peut majorer et y aller à la grosse louche à l'aide de la norme de l'application 
	linéaire $A(t)$. Je peux en effet dire que le carré d'une somme est majoré par le 
	carré des éléments de la matrice
	\begin{equation}
	\|\vec{f}(t,\vec{y}) - \vec{f}(t,\vec{z})\| \leq \|\vec{y}-\vec{z}\| \underbrace{
	\max_{\|\vec{u}\| = 1}
	\|A(t)(\vec{u})\|}_{|||A(t)||| \leq \sqrt{\sum_{ij} (a_{ij}(t))^2} := \Lambda(t)}
	\qquad (t\ \text{ fixé!})
	\end{equation}
	Dernier souci : $\Lambda(t)$ n'est pas constante. Heureusement, si on prend le 
	suprémum des normes quand $t$ est confiné à un compact, ce suprémum existe dans 
	$\mathbb{R}$.
	

\section{Théorèmes d'existence et d'unicité pour Cauchy}
	\subsection{Méthode des approximations successives de Picard}
	Considérons l'opérateur intégral\footnote{Revient très souvent à l'examen!} $T : 
	z\rightarrow T(z)$ où 
	\begin{equation}
	T(z)|_t := y_0 + \int_{t_0}^t f(\tau,z(\tau))d\tau
	\end{equation}
	relatif au \textbf{problème de Cauchy} $\left\{\begin{array}{ll}
	y'(t) &= f(t,y)\\
	y(t_0) &= y_0
	\end{array}\right.$ dont les solutions sont les points fixes de $T$. Si on arrive 
	à prouver que $T$ est contractant et que l'on est dans un espace de Banach alors 
	on prouve que cet opérateur admet un et un seul point fixe, soit une et une seule 
	solution pour le problème de Cauchy. De plus, d'où que l'on parte, on convergera 
	vers cette sainte solution.
	
	\subsection{Espace normé, cylindres internes et de sécurité}
	Il faut travailler dans un espace de Banach, c'est-à-dire un espace vectoriel 
	réel normé et complet. Je m'intéresse à des fonctions au moins dérivables, et 
	donc continue : $V = C^0(I,\mathbb{R}^m)$ (où $I$ est un intervalle autour de $t_0$ 
	à préciser) muni de la norme suprémum :
	\begin{equation}
	\| \|_\infty : \vec{y} \rightarrow \|\vec{y}\|_\infty = \sup_{t\in I}\|\vec{y}(t)
	\|
	\end{equation}
	On a maintenant un espace de Banach dont la convergence sera même uniforme. Il faut 
	maintenant définir $E$. Je travaille dans un intervalle centrée sur $t_0$ de demi-
	côté $l$ et $r$. 
	\begin{equation}
	C = [t_0-l, t_0+l]\times\vec{B}(\vec{y_0},\vec{r}) \subseteq \mathcal{U}
	\end{equation}
	De façon préventive, le cylindre sera dit \textit{de sécurité} si $l.\sup_C \|\vec{f}
	\| \leq r$. Autrement dit, le maximum de la pente en valeur absolue ne dépasse pas 
	$r$ ; on n'en sortira pas.\\
	
	Dès lors $E :=$ ensemble des fonctions $y\in C^0(\underbrace{[t_0-l,t_0+l]}_{:= I}, 
	\mathbb{R}^m)$ telles que gph($y$) $\subset C$, le cylindre de sécurité. C'est-à-
	dire : $\forall t \in I : \| y(t) - y_0\| \leq r$. On dit bien que $y(t)$ est à une 
	distance de $y_0$ qui ne dépassera jamais $r$.
	
	\subsection{$T$, opérateur interne et lipschitzien dans $E$}
	La première chose à vérifier est que l'on reste toujours dans $E$
	\begin{enumerate}
	\item $T$ est un opérateur interne à $E$ : $y\in E \rightarrow T(y) \in E$.\\
	Si $y \in C^0(I,\mathbb{R}^m) \Longrightarrow T(\vec{y}) \in C^0(I,\mathbb{R}^m)$
	Comme le graphe de $T(y)$ reste dans $C$ (et on majore l'intégrale comme d'hab) :
	\begin{equation}
	\begin{array}{ll}
	\|y(t)-y_0\| \leq r \Longrightarrow \|T(y)|_t - y_0\| &= \|\int_{t_0}^t f(\tau,y(
	\tau))d\tau\| \\
	 & \leq |t-t_0| \underbrace{\sup_{\tau \in I} \|f(\tau, y(\tau)) \|_m}_{\mu}\\
	 & \leq l\mu\\
	 & \leq r\qquad \text{ (par def. du cylindre de sécurité)}
	\end{array}
	\end{equation}
	
	\item $T$ est lipschitzien (\dots si $f$ l'est par rapport à $y$).\\
	Il faut avant tout que $f$ ne varie pas trop vite par rapport à $y$, c'est à dire 
	que pour un même $t$, la pente ne devrait pas varier trop vite.  Je dois prouver 
	que la distance entre les images est inférieure à une constante*... (def. lips.)
	\begin{equation}
	\begin{array}{ll}
	\| T(y) - T(z)\|_\infty &= \sup_{t\in I} \left\|\int_{t_0}^t(f(\tau,y(\tau))-f(\tau, 
	z(\tau)))d\tau\right\|\\
	 &\leq \sup_{\tau\in I}\|f(\tau,y(\tau))-f(\tau,z(\tau))\|.\sup_{t\in I}|t-t_0|\\
	 &\leq \Lambda.\sup_{\tau\in I}\|y(\tau)-z(\tau)|.l\\	 
	\end{array}
	\end{equation}
	où $\Lambda$ est une constante de Lipschitz de $f_C$ relativement à $y$. On a ici 
	majoré pour faire apparaître la définition d'une fonction lipschitzienne. D'où
	\begin{equation}
	\|T(y)-T(z)\|_\infty \leq \Lambda \|y-a\|_\infty
	\end{equation}
	C'est-à-dire que $T$t est $\Lambda l$-lipschitzien.
	\end{enumerate}
	
	\subsection{$T$ est contractant dans $E^*$ (de haute sécurité)}
	Si $l$ est suffisament petit, on rentre dans les conditions de haute sécurité. Le 
	cylindre :
	\begin{equation}
	C^* [t_0-l^*, t_0+l^*]\times\vec{B}(y_0,r)
	\end{equation}
	est dit de \textbf{haute sécurité} si de plus $\Lambda l^* =: \alpha < 1$.\\
	Si l'on définit l'ensemble $E^*$ des fonctions admissibles associé au cylindre 
	$C^*$ alors $T : E^* \rightarrow E^*$ est une contraction.
	
	\subsection{Théorème d'existence et d'unicité locale}
	Compte-tenu des deux sections précédentes, le principe de contraction de Banach 
	peut s'appliquer. On conclut à l'existence et l'unicité de la solution du problème 
	de Cauchy dont le graphe est inclus dans $C^*$.
	
	
	\theor{\ \\
	Si $ \vec f : {\cal U} \subseteq \mathbb{R} \times \mathbb{R}^m \rightarrow
	\mathbb{R}^m :
 ( t , \vec y) \to \vec f ( t, \vec y) $ 
est continue et lipschitzienne en $\vec y$ au {\bf voisinage} de
 $( t_0 , \vec y_0) \in $ {\rm int }${\cal U}$, 
alors le problème de Cauchy ${\displaystyle  \left \{ \begin{array}{l}
\vec y \:'  = \vec f ( t , \vec y ) \\
\vec y (t_0)  = \vec y_0
\end{array} \right . }$ admet une seule courbe intégrale au
voisinage de $(t_0 , \vec y_0) $.}
	
	\subsection{Théorème d'existence et d'unicité d'une solution maximale}
	C'est possible de l'étendre, mais ce n'est pas vu !
	
	
	
	
	
	
	
	
	
	
	
	
	
	
	
	
	
	
	
	
	
	
	
	
	
	
	
	
	
	
	
	