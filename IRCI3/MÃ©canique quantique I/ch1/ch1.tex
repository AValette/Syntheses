\chapter{Notation de Dirac}
Inclure les notes de Terence
\section{Vecteurs d'état et espace de Hilbert}
Le \textit{vecteur d'état} se dénomme \textit{ket} et est noté :
\begin{equation}
\begin{array}{ll}
\ket{\psi} & \in \mathcal{E}\\
&\in \mathcal{E}_H
\end{array}
\end{equation}
où $\mathcal{E}$ est l'espace des états et $\mathcal{E}_H$ l'espace de Hilbert. 
Notons que $\mathcal{E} \subset \mathcal{E}_H$. Par abus de langage, nous 
désignerons souvent l'espace des états comme étant l'espace de Hilbert, ce qui 
n'est en toute rigueur pas exact ($\mathcal{E}_H$ contient des états non-physiques).
L'espace de Hilbert est un espace complet (si on définit une suite d'état, celle-ci 
convergera vers un état) muni d'un produit scalaire (défini à la section suivante).\\

Pourquoi définir un vecteur d'état ? En physique classique l'état d'un système 
ne pose pas de problèmes particuliers. A l'inverse, en physique quantique, la notion 
même pose déjà un problème, contraignant l'utilisation de vecteurs d'état. La raison 
physique de leur utilisation vient au principe d'incertitude d'Heisenberg. En effet, 
il nous est impossible de décrire la particule par le couple position/impulsion d'où 
la motivation à l'utilisation de ces vecteurs.\\

A la base de la physique, le \textbf{principe de superposition} nous dit que la 
combili (de coefficients complexes) de deux vecteurs d'états, soit deux kets, est 
de nouveau un ket, soit un état 100\% admissible.
\begin{equation}
\ket{\psi_1}, \ket{\psi_2} \in\mathcal{E},\qquad \ket{\lambda_1\psi_1+\lambda_2\psi_2} 
\equiv \lambda_1\ket{\psi_1}+\lambda_2\ket{\psi_2}\qquad \forall \lambda_1,\lambda_2\in
\mathbb{C}
\end{equation}
Il s'agit de la \textit{linéarité de la physique quantique} avec laquelle on peut, par 
exemple, décrire le phénomène d'interférences.


\section{Produit scalaire entre deux kets}
Le produit scalaire entre deux kets se note
\begin{equation}
\bra{\psi_2}\ket{\psi_1}
\end{equation}
\newpage
Les propriétés de bases de ce produit scalaires sont bien connues :
\begin{equation}
\begin{array}{llll}
\bullet & \bra{\psi}\ket{\psi} &= 0\\
\bullet & \bra{\psi_1}\ket{\psi_2} &= \bra{\psi_2}\ket{\psi_1}^*\\
\bullet & \bra{\psi}\ket{\lambda_1\psi_1+\lambda_2\psi_2} &= \lambda_1\bra{\psi_1}\ket{\psi_2}
+\lambda_2\bra{\psi_1}\ket{\psi_2}\quad \forall \lambda_i\in\mathbb{C}.\quad
& \text{Linéarité (à gauche)}\\
\bullet & \bra{\lambda_1\psi_1+\lambda_2\psi_2}\ket{\psi} &= \bra{\psi}\ket{\lambda_1
\psi_1+\lambda_2\psi_2}^* \quad & \text{Antilinéarité (à gauche)}\\
&  &= (\lambda_1\bra{\psi}\ket{\psi_1}+\lambda_2\bra{\psi}\ket{\psi_2})^*\\
&  &= \lambda_1^*\bra{\psi_1}\ket{\psi} +  \lambda_2^*\bra{\psi_2}\ket{\psi}\\
\bullet & \|\psi\| = \|\ket{\psi}\| = \sqrt{\bra{\psi}\ket{\psi}}>0
\end{array}
\end{equation}

Il est intéressant de s'intéresser à la \textit{"représentation"} d'un ket au sein 
d'un espace de Hilbert. Considérons l’exemple suivant (qui reviendra souvent).\\

\textsc{Exemple}\\
Considérons un espace de Hilbert de dimension $n$. Les vecteurs d'états, les ket, ne sont rien 
d'autres que des vecteurs colonnes dans cet espace de dimension $n$. Soit
\begin{equation}
\ket{u} = \left(\begin{array}{c}
u_1\\
u_2\\
\vdots\\
u_n
\end{array}\right),\qquad\qquad \ket{v} = \left(\begin{array}{c}
v_1\\
v_2\\
\vdots\\
v_n
\end{array}\right),\qquad u_1,v_i\in\mathbb{C}
\end{equation}
Le produit scalaire entre ces deux ket est donné par
\begin{equation}
\bra{v}\ket{u} = \sum_{i=1}^n v_i^*u_i = \underbrace{(v_1^*\ v_2^*\ \dots\ v_n^*)}_{(*)}
\left(\begin{array}{c}
u_1\\
u_2\\
\vdots\\
u_n
\end{array}\right)
\end{equation}
On va définir $(*)$ comme étant un \textit{"complémentaire au ket"}, $\bra{v}$ que l'on 
nomme \textit{bra}. Ce bra appartient à un espace dual, ce qui est le sujet de la 
section suivante.


\section{Espace dual $\mathcal{E}^*$, vecteur "bra"}
Le bra est une forme linéaire : c'est une application qui va depuis l'espace des état 
(ou de Hilbert, pas de différence dans ce cours) vers $\varphi(\psi)$, un nombre complexe.
\begin{equation}
\varphi : \ket{\psi}\in\mathcal{E} \leadsto \varphi(\ket{\psi})\ \in \mathbb{C}
\end{equation}
Cette forme linéaire fait correspondre à chaque état un nombre complexe. La superposition 
est également vérifiée d'où le "linéaire".
\begin{equation}
\varphi(\ket{\lambda_1\psi_1 + \lambda_2\psi_2}) = \lambda_1\varphi(\ket{\psi_1})+
\lambda_2\varphi(\ket{\psi_2})\qquad \forall \lambda_1,\lambda_2\in\mathbb{C}
\end{equation}
où $\varphi \in \mathcal{E}^*$.\\

Il semble dès lors intéressant d'introduire un nouvel "objet" : 
\begin{equation}
\left\{\begin{array}{ll}
\varphi \in \mathcal{E}^*\\
\bra{\varphi}
\end{array}\right.
\end{equation}
Il s'agit de l'ensemble de toutes les formes linéaires, ensemble qui forme un espace dual. 
L'intérêt réside dans un isomorphisme : on peut associer à chaque état de l'espace des états 
un bra de l'espace dual.\\

Ceci étant dit, il faut caractériser et montrer comment cette application agit sur les espaces. 
\begin{equation}
\forall \ket{\psi} \in \mathcal{E},\qquad \underline{\varphi(\ket{\psi}) =\bra{\varphi}\ket{\psi}}
\end{equation}
Cette application peut ainsi être écrite comme un produit scalaire. Il s'agit de la forme 
linéaire $\varphi$ qui s'applique à $\psi$ et qui donne un nombre complexe. Il existe une 
autre façon de voir ceci. On peut le voir comme le produit scalaire entre deux ket ou encore 
comme un bra (forme linéaire qui appliquée à un ket qui donnera un complexe) et un ket.\\
 
Comme précisé, il s'agit d'une forme \textbf{linéaire} :
\begin{equation}
\begin{array}{ll}
\varphi(\ket{\lambda_1\psi_1+\lambda_2\psi_2}) &= \bra{\varphi}\ket{\lambda_1\psi_1+\lambda_2\psi_2}\\
&= \lambda_1\bra{\varphi}\ket{\psi_1}+\lambda_2\bra{\varphi}\ket{\psi_2}\\
&= \lambda_1\varphi(\ket{\psi_1})+\lambda_2\varphi(\ket{\psi_2})
\end{array}
\end{equation}
L'espace dual est également un espace de Hilbert : toutes les propriétés de 
linéarité seront retrouvées. Ainsi, toute combili (complexe) de forme 
apparentent à $\mathcal{E}^*$ forme une troisième forme appartenant à 
$\mathcal{E}^*$.
\begin{equation}
\text{Si } \bra{\varphi_1},\bra{\varphi_2}\in\mathcal{E}^*, \text{ alors }\ 
\lambda_1\bra{\varphi_1} + \lambda_2\bra{\varphi_2} \in\mathcal{E}^*\qquad 
\forall \lambda_i\in\mathbb{C}
\end{equation}
On peut ainsi démontrer que $\mathcal{E}^*$ est un espace vectoriel.
\begin{equation}
\begin{array}{ll}
\forall \ket{\psi} : (\lambda_1\bra{\varphi_1} + \lambda_2\bra{\varphi_2})\ket{\psi} 
&= \lambda_1\bra{\varphi_1}\ket{\psi}+\lambda_2\bra{\varphi_2}\ket{\psi} \\
&= \lambda_1\bra{\psi}\ket{\varphi_1}^*+\lambda_2\bra{\psi}\ket{\varphi_2}^*\\
&= (\lambda_1^*\bra{\psi}\ket{\varphi_1}+\lambda_2^*\bra{\psi}\ket{\varphi_2})^*\\
&= \bra{\psi}\ket{\lambda_1^*\varphi_1+\lambda_2^*\varphi_2}^*\\
&= \bra{\lambda_1^*\varphi_1+\lambda_2^*\varphi_2}\ket{\psi}
\end{array}
\end{equation}
Nous avons donc bien un espace vectoriel (ce qui est clairement visualisable 
dans l'équation ci-dessous). La dernière relation applique un certain bra à 
n'importe que $\psi$. En terme de bra, on peut alors écrire
\begin{equation}
\underline{\lambda\bra{\varphi_1} + \lambda_2\bra{\varphi_2} = \bra{\lambda_1^*\varphi_1
+\lambda_2^*\varphi_2}}\qquad \lambda_1,\lambda_2\in\mathbb{C}
\end{equation}


On vient de voir qu'à n'importe quel bras je peux associer un ket. Il serait 
dès lors intéressant de trouver le ket correspondant à ce bra. Mais avant, on va définir 
la notion d'opérateur s'appliquant dans l'espace de Hilbert. \\

Il est possible de se représenter de façon plus précise ce qu'est un bra en 
se souvenant de l'exemple donné avec un espace de Hilbert de dimension $n$. Dans 
un tel espace, un bra n'est qu'un vecteur ligne complexe conjugué.

\subsection{Opérateurs linéaires (agissant dans $\mathcal{E}$)}
Un opérateur linéaire est une application qui fait correspondre un ket à un ket, à 
la différence de la forme qui fait correspondre un ket à un complexe.

\begin{equation}
\ket{\psi} \in \mathcal{E} \leadsto \hat{A}\ket{\psi} \in \mathcal{E}
\end{equation}
Il est coutume d'indiquer les opérateurs linéaires par un chapeau. La sainte 
superposition reste d'actualité :
\begin{equation}
\hat{A}\ket{\lambda_1\psi_1+\lambda_2\psi_2} = \lambda_1\hat{A}\ket{\psi_1}+
\lambda_2\hat{A}\ket{\psi_2}
\end{equation}
Pas mal de propriétés valent la peine d'être énoncées :
\begin{equation}
\begin{array}{llll}
\bullet & (\hat{A}+\hat{B})\ket{\psi} &= \hat{A}\ket{\psi}+\hat{B}+\ket{\psi}\\
\bullet & (\hat{A}.\hat{B})\ket{\psi} &= \hat{A}(\hat{B}\ket{\psi})\qquad & \text{ Opérateur 
produit $\hat{A}.\hat{B}$}
\end{array}
\end{equation}
Nous pouvons voir cet opérateur produit comme une notation efficace. Il ne faut 
cependant pas perdre à l'idée que, en toute généralité, $\hat{A}$ et $\hat{B}$ 
ne commutent pas. On définit alors le commutateur :
\begin{equation}
[\hat{A},\hat{B}] = \hat{A}\hat{B} - \hat{B}\hat{A} \neq 0
\end{equation}
Comme $\hat{A}$ et $\hat{B}$ sont des opérateurs, la différence des opérateurs 
est toujours un opérateur, le commutateur est bien un opérateur. Il jouit des 
propriétés suivantes :
\begin{equation}
\begin{array}{lll}
\bullet & [\hat{B},\hat{A}] &= -[\hat{A},\hat{B}]\\
\bullet & [\hat{A},\hat{B}+\hat{C}] &= [\hat{A},\hat{B}]+[\hat{A},\hat{C}]\\
\bullet & [\hat{A},\hat{B}.\hat{C}] &= \hat{B}.[\hat{A},\hat{C}]+[\hat{A},
\hat{B}].\hat{C}
\end{array}
\end{equation}
On peut montrer qu'un opérateur linéaire peut se représenter comme une 
matrice. Pour l'illustrer, reconsidérons notre précédent exemple.\\

\textsc{Exemple}\\
Soit un espace de Hilbert de dimension $n$. Soit
\begin{equation}
\ket{u} = \left(\begin{array}{c}
u_1\\
u_2\\
\vdots\\
u_n
\end{array}\right),\qquad \ket{v} = \hat{A}\ket{u}\quad ; \left(\begin{array}{c}
v_1\\
v_2\\
\vdots\\
v_n
\end{array}\right) = \underbrace{\left(\begin{array}{ccc}
a_{11} & \dots & a_{1n}\\
\vdots &\ddots &\vdots\\
a_{n1} & \dots & a_{nn}
\end{array}\right)}_{\hat{A}}\left(\begin{array}{c}
u_1\\
u_2\\
\vdots\\
u_n
\end{array}\right)
\end{equation}
De par cette représentation, on peut aisément comprendre que la non-commutation 
vient du fait que les différentes lignes et colonnes de $\hat{A}$ ne peuvent 
être commutées. Intéressons-nous aux éléments de la matrice de cet opérateur.



\section{"Élément de matrice" d'un opérateur $\hat{A}$}
Comme précédemment, définissons un nouvel "objet" :
\begin{equation}
\ket{\psi} \text{ et } \left\{\begin{array}{ll}
\ket{\varphi} &\in \mathcal{E}\\
\bra{\varphi} &\in \mathcal{E}^*
\end{array}\right., \quad \bra{\varphi}\hat{A}\ket{\psi} = \bra{\varphi}(
\hat{A}\ket{\psi})
\end{equation}
Les parenthèses permettent de voir ça "tel un produit scalaire". Revenons 
à notre précédent problème : quel est finalement ce ket ? Lors de l'écriture 
d'un élément de matrice, il serait intéressant de pouvoir le voir comme un 
opérateur appliqué à un ket. Une autre vision est celle d'un opérateur 
qui agit sur un bra, définissant un nouveau bra qui cette fois, agit sur 
$\psi$. Revenons à notre exemple.

\newpage
\textsc{Exemple}\\
Soit
\begin{equation}
\ket{u} = \left(\begin{array}{c}
u_1\\
u_2\\
\vdots\\
u_n
\end{array}\right), \qquad \bra{v} = (v_1^*\ v_2^*\ \dots\ v_n^*), \qquad 
\hat{A}\ket{u} = \left(\begin{array}{ccc}
a_{11} & \dots & a_{1n}\\
\vdots &\ddots &\vdots\\
a_{n1} & \dots & a_{nn}
\end{array}\right)\left(\begin{array}{c}
u_1\\
u_2\\
\vdots\\
u_n
\end{array}\right)
\end{equation}
Nous avons alors
\begin{equation}
\bra{v}\left(\hat{A}\ket{u}\right) = \underbrace{(v_1^*\ v_2^*\ \dots\ v_n^*)\left(\begin{array}{ccc}
a_{11} & \dots & a_{1n}\\
\vdots &\ddots &\vdots\\
a_{n1} & \dots & a_{nn}
\end{array}\right)}_{\bra{?}\ket{u}}\left(\begin{array}{c}
u_1\\
u_2\\
\vdots\\
u_n
\end{array}\right)
\end{equation}



\section{Opérateur adjoint}
A toute opérateur $\hat{A}$, on peut associer un nouvel opérateur noté $\hat{A}^\dagger$. 
Soit $\ket{\psi}$ :
\begin{equation}
\hat{A} \text{ agit dans }\mathcal{E} ; \ket{\psi'} = \hat{A}\ket{\psi}
\label{eq:16}
\end{equation}
Chaque ket est associé à un bra ; dans ce cas ci il s'agit de $\bra{\psi'}$ et
$\bra{psi}$. Existe-t-il une relation entre ces bra ? Mais à quoi cet objet 
correspond-t-il ? Un bra est une forme linéaire, il faut déterminer comment 
agit $\psi'$ sur n'importe quel ket de l'espace.
\begin{equation}
\begin{array}{lll}
\forall \ket{\psi} \in \mathcal{E} : \psi'\left(\ket{\varphi}\right) &\equiv 
\bra{\psi'}\ket{\varphi} & \qquad \text{Prop. p.scal.}\\
&= \bra{\varphi}\ket{\psi'}^*\\
&= \bra{\varphi}\hat{A}\ket{\psi}^* & \qquad \text{Def. de }\psi', \text{ def. op. adj.}\\
&= \bra{\psi}\hat{A}^\dagger\ket{\varphi}& \qquad \text{(*)}
\end{array}
\end{equation}
Pour arriver à $(*)$, on peut remplacer $\hat{A}$ par son adjoint si l'on 
permute les termes et considère le complexe conjugué.
La conclusion de tous cela - modulo la définition de l'opérateur adjoint - est 
que l'on voit que l'on peut réécrire le $\bra{\psi'}$ en terme de $\bra{\psi}$.
\begin{equation}
\underline{\bra{\psi'} = \bra{\psi}\hat{A}^\dagger}
\end{equation}
Cette relation ressemble assez fortement à \autoref{eq:16} ou $\hat{A}\rightarrow
\hat{A}^\dagger$.
De façon générale on peut voir qu'un opérateur linéaire peut être entièrement 
caractérisé par ses éléments de matrice, exactement comme une matrice est 
caractérisée par tous ses éléments. Pour parvenir à ce résultat, nous avons 
utilisé la définition d'un opérateur adjoint :
\begin{equation}
\forall \ket{\psi} \text{ et } \ket{\varphi} \in \mathcal{E},\qquad 
\bra{\psi}\hat{A}^\dagger \ket{\varphi} = \bra{\psi}\hat{A}\ket{\varphi}^*
\end{equation}

\textsc{Exemple}\\
Comme toujours, prenons notre espace de Hilbert de dimension $n$.
\begin{equation}
\bra{v}\hat{A}\ket{u} = \underbrace{(v_1^*\ v_2^*\ \dots\ v_n^*)\left(\begin{array}{ccc}
a_{11} & \dots & a_{1n}\\
\vdots &\ddots &\vdots\\
a_{n1} & \dots & a_{nn}
\end{array}\right)}_{(*)}\left(\begin{array}{c}
u_1\\
u_2\\
\vdots\\
u_n
\end{array}\right)
\end{equation}
Le "but" est que $(*)$ devienne notre nouveau bra $(w_1\ w_2\ \dots\ w_n)$ :
\begin{equation}
\left(\begin{array}{c}
w_1\\
w_2\\
\vdots\\
w_n
\end{array}\right) = \underbrace{\left(\begin{array}{ccc}
a_{11}^* & \dots & a_{1n}^*\\
\vdots &\ddots &\vdots\\
a_{n1}^* & \dots & a_{nn}^*
\end{array}\right)}_{\hat{A}^\dagger}\left(\begin{array}{c}
v_1\\
v_2\\
\vdots\\
v_n
\end{array}\right)
\end{equation}
Pour obtenir le bra, nous avons réalisé une opération semblable à celles 
réalisées en algèbre linéaire, à savoir pris le complexe conjugé de la matrice 
conjugué après inversion et transposée du vecteur.\footnote{Mieux expliciter plz}.
L'opérateur adjoint n'est rien d'autre que de la matrice adjointe. Le fait 
de permuter les lignes et les colonnes ne faisait qu'inverser les bra et ket.
Il en découle des propriétés intéressantes :
\begin{multicols}{2}
\begin{itemize}
\item[$\bullet$] $\left(\hat{A}^\dagger\right)^\dagger = \hat{A}$
\item[$\bullet$] $\left(\lambda\hat{A}\right)^\dagger = \lambda^*\hat{A}$
\item[$\bullet$] $\left(\hat{A}+\hat{B}\right)^\dagger = \hat{A}^\dagger+\hat{B}^\dagger$
\item[$\bullet$] $\left(\hat{A}.\hat{B}\right) = \hat{B}^\dagger.\hat{A}^\dagger$
\end{itemize}
\end{multicols}
A titre d'exercice, démontrons la dernière propriété
\begin{equation}
\begin{array}{ll}
\bra{\psi}\left(\hat{A}.\hat{B}\right)^\dagger\ket{\varphi} &= \bra{\varphi}\hat{A}.
\hat{B}\ket{\psi}^*\\
&= \left(\left(\bra{\varphi}\ket{\hat{A}}\right)\left(\bra{\hat{B}}\ket{\psi}\right)\right)^*\\
&= (\bra{\psi}\hat{B}^\dagger)(\hat{A}^\dagger\ket{\varphi})\\
&= \bra{\psi}\hat{B}^\dagger\hat{A}^\dagger\ket{\varphi}
\end{array}
\end{equation}

\section{Opérateurs hermitiens/auto-adjoints}
Par définition
\begin{equation}
\hat{A} = \hat{A}^\dagger
\end{equation}
Dès lors
\begin{equation}
\begin{array}{ll}
\bra{\psi}\hat{A}\ket{\varphi} &= \bra{\varphi}\hat{A}\ket{\psi}^*\\
\bra{\psi}\hat{A}\ket{\psi} &= \bra{\psi}\hat{A}\ket{\psi}^* \underline{\in \mathbb{R}}
\end{array}
\end{equation}
Énonçons quelques propriétés intéressantes
\begin{equation}
\begin{array}{lll}
\forall \hat{A},\hat{B} \text{ hermitiens },\qquad & \hat{A}+\hat{B} & \text{hermitien}\\
& \hat{A}.\hat{B} & \text{hermitien ssi } [\hat{A},\hat{B}] = 0
\end{array}
\end{equation}
On peut justifier la dernière propriété de la façon suivante :
\begin{equation}
\begin{array}{lll}
(\hat{A}.\hat{B})^\dagger &= \hat{B}^\dagger.\hat{A}^\dagger \\
&= \hat{A}^\dagger.\hat{B}^\dagger & \text{vrai ssi } [\hat{A}^\dagger,\hat{B}^\dagger]=0=-
\underbrace{[\hat{A},\hat{B}]^\dagger}_{=0}\\
&= \hat{A}.\hat{B}
\end{array}
\end{equation}
Le produit position et impulsion n'est pas un opérateur hermitien (ces deux états ne 
commutent pas) ; ce n'est donc pas une quantité observable en physique quantique.