\chapter{Polynômes orthogonaux, fonctions spéciales et résolution d'ED 
par séries de puissances}

\setcounter{section}{16}
\section{Polynômes orthogonaux}
\setcounter{subsection}{1}
\subsection{Espaces à produit scalaire}
Pour toute fonction $w$ continue et positive sur $[a,b]$ :
\begin{equation}
	<a,b> := \int_a^b w\ f\ g^*
\end{equation}
Celui-ci défini un produit scalaire hermition sur l'EV $C^0([a,b], 
\mathbb{K})$ et un presque" produit scalaire sur\footnote{Fonction de 
	carré sommable si $\int_a^b |f|^2 w < +\infty$.} $L^2([a,b],\mathbb{
	K})$.\\
	
Soit $E$ un espace pré-hilbertien et $E_n$ un sous-espace de dimension 
$n$ de $E$. Notons $pr_{E_n}\ f$ la projection de $f$ sur $E_n$.

\proposition{$pr_{E_n}\ f$ existe ($n<\infty$) et est la meilleure 
	approximation de $f\in E$ par un élément de $E_n$ pour $\|\ \|$.}
	
	
\subsection{Polynômes orthogonaux}
Considérons une fonctions poids $w \in C^0(]a,b[,\mathbb{R}_0^+)$ 
telle que $\forall n \in \mathbb{N} : \int_a^b |x|^n w(x)dx$ converge. 
Soit $E$ l'EV des fonctions continues dans $]a,b[$ et de carré sommable 
pour le produit scalaire
\begin{equation}
	<f,g> := \int_a^b f(x)g^*(x)w(x)dx
\end{equation}
Notons $\|\ \|_2$ la norme $L_2$, la norme en moyenne quadratique. Donc 
\begin{equation}
	E = C^0(]a,b[,\mathbb{R})\cap L_2(]a,b[, \mathbb{R})
\end{equation}
Un polynôme $p_n$ de degré $n$ est dit \textbf{unitaire} (ou \textbf{
	monique}) ssi le coefficient de $x^n$ dans $p_n(x)$ vaut 1.\\
	
\theor{\ \\
	$\forall w, \exists 1!$ suite de polynôme unitaire $(p_n)_{n\in\mathbb{N}}
	$ tels que
	\begin{equation}
		\left\{\begin{array}{l}
		\text{deg } p_n = n\\
		p_n \perp p_m\ \text{ si } n \leq m
		\end{array}\right.
	\end{equation}}
	
\begin{proof}\ \\
	$\forall n$ les polynômes $1,x,x^2,\dots,x^n$ sont L.I. dans $\mathcal{P}$. 
	Il s'agit d'une suite de polynômes unitaires de degré 0,1,dots, $n$. En 
	appliquant Gram-Schmidt on obtient une suite satisfaisant aux hypothèses.
	\begin{equation}
		\begin{array}{ll}
			p_0 & := 1                                                     \\
			p_n & := x^n - \text{proj }_{\mathcal{P}_{n-1}} x^n            \\
			    & = x^n - \sum_{k=0}^{n-1} \frac{<x^n, p_k>}{<p_k,p_k>}p_k 
		\end{array}
	\end{equation}
\end{proof}
Voir le théorème 17.31 page 113 également, un peu la flemme.
	
	
	
\theor{\ \\
	$\forall w|_{[a,b]}, p_n$ possède $n$ zéros distincts dans $]a,b[$.}\ 
	
\begin{proof}\ \\
	Soit $x_1, \dots, x_k$ les zéros de $p_n$ dans $]a,b[$ et leurs 
	multiplicités $m_1, \dots, m_k$. On a que $m_1 + \dots + m_k 
	\leq n = $deg $p_n$. Définissons $\epsilon_i := \left\{\begin{array}{ll}
	0 & \text{si $m_i$ pair}\\
	1 & \text{si $m_i$ impair}
	\end{array}\right.$ et construisons le produit suivant 
	\begin{equation}
		q(x) :=\Pi_{i=1}^k (x-x_i)^{\epsilon_i}\qquad \text{ deg } q \leq k \leq n
	\end{equation}
	Je multiplie $p_n$ par $q$ : si l'exposant est impair je l'augmente d'une 
	unité et s'il est pair je ne fais rien. Le polynôme $p_nq$ a pour zéros 
	$x_1, \dots, x_k$ avec comme multiplicité $m_1+\epsilon_1+\dots+m_k+
	\epsilon_k$, toutes paires. Dans $]a,b[\backslash\{x_1,\dots,x_k\}, p_nq$ 
	est de signe constant\footnote{??}. On a alors 
	\begin{equation}
		<p_n, q> = \int_a^b p_nqw \neq 0
	\end{equation}
	Or $p_n \perp \mathcal{P}_{n-1}$ et deg $q \leq n$. On en déduit que deg $q 
	= n \Longrightarrow k = n$ et tous les $m_i = 1$.
\end{proof}
	
	
\theor{\ \\
	$\forall f \in E, \forall n \in \mathbb{N}, \exists 1!$ polynôme $q_n \in 
	\mathcal{P}_n$ tel que $\| f-q_n\| = \min \{\|f-p\|_2; p \in \mathcal{P}_n
	\} =: d_2(f, \mathcal{P}_n)$}\ \\
On dit que $q_n$ est le polynôme de meilleure approximation quadratique de 
$f$ à l'ordre $n$.
\begin{proof}
	Cf. Analyse II
\end{proof}
	
\theor{\ \\
	Si $]a, b[$ est \textbf{borné} (!), alors $\forall f \in E : \lim\limits_{n
		\rightarrow	\infty} \| f - q_n \|_2 = 0$}
Cela signifie que l'ensemble $(p_k)_{k\in\mathbb{N}_0}$ est complet 
relativement à $E$ ; le développement de Fourier va converger quadratiquement.
	
\begin{proof}\ \\
	Pour se simplifier la vie, on travaille dans un fermer et $f$ est continue en 
	$a$ et $b$ : $f \in C^0([a,b])$. Considérons $r_n$, un polynôme de degré au 
	plus $n$ étant la meilleure approximation \textsc{uniforme} de $f$ dans 
	$\mathcal{P}_n$ (alors que $q_n$ est le champion en quadratique). On a donc
	\begin{equation}
		\begin{array}{ll}
			\|f - q_n\|_2^2 \leq \|f-r_n\|_2^2 & = \int_a^b |f-r_n|^2w              \\
			                                   & \leq \| f-r_n\|_\infty^2\int_a^b w 
		\end{array}
	\end{equation}
	En effet\footnote{On peut majorer cette fonction par le suprémum de ce carré 
	et le sortir de l’intégrale. (On majore l'intégrale comme d'hab?)}, par 
	définition $\|f-r_n\|_\infty^2 = \sup_{[a,b]} |f-r_n|^2$. Par le théorème de 
	Weirestrass $\| f - r_n\|_\infty^2 \rightarrow 0$ quand $n \rightarrow\infty$.
\end{proof}
	



\section{Théorème d'approximation de Weierstrass}
\subsection{Fonctions continues comme limites uniformes de polynômes}
\theor{\ \\
	Soit $f : [a,b] \rightarrow \mathbb{R}$ une fonction continue. Pour tout $
	\epsilon>0$, il existe une fonction polynomiale $p$ telle que $\|f-p\|_\infty 
	< \epsilon$}\ \\
Autrement dit, $\forall \epsilon > 0, \exists$ polynôme $p(x)$ tel que 
\begin{equation}
	\forall x \in [a,b] : p(x) - \epsilon \leq f(x) \leq p(x)+\epsilon
	+
\end{equation}
	
\subsection{Suites de Dirac}
On définira l'outil de Dirac 
\begin{equation}
	\varphi_n(x) := \left\{\begin{array}{ll}
	0 & \text{ si } x \notin [-1,1]\\
	\mu_n(1-x^2)^n & \text{ si } x \in [-1,1]	
	\end{array}\right.
\end{equation}
où $\mu_n$ est un réel tel que $\int_{-1}^1 \varphi_n = 1$ et donc 
\begin{equation}
	\int_{-\infty}^\infty \varphi_n = 1
\end{equation}
On peut prouver qu'il $\exists\mathcal{C} : \mu_n \leq \mathcal{C}\sqrt{n}$.\\
	
	
\subsection{Polynôme $p_n$ d'approximation "à la Dirac"}
Supposons que $a$ et $b$ sont compris entre 0 et 1 (par changement de variable 
par exemple), ceci n'est pas restrictif. Définitions le produit de convolution 
suivant (et utilisons l'outil de Dirac) :
\begin{equation}
	\begin{array}{ll}
		\forall \xi \in [a,b] : p_n(\xi) & := \int_0^1 f(x)\varphi_n(x-\xi)dx    \\
		                                 & = \mu_n\int_0^1 f(x)(1-(x-\xi)^2)^ndx 
	\end{array}
\end{equation}
où $(1-(x-\xi)^2)^n$ est un polynôme en $\xi$ de degré $2n$.\\
Deux cas :
\begin{enumerate}
	\item Si $x \notin V_\xi$, alors $\varphi_n(x-\xi).f(x) \approx 0$ au point 
	      que l'$\int = 0$. En effet, $\varphi_n(x-\xi)$ est très centrée autour de 
	      $\xi$. Si on en est loin, le produit est forcément proche de zéro.
	\item Si $x \in V_\xi$, on remplace brutalement $f(x)$ par $f(\xi)$ et on peut 
	      le sortir de l'intégrale. Il ne reste que l'intégrale de $\varphi_n$ près de $x$ 
	      et celle ci $\approx 1$
	      \begin{equation}
	      	\int_{V_\xi} f(x).\varphi_n(x-\xi)dx \approx f(\xi)\int_{V_0}\varphi_n \approx 
	      	f(\xi)
	      \end{equation}
\end{enumerate}
	
\setcounter{section}{8}
\section{L'ELD de Tchebychev}
\subsection{Présentation trigonométrique des polynômes de Tchebychev}
On peut utiliser la formule du binôme de Newton sur le deuxième membre de la 
formule d'Euler $\cos(n\theta) + i\sin(n\thetaà = (\cos\theta + i\sin\theta)^n
$. On obtient alors : 
\begin{equation}
	\cos^n\theta + n\cos^{n-1}\theta(i\sin\theta) + \left(\begin{array}{c}
	n\\
	2
	\end{array}\right) \cos^{n-2}\theta(i\sin\theta)^2 + \dots + (i\sin\theta)^n
\end{equation}
En identifiant les parties réelles des deux membres :
\begin{equation}
	\cos(n\theta) = \text{ combili}(\cos^{n-2k}\theta\underbrace{(i\sin\theta)^{2
		k}}_{(-1)^k(1-\cos^2\theta)^k})
\end{equation}
soit un polynôme en $\cos\theta$.\\
On définit le $n^e$ polynôme de Tchebychev à partir de $\cos n\theta =: T_n(\cos
\theta)$ où $T_n$ est défini sur $[-1,1]$. Autrement dit :
\begin{equation}
	T_n(x) = \cos(n\arccos x)
\end{equation}
Ceci revient par exemple à considérer $T_n(\cos\theta)$ comme le développement de $\cos n\theta$ sous forme de polynôme en $\cos\theta$.
Le polynôme $T_n : \mathbb{R}\rightarrow\mathbb{R}$ de degré $n$ en $x\in\mathbb{R}$ 
est obtenu en prolongeant $T_n|_{[-1,1]}$. Il n'est \textbf{pas} unitaire et est 
normé pour la norme sup.
	
\begin{equation}
	\|T_n\|_{\infty,[-1,1]} = \sup_{x \in [-1,1]}|T_n(x)| = \sup_{\theta \in [0,\pi]} 
	|\cos n\theta| = 1
\end{equation}		
	
La fonction $y = \cos(n\theta)$ est solution de 
\begin{equation}
	\frac{d^2y}{d\theta^2} + n^2y = 0
\end{equation}
En posant $x:= \cos\theta \rightarrow \tilde{y}(x) = y(\cos n\theta)$ on obtient 
l'\textbf{EDL de Tchebychev}
\begin{equation}
	(1-x^2)\frac{d^2\tilde{y}}{dx^2} - x\frac{d\tilde{y}}{dx} + n^2\tilde{y} = 0
\end{equation}
Une telle EDL (si $n^2 \in \mathbb{R}^+$) admet une solution $\tilde{y}$ dont la 
dérivée est bornée pour $x\rightarrow 1 \Leftrightarrow n^2$ est un carré parfait 
$\Leftrightarrow \tilde{y}$ est un polynôme de Tchebychev de degré $n$.
	
\setcounter{subsection}{3}
\subsection{Quelques propriétés des polynômes de Tchebychev}
Voir syllabus page 54 et slide 26.\\
Les polynômes unitaires de Tchebychev sont $2^{1-n}T_n$, c'est-à-dire que $\|T_n
\|_{\infty, [-1,1]} = \|\cos(n\theta)\|_{\infty, [0,\pi]} = 1$. Donc 
\begin{equation}
	\|2^{1-n}T_n\|_{\infty,[-1,1]} = 2^{1-n} = \left(\frac{1}{2}\right)^{n-1}
\end{equation}
Ce polynôme réalise le minimum pour cette norme, c'est le "champion" : celui 
possédant la plus petite distance dans l'espace des polynômes unitaires est 
donné par Tchebychev.\\
On en tire la \textit{Proposition du \textbf{minimax}}\\
\proposition{Dans $\mathcal{P}\mathcal{U}_n$, les polynômes unitaires de degré 
	$n > 0$ :
	\begin{enumerate}
		\item $2^{1-n}T_n$ minimise $\|\ \|_{\infty,[-1,1]}$ dans l'ensemble des 
		      polynômes unitaires.
		\item $\|2^{1-n}T_n\|_{\infty,[-1,1]} = 2^{1-n} = \left(\frac{1}{2}\right)^{n-1}$
	\end{enumerate}
}
\begin{proof} [Démonstration par l'absurde]\ \\
	Supposons $P$, un polynôme unitaire dont la norme est plus petite, encore 
	meilleur que celui de Tchebychev : $\|P\|_{\infty, [-1,1]} < 2^{1-2}$. L'équation
	\begin{equation}
		2^{1-n}T_n(x) = 2^{1-n}\cos n\theta
	\end{equation}
	prends les valeurs $2^{1-n}, -2^{1-n}, 2^{1-n}, \dots, (-1)^n2^{1-n}$ aux points 
	$\theta = 0, \pi/n, 2\pi/n, \dots, n\pi/n$. Comme $P$ est la meilleure 
	approximation, la fonction $Q(x)$ ci-dessous est forcément positive (l'autre 
	terme étant forcément plus grand)
	\begin{equation}
		Q(x) := 2^{1-n}T_n(x) - P(x)
	\end{equation}
	$Q(x)$ a le même signe que $2^{1-n}$ en ces points, donc $Q$ à au moins $n$ 
	zéros sur $[-1,1]$. Or, deg $Q \leq n-1.$ (car $2^{1-n}T_n$ et $P$ sont unitaires, 
	du coup le terme avec le facteur 1 disparaît d'office : degré $n-1$). On a 
	contradiction.
\end{proof}
	

\setcounter{section}{18}
\section{Quadratures et polynômes orthogonaux}
\setcounter{subsection}{6}
\subsection{Quadrature par interpolation}
On préfère le mot quadrature à interpolation. Ces méthodes consistent à 
approcher l'intégrale $\int_a^b fw$ par une formule d'interpolation
\begin{equation}
	\int_a^b fw \approx \sum_{k=0}^l A_k f(x_k)
\end{equation}
où les $A_k$ sont les coefficients d'interpolation.
\subsection{Méthode de quadrature $M_{l,X}$}
Supposons que l'on possède $l$ points $x_0,\dots, x_l$. Soit $P_l(x_k) \approx 
f(x)$, le polynôme d'interpolation de degré $\leq l$ par les $(x_k, f(x_k))$. 
Au différents points connus, $f(x_k) = P_l(x_k) \forall k =0,\dots,l$.\\
On associe un \textit{polynôme de Lagrange} aux points $x_0,\dots,x_l$. Soit 
$L_k(x) :=$ le polynôme de degré $l$ tel que $\left\{\begin{array}{ll}
L_k(x_k) &= 1\\
L_k(x_j) &= 0\ \forall j \neq k
\end{array}\right.$ On l'écrit 
\begin{equation}
	L_l(x) = \prod_{j=0,\dots,l;j\neq k} \frac{x-x_j}{x_k-x_j}
\end{equation}
Le \textbf{polynôme d'interpolation} est combili des polynômes de Lagrange 
\begin{equation}
	P_l(x) = \sum_{k=0}^l f(x_k)L_k(x)
\end{equation}
Par exemple, pour $l=1 : P_1(x) = f(x_0)L_0(x) + f(x_1)L_1(x)$. Cette fonction 
vérifie bien les points donnés : $P_1(x_0) = f(x_0)*1 + 0, \dots$. Cette fonction 
$P_l(x)$ va alors servir à approcher des fonctions dont on connaît une série 
de points.\\
Ici l'idée est d'approcher $\int_a^b fw$ par $\int_a^b P_lw$ :
\begin{equation}
	\begin{array}{ll}
		\int_a^b fw \approx \int_a^b P_l(x)w(x) dx & = \int_a^b \sum_{k=0}^l f(x_k)              
		L_k(x)w(x)dx\\
		                                           & = \sum_{k=0}^l f(x_k)\int_a^b L_k(x)w(x) dx \\
		                                           & = \sum_{k=0}^l f(x_k)\ A_k := M_{l,X}(f)    
	\end{array}
\end{equation}
Il s'agit de la méthode $M_{l,\{x_0,\dots,x_l\}}$. L'approximation est fournie 
par la somme ci-dessous où les $A_k$ sont calculés indépendamment de $f$.\\
	
\lemme{L'ordre de toute méthode de quadrature $M_{l,X}$ (par interpolation) est 
	$\geq l$.}\ \\
Cela signifie que si on prend un polynôme de degré au plus $l$, l'approximation 
sera parfaite.
	
\begin{proof}\ \\
	Les points $x_0, \dots, x_l$ sont donnés. Si $f$ est un polynôme de degré $\leq 
	l$, alors $f$ \textbf{est} le polynôme d'interpolation de $f$ par les points 
	$x_0,\dots, x_l$. Donc
	\begin{equation}
		\int_a^b fw = \sum_{k=0}^l f(x_k)A_k
	\end{equation}
\end{proof}
	
\subsection{Méthode de Gauss}
\proposition{La méthode de quadrature par interpolation $M_{l,X}$ est d'ordre 
	$2l+1$ ssi les points d'interpolations $x_0,\dots,x_l$ sont les zéros du 
	$(l+1)^{\text{ème}}$ polynôme (noté $P_{l+1}$) dans la suite des polynômes 
	unitaires orthogonaux 	de degrés croissants pour le produit scalaire $<f,g> 
	= \int_a^b fgw$.}\ \\
Si $w=1$, on parlera de la \textit{méthode de Gauss}.
	
\begin{proof}\ \\
	Définissons le polynôme unitaire $\alpha$ de degré $l+1$ : $\alpha(x) := (x-x_0)
	(x-x_1)	\dots(x-x_l) \perp_w \mathcal{P}_l$. Par hypothèse si l'ordre $\geq 2l+1$ 
	j'obtiens la solution exacte :
	\begin{equation}
		\forall f \in \mathcal{P}_{2l+1} : \int_a^b fw = \sum_{k=0}^l f(x_k)A_k\qquad 
		\text{ où } A_k = \int_a^b L_k(x)w(x)dx
	\end{equation}
	Notons que comme $\alpha'(x) = \prod_{j\neq l} (x_k-x_j)$, on a 
	\begin{equation}
		L_k(x) = \frac{\alpha(x)/(x-x_k)}{\alpha'(x)}
	\end{equation}
	Si $Q \in \mathcal{P}_l$ un polynôme quelconque\footnote{Attention, $\mathcal{P}_l$ 
		ne désigne pas seulement les polynômes unitaires!} de degré $\leq l : \alpha Q \in 
	\mathcal{P}_{2l+1}$, d'où
	\begin{equation}
		\int_a^b \alpha Q w = \sum_{k=0} \underbrace{\alpha(x_k)}_{=0}Q(x_k)A_k = 0
	\end{equation}
	$\Longrightarrow \forall Q \in \mathcal{P}_l : \alpha \perp_w Q \Rightarrow 
	\alpha \perp_w \mathcal{P}_l \Rightarrow \alpha = P_{l+1}$.
\end{proof}
	
	
\proposition{Toute méthode de quadrature par interpolation en $x_0, \dots, 
	x_l$ est d'ordre $\leq 2l+1$.}	\ \\
	
\begin{proof} [Démonstration par l'absurde]\ \\
	Par le même raisonnement : $\alpha \perp_m \mathcal{P}_{l+1}$ et donc $\alpha 
	\perp_m \alpha$, ce qui est absurde.
\end{proof}
	
\theor{\ \\
	Les méthodes de quadrature par interpolation polynomiales en $l+1$ points 
	sont d'ordre $\sigma : l \leq \sigma \leq 2l+1$.}\ \\
	
\begin{proof}
	Voir les trois lemmes précédents.
\end{proof}
	
\theor{\ \\
	La méthode $M_{l,X}$ est d'ordre maximal $(2l+1)$ ssi les points d'interpolation 
	$x_0, \dots, x_l$ sont les zéros du polynôme $P_{l+1}$ de degré $l+1 \perp_m 
	\mathcal{P}_l$ pour le produit scalaire\dots}\ \\

\begin{proof}
	Voir lemmes précédents	
\end{proof}
	
\proposition{Si $\alpha \perp_m \mathcal{P}_l$, alors la méthode est d'ordre 
	$\geq 2l+1$.}
	
\begin{proof}
	Voir page 136.
\end{proof}
	
	
	
	
	
	
	
	