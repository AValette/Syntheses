
\section*{TP 7 : Composition de moments cinétiques}
\begin{itemize}
	\item $\vec{L} = \vec{r}\times \vec{p} = -i\hbar (\vec{r}\times \vec{\nabla})$
	
	\item $\vec{S} = spin$ et $s$ est le nombre quantique de spin qui soit 
	\begin{equation}
		s \in \frac{\mathbb{N}}{2} = \left\{ \frac{1}{2}, \frac{3}{2},\dots \right\} \, (fermions) \qquad soit \qquad s \in \mathbb{N} \, (bosons)
	\end{equation}
	\item $m_s$ : projection du spin sur l'axe z et varie par pas de 1 entre $-s \leq m_s \leq s$
	
	\item $l$ : nombre quantique de moment cinétique orbital ($l \in \mathbb{N}$)
	
	\item $m$ : nombre quantique magnétique avec $m \in \mathbb{Z} \ |\, -l \leq m \leq l$
	
	\item Pour un nombre qjuantique quelconque $\underbrace{\vec{j}}_{j,m} = \underbrace{\vec{j_1}}_{j_1,m_1}+\underbrace{\vec{j_2}}_{j_2,m_2}$ alors on utiilse la \textbf{relation triangulaire} 
	\begin{equation}
	|j_1 -j_2| \leq j \leq j_1+j_2
	\end{equation}
	
	\item Le nombre de valeur possible pour $m$ est toujours $2 j +1$ valeurs
	
	\item Dans le cas d'un électron plongé dans un champ magnétique, il faut tenir compte de l'action de ce champ dans le hamiltonien en rajoutant un terme $W$ au hamiltonien de l'état non perturbé $H_0$
	\begin{equation}
		H = H_0 + W
	\end{equation}
	avec $W = -\vec{M}\vec{B}$ et $\vec{M} = -\frac{e}{2m_e}(\vec{L}+g.\vec{S})$. g est le facteur gyromagnétique $\approx 2$
\end{itemize}