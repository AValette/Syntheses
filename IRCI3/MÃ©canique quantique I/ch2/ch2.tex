\chapter{Principes fondamentaux de la mécanique quantique}

 Nous allons ici reformaliser les bases en nous basant sur le formalisme de Dirac. 
 On souhaite décrire l'\textit{état} du système, la \textit{mesure} et
 l'\textit{évolution} temporelle. Nous avions vu au début du premier chapitre 
 que la façon de définir une mesure était quelque peu particulière. Nous 
 allons ici nous baser sur l'interprétation de \textsc{Copenhagen} (Niels Bohr) mais 
 il faut savoir qu'il y en existe d'autre (\textsc{Bohm} (interprétation de l'onde 
 pilote), interprétation des \textsc{mondes multiples}, \dots).\\

  Cette interprétation 
 pose des  problèmes d'interprétation mais est bonne d'un point de vue pragmatique (
 les résultats expérimentaux correspondent très bien avec la théorie). Dès lors, si 
 l'on n'essaye pas d'interpréter la chose, tout fonctionne très bien\\
 
 Dans ce cours nous allons nous limiter aux \textit{états purs} : idéalisation 
 de la description s'il n'y a aucun bruit. Il existe évidemment des 
 \textit{états mixtes} (voir cours MA1) qui tient compte du bruit. 
 
 \section{1er principe : État d'un système}
 \subsection{Premier principe}
 Un état sera défini par un ket $\ket{\psi(t)} \in \mathcal{E}_H$. Cet 
 état doit être normé ($\|\psi(t)\|^2 = 1 \forall t$). Ceci a pour 
 conséquence immédiate le principe de superposition, toute combili d'état 
 étant un état possible
 \begin{equation}
 \ket{\psi} = \sum_i\ket{\psi_i},\qquad \sum_i |c_i|^2 = 1
 \end{equation}
 En effet, ceci est nécessaire pour la normalisation de la fonction d'onde
 \begin{equation}
 \begin{array}{ll}
 \|\psi\| &= \sum_{ij} c_j^*c_i\overbrace{\bra{\psi_j}\ket{\psi_i}}^{\delta_{ij}}\\
 &= \sum_i |c_i|^2 \\
 &= 1
 \end{array}
 \end{equation}
 L'état d'un système est déterminé à une indétermination de phase près : on 
 défini la phase globale
 \begin{equation}
 e^{i\delta}\ket{\psi}
 \end{equation}
 Cet état sera totalement indistinguable de l'état $\ket{\psi}$. Cette phase 
 globale disparaît d'ailleurs dans le traitement plus général des états mixtes. 
 Cette phase globale n'a pas d'interprétation physique. Par contre un phase 
 locale pondérant les différents états d'une superposition est pourvue de 
 sens physique (interférences)
 \begin{equation}
 \sum_j c_je^{i\delta_j}\ket{\psi_j}\quad\neq\quad \sum_j c_j\ket{\psi_j}
 \end{equation}
 
 Avant de s'attaquer à un problème, il faut s'intéresser au nombre de 
 degrés de liberté du système.
 \begin{equation}
 \text{Degré de liberté}\quad \rightsquigarrow\quad \mathcal{H}
 \end{equation}
 où $\mathcal{H}$ est l'espace de Hilbert. A chaque degré de liberté, on 
 confond un espace de Hilbert donnant lieu à des produits tensoriels d'espace 
 de Hilbert.
 
 \subsection{Structure de l'espace de Hilbert}
 Prenons l'exemple d'une particule à une dimension. En terme de fonction d'onde
 (qui se traduit aisément en notation de Dirac)
 \begin{equation}
 \psi(x) = \sum c_n\phi_n(x)\quad \rightarrow\quad \ket{\psi} = \sum_n c_n
 \ket{\phi_n}
 \label{eq:3.7}
 \end{equation}
 En étudiant les fonctions propres de l'Hamiltonien on peut écrire la fonction 
 d'onde sous la forme \autoref{eq:3.7}.\\
 A deux dimensions
 \begin{equation}
 \psi(x,y) = \sum_{n,m} c_{n,m}\phi_n(x)\phi_m(y)\quad\rightarrow\quad
 \ket{\psi} = \sum_{n,m} c_{n,m} \underbrace{\ket{\phi_n}\otimes\ket{\phi_m}}_{(*)}
 \end{equation}
 où $(*)$ est un produit tensoriel entre deux ket. Il en résultera un autre ket, 
 mais appartenant à un espace de Hilbert plus grand (ceci est équivalent à 
 $|\phi_n\ket{\phi_m}$. Plus généralement, on peut également exprimer une base :
  cela pourrait être n'importe quel fonction de la base multipliée par une autre. Ce p
 produit forme une base des fonctions d'ondes à deux dimension. \\
 
 Arrêtons avec les exemples et considérons deux ket de deux espaces de Hilbert
 \begin{equation}
 \left.\begin{array}{ll}
 \ket{u} \in \mathcal{E}_H\\
 \ket{v} \in \mathcal{F}_H 
 \end{array}\right\}\quad\rightarrow\quad \ket{u}\otimes\ket{v} \in\mathcal{G}_H 
 \equiv \mathcal{E}H_\otimes\mathcal{F}_H
 \end{equation}
 où $\otimes$ désigne le produit tensoriel. De façon encore plus générale :
 \begin{equation}
 \left.\begin{array}{l}
 \text{Si } \ket{e_n} \text{forme une base de } \mathcal{E}_H\\
 \text{Si } \ket{f_n} \text{forme une base de } \mathcal{E}_F 
 \end{array}\right\}\Rightarrow \left\{\ket{e_n}\otimes\ket{f_n}\right\} \begin{array}{l}
 \text{ forme  une base de l'espace de Hilbert }\\
 \text{  de produit $\mathcal{E}_H\otimes\mathcal{E}_F$}
 \end{array}
 \end{equation}
 On peut ainsi écrire tout $\psi$
 \begin{equation}
 \ket{\psi} = \sum_{n,m} \ket{e_n}\otimes\ket{f_m}
 \end{equation}
 
 Il en découle une série de propriétés. Par exemple
 \begin{equation}
 \dim(\mathcal{E}_F\otimes\mathcal{F}_H) = \dim(\mathcal{E}_F).
 \dim(\mathcal{F}_H)
 \end{equation}
 Le produit sera simplement le produit scalaire espace par espace
 \begin{equation}
 11
 \end{equation}
 Même si cet état est parfaitement possible, certains états ne peuvent 
 \textbf{jamais} s'écrire sous cette forme la. C'est le principe d'\textit{intrication 
 quantique }: l'état quantique ne peut pas se voir comme un produit 
 tensoriel c'est-à-dire une situation ou on ne peut pas décrire les deux particules 
 de façon séparées. Il faudrait les décrire simultanément et l'on n'arriverait donc jamais \\
 à écrire $\psi$ sous cette forme.\\
 
 Petite note supplémentaire: on peut considérer cet exemple de produit tensoriel dans 
 le cas d'un système à deux degrés de liberté discrets.
 \begin{equation}
 \begin{array}{ll}
 \ket{u} &= \left(\begin{array}{c}
 u_1\\
 u_2
 \end{array}\right)\\
 &\\
  \ket{v} &= \left(\begin{array}{c}
 v_1\\
 v_2
 \end{array}\right) 
 \end{array}\qquad \ket{u}\otimes\ket{v} = \left(\begin{array}{c}
 u_1v_1\\
 u_1v_2\\
 u_1v_3\\
 u_2v_1\\
 u_2v_2\\
 u_2v_3  
 \end{array}\right)
 \end{equation}
 Le produit tensoriel donne bien lieu à toutes les combinaisons possibles.
 
 
 \section{2e principe : Mesure}
 \subsection{Observable}
 A toute grandeur physique mesurable $A$ on peut associer $\hat{A}$ un 
 opérateur linéaire hermitien qui agit dans $\mathcal{E}_H$. Ceci étant dit 
 nous savons que pour toute grandeur mesurable il doit exister un certain opérateur
 hermitien. Ceci ne dit rien sur cet opérateur, mais les règles de 
 correspondances permettent de passer d'un opérateur classique à quantique. 
 Tout ce qui existe en classique existe en quantique, l'inverse n'est pas
 vrai (spin).
 
 \subsection{Principe de quantification}
 Les seuls résultats de la mesure de l’observable $\hat{A}$ sont les 
 valeurs propres $a_n$ de l'observable. Ceci est tout aussi vrai pour les 
 systèmes liés (boîte bien quantifiée) que pour les non liés (cas plus classique ou 
 continuum d'état, tout est observable).
 
 \subsection{Principe de décomposition spectrale}
 La probabilité d'obtenir un résultat $a_n$ est donné par l'élément de 
 matrice diagonale de l'opérateur projection associé
 \begin{equation}
 \underline{\mathbb{P}(a_n) = \bra{\psi}\hat{P_n}\ket{\psi}}
 \end{equation}
 avec, pour rappel 
 \begin{equation}
 \hat{A}\ket{\psi_n^i} = a_n\ket{\psi_n^i},\qquad \hat{P_n} = \sum_{i=1}^{g_n} 
 \ket{\psi_n^i}\bra{\psi_n^i}
 \end{equation}
 De façon équivalente, en substituant $\hat{P}_n$, on obtient
 \begin{equation}
 \mathbb{P}(a_n) = \sum_{i=1}^{g_n} \left|\bra{\psi_n^i}\ket{\psi}\right|^2
 \end{equation}
 Si $g_n = 1$, on retombe sur la \textbf{règle de Born}
 \begin{equation}
 \mathbb{P}(a_n) = |\bra{\psi_n}\ket{\psi}|^2
 \end{equation}
 Mais que se passe-t-il directement après la mesure?
 
 \subsection{Réduction du paquet d'onde}
 Juste après la mesure, le système dans un nouvel état $\ket{\psi'}$ (qu'il 
 faut normaliser):
 \begin{equation}
 \begin{array}{ll}
 \ket{\psi'} &= \dfrac{\ket{\psi_n}}{\|\psi_n\|}\qquad\qquad \text{où (*) } \ket{\psi_n} 
 = \hat{P_n}\ket{\psi}\\
 &= \dfrac{\ket{\psi_n}}{\sqrt{\bra{\psi}\hat{P_n}\ket{\psi}}} = \dfrac{\ket{\psi_n}}{
 \sqrt{\mathbb{P}(a_n)}}
 \end{array}
 \end{equation}
 Petite remarque sur (*) : il s'agit de la projection sur l'espace propre associé à 
 l'état mesuré. Cette projection donne lieu à un nouveau ket donnant une probabilité. Si 
 celui-ci est normalisé, il s'agit de la \textit{réduction du paquet d'onde}.\\
 
 Compte-tenu de ceci, on peut ré-écrire la probabilité\footnote{Par identification avec 
 la première égalité, $\|\psi_n\| = \sqrt{\mathbb{P}(a_n)}$.}
 \begin{equation}
 \underline{\mathbb{P}(a_n) = \|\psi_n\|^2}
 \end{equation}
 
 Au niveau de l'interprétation : comment interpréter ces probabilités 
 qui apparaissent ? Il s'agit d'un problème toujours non résolu : les probabilités 
 observées sont liées à la connaissance du système (la fonction d'onde est objet de type 
 théorie des probabilités) ou faut il comprendre la fonction d'onde comme un objet 
 physique existant, comme une onde EM ? \\
  
 La limite de la fréquence d'apparition de $a_n$ n'est rien d'autre que 
 $\mathbb{P}(a_n)$. Pour une mesure particulière la mécanique quantique ne 
 donne pas précisément  la valeur observée mais juste la "chance" de pouvoir 
 l'observer. C'est la raison pour laquelle notamment Einstein disait que la 
 mécanique quantique était incomplète : les probabilités ne feraient que 
 cacher un mécanisme sous-jacent. Aujourd'hui, nous savons que ceci n'est pas 
 correct : il n'existe pas de variable cachée dont on ne connaît pas la 
 mécanique (démontrable expérimentalement) : ces propriétés sont intrinsèques 
 à cette théorie.\\
 
 Remarquons que la probabilité correspond bien aux trois axiomes des probabilités
 \begin{enumerate}
  \item \begin{equation}
 \sum_n \mathbb{P}(a_n) = \sum\bra{\psi}\hat{P_n}\ket{\psi} = \bra{\psi}\overbrace{
 \sum_n \hat{P_n}}^{\mathbb{1}}\ket{\psi} = \bra{\psi}\ket{\psi} = 1
 \end{equation}
 \item\begin{equation}
 \mathbb{P}(a_n) = \sum_{i=1}^{g_n} \underbrace{\left|\bra{\psi_n^i}\ket{\psi_n}
 \right|^2}_{\geq 0} \geq 0
 \end{equation}
 \item  
 \begin{equation}
 \mathbb{P}(a_n) \leq 1 \text{ par propriété de l'opérateur projecteur (valeurs propres : 0,1.)}
 \end{equation}
 \end{enumerate}
 Que se passe-t-il si l'on mesure immédiatement après la première mesure, la 
 même observable ($\longrightarrow$ signifie l'application de $\hat{A}$) ?
 \begin{equation}
 \begin{array}{ll}
 \ket{\psi} \longrightarrow \ket{\psi'} = \dfrac{\ket{\psi_n}}{\|\psi_n\|} \longrightarrow
 \mathbb{P}'(a_n) &= \bra{\psi'}\hat{P_n}\ket{\psi'}\\
 &= \frac{1}{\|\psi_n\|^2}\bra{\psi_n}\hat{P_n}\ket{\psi_n}\\
 &= \frac{1}{\|\psi_n\|^2}\bra{\psi_n}\ket{\psi_n} = \frac{1}{\|\psi_n\|^2}\|\psi_n\|^2=1
\end{array}
 \end{equation}
 car l'opérateur projecteur est idempotent. Si l'on effectue deux fois la même mesure, on 
 est ainsi certain de retrouver la même mesure si 'on effectue la seconde mesure immédiatement 
 après la première.
 
 \subsection{Reproductibilité de la mesure}
 Imaginons que pour un système dans un état quelconque, le résultat d'une mesure soit $a_n$. 
 Pour que ceci ai un sens, comme nous venons de le voir, il faut que si on effectue une 
 mesure directement après la première, celle-ci soit identique.\\
 
 On définit la valeur moyenne de l'observable $\hat{A}$
 \begin{equation}
\begin{array}{lll}
 \langle a\rangle = \sum_n a_n\mathbb{P}(a_n) &= \sum_n a_n\bra{\psi}\hat{P_n}\ket{\psi} &=  
 \bra{\psi}\left(\sum_n a_n\hat{P_n}\right)\ket{\psi}\\
 &= \bra{\psi}\hat{A}\ket{\psi} &= \langle \hat{A}\rangle_\psi \equiv \langle\hat{A}\rangle
\end{array}
 \end{equation}
 Ceci désigne un valeur moyenne, c'est l'élément de matrice diagonal de 
 l'observable $\hat{A}$. On peut montrer en partant de cette 
 expression de la valeur moyenne que seules les valeurs propres peuvent apparaissent 
 comme mesure.\\
 Définissons la variance :
 \begin{equation}
 \begin{array}{ll}
 \langle a\rangle = \bra{\psi}\hat{A}\ket{\psi},\qquad  \langle a^2\rangle = \bra{\psi}\hat{A}^2
 \ket{\psi}, \qquad \Delta a^2 &=  \langle a^2\rangle - \langle a\rangle^2\\
 &= \bra{\psi}\hat{A}^2\ket{\psi} - \bra{\psi}\hat{A}\ket{\psi}^2
 \end{array}
 \end{equation}
 Comme nous avons 
 \begin{equation}
 \ket{\psi} = \sum_n c_n \ket{\psi_n}\quad\text{ où }\quad \hat{A}\ket{\psi_n}=a_n\ket{\psi_n}
 \end{equation}
 On peut alors écrire
 \begin{equation}
 \begin{array}{ll}
 \langle a \rangle &= \left(\sum_{n'} c_{n'}^*\bra{\psi_{n'}}\right)\hat{A}\left(
 \sum_n c_n\ket{\psi_n}\right)= \sum_{n,n'} c_{n'}^*c_n a_n\delta_{n,n'} = \sum_n |c_n|^2 a_n\\
 \langle a^2 \rangle &= \left(\sum_{n'} c_{n'}^*\bra{\psi_{n'}}\right)\hat{A}\left(
 \sum_n c_n\ket{\psi_n}\right)= \sum_{n,n'} c_{n'}^*c_n a_n^2\delta_{n,n'} = \sum_n |c_n|^2 a_n^2 
 \end{array}
 \end{equation}
 où $|c_n|^2 = p_n$. En utilisant ces expression, on peut ré-écrire la variance
 \begin{equation}
 \Delta a^2 = \langle a^2\rangle - \langle a\rangle^2 = \sum_n |c_n|^2 a_n^2 - 
 \left(\sum_n |c_n|^2 a_n\right)^2
 \end{equation}
 Annulons cette expression%Pq le 2e ssi ??
 \begin{equation}
 \Delta a^2 = 0\quad \Leftrightarrow\quad |c_n|^2 = \delta_{n,m} \Leftrightarrow
 \ket{\psi} = \ket{\psi_m}
 \end{equation}
 Ceci signifie que $|c_n|^2$ vaudra 1 en un point $m$ et 0 sinon.
 Les seuls état à donner une variance nuls sont les états propre.  Lors d'une 
 seconde mesure, la variance doit forcément être nulle : ceci montre que seules 
 les valeurs propres peuvent être observée et qu'il faut que juste après la 
 mesure, on ai l'état propre de la grandeur mesurée\\
 
 De façon générale, si on prend toujours pour acquis que la valeur moyenne 
 d'une quantité physique est donné par l'élément de matrice diagonal 
 de l'opérateur en question :
 \begin{equation}
 \langle a^m\rangle = \bra{\psi}\hat{A}^m\ket{\psi} = \sum_n |c_n|^2 a_n^m\qquad
  \forall m
 \end{equation}
 On reconnaît l'expression du moment d'ordre $m$ d'une distribution de probabilité 
 classique. Comme ceci est vrai pour tout $m$, alors forcément
 \begin{equation}
 \mathbb{P}(a_n) = p_n = |c_n|^2
 \end{equation}
 La probabilité est donnée par le module carré du coefficient, tout ça 
 en faisant une seule hypothèse.
  
 \subsection{Relation d'incertitude de Heisenberg}
 Initialement, considérons un ket $\ket{\psi}$ ainsi que deux observables qui 
 à priori ne commutent pas
 \begin{equation}
 \begin{array}{lll}
 \hat{A}\rightarrow\hat{A}'\qquad\qquad &\langle a\rangle &= \bra{\psi}\hat{A}\ket{\psi}\\
 \hat{B}\rightarrow\hat{B}'\qquad\qquad &\Delta a^2 &= \bra{\psi}\hat{A}^2\ket{\psi}
 -\bra{\psi}\hat{A}\ket{\psi}^2
 \end{array}
 \end{equation}
 où $\hat{A}' = \hat{A}-\langle a\rangle \leftrightarrow \langle a'\rangle=0$. Remarquons 
 \begin{equation}
 \begin{array}{ll}
 \left(\Delta a'\right)^2 &= \bra{\psi}\left(\hat{A}-\langle a\rangle\right)
 \left(\hat{A}-\langle a\rangle\right)\ket{\psi} -\langle a'\rangle\\
 &= \bra{\psi}\hat{A}^2\ket{\psi} - \langle a\rangle\bra{\psi}\hat{A}\ket{\psi}-\langle a
 \rangle\underbrace{\bra{\psi}\hat{A}\ket{\psi}}_{\langle a\rangle} + \langle a\rangle^2\bra{\psi}\ket{\psi}\\
 &= \bra{\psi}\hat{A}^2\ket{\psi} - \langle a\rangle^2 = \Delta a^2
 \end{array}
 \end{equation}
 Ceci montre que la variance de $a'$ correspond à celle de $a$ (la variance reste 
 inchangée pour une translation). Le même résultat peut être obtenu pour $\hat{B}'$.\\
 
 
 Nous allons maintenant préparer un grand nombre de systèmes. Sur une partie de ceux-ci, 
 observons $\hat{A}$ ou $\hat{A}'$ et sur l'autre $\hat{B}$ ou $\hat{B}'$ afin d'en 
 déduire les variances. L'objectif est de montrer que ces deux variances sont liées et 
 finalement, qu'elles ne peuvent être petites simultanément.\\
 Passons par une astuce mathématiques en définissant un opérateur linéaire mais pas 
 forcément hermitien :
 \begin{equation}
 \hat{C} \equiv \hat{A}' + i\lambda \hat{B}',\qquad
  \hat{C}^\dagger \equiv \hat{A}' - i\lambda \hat{B}'\qquad\qquad\lambda\in\mathbb{R}
 \end{equation}
 Omettons ici les $\hat{\ }$  afin d'éviter trop de lourdeur. Nous avons
 \begin{equation}
 \begin{array}{ll}
 \|C\ket{\psi}\|^2 &= \bra{\psi}C^\dagger C\ket{\psi}\\
 &= \bra{\psi}\left(A'-i\lambda B'\right)\left(A'+i\lambda B'\right)\ket{\psi}\\
 &= \underbrace{\bra{\psi}A^{'2}\ket{\psi}}_{\Delta a^2}\underbrace{+i\lambda\bra{\psi}
 A'B'\ket{\psi}-i\lambda\bra{\psi}B'A'\ket{\psi}}_{\lambda\bra{\psi}i[A',B']\ket{\psi}} 
 + \underbrace{\lambda^2\bra{\psi}B^{'2}\ket{\psi}}_{ \Delta b^2}
 \end{array}
 \end{equation}
 Nous pouvons montrer que le commutateur de deux opérateurs hermitien est lui-même 
 hermitien lorsqu'il est multiplié par $i$ :
 \begin{equation}
 \begin{array}{ll}
 \hat{D} = i[A',B']\qquad \rightarrow\quad \hat{D}^\dagger &=-i(A'B'-B'A')^\dagger\\
 &= -i(B^{i\dagger}A^{'\dagger}-A^{i\dagger}B^{'\dagger}) = i[A',B']
 \end{array}
 \end{equation}
 Nous savons que
 \begin{equation}
 \Delta b^2\lambda^2 + \bra{\psi}D\ket{\psi}\lambda + \Delta a^2 \geq 0
 \end{equation}
 Ou encore
 \begin{equation}
 \bra{\psi}D\ket{\psi}^2 - 4\Delta a^2\Delta b^2 \leq 0
 \end{equation}
 Dès lors
 \begin{equation}
 \begin{array}{ll}
 \Delta a^2 \Delta b^2 &\geq \dfrac{1}{4}\left(\bra{\psi}i[A',B']\ket{\psi}\right)^2\\
 &\geq \dfrac{1}{4}\left|\bra{\psi}[A',B']\ket{\psi}\right|^2
 \end{array}
 \end{equation}
 Or, on peut trivialement montrer que $[A',B'] = [A-\langle a\rangle, B-\langle b\rangle]
 =[A,B]$. On en tire la \textbf{relation d'incertitude de Robertson} (Heisenberg généralisée) 
 \begin{equation}
 \Delta a \Delta b \geq \dfrac{1}{2}\left|\bra{\psi}[A,B]\ket{\psi}\right|
 \end{equation} 
 Ceci donnera un nombre strictement positif, les deux ne peuvent donc pas être très petits. 
 On retrouve facilement la relation d'incertitude de Heisenberg :
 \begin{equation}
 \left\{\begin{array}{ll}
 \hat{x}\\
 \hat{p}
 \end{array}\right.\quad [\hat{x},\hat{p}] = i\hbar\qquad\rightarrow\quad \Delta \hat{x}\Delta
 \hat{p} \geq \dfrac{1}{2}|\bra{\psi}i\hbar\ket{\psi}| \geq \dfrac{\hbar}{2}
 \end{equation}
 
 
 
 
 
 
 
 
 
 
 
 
 
 
 
 
 
 