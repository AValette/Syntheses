\chapter{Propriétés des substances pures} 
\section{Substances pures}
Une \textbf{substance pure} est une substance de composition chimique 
homogène et stable, par exemple l'eau liquide ou un mélange eau/glace. 
Même si ce n'est pas le cas pour l'air, en l'absence de réaction 
chimique et de changement de phase, on peut le considérer comme telle 
: \textbf{substance pseudo-pure}.

\section{Équilibre des phases d'une substance pure}
Considérons de l'eau dans un cylindre contenu par un piston, le tout à 
20$^\circ$C. Si je chauffe l'eau, la pression exercée sera toujours la 
même. Tant qu'il reste une goutte d'eau, la température ne dépasse pas 
les 100$^\circ$C (pression et température constante), mais dès que 
celle-ci sera weg le gaz pourra se dilater et la température augmenter.\\

\noindent
\textbf{Attention !} Ceci illustre bien que durant un changement de 
phase, l'échange d'énergie n'est \textit{pas} lié à l'augmentation de 
la température ! Ce n'est pas parce que cela ne chauffe pas que de l'
énergie n'est pas dépensée : le changement de phase en consomme.\\
On définit alors
\begin{description}
\item[Température de saturation] Température à laquelle la vaporisation 
se produit pour une pression donnée
\item[Pression de saturation] La même chose, mais pour une température 
donnée.
\item[Courbe de vaporisation] Relation fonctionnelle liant pression et 
température.
\end{description}

\begin{center}
\textit{Inclure graphe. Si l'on fait varier la pression pour $T$, la 
pression de saturation est celle ou l'on passe de l'état liquide à 
vapeur (Spoil : loi de Clapeyron)}
\end{center}

Tant que l'on parle de saturation, un \textbf{liquide saturé} est une 
substance à l'état liquide dans les contions $(p,T)$ de saturation. 
Une substance à l’état liquide à une température inférieure à la 
température de saturation à la pression donnée (et par conséquent à 
une pression supérieure à la pression de saturation à la température 
donnée) est appelée \textbf{liquide refroidi} ou \textbf{comprimé}.\\

\noindent
On peut également s'amuser à définir le \textbf{titre en vapeur} $
x = m_v/m$ où $m$ est la masse totale et $m_v$ la masse de vapeur. On 
parlera de \textbf{vapeur saturée} lorsqu'un état vapeur est dans les 
conditions de saturation et de \textbf{vapeur surchauffée} lorsque 
l'état vapeur est à température supérieure à la température de 
saturation.
\newpage
\noindent
\textit{Par exemple}, un liquide est saturé lorsqu'il bout. Au moment 
ou il est totalement évaporé, on sera en vapeur saturée.

\begin{center}
$AB$ : chauffage de l'eau, $BC$ : vaporisation, $CD$ : chauffage 
vapeur.
\end{center}
Si l'on augmente la pression, on obtient la courbe $EFGH$. Pour une 
pression encore plus élevée, l'étape de vaporisation à température 
constante n'est plus (RIP) : $N$ est un point d'inflexion, c'est le 
\textbf{point critique}. Après la pression critique ($PQ$) l'
évolution de la température est continue et on ne peut plus 
dinstinguer la phase liquide de la phase vapeur. 

En partant de la glace à 0,26 kPa\footnote{Pour de la glace à 0.1 MPa, 
elle fond puis s'évapore pas de stress.}, la température s'élève jusqu'à 
-10$^\circ$C puis le liquide passe directement en vapeur : \textbf{
sublimation}. Le \textbf{point triple} marque le frontière entre 
les processus de fusion, d'évaporation et de sublimation.\\

\noindent
Notons qu'une substance peut exister sous plusieurs phases solides : 
un changement d'une telle phase à une autre est une transformation 
\textbf{allotropique}.