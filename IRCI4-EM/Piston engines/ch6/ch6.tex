
\chapter{Combustion in CI engine}
	Diesel had many contacts in UE. In Belgium they were located in Gent, we get the license to build the diesel engine, and we succeeded to make the more powerful engine in those days (proud of us). 
	
\section{Classification}
	They can be classified on basis of the rpm. Fast (500-5000) and slow (100-400) running. The limitation of the rotation speed is due to the much higher compression in the engine, forcing the use of other materials like steel and not aluminum, increasing the inertia, the ignition delay also plays a role in the speed limitation. 2 strokes engines are more efficient (with compressor and with valves), thus we classify also in term of strokes and number of pistons in cylinder. On basis of the combustion chamber, we had a separated combustion chamber for indirect diesel injection. The direct injection is much more efficient. Finally, we differentiate them in functino of the construction mechanism (piston + piston rod + cross head or piston + connecting rod). 
	
\paragraph{Differences with SI}
	We have auto ignition instead of spark ignition. The fuel is different (diesel with Cetane number for the ignition delay) and higher temperature and pressure. Consequences on the material used. We work with lean mixture, excess of air wrt diesel (evaporation in hot air). High efficiency, for a fast running it goes up to 35\%, for large engines, slow ones have an efficiency of 40-45\%. 
	
\section{Combustion process}
\subsection{Droplets}
	The fuel is injected in form of droplets directly in hot air, the size of these is important in the combustion process. The longer the combustion delay, the more time the combustion will take. It depends on parameters: the CN of the fuel, the pressure, the temperature, the amount of turbulence, the size of the droplets, enough oxygen or not. 
	
	\wrapfig{7}{l}{4}{0.3}{ch6/1}{ch6/1}
	This is the cylinder, in the middle is an injecter for the droplets that go in the hot air. They will start burning at a certain time, then we continue to inject fuel. When the first droplet starts burning we are still adding fuel, the most important delay is the one of the first droplets since the next ones will be in higher temperature. The combustion chamber is designed to increase turbulence. 
	
	\ \\
	To ignite the droplet, we need high temperature and high pressure thus $\epsilon$. The droplet size also influences, if $d$ is its diameter and we qualify $Q_{required} = d^2$ and $Q_{received}=d^3$, we must respect: 
	
	\begin{equation}
	k_1 d^2 > k_2 d^3 \qquad \Rightarrow d < \frac{k_1}{k_2}
	\end{equation}

	The diameter must be small to easily ignite. On the other hand, the droplets have to find enough air and this requires larger droplets (they also go further). The diameter is also function of the input nozzle and pressure. The pressure is also important, the size of droplets express a driving force, the more we apply a pressure in the injection, the further it will go in the combustion chamber. We try thus to have the highest input pressure. 
	
\subsection{Air fuel ratio}
	\wrapfig{7}{l}{3}{0.3}{ch6/2}{ch6/2}
	If we increase the power, we inject more fuel so we are getting less lean. But the regime of a Diesel engine is lean mixture. Here we have a problem because in SI engine the mixture is more or less heterogeneous but here we can have $\lambda$ concentrations in the chamber. For example, on a simple droplet $\lambda$ can go from 0.3 on a region to 1.5 on another region. 
	
	\ \\ 
	On the figure we can see that the droplet is not a simple ball but we have the droplet, the evaporated envelop and the flame zone. 
	
\subsection{Ignition delay}
	\wrapfig{7}{r}{6}{0.3}{ch6/3}{ch6/3}
	The droplet must evaporate, realize the cracking process (carbonized particles separate), the time between injection and combustion is called \textbf{ignition delay}. This is one cause of the rpm limitation and depends on the Cetane number, the droplet size and the combustion chamber design. 
	
	\paragraph{Cetane number}
	It gives an information on the ignition delay, the higher it is, the lower is the delay and is about 50 for ordinary diesel and 60-70 for super diesel. Ignition delay becoming shorter, the rpm can be higher. The only advantage to have super diesel is when we start the engine (cold). In ships they use heavier fuel, CN is about 30, engine running at very low speed so we have much more time to ignite the fuel (lower price). 

	\paragraph{Combustion chamber}
	The combustion chamber in Otto engine should avoid the auto ignition. In Diesel one must promote the auto ignition! To promote the ignition we can create turbulence. 
	
\subsubsection{Separate combustion chamber}
	\wrapfig{5}{l}{5}{0.25}{ch6/4}{ch6/4}
	\paragraph{Pre-chamber} 
	The pre-chamber allows to make a "pre-combustion" to cancel the ignition delay. Since there is low amount of air, the combustion is slow, the pressure too and it increases gradually (high pressure don't needed since the combustion already started). Less emissions of NOx and soot. Less thermal load on the engine. 
	Then the burning droplets are transferred to the combustion chamber. This can lead to diesel knock because of the bulk fuel. We have to pump the fuel in the pre-chamber, increasing the losses. Difficult cold start since it is in the cylinder head and has more thermal inertia (cooling systems).
	
	\paragraph{Whirl chamber}
	\wrapfig{4}{l}{2.5}{0.2}{ch6/5}{ch6/5}
	The idea is the same, but it is not completely detached from the combustion chamber, it is just a passage before going to the chamber. The advantage is that there is a bit less loss than the previous case. 
	
	\ \\\\
	
\subsubsection{Combustion chamber in the piston}
	\wrapfig{5}{r}{2}{0.25}{ch6/6}{ch6/6}
	 The advantages are that it is a simple construction, the losses are low (high output) and excellent cold start properties. But due to higher temperatures we have more soot and NOx, more vibration, larger ignition delay (lower speed), injection pressure must be higher and nozzle with several holes necessary, the cooling is a bigger issue. 
	 
	 \ \\
\paragraph{Glow plug}
	\wrapfig{7}{l}{3}{0.25}{ch6/7}{ch6/7}
	Pre-heating systems are possible to counter cold start by using flame heater or \textbf{glow plug}. This is shown on the picture, heating electrode. It is important because at start, the engine is cold, so we have compression losses, large ignition delays leading to knocks. 
	
\section{The injection system}
	Its role is to inject the fuel in the adapted quantity as droplets in the combustion chamber at the correct moment. Management of the drop size, penetration (inlet pressure), and quantity. The injection can be performed in one step or in several steps. The second consist in performing a \textbf{pilot injection} then inject the real quantity and sometimes a bit more after to burn the soot. \\
	
	The injection system is composed of the fuel tank, fuel filter, low pressure fuel pump, high pressure injection pump and injection nozzles. \\
	
	As types of injection, former days we had the: inline fuel injection pump with mechanical regulation and axial piston distribution injection pump controlled mechanically. Now: common rail injection, pump nozzle unit, electronically controlled.
	
	\minifig{ch6/8}{ch6/9}{0.3}{0.3}{0.3}{0.4}
	\minifig{ch6/10}{ch6/11}{0.3}{0.3}{0.3}{0.4}
	
	Above is plotted the rail injection, the nozzle principle, the unit injector system and unit injector (I don't really understand the difference if you want to explain it here don't hesitate). 
	
\section{Diesel engine benefits}
	High torque, operational safety, production cost, economy of operation fuel, Reliability. The major advantages for the efficiency above the SI engine are: higher compression ratio, air excess, no throttle (no loss), lower consumption. Increased compression ratio improves thermal efficiency and lowers specific fuel consumption, but increases pumping, friction and compression/ expansion losses. For example, $\epsilon$ going from 10 to 20 increases the thermal efficiency from 26 to 40\% but decreases mechanical efficiency from 83 to 75\%.
	
	\minifig{ch6/12}{ch6/13}{0.35}{0.326}{0.49}{0.49}