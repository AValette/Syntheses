%%%%%%%%%%%%%%%%%%%%%%%%% %
% CH3 : Inviscid incompressible potential flows %
%%%%%%%%%%%%%%%%%%%%%%%%% %

\chapter{Inviscid incompressible potential flows}

\section{Introduction}
	\subsection{Governing equations}
		Let's remind that we have a set of 4 equations and 4 unknowns. It's inviscid so there are no stresses. This is decoupled from the energy equation, for constant density flows, the hydrodynamic problem gets decoupled from the thermal problem. So when we speak about compressible fluid we have to couple them. 
		\begin{equation}
			\rho = cst \qquad\qquad \nabla \vec{u} = 0 \qquad\qquad \rho \left[\frac{\D \vec{u}}{\D t} + \vec{u} \nabla \vec{u} \right] = -\nabla p +\rho \vec{F}
		\end{equation}
		
	\subsection{Bernouilli equation}
		The flow is barotropic, in additoin let's consider that the flow is steady and that the force derives from a potential $\vec{F} = -\nabla \Phi$ (F = 0 for most applications). We have the Bernouilli equation
		\begin{equation}
			\epsilon _m = \frac{p}{\rho}+k + \cancel{\Phi} = cst = \frac{p}{\rho} +\frac{u^2}{2}
		\end{equation}
		The mechanical energy is constant on a streamline. We can give an interpretation to the constant by calling it $\frac{p_t}{\rho} = \frac{p^0}{\rho}$. Physically, $p^0$ is the pressure where u = 0 and is called \textbf{stagnation pressure}. The constant differs from constant to constant. Indeed we found that $\vec{\omega}\times \vec{u} = -\nabla \epsilon _m$, so if the fluid is rotational $\epsilon _m$ will differ from streamline to another, meaning that the stagnation pressure differs. 
		
	\subsection{Irrotational (potential) flow}
		In that case we have that $\omega = 0$ so $u = \nabla \phi$ 