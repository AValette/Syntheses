\chapter{EDP du premier ordre}
\section{Introduction}
Une \textbf{EDP du premier ordre} en la fonction inconnue $u : D \subseteq 
\mathbb{R}^m\rightarrow\mathbb{R} : \vec{x}\mapsto u(x)$ est une 
expression de la forme
\begin{equation}
f\left(x,y,u,\dfrac{\partial u}{\partial  x},\dfrac{\partial  u}{x}\right)
\end{equation}
où $f \in C^1$. Une solution d'une telle expression $u$ sera telle que $
f(\vec{x},u(\vec{x}),\vec\nabla(\vec{x}))=0$. On restreindra notre étude 
au cas ou $\dfrac{\partial f}{\partial u}=0$ avec $f$ linéaire en $\vec 
\nabla u$ : les EDP linéaire  homogène :
\begin{equation}
\alpha(x,y)\dfrac{\partial u}{\partial x}+\beta(x,y)\dfrac{\partial u}{
\partial y}+\gamma(x,y)u = 0
\end{equation}
Cette équation peut s'écrire $L(u)=0$ où $L : u\mapsto \alpha(x,y)
\dfrac{\partial u}{\partial x}+\beta(x,y)\gamma(x,y)u$. On dira que 
$L$ est un opérateur différentiel. Le noyau de $L$ est un EV réel qui 
n'est rien d'autre que l'ensemble des solutions (ou combili) de l'EDP.\\

On retrouvera aussi notre EDP préférée d'Analyse II : l'EDP quasi-linéaires 
\begin{equation}
\alpha(x,y)\dfrac{\partial u}{\partial x}+\beta(x,y)\dfrac{\partial u}{
\partial y} = \gamma(x,y,u)
\end{equation}
On rencontrera également le cas de trois variables indépendantes : $u(x,
y,z)$. On aura alors une EDP complètement homogène :
\begin{equation}
\alpha(x,y,z)\dfrac{\partial u}{\partial x}+\beta(x,y,z)\dfrac{\partial 
u}{\partial y} + \gamma(x,y,z)\dfrac{\partial u}{\partial z}+
\underbrace{\delta(x,y,z)u}_{=0}=0
\end{equation}
Cela nous inspire déjà le fameux $\vec \nabla u \perp\vec{F}$ du chapitre 
20.

\section{EDPL hyper-homogène}
	\setcounter{subsection}{1}
	\subsection{La solution générale}
	Dans le cas ou $F_0=0$ et $F_i$ est le coefficient des $\frac{\partial u}{
	\partial x_i}$, ces $F_i$ forment le champ de vecteur $\vec{F}=(F_1,\dots,
	F_m)$ sur $D$ de sorte qu'on puisse écrire l'EDP\footnote{Si $u$ possède 
	$m$ variable, sa représentation à besoin de $m+1$ axes.} :
	\begin{equation}
	\vec{F}\perp\vec\nabla u
	\end{equation}
	Les coefficients de l'EDP sont les composantes d'un champ de vecteur 
	$\vec{F}$ dans l'espace des variables. On a vu au précédent chapitre 
	que cette relation caractérise $u$ comme des intégrales premières de 
	$\vec F$\\
	
	On définit alors les \textbf{courbes caractéristiques} de l'EDP comme les 
	lignes de champ de $\vec{F}$, mais surtout les courbes intégrales du 
	système 
	\begin{equation}
	\dfrac{dx_1}{F_1}=\dots=\dfrac{dx_n}{F}
	\end{equation}
	Ceci forme le \textbf{système différentiel des caractéristiques.}. Les 
	solutions de l'EDP $\vec\nabla u \perp\vec{F}$ sont les intégrales 
	premières de ce système, soit des fonctions qui sont constante tout le 
	long de toute ligne de champ de $\vec{F}$.\\
	
	La résolution d'une telle EDPL est équivalente à la recherche de $m-1$ 
	solutions particulières fonctionnellement indépendantes de sorte que 
	la S.G. est :
	\begin{equation}
	u = G(u_1,\dots,u_{m-1})
	\end{equation}
	\textsc{Exemple.} Voir section 21.2.3 et 21.2.4
	
	\setcounter{subsection}{4}
	\subsection{Problème de Cauchy pour $\vec\nabla u \perp \vec F$}
	Reprenons notre équation 
	\begin{equation}
	F_1(\vec{x})\dfrac{\partial u}{\partial x_1}+\dots+F_n(\vec{x})
	\dfrac{\partial u}{\partial x_n}=0
	\end{equation}
	Supposons que les courbes caractéristiques sont connues. Si $u : 
	D\subseteq \mathbb{R}^n\rightarrow\mathbb{R}$ est solution de 
	l'EDP, alors $u$ est connue si on connaît sa valeur sur toute 
	courbe caractéristique.\\
	Autrement dit, si l'on a toutes les caractéristiques il suffit 
	de connaître la valeur de $u$ en un point pour chaque caractéristique 
	pour déterminer la valeur de $u$ dans tout l'espace (ou tout domaine).
	Par exemple, en $2D$, il suffit de donner la valeur de $u$ le long 
	d'une transversale à une courbe caractéristique.\\
	
	Soit $H$, une hypersurface coupant chaque caractéristique en 
	\textbf{un} point $\Rightarrow$ $u$ est univoquement déterminé par 
	la C.I. $u|_H$ donné\footnote{Qui sera $C^1$.}.
	
\section{EDP quasi linéaire sur $\mathbb{R}^m$}
	\subsection{Interprétation géométrique pour $m=2$}
	Considérons 
	\begin{equation}
	F_x|_{(x,y,z)}\dfrac{\partial z}{\partial x}+F_y|_{(x,y,z)}\dfrac{
	\partial z}{\partial y} = F_z|_{(x,y,z)}
	\end{equation}
	La fonction $Z : (x,y)\mapsto Z(x,y)$ est solution de l'EDPQL ssi 
	$\forall (x,y)\in D$ : 
	\begin{equation}
	F_x|_{(x,y,Z(x,y))}\left.\dfrac{\partial z}{\partial x}\right|_{
	(x,y,Z(x,y))}+	F_y|_{(x,y,Z(x,y))}\left.\dfrac{\partial z}{\partial 
	y}\right|_{(x,y,Z(x,y))} = F_z|_{(x,y,Z(x,y))}	
	\end{equation}
	L'équation du plan tangent au graphe de $Z$ en $(x_0,y_0,Z(x_0,y_0))$ :
	\begin{equation}
	\left.\dfrac{\partial Z}{\partial x}\right|_{(x_0,y_0)}(x-x_0)+
	\left.\dfrac{\partial Z}{\partial y}\right|_{(x_0,y_0)}(y-y_0)=z-z_0
	\end{equation}
	En graphe de $Z$ en ce point est orthogonal aux vecteurs 
	\begin{equation}
	\text{gph }Z \perp \left(\left.\dfrac{\partial Z}{\partial x}\right|_{
	(x_0,y_0)}, \left.\dfrac{\partial Z}{\partial y}\right|_{(x_0,y_0)},
	-1\right) \perp 	\vec{F}_{x,y,Z(x,y)}
	\end{equation}
	La deuxième $\perp$ est donné par l'EDP. On peut voir le vecteur 
	ci-dessus comme le vecteur directeur du plan. Dès lors, on en déduit 
	que $\vec{F}$ est tangent au graphe de $Z$.\\
	\textit{$Z$ est solution de l'EDP ssi son graphe est (en chacun de 
	ses points) tangent au champ de vecteur $\vec{F}$}.\\

	 Le graphe de $Z$ 
	n'est rien d'autre qu'une réunion de lignes de champ de $\vec{F}$. Il 
	forme un ensemble de niveau d'une intégrale première des du système 
	des caractéristiques de $F$ et est donc le système caractéristique de 
	l'EDP.
	
	\subsection{Résolution pour $m=2$ (= EDPQL)}
	\proposition{$Z : (x,y)\mapsto Z(x,y)$ est localement solution de 
	\begin{equation}
	F_x\dfrac{\partial z}{\partial x}+	F_y\dfrac{\partial z}{\partial y}
	=F_z
	\end{equation}
	ssi $\exists \mathcal{U} : (x,y,z) \mapsto \mathcal{U}(x,y,z)$ 
	intégrale première du système des caractéristiques
	\begin{equation}
	\dfrac{dx}{F_x}=\dfrac{dy}{F_y}=\dfrac{dz}{F_z}
	\end{equation}
	et $\exists c\in\mathbb{R}$ t.q. $Z$ est localement solution de 
	\begin{equation}
	\mathcal{U}(x,y,Z(x,y)) = c
	\end{equation}}\
	
	
	\begin{proof}
	Vue ? 
	\end{proof}
	
	En pratique :
	\begin{itemize}
	\item[$\bullet$] On écrit le SD des lignes.
	\item[$\bullet$] On cherche deux intégrales premières indépendantes 
	$\mathcal{U}_1$ et $\mathcal{U}_2$.
	\item[$\bullet$] On l'écrit l'intégrale première $g(\mathcal{U}_1,
	\mathcal{U}_2) =c$
	\item[$\bullet$] On résout $g(\mathcal{U}_1,\mathcal{U}_2) =c$ par 
	rapport à $z$ (si possible)
	\end{itemize}
	\textsc{Exemple}. Section 214.3.3 ou slide 14.
	
	
	\setcounter{subsection}{3}
	\subsection{Condition de Cauchy pour EDPQL}
	On a vu que si $Z : (x,y)\mapsto Z(x,y)$ est une solution de l'EDPQL 
	alors le graphe de $Z$ est une réunion des lignes de champ de $\vec 
	F$. Qu'est ce qu'une bonne C.I. ? Il suffit de considérer une courbe 
	transversale $\Gamma$ transversale aux caractéristiques de sorte à ce 
	que le graphe de $Z$ s'appuie sur $\Gamma$.\footnote{Ceci semble naturel 
	lorsqu'on se rappelle que l'on a ramené la résolution de l'EDPQL à la 
	résolution de l'EDPLHH : $\vec\nabla \mathcal{U}\perp\vec{F}$.}\\
	On demande à ce que cette restriction soit au moins de classe $C^1$ 
	et $z$ non tangent à une courbe caractéristique.
	
	
	
	
	
	
	
	
	
	
	
	
	
	
	
	
	
	
	
	
	
	
	
	
	
	
	
	
	
	
	
	
	
	
	
	
	
	
	
	
	
	
	
	
	
	
	
	