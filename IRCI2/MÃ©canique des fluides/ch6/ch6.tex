\chapter{Théorèmes des travaux virtuels}
\section{Pourquoi les traveaux virtuels ?}
	\subsection{Problème à résoudre}
	Récapitulons les différentes forces, contraintes et déplacements rencontrés
	jusqu'ici :
	\begin{itemize}
	\item $f_i$ ; les forces de volumes
	\item $T_i^{(n)}$ ; les forces de surfaces associées à $\vec{n}$
	\item $u_i$ ; les déplacements résultant de ces forces
	\item $\tau_{ij}$ ; les contraintes résultant de ces forces	
	\end{itemize}
	
	\subsection{Equations à résoudre}
	Le théorème des travaux virtuels permettra notamment de résoudre des :
	\begin{itemize}
	\item Equations d'équilibre en volume 
		\begin{itemize}
		\item Translation : $\tau_{ji,j} + f_i = 0$
		\item Rotation : $\tau_{[ij]} = 0$
		\end{itemize}
	\item Des lois de comportement
	\item Des équations de compatibilité
	\end{itemize}
	
\section{Travail virtuel}
	\subsection{Travail virtuel des forces extérieurs}
	Considérons deux ensembles : 
	\begin{enumerate}
	\item Virtuel : les déplacement $u_i'$
	\item Réel : les forces de volume $f_idV$ et de surface $T_i^{(n)}dS$
	\end{enumerate}
	Le travail virtuel des forces extérieures est \textbf{défini} par :
	\begin{equation}
	T_{ext}' \equiv \int_V f_iu_i'\ dV + \oint_S T_i^{(n)}u_i'\ dS
	\end{equation}
	
	\subsection{Calculs préliminares}
	On va ici restreindre les déplacements virtuels à des déplacement \textbf{
	infinitésimaux} de corps \textbf{indéformables}. Comme pour le cours de 
	\textit{Mécanique Rationelle II}, nous avons :
	\begin{equation}
	\vec{u_P'} = \vec{u_Q'} + \vec{\theta'}\times\vec{QP}
	\end{equation}
	La rotation \textbf{doit} être infinitésimale afin de pouvoir conserver le
	produit vectoriel. On a dès lors :
	\begin{equation}
	T_{ext}' \equiv \int_V \vec{f}(\vec{u_Q'} + \vec{\theta'}\times\vec{QP})\ dV
	 + \oint_S \vec{T}^{(n)}(\vec{u_Q'} + \vec{\theta'}\times\vec{QP})\ dS
	\end{equation}	
	On réorganise :
	\begin{equation}
	T_{ext}' \approx \vec{u_Q'}\underbrace{\left[\int_V \vec{f}\ dV + \oint_S
	 \overline{T}^{(n)}\ dS \right]}_{\vec{R}} + \vec{\theta'}\underbrace{\left[
	 \int_V (\overline{QP}\times\vec{f})\ dV + \oint_S (\overline{QP}\times
	 \overline{T}^{(n)}\ dS\right]}_{\vec{C_Q}}
	\end{equation}
	On reconnaît dans cette expression la résultante des forces, ainsi que le 
	moment résultant des forces par rapport au point $Q$ nous permettant de 
	ré-écrire :
	\begin{equation}
	T_{ext}' \equiv \vec{u_Q'}.\vec{R} + \vec{\theta'}.\vec{C_Q}
	\end{equation}
	

\section{Théorème}
\theor{\textsc{trav.virt. pour des dép. infinitésimaux de corps indéformables}\\
A l'équilibre, le travail virtuel des forces extérieures est nul pour tout 
déplacement virtuel \textbf{infinitésimal} de corps indéformable, et 
réciproquement.
\begin{equation}
T_{ext}' \equiv \vec{u_Q'}.\vec{R} + \vec{\theta'}.\vec{C_Q}
\end{equation}}\ 
	
	\subsection{Le théorème direct}
	Si l'on se trouve à l'équilibre, alors le travail virtuel des forces 
	extérieures est nul pour tout déplacement virtuel infinitésimal de 
	corps indéformable.\\
	Equilibre $\rightarrow \vec{R}=\vec{0}, \vec{C_Q}=\vec{0} \Rightarrow T_{
	ext}' = 0$ et ce quelque soit le déplacement virtuel de corps indéformable.
	
	\subsection{Le théorème réciproque}
	Forcément, ici c'est $T_{ext}' = 0 \Rightarrow \vec{U_Q'}.\vec{R} + 
	\vec{\theta'}.\vec{C_Q} = 0\ \ \ \forall \vec{u_Q'},\vec{\theta'}$.\\
	Ce résultat s'obtient facilement : pour une translation virtuelle 
	quelconque $\vec{\theta'}=\vec 0$. Pour que l'équation ci-dessus soit bien nulle,
	il faut forcément que $\vec{R}=\vec{0}$. En choississant une rotation
	virtuelle quelconque $\vec{u_Q'}=\vec{0}$ et le raisonnement est similaire.
	
	
	
	
	
	
	
	
	
	
	
	
	
	
	
	
	
	
	
	
	
	
