\chapter{Les machines électriques - Généralités}
\section{Introduction}
	\subsection{Classification des machines électriques}
	Tout se base sur l'interaction des courants électriques et des champs 
	magnétiques. On classes ces bèbètes en trois catégories :
	
	\begin{enumerate}
	\item \textit{Les machines génératrices.} Elles se basent 
	sur l'induction d'un courant électrique dans un circuit conducteur par
	\textbf{déplacement relatif} de celui-ci et d'un champ magnétique. La 
	dynamo et l'alternateur sont de ce type.
	\item \textit{Les moteurs électriques.} Ils sont basés sur 
	l'obtention d'un effort mécanique sur un circuit traversé pour un 
	courant extérieur (pouvant donner lieux à un champ magnétique). Nous 
	pouvons parler ici du moteur à courant continu ou alternatif.
	\item \textit{Les machines transformatrices.} Leur rôle est 
	de modifier la grandeur des courants et tensions alternatifs. Le 
	transformateur est le grand classique.
	\end{enumerate}
	
	\subsection{Intérêt des moteurs électriques}
	Ceux-ci ont pas mal d'avantages sur les moteurs thermiques : moins 
	polluants, bruyants, démarrent seuls, facilité d'emploi, régularité 
	du couple utile, possibilité de l'inversion du sens de rotation, 
	fort couple vitesse à faible vitesse et même à l'arrêt, \dots Le 
	dernier point est très important car cela les affranchis, par 
	exemple, de boîte à vitesses. En effet, les moteurs thermiques ne 
	disposent pas de cette jouissante propriété et possèdent tout un 
	dispositif mécanique à engrenages, dissipant de l'énergie.
	
	\subsection{Le moteur asynchrone}
	C'est \textbf{le} moteur le plus utilisé. Il fonctionne directement 
	en tension alternative. Celle-ci génère un courant circulant dans le 
	stator constituant la seule source externe de champ magnétique, le 
	rotor n'a pas à être relié à une source d'énergie. Cependant, il existe 
	des courants rotorique mais ceux-ci sont \textbf{induits} : on parle 
	parfois de \textbf{moteur d'induction}. Ce moteur équipe la quasi 
	totalité des machines-outils classique (tours, fraiseuses, \dots)
	
	\subsection{Le moteur synchrone}
	Afin de les utiliser, il faut d'abord les faire "roter" à leur 
	vitesse nominale avant de les coupler au réseau, nécessitant un 
	moteur auxiliaire. La seule différence avec le moteur asynchrone 
	se situe dans la conception du rotor. Ce-dernier est constitué 
	d'aimants (ou alimenté en courant continu). Après le démarrage, le 
	moteur tourne en synchronisme avec le champ tournant. Ces moteurs 
	ne dépendent donc que du réseau qui les alimente et sont ainsi 
	utilisés lorsqu'une rotation uniforme est primordiale.\footnote{
	Différence à plus expliciter plz}
	
	
	\subsection{Les moteurs à courant continu}
	Ils sont les champions dans les très faibles puissances (jouets, 
	essuie-glaces\dots). Leur atout majeur est de posséder une 
	remarquable capacité de variation de vitesse. Ils jouent un rôle 
	important dans la traction électrique et sont utilisés en tant que 
	moteurs "série".
	
	\subsection{Les autres types de machines électriques}
		\textsc{Les moteurs universels}\\
		On les trouve dans les robots ménagers, ventilateurs, \dots  
		C'est le moteur de la vie domestique. Leur vitesse chute 
		rapidement lorsqu'un couple trop important leur est demandé.\\
		
		\textsc{Les moteurs pas à pas}\\
		Utilisés dans les dispositifs à positionnement précis et ont 
		l'avantage d’être très simple à la conception.
		
	\subsection{Associations moteurs - électronique}
	Les moteurs à faible puissance, ou les synchrones auto-pilotés pour 
	les fortes sont souvent associés à des équipement électroniques.
	
	
\section{Méthodes d'étude des machines électriques}
	\subsection{Généralités}
	Pour étudier les machines, deux méthodes s'offrent à nous :
	\begin{enumerate}
	\item \textit{La méthode de Kirchhoff}. On écrit les équations des 
	circuits, la conservation et l'énergie et on déduit le reste. Le 
	dispositif décrit par les équations se présente comme une boite noire 
	s'incluant dans une chaîne de régulation. C'est l'optique de l'
	\textbf{automaticien}.
	\item \textit{La méthode de Maxwell}. On part des grandeurs physiques 
	et on calcule le reste. C'est l'optique du \textbf{constructeur}.
	\end{enumerate}
	Si l'on se base sur le critère de l'utilité pratique, en Belgique, il 
	est plus intéressant de choisir la méthode de Kirchhoff. D'un point de 
	vue formation, cette méthode est également plus "simple" (car 
	systématique). La préférence va ainsi pour Kirchhoff, mais n'oublions 
	pas pour autant la seconde!
	
	\subsection{Choix du phénomène physique exploité}
	On peut concevoir des moteurs capacitifs (loi de Coulomb) ou inductif (
	Laplace). Quasi tous les moteurs sont de type inductif car la densité 
	d'énergie potentielle magnétique ($1/2B^2/\mu_0$) est 10,000 fois 
	supérieure à la densité d'énergie potentielle électrique ($1/2\epsilon_0
	E^2$).\\
	Le dispositif magnétique le plus simple est l'électro-aimant dont 
	la force est donnée par $f_{em} = \frac{1}{2} I^2\frac{dL(x)}{dx}$ où I 
	est le courant parcourant le circuit.
	
	
	
\section{Rappel des lois de l'électromagnétisme}
	\subsection{Loi de la force magnétomotrice (f.m.m.)}
	Elle intervient dans le calcul des ampère-tours nécessaires pour 
	magnétiser un circuit magnétique. Sous sa forme locale 
	\begin{equation}
	\rot \vec{H} = \vec{J}_t
	\end{equation}
	où $\vec{H}$ est le champ magnétique local et $\vec{J}_t$ la 
	densité de courant. On peut également exprimer cette loi 
	\begin{equation}
	\mathcal{F} = \oint \vec{H}.\vec{dl} = \sum i
	\end{equation}
	où $\mathcal{F}$ est la force électromotrice le long d'un contour 
	fermé embrassant un faisceau de conducteurs parcourus par des 
	courants $i$.
	
	\subsection{Loi de Maxwell}
	Elle exprime la force électromotrice induite dans un \textbf{circuit}. 
	Sous sa forme locale 
	\begin{equation}
	\rot \vec{E} = -\frac{\partial \vec{B}}{\partial t}
	\end{equation}
	Sous sa forme intégrale 
	\begin{equation}
	e = ri = \oint \vec{E}.\vec{dl} = -\frac{d\Phi}{dt}
	\end{equation}
	La f.e.m. induite $e$ fait alors circuler un courant $i$. Un 
	accroissement du flux fait ainsi circuler un courant négatif
	\footnote{Trigonométrique}. Si le circuit est fixe et l'induction 
	variable on parle de f.e.m. \textbf{induite}. Si le circuit est 
	mobile, on dira \textbf{engendrée}. Dans ce dernier cas, on écrit 
	alors la loi sous la forme
	\begin{equation}
	de = -[\vec{B}\times\vec{v}].\vec{dl}
	\end{equation}
	où $\vec{v}$ est la vitesse relative par rapport à un champ d'induction 
	$\vec{B}$ d'un élément de longueur $\vec{dl}$ du circuit électrique 
	considéré.\\
	\textsc{Exemple}. Soit un conducteur linéaire de longueur $l$ se 
	déplaçant à vitesse constante $\vec{v}$ dans un champ d'induction $\vec{
	B}$. Celle loi devient 
	\begin{equation}
	e = -[\vec{B}\times\vec{v}].\vec{l}
	\end{equation}
	Si le déplacement se fait normalement à son axe et à la direction 
	du champ (en valeur absolue) : $e = Blv$.
	
	
	\subsection{Loi de Laplace}
	Elle donne l'expression de la force sur un conducteur parcouru par 
	un courant plongé dans un champ d'induction $\vec{B}$.
	
	
\section{Principes de fonctionnement des machines électriques}
	\subsection{Éléments constitutifs des machines électriques}
	Quasi toutes contiennent un élément fixe dénommé \textbf{stator} 
	et un organe mobile, le \textbf{rotor}, séparés par un entrefer. 
	L'\textbf{inducteur} est l'organe destiner à créer le flux 
	magnétique, ou par des aimants permanents ou par des courants 
	électriques.
	
	
	\subsection{Machines hétéropolaires}
		\subsubsection{Principes de fonctionnement}
		Hétéropolaire signifie que $\vec{B}$ n'a pas le même signe 
		partout dans l'entrefer. Considérons le dispositif suivant, 
		constitué d'un stator métallique portant un circuit inducteur 
		de $N_S$ spires parcourues par un courant continu $i_s$ et 
		un rotor lisse composé de la spire $11'$ constituée de deux 
		conducteurs diamétralement opposés. Les bornes $1$ et 1' sont 
		connectées à de disques conducteurs.\\
		
		\textsc{Méthode des champs}\\
		Considérons $i_r=0$. On suppose le fer parfait, de perméabilité 
		infinie impliquant que tous les ampère-tours se concentrent dans 
		l'entrefer\footnote{En effet, $H_{fer} = B/\mu$ où $\mu = \infty$. 
		On peut donc négliger le champ dans le fer.}. En considérant un 
		contour fermé traversant l'entrefer 
		\begin{equation}
		N_Si_S = 2H\delta(\beta)
		\end{equation}
		où $\delta(\beta)$ est la largeur de l'entrefer. On a donc
		\begin{equation}
		B(\beta) = \mu_0H = \mu_0\frac{N_SI_S}{2\delta(\beta)}
		\end{equation}
		Si le rotor tourne à vitesse $\Omega_r$ constante, il apparaît une 
		f.e.m. aux bornes du conducteur valant
		\begin{equation}
		e_1 = Blv = B(\beta)lR\Omega_r
		\end{equation}
		où $R$ est le rayon du rotor. Comme $e_{1'} = -e_1$, on a 
		\begin{equation}
		e_r = e_1-e_{1'} = 2lRB(\beta)\Omega_r
		\end{equation}
		On retrouvera cette relation pour tous les types de machines : la 
		f.e.m. engendrée est $\propto$ flux*vitesse : $e_r = c^{te}B(
		\Omega_rt)$ où $e_r(t)$ est une fonction périodique qui reproduit 
		dans le temps la répartition spatiale de l'induction. \\
		Notons que pour éviter les problèmes de glissements, on peut 
		échanger les emplacement de l'inducteur et de l'induit avec un 
		structure à pôles lisses ou saillants.\\
		
		\textsc{Méthode des circuits}\\
		Vu en cours?
		
		
		\subsubsection{Spire non diamétrale}
		Si 1' est décalé de $\theta_1$ par rapport à 1'' d'une spire 
		diamétrale, la connexion d'extrémité est plus courte (bien) 
		mais la tension à ses bornes est plus faible (bof).
		
		\subsubsection{Enroulement}
		D'un point de vue économique, il est préférable de considérer 
		plusieurs spires. Soit 22', une spire décalée de $\theta_1$ 
		par rapport à 11' ; le phaseur à ses bornes est lui aussi 
		déphasé de $\theta_1$ :
		\begin{equation}
		\underline{E_{2'2}} = \underline{E_{1'1}}e^{j\theta_1}
		\end{equation}
		\textbf{Attention !} On peut mettre ces deux spires en série, 
		mais pas en parallèle\footnote{Pq ?}.\\
		
		On peut ajouter $m$ spires sur un arc $\theta_m$ du rotor. 
		Néanmoins, 	il n'est pas économique de dépasser $\theta_m = 
		\pi/3$, l'accroissement de tension étant faible. Pour $\theta_m = 
		\pi/3$, le rotor peut accueillir trois enroulements indépendants 
		aux bornes desquels on peut obtenir une f.e.m. d'amplitudes 
		égale, mais décalée de $2\pi/3$ : c'est le système \textbf{triphasé 
		équilibré}. Une telle machine a la particularité de posséder l'induit sur 
		le stator. Il s'agit d'une machine synchrone à rotor lisse à 
		une paire de pôles (non-étudié ici).\\
		
		Au premier chapitre, nous avons vu comment connecter les enroulements 
		pour garantir une distribution économique de l'énergie. Si des 
		impédances égales sont branchées sur les enroulements, le système 
		est triphasé équilibré : un moteur synchrone connecté à ce réseau 
		entraînera une vitesse constante.
		
		
		\subsubsection{Machine à courant continu}
		Les extrémité de la spire sont connectées à un secteur conducteur 
		tournant, isolé du précédent. Sur ces secteurs appelés \textit{lames 
		de collecteur} reposent deux balais fixes diamétralement opposés. La 
		commutation est le passage d'un valai d'un secteur à un autre. En bref, 
		si un conducteur en forme de spire, parcouru par un courant, est placé
		dans un champ magnétique, il est soumis à des forces de Laplace. Ces forces 
		créent un couple de rotation qui fait tourner la spire sur son axe. Quand 
		la spire a fait un demi tour, il faut inverser la polarité pour inverser le 
		sens des forces et continuer le mouvement. ce sera le rôle du collecteur.
		
		\subsubsection{Machine à plusieurs paire de pôles}
		Simple généralisation : la période d'induction n'est plus de $2\pi$ mais 
		de $2\pi/p$.
		
		
		
\section{Composants des machines électriques}
Pour canaliser le champ magnétique on utilise du fer : on forme un circuit 
magnétique, généralement en cuivre. Pour séparer les composants, un isolant 
est utilisé. Comme ça chauffe, il sera nécessaire de refroidir toute machine 
électrique.

	\subsection{Circuit magnétique}
	Son rôle est de conduire le flux qui devra agir sur les courants circulant 
	dans le circuit électrique placé au milieu de l'entrefer. Ce circuit est 
	constitué d'un solide de forte perméabilité magnétique forçant\footnote{Une 
	partie parvient tout de même à s'échapper : le flux de dispersion magnétique.} 
	le trajet des lignes de champs d'où le nom \textit{circuit magnétique} par analogie 
	à l'\textit{électrique}.\\
	
	\textsc{Matériaux utilisés}\\
	L'acier, la fonte, le fer, \dots Le plus important et ce peu importe le 
	matériau est la loi qui lie l'induction au champ magnétique. Ce n'est 
	pas quelque chose de linéaire : la perméabilité d'un matériau varie en 
	fonction du champ qui lui est appliqué. Pour représenter ça, on regarde 
	les \textit{courbes de magnétisation.}
	
	
	\subsection{Circuit électrique}
	\textsc{Rappel.} L'\textbf{inducteur} est chargé de créer le flux utile et l'
	\textbf{induit} les f.e.m.\footnote{Wiki : L'inducteur est un organe 
	électrotechnique, généralement un électroaimant, ayant comme fonction d'induire 
	un champ électromagnétique dans un induit servant à chauffer toutes sortes de 
	conducteurs comme des métaux de toutes sortes.}
	
		\subsubsection{Disposition des enroulements}
		\textsc{Inducteur}\\
		Il peut être situé au rotor. On utilise des aimants permanents pour les 
		petites puissances.\\
		
		\textsc{Induit}\\
		Les conducteurs sont généralement isolés entre eux. On utilise souvent 
		un bobinage.
		
		\subsubsection{Groupement des conducteurs}
		L'association des conducteurs d'une machine est le \textbf{bobinage}. 
		\begin{description}
		\item[Conducteurs.] On peut utiliser un conducteur (massif ou creux) pour 
		véhiculer un courant $I$.
		\item[Spire.] Constituée de deux conducteurs
		\item[Bobine.] Lorsqu'il y a plusieurs conducteurs par encoche.
		\item[Phase.] Un groupe de bobines associées en série ou en parallèle.
		\end{description}
		
	\subsubsection{Pertes joules dans les conducteurs}
	Étudions la densité de courant dans un conducteur d'encoche. Les courants 
	circulants engendrent des pertes ($P_J = RI^2$ où $R=\rho l/S$ en continu).
	Si le conducteur est massif, en continu, la densité de courant $J=I/S$ est 
	constante. Démontrons par l'absurde que ce n'est pas le cas en alternatif.\\
	
	Soit un conducteur massif dans une encoche et supposons $J = \text{cste}$ :
	\begin{equation}
	\begin{array}{lll}
	
	& H(x)d + 0 &= \displaystyle\int_0^x Jedx\\
	\Leftrightarrow& H(x) &= \dfrac{Je}{d}x
	\end{array}
	\end{equation}
	où $H$ varie linéairement avec $x$. Le flux embrassé par le circuit constitué 
	par cette partie du conducteur et le retour à l’infini vaut, par unité de 
	longueur :
	\begin{equation}
	\begin{array}{ll}
	\dfrac{\Delta \phi(x)}{\Delta l} &= \text{ cste } - \int_0^x B(x)dx\\
 	&= \text{ cste } - \mu_0\int_0^x H(x) dx\\
	&= \text{ cste } - \dfrac{\mu_0Je}{d}\dfrac{x^2}{2}
	\end{array}
	\end{equation}
	La chute de tension inductive par unité de longueur du conducteur dépend de 
	la position du filet de courant (et de la fréquence) : il n'est pas correct 
	de supposer $J = \text{cste}$ : la densité de flux est plus importante à la 
	"surface".\\
	La fréquence augmente également les pertes (résistance plus importante). On 
	peut utiliser des barre de Roebel obligeant le courant à passer à la surface 
	et en profondeur de l'encoche pour contrer au maximum cet effet.
	
	\subsection{Isolation des machines}
		\subsubsection{Loi de Montsinger - Vieillissement des isolants}
		La température déteriore la qualité de l'isolant. Une loi expérimentale 
		décrit cet effet
		\begin{equation}
		t = ab^{-\theta}
		\end{equation}
		où $t$ est la durée de vie, $a,b$ des constantes pour un isolant donné 
		et $\theta$ la température. On peut l'écrire :
		\begin{equation}
		\log t = \log a - \theta\log b
		\end{equation}
		où $\log t$ est une fonction linéaire de $\theta$. Ainsi, élever la 
		température de 6 à 10$^\circ$ réduit la durée de vie de moitié!
		
	\subsection{Refroidissement}
		\subsubsection{Agents de refroidissement}
		Le plus courant est d'utiliser l'air, un ventilateur. Des ventilations 
		intérieures ou extérieure (enceinte close) pour un atmosphère fort 
		pollués existent également. L'hydrogène peut également refroidir en 
		le faisant circuler dans les conducteurs. Les diélectriques liquides 
		sont également une option.
		
\section{Grandeurs caractéristiques des machines électriques}
	\subsection{Grandeurs nominales}
	Une grandeur physique est dite \textit{nominale} lorsque l'appareil peut 
	fonctionner indéfiniment à celle-ci sans subir d'usure (avec un coefficient 
	de sécurité).\\
	La \textit{puissance nominale} est plus subtile, il faut savoir de quoi on 
	parle :
	\begin{itemize}
	\item[$\bullet$] C'est la puissance électrique développable (en kW) à ses 
	bornes si on parle d'une génératrice à courant continu.
	\item[$\bullet$] Pour un alternateur, c'est la puissance électrique 
	apparente développable (en kVA)
	\item[$\bullet$] Pour un moteur, cil s'agit de la puissance mécanique 
	disponible (en kW)
	\end{itemize}
	
	
	\subsection{Rendements des machines}
	Par définition, on défini le rendement 
	\begin{equation}
	\eta = \frac{P_u}{P_a}
	\end{equation}
	où $P_u$ est la puissance utile à la sortie et $P_a$ la puissance 
	absorbée. On peut écrire cette formule
	\begin{equation}
	\eta = \frac{P_u}{P_a} = \frac{P_a-p}{P_a} = \frac{P_u}{P_u+p}
	\end{equation}
	où $p$ est la perte. Celles-ci ont un terme fixe et un terme variable avec 
	la puissance apparente $S$. On utilisera la dernière égalité, plus précise\footnote{
	Pour les machines génératrices, $P_a$ est mécanique et $P_u$ électrique. C'est 
	l'inverse pour 	une machine motrice}. Les pertes peuvent être mécaniques, 
	due aux fer, \dots\\
	
	Pour le cuivre, les pertes sont variable en fonction de $I$ : $P_{p,Cu} = 3
	RI^2$. Comme la tension est supposée constante : $P_{p,Cu} = kV^2I^2 = kS^2$. 
	Les pertes totales s'expriment ainsi de la forme $p = a+bS^2$ et, par exemple, 
	on peut écrire l'expression du rendement :\footnote{???}
	\begin{equation}
	\eta = \frac{P_a-p}{P_a} = \dfrac{S\cos\varphi-a-bS^2}{S\cos\varphi}
	\end{equation}
	La valeur maximale du rendement est donnée par
	\begin{equation}
	\frac{d\eta}{dS} = 0\qquad \Leftrightarrow\qquad S = \sqrt{\frac{a}{b}}
	\end{equation}
	La forme de la courbe rendement souhaitée dépend de ce que l'on veut faire.\\
	Pour mesurer le rendement, il est plus intéressant de mesurer les pertes que 
	de mesurer la puissance électrique et mécanique. On peut le faire en mesurant 
	l'échauffement ou en mesurer chaque perte une à  une, par essai.
	
	
	\subsection{Caractéristiques des machines tournantes}
	A lire\\
	
	\textsc{Génératrice}\\
	A lire
		
		
		
		
		
		
		
		
		
		
		
		
		
		
