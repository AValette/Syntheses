\chapter{Lignes de champ et intégrales premières}
\section{ED exactes sur $\mathbb{R}^2$ et fonction potentielle}
	\subsection{Cas des EDO à variables séparées}
	Soit $y' = f(x).g(x)$. La résolution de cette équation est 
	maintenant bien connue :
	\begin{equation}
	f(x)\text{dx}-\frac{1}{g(y)}\text{dy} = 0 \Leftrightarrow 
	G(y) := \int\frac{dy}{g(y)}=\int f(x)dx =: F(x) + \text{cste} 
	\equiv F(x) - G(y) = \text{cste}
	\end{equation}
	Posons $h(x,y) = F(x) - G(y)$ : $h$ est une \textbf{intégrale 
	première} de cette ED car $h$ est constante tout le long de la 
	courbe intégrale. La 1-forme différentielle de $h$ peut s'écrire 
	\begin{equation}
	dh(x,y) = f(x)dx - \frac{1}{g(y)}dy
	\end{equation}
	Comme ED $\Leftrightarrow dh = 0$, l'ED est dite exacte et 
	$h$ est une fonction potentielle. La solution générale de cette 
	ED n'est rien autre que $h(x,y)=c$.
	
\section{ED non exacte, FI et intégrale première}
	\subsection{ED non exacte}
	L'ED suivante sous sa forme générale :
	\begin{equation}
	P(x,y)dx + Q(x,y)dy = 0
	\end{equation}
	est \textbf{exacte} $\displaystyle\Leftrightarrow \exists h : 
	\frac{\partial h}{\partial x} = P(x,y), \frac{\partial h}{
	\partial y} = Q(x,y) \Leftrightarrow dh(x,y)=0$. Autrement dit, 
	si il existe une fonction potentielle exacte pour cette ED. 
	Cette ED peut ne pas être exacte mais en la multipliant par une 
	fonction $\mu(x,y)$ elle peut le devenir. Notre ED est alors 
	essentiellement\footnote{Si l'on rejette $\mu(x,y)=0$.} 
	équivalente à
	\begin{equation}
	\mu Pdx + \mu Pdy = 0
	\end{equation}
	où $\mu$ est un \textbf{facteur intégrant}. Bien sur, cette ED 
	doit avoir les mêmes solutions. Cette ED 
	sera \textbf{exacte} $\displaystyle\Leftrightarrow \exists h : 
	\frac{\partial h}{\partial x} = \mu P(x,y), \frac{\partial h}{
	\partial y} = \mu Q(x,y) \Leftrightarrow dh(x,y)=0$.

	\defi{$\mu$ est un facteur intégrant de 
	\begin{equation}
	F_xdx + F_ydy = 0
	\end{equation}	
	ssi $F_xdx + F_ydy$ est exacte.}\
		
	Les courbes intégrales sont $h(x,y) = c$ qui sont aussi les 
	intégrales premières de l'ED (avec et sans $\mu$). \textit{L'
	intégrale première est donc une fonction non constante qui est 
	constante le long de toute courbe intégrable.} 
	\textsc{Exemple.} Voir exemple 20.2, page 8, slide 4.
	
	\subsection{Existence et multiplicité des facteurs intégrants}
	Supposons que l'ED suivante 
	\begin{equation}
	P(x,y)dx + Q(x,y)dy = 0
	\end{equation}
	soit \textbf{non exacte}, c'est-à-dire\footnote{Notation intéressante 
	car elle permet d'effectuer simplement le rotationnel} $\nexists h(x,y) : 
	\vec \nabla h = P\vec{1_x}+Q\vec{1_y}$. Comment savoir s'il existe 
	un facteur intégrant  rendant cette ED exacte ? C'est-à-dire si il 
	$\exists h(x,y) : \vec\nabla h = (\mu P,\mu Q)$ ?
	
	\subsection{EDP des facteurs intégrants}	
	\subsubsection{a.}
	La fonction $\mu(x,y)$ est facteur intégrant de $P(x,y)dx+Q(x,y)dy = 0$
	\begin{equation}
	\begin{array}{lll}
	\Leftrightarrow &(\mu P, \mu Q) \text{ dérive d'un potentiel (
	rotationnel nul)}\\
	\Leftrightarrow & \dfrac{\partial}{\partial y}	(\mu P) = \dfrac{
	\partial}{\partial x}(\mu Q) & \text{(dans un domaine simplement 
	connexe)}\\
	\Leftrightarrow & P\dfrac{\partial\mu}{\partial y}-Q\dfrac{\partial\mu}{
	\partial x} = \left(\dfrac{\partial Q}{\partial x}-\dfrac{\partial P}{
	\partial y}\right)\mu
	\end{array}
	\end{equation}
	la dernière ligne est une \textbf{EDP linéaire du premier ordre} en 
	$\mu$ dont la résolution n'est pas simple mais considérons le cas 
	particulier très simple où $\mu(x)$ est indépendant de $y$.\\
	
	\proposition{Si $\displaystyle \left.\left(\frac{\partial Q}{\partial x}
	-\frac{\partial P}{\partial y}\right)\right/Q =: \varphi(x)$ est 
	indépendante de $y$ et si $\phi$ est une primitive de $\varphi 
	$, alors $\mu(x) := e^{-\phi(x)}$ est un F.I.}
	
	
%	st F.I. ssi
	
	\begin{proof}\ \\
	Avec un facteur 
	intégrant ne dépendant que de $x$, $\mu(x)$, l'EDP linéaire du premier 
	ordre en $\mu$ devient\footnote{$\partial \mu/\partial y = 0$ et $
	\partial\mu/\partial x = \mu'$.}
	\begin{equation}
	\mu' = -\left.\mu\left(\frac{\partial Q}{\partial x}-\frac{\partial P}{
	\partial y}\right)\right/Q
	\end{equation}
	Cette équation à variable séparée ne dépend que de $x$. On peut alors 
	écrire $\frac{\mu'}{\mu} = -\phi$ 
	\begin{equation}
	\Leftrightarrow -\mu'=\varphi\mu =-\varphi 
	\Leftrightarrow \mu(x) = e^{-\phi(x)}
	\end{equation}
	Ceci prouve qu'il existe un F.I. $\mu(x)$.
	\end{proof}
	La réciproque est vraie : si l'ED admet un F.I. $\mu(x)$ alors $(\dots)$ 
	est indépendant de $y$.
	
\setcounter{section}{1}
\setcounter{subsection}{4}
	\subsection{Les courbes intégrales vues comme courbes de niveau}
	Voir slide 10, il montre que $\vec\nabla h \perp (dx,dy)$.

\setcounter{section}{2}
\setcounter{subsection}{2}
	\subsection{EDP des facteurs intégrants (suite)}
	\subsubsection{b.}
	Si $Q\neq0$, l'équation $Q(x,y)=0$ décompose le problème $oxy$ en 
	secteurs délimités par les courbes incluses dans l'ensemble de 
	niveau $Q(x,y)=0$. Ainsi, dans chacun de ces secteurs $P(x,y)dx + 
	Q(x,y)dy = 0 \Leftrightarrow y' = -P/Q$.\\
	
	\proposition{Si $-P/Q$ est $\left\{\begin{array}{ll}
	\text{continue}\\
	\text{localement lipschitzienne en $y$}
	\end{array}\right.$, alors il existe (localement) \textbf{une} 
	solution à
	\begin{equation}
	\left\{\begin{array}{ll}
	y' &= -P/Q\\
	y(x_0) &= y_0
	\end{array}\right.
	\end{equation}
	où la dernière ligne est valable $\forall(x_0,y_0)$ dans le 
	secteur dans lequel on travail.}\ \\
	On a changé notre problème en un problème de Cauchy et utilisé 
	notre unicité locale, fournissant l'existence et l'unicité!\\
	La solution de ce problème, donc la courbe intégrale correspondante  
	est $y=y(x;x_0, y_0) = y(x;y_0)$ avec $x_0$ \textbf{fixé}! La 
	famille des courbe est $y = y(x;c)$ et la résolution par rapport à 
	$c$ donne $c=h(x,y)$ impliquant que $h$ est intégrale première.\footnote{
	A éclaircir/reformuler.}
	
	\subsubsection{c.}
	Résumons : $h(x,y)$ est intégrale première de $Pdx+Qdy=0$ ssi ses 
	courbes intégrales $\equiv h (x,y)=c$ ssi cette ED est (à un facteur 
	près)
	\begin{equation}
	\frac{\partial h}{\partial x}dx + \frac{\partial h}{\partial y}dy=0
	\end{equation}
	Soit $g \in C^1(\mathbb{R},\mathbb{R})$ avec $g'\neq0$. Multiplions 
	notre dernière équation par $g'$ :
	\begin{equation}
	g'|_{h(x,y)}\frac{\partial h}{\partial x}dx + g'|_{h(x,y)}\frac{
	\partial h}{\partial y}dy = 0
	\end{equation}
	Cette fonction n'est rien d'autre que la dérivée de la fonction 
	suivante 
	\begin{equation}
	g(h(x,y)) = c \equiv \text{ ses courbes intégrales}
	\end{equation}
	qui n'est rien d'autre que ses courbes intégrales !\\
		
	\proposition{Si $h$ est intégrale première alors $g\circ h$ est 
	intégrale 
	première}\
	
	\corollaire{S'il existe au moins une intégrale première, alors 
	il en existe 
	une infinité.}
	
\section{Lignes de champ dans $\mathbb{R}^2$ et point singulier}
	\subsection{Ligne de champ vue comme image, ensemble de niveau ou 
	graphe d'une fonction}
	Soit $\vec{F} = F_x\vec{1_x}+F_y\vec{1_y}$, un champ de vecteurs dans 
	$\mathbb{R}^2$.\\
	
	\defi{$\mathcal{C}$ est une \textbf{ligne de champ} de $\vec{F}$ ssi
	\begin{equation}
	\forall(x,y)\in\mathcal{C}, \vec{F}(x,y)\text{ est tangent à $C$ en $
	(x,y)$}
	\end{equation}}\
	
	Si $\mathcal{C}$ admet une paramétrisation $\vec{\gamma} :t\mapsto
	(x(t),y(t))$ (de classe $C^1$ telle que la dérivée $\neq 0$), alors 
	$\forall t, \exists k(t)\in\mathbb{R} : \vec{\gamma'}(t) = k(t)\vec{F}
	(\gamma(t))$. On peut écrire ceci sous la forme d'un système qui peut 
	encore donner\footnote{??} :
	\begin{equation}
	\dfrac{dy/dt}{dx/dt} = \dfrac{F_y}{F_x} 
	\end{equation}
	Considérons des $\mathcal{C}$ particuliers :
	\begin{itemize}
	\item[$\bullet$] Si $\mathcal{C} \equiv y=y(x)$
	\begin{equation}
	\frac{dy}{dx}=\frac{F_y}{F_x}
	\end{equation}
	\item[$\bullet$] Si $\mathcal{C} \equiv u(x,y) = \text{cste}$
	\begin{equation}
	\frac{dx}{F_x} = \frac{dy}{F_y}
	\end{equation}
	\end{itemize}
	On peut bien évidemment généraliser tout ceci à $\mathbb{R}^3$ :
	\begin{itemize}
	\item[$\bullet$] Paramétriques ($F_x \neq 0$)
	\begin{equation}
	\left\{\begin{array}{ll}
	\frac{dy/dt}{dx/dt} &=\frac{F_y}{F_x}\\
	\frac{dz/dt}{dx/dt} &= \frac{F_z}{F_x}
	\end{array}\right.
	\end{equation}
	\item[$\bullet$] Cartésienne (graphe)($F_x\neq0$)
	\begin{equation}
	\left\{\begin{array}{ll}
	\frac{dy}{dx} &= \frac{F_y}{F_x}\\
	\frac{dz}{dx} &= \frac{F_z}{F_x}
	\end{array}\right.
	\end{equation}	
	\item[$\bullet$] Cartésienne (ens. de niveau) ($F_x,F_y,F_z\neq0$)
	\begin{equation}
	\frac{dx}{F_x}=\frac{dy}{F_y}=\frac{dz}{F_z}\quad\Leftrightarrow\quad
	\left\{\begin{array}{ll}
	du &= 0\\
	dv &= 0	
	\end{array}\right.
	\end{equation}	
	\end{itemize}
	\textsc{Exemples.} Slide 17 et 18 et section 20.3.2.
	Comme $\mathbb{R}^3$ c'est pour les n00b, passons à $m$ dimensions!
	
\section{Lignes de champ et intégrale première dans $\mathbb{R}^m$}
	\setcounter{subsection}{2}
	\subsection{Lignes de champ : cas $mD$}
	On peut adopter quatre points de vues  pour ses lignes de champ :
	\begin{enumerate}
	\item orbites des solutions de :
	\begin{equation}
	\vec{y'} = \vec{F}(\vec{y})
	\end{equation}
	(vision vision paramétrique qui caractérise les lignes de champ de $\vec{F}$ 
	comme des orbites de solution du SD autonome en $\vec{y}(t)$)
	\item courbes intégrales $(F_m\neq0)$ :
	\begin{equation}
	\frac{dy_1}{F_1}=\dots=\frac{dy_m}{F_m}
	\end{equation}
	\item courbes intégrales de 
	\begin{equation}
	\begin{array}{lll}
	\frac{dy_1}{dy_m} &= \frac{F_1}{F_m} &=: f_i\\
	\vdots\\
	\frac{dy_{m-1}}{dy_m} &= \frac{F_{m-1}}{F_m} &=: f_{m-1}
	\end{array}
	\end{equation}
	\item courbes intégrales de 
	\begin{equation}
	\frac{d\vec{z}}{dx} = \vec{f}(x,\vec{z})
	\end{equation}
	\end{enumerate}
	\textbf{Attention section pas claire! A retravailler !}
	
	\subsection{Intégrales première ou invariant d'un SD}
	\defi{Dans les notations $\vec{z'} =\vec{f}(x,\vec{z})$, la fonction 
	$u : \mathbb{R}^m\rightarrow\mathbb{R}$, non constante, est une intégrale 
	première du SD ssi 
	\begin{equation}
	\forall \text{ sol. } \vec{\varphi}(x), \exists c \in\mathbb{R}, \forall 
	x :u(x,\vec{\varphi}(x)) = c
	\end{equation}}\ 
	
	Autrement dit, $u$ est une \textit{fonction qui reste constante tout le long 
	de toute courbe intégrale.}\\
	Pour une même valeur fixée atteinte par $u$, on a plusieurs courbe donnant 
	une hypersurface, fibrée par des courbes. L'intégrale première permet de 
	repérer une courbe intégrale mais pas de la caractériser.\\
	
	Plus généralement :\\
	\defi{Une \textbf{intégrale première d'un SD} sur $D\subseteq \mathbb{R}^m$ 
	est une fonction non constante
	\begin{equation}
	u : \mathcal{U} \subseteq D \rightarrow \mathbb{R} = \vec{y}\mapsto u(\vec{y})
	\end{equation}
	qui reste constante tout le long de \textit{toute courbe intégrale}.}\ \\
	
	La notion d'intégrale première (ou \textbf{invariant}) d'une \textbf{ED d'
	ordre supérieur} découle du fait qu'une telle ED est équivalent au SD d'ordre 
	1 en les fonctions inconnues : on peut appliquer ce que nous venons de voir.\\

	Cette fonction $u$ a pour interprétation physique d'être un invariant du 
	système (énergie totale, moment cinétique, ...). Une loi de conservation 
	implique la connaissance d'une intégrale première et est un premier par 
	vers la résolution.
	
	\setcounter{subsection}{5}
	\subsection{Fonctions indépendantes}
	\defi{$u_1,\dots,u_p$ sont \textbf{fonctionnellement indépendantes} ssi 
	aucune n'est fonction des autres localement ssi $\nexists G$, une fonction 
	non constante tel que $G(u_1,\dots,u_p)=0$.}\ \\
	
	\proposition{Si $\exists \vec{x_0} : \vec\nabla u_1|_{\vec{x_0}},\dots,
	\vec\nabla u_p|_{\vec{x_0}}$ sont linéairement indépendant, alors 
	$u_1,\dots,u_p$ sont fonctionnellement indépendant}