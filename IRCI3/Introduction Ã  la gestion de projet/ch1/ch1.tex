\chapter{Fondamental de la gestion de projet}
\section{Introduction}
Un projet a un début et une fin qui a pour objectif de délivré un résultat final.  Un projet comporte trois
grandes phases
\begin{enumerate}
\item \textit{Initialisation}. La raison d'etre, les objectifs, le timing, les risques, la plannification
\item \textit{Exécutions des taches}. Effectuer une tache $\to$ monitoring $\to$ controle $\to$ effectuer une 
tache
\item \textit{Cloture}. Délivrer résultat, mettre fin aux activités et aux contrats.
\end{enumerate}

\section{Qu'est ce que le project management}
\begin{description}
\item[Project management] planifier organiser, sécuriser et manager les ressources. C'est l'applications de 
connaissances, techniques,\dots pour que le projet arrive à sa fin
\end{description}
Sans ces méthodes, les gens ont différentes idées et le projet serait un peu décousu. Son role est un peu 
de jouer le chef d'orchestre. \\

Un projet mature se compose d'un portfolio, un programme et de projets. 
\begin{description}
\item[Portfolio] Ensemble de projets ou de programmes qui sont groupés ensemble pour faciliter leur 
gestion.
\item[Programme] Groupe de projet commun qui son coordonné ensemble pour avoir des bénéfices qui ne seraient 
pas atteignables si on les gérait seuls.
\end{description}
Un projet peut faire partie d'un programme, mais forcéement un programme se compose toujours de projet. Pour 
etre efficace dans le gestion de projet, le project manager doit avoir trois grande caractéristiques
\begin{enumerate}
\item \textit{Connaissance}. Il connait le management
\item \textit{Performance}. Il sait ce qu'il faut faire pour vaincre
\item \textit{Personalité}. Son leadership, son abilité à diriger une équipe,\dots
\end{enumerate}

Un projet \textbf{traditionnel} consiste en l'accomplissement d'une série d'étape (phased approach)
\begin{itemize}
\item Initialisation du projet
\item Planning du projet  et conception
\item Exécution
\item Monoritong et contrôle
\item Fin du projet
\end{itemize}

Il existe d'autre approches, qui dépendent du projet
\begin{description}
\item[Chaine critique] : méthode d'organisation qui met l'accent sur les ressources nécessaire pour exécuter un projet
\item[Chemin critique] : méthode du chemin critique, on y reviendra
\item[Agile] : basé sur le principe "human interaction management". Cette méthode se fonde sur la collaboration entre humain ce qui contraste  nettement avec l'approche traditionnelle. Le projet est ainsi vu comme une série de petite taches à exécuter sur le moment que comme un gros projet déjà tout planifier.\\
Wiki : Le management agile peut être vu comme une organisation de type holistique et humaniste basée essentiellement sur la motivation rationnelle des ressources humaines. Son émergence, au début des années 1990, a été portée par la vague des nouvelles technologies (NTIC).\\

Ses valeurs et principes combinent des aspects sociologiques et technologiques à une approche industrielle1. Le management Agile s’oppose aux fondements du taylorisme : parcellisation du travail, déresponsabilisation globale ainsi que d'autres principes de productivité individuelle dont la mise en œuvre devient difficilement défendable dans les pays industrialisés, compte tenu du coût des ressources humaines2 .
\end{description}

Le \textbf{cycle de vie du projet} est souvent séquentiel. Souvent, on retrouve la structure suivante et ce
malgré les grosses différences de tailles/complexité entre projet
\begin{itemize}
\item Commencement du projet
\item Organisation et préparation
\item Réalisation du projet, travail
\item Finalisation
\end{itemize}



\subsection{Project phase}
Il s'agit de division dans un projet ou un controle supplémentaire est nécessaire. La structure par 
phase d'un projet suggère en effet qu'on puisse le segmenter en morceau afin d'avoir plus de liberté de 
controle, management, ...


\chapter{Intiating \& planning}
Avant de commencer, il est important de consacrer du temps pour brainstormer et répondre à certaines 
questions :
\begin{description}
\item[Pourquoi?] Pourquoi faire ce projet ? Il faut s'assurer à intervalle régulier ce que ce que l'on 
fait à du sens
\item[Quoi?] Les objectifs et surtout le respect des contraintes
\item[Qui?] Besoin d'expert ? Définit le type d'organisation (voir après)
\item[Comment et quand?] Faire un plan, plannifier
\end{description}
Cette initialisation est \textbf{critique} pour que les objectifs du projet soient réalisés. Il faut 
en effet que les objectifs répondent au cadre, au budget et au timing. Il faut s'assurer de certaines choses :
\begin{itemize}
\item Analyser le besoin et les objectifs
\item Contextualiser la demande par rapport à la situation actuelle
\item Analyser les couts et bénéfices, les inclure
\item Faire attention au personnel (délocaliser, c'est se mettre les syndicats à dos)
\item Inclure des milestones (étapes importantes) date de remise, coûts, ... 
\end{itemize}

\subsection{Functional team structure}
En trois partie
\begin{enumerate}
\item \textit{Engineering} Conception du produit, des outils pour la chaine de prod, ...
 \item \textit{manufacturing} Ouvrier
 \item \textit{Marketing} Commecrieux, fait rentrer de l'argent, trouver des gens qui vont échanger argent contre poduit
\end{enumerate}
On a ainsi divisé le projet en trois unités fonctionelle, chacune d'elle avait un chef mais c'était 
problématique pour gérer et coordiner tout, chacun avait ses prores envies
\subsection{Lightweight team structure}
Le chef de projet est assez jeune (junior) et n'est dans aucune des unités fonctionnelle, il va travailler avec 
des agent de liaison qui sont dans chacunes des unités.  Le project manager n'a pas d'allocution de pouvoir, 
il ne peut \textbf{pas} obliger les gens à travailler mais il peut les motiver. Son rôle est d'influence mais ce n'est pas facile car il n'a aucun pouvoir


\subsection{Heavyweight team strucyture}
Ici le chef de projet est un senior qui est reconnu, il a de l'influence sur les ressources de chaque département. Il bosse avec une équipe de chaque département. Il existe une intégrité plus forte pour arriver aux objectifs. Ses "équipes de liaisons" sont constituées de gens qui ne sont pas les plus mauvais, qui sont motivés et intelligents. Le chef de projet a ici le controle primaire sur les différents département

\subsection{Autonomous team structure}
On prend les meilleurs de chaque département et on les fait bosser sur des trucs. On a des gens super forts ensemble : meilleurs compétances, capacité de focalisation, ... Mais il faut mettre des lignes directrices fortes sinon il n'en font qu'à leur tete. L'avantage de prendre les meilleurs est qu'ils auront un focus sur l'objectif et seront pas distrait. Le problème c'est qu'apres ils veulement plus retourner au cas d'avant. C'est ce que faisait Jobs. 


\subsection{Réunion}
Il faut sécuriser T, C et Q (temps cout et qualité) dans un environnement complexe. Ca passe par une gestion des réunions. Pour que tout se passe bien, il faut que
\begin{enumerate}
\item Un chef. Il s'occupe du cadrage et dit si on sort du cadre. 
\item Un secrétaire. Il prend notes des conclusion, décisions et actions
\item Un facilitateur. S'assure que le déroulement de la réunion est assez fluide et que quelqu'un ne prend 
pas trop de place
\item Time keeper. En accord avec le facilitateur et le chef de projet, s'assure que le timing est respecter
\end{enumerate}
Pour que ça fonctionne 	il faut choisir les bonnes personnes. Il faut s'assurer que le rapport du secrétaire est sorti dans les 24h. 


\subsection{Critical path}
On a un début une fin et un ensemble d'activité à faire, chemins à respectés



































